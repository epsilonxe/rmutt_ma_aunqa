\chapter{ภาคผนวก}

\section{ข้อมูลพื้นฐานหลักสูตร (Commom Data Set)}
\begin{longtable}{|c|p{0.7\textwidth}|c|}
\caption{ข้อมูลพื้นฐานหลักสูตร (Common Data Set)\hspace*{5.8cm}}\\
\hline
	\cellcolor{gray!50}{\textbf{ลำดับ}}&\multicolumn{1}{c|}{\cellcolor{gray!50}{	\textbf{ชื่อข้อมูลพื้นฐาน}}}&	\cellcolor{gray!50}{\textbf{CdsValues}}\\\hline
	\endfirsthead
	
	\caption[]{(ต่อ) ข้อมูลพื้นฐานหลักสูตร (Common Data Set)}
	\\
	\hline
	\cellcolor{gray!50}{\textbf{ลำดับ}}&\multicolumn{1}{c|}{\cellcolor{gray!50}{	\textbf{ชื่อข้อมูลพื้นฐาน}}}&	\cellcolor{gray!50}{\textbf{CdsValues}}\\\hline
	\endhead
	
	\multicolumn{3}{|l|}{\textbf{ชุดข้อมูลที่ 1}}\\\hline
	1&\cellcolor{red!10}{จำนวนหลักสูตรที่เปิดสอนทั้งหมด}&\cellcolor{red!10}{\textbf{1}}\\\hline
	2&-ระดับปริญญาตรี& 1\\\hline
	3&- ระดับ ป.บัณฑิต& -\\\hline
	4&- ระดับปริญญาโท& -\\\hline
	5&- ระดับ ป.บัณฑิตขั้นสูง& -\\\hline
	6&- ระดับปริญญาเอก& -\\\hline
	
	\multicolumn{3}{|l|}{\textbf{ชุดข้อมูลที่ 2}}\\\hline
	7&\cellcolor{red!10}{จำนวนหลักสูตรที่จัดการเรียนการสอนนอกสถานที่ตั้ง}& \cellcolor{red!10}{\textbf{-}}\\\hline
		8&-ระดับปริญญาตรี& -\\\hline
	9&- ระดับ ป.บัณฑิต& -\\\hline
	10&- ระดับปริญญาโท& -\\\hline
	11&- ระดับ ป.บัณฑิตขั้นสูง& -\\\hline
	12&- ระดับปริญญาเอก& -\\\hline
	
		
	\multicolumn{3}{|l|}{\textbf{ชุดข้อมูลที่ 3}}\\\hline
	13&\cellcolor{red!10}{จำนวนนักศึกษาปัจจุบันทั้งหมดทุกระดับการศึกษา}&  \cellcolor{red!10}{\textbf{59}}\\\hline
	14&- จำนวนนักศึกษาปัจจุบันทั้งหมด – ระดับปริญญาตรี&59\\\hline
	15&- จำนวนนักศึกษาปัจจุบันทั้งหมด - ระดับ ป.บัณฑิต& -\\\hline
	16&- จำนวนนักศึกษาปัจจุบันทั้งหมด – ระดับปริญญาโท& -\\\hline
	17&- จำนวนนักศึกษาปัจจุบันทั้งหมด - ระดับ ป.บัณฑิตขั้นสูง& -\\\hline
	18&- จำนวนนักศึกษาปัจจุบันทั้งหมด - ระดับปริญญาเอก & -\\\hline
	
	\multicolumn{3}{|l|}{\textbf{ชุดข้อมูลที่ 4}}\\\hline
	19&\cellcolor{red!10}{จำนวนอาจารย์ประจำทั้งหมด รวมทั้งที่ปฏิบัติงานจริงและลาศึกษาต่อ}&\cellcolor{red!10}{\textbf{18}}\\\hline
	20&-จำนวนอาจารย์ประจำทั้งหมดที่ปฏิบัติงานจริงและลาศึกษาต่อ\newline วุฒิปริญญาตรีหรือเทียบเท่า &-\\\hline
	21&-จำนวนอาจารย์ประจำทั้งหมดที่ปฏิบัติงานจริงและลาศึกษาต่อ\newline วุฒิปริญญาโทหรือเทียบเท่า& 9 \\\hline
	22&-จำนวนอาจารย์ประจำทั้งหมดที่ปฏิบัติงานจริงและลาศึกษาต่อ\newline วุฒิปริญญาเอกหรือเทียบเท่า & 9\\\hline
	
	23&\cellcolor{red!10}{จำนวนอาจารย์ประจำทั้งหมดที่ดำรงตำแหน่งอาจารย์}&\cellcolor{red!10}{\textbf{10}}\\\hline
	24&-จำนวนอาจารย์ประจำ (ที่ไม่มีตำแหน่งทางวิชาการ) ที่มีวุฒิปริญญาตรี หรือเทียบเท่า &-\\\hline
	25&-จำนวนอาจารย์ประจำ (ที่ไม่มีตำแหน่งทางวิชาการ) ที่มีวุฒิปริญญาโท หรือเทียบเท่า& 6 \\\hline
	26&-จำนวนอาจารย์ประจำ (ที่ไม่มีตำแหน่งทางวิชาการ) ที่มีวุฒิปริญญาเอก หรือเทียบเท่า & 4\\\hline
	
	27&\cellcolor{red!10}{จำนวนอาจารย์ประจำทั้งหมดที่ดำรงตำแหน่งผู้ช่วยศาสตราจารย์}&\cellcolor{red!10}{\textbf{7}}\\\hline
		28&-จำนวนอาจารย์ประจำตำแหน่งผู้ช่วยศาสตราจารย์ ที่มีวุฒิปริญญาตรี หรือเทียบเท่า &-\\\hline
	29&-จำนวนอาจารย์ประจำตำแหน่งผู้ช่วยศาสตราจารย์ ที่มีวุฒิปริญญาโท หรือเทียบเท่า& 3 \\\hline
	30&-จำนวนอาจารย์ประจำตำแหน่งผู้ช่วยศาสตราจารย์ ที่มีวุฒิปริญญาเอก หรือเทียบเท่า & 4\\\hline
	
	31&\cellcolor{red!10}{จำนวนอาจารย์ประจำทั้งหมดที่ดำรงตำแหน่งรองศาสตราจารย์}&\cellcolor{red!10}{\textbf{1}}\\\hline
	32&-จำนวนอาจารย์ประจำตำแหน่งรองศาสตราจารย์ ที่มีวุฒิปริญญาตรี หรือเทียบเท่า &-\\\hline
	33&-จำนวนอาจารย์ประจำตำแหน่งรองศาสตราจารย์ ที่มีวุฒิปริญญาโท หรือเทียบเท่า& - \\\hline
	34&-จำนวนอาจารย์ประจำตำแหน่งรองศาสตราจารย์ ที่มีวุฒิปริญญาเอก หรือเทียบเท่า & 1\\\hline
	
	35&\cellcolor{red!10}{จำนวนอาจารย์ประจำทั้งหมดที่ดำรงตำแหน่งศาสตราจารย์}&\cellcolor{red!10}{\textbf{-}}\\\hline
	36&-จำนวนอาจารย์ประจำตำแหน่งศาสตราจารย์ ที่มีวุฒิปริญญาตรี หรือเทียบเท่า &-\\\hline
	37&-จำนวนอาจารย์ประจำตำแหน่งศาสตราจารย์ ที่มีวุฒิปริญญาโท หรือเทียบเท่า& -\\\hline
	38&-จำนวนอาจารย์ประจำตำแหน่งศาสตราจารย์ ที่มีวุฒิปริญญาเอก หรือเทียบเท่า & -\\\hline
	
	\multicolumn{3}{|l|}{\textbf{ชุดข้อมูลที่ 5}}\\\hline
	39&\cellcolor{red!10}{จำนวนอาจารย์ผู้รับผิดชอบหลักสูตรแยกตามวุฒิการศึกษา}&\cellcolor{red!10}{\textbf{5}}\\\hline
	40&-ระดับปริญญาตรี&-\\\hline
	41&-ระดับ ป.บัณฑิต& - \\\hline
	42&-ระดับปริญญาโท & 2\\\hline
	43&-ระดับ ป.บัณฑิตขั้นสูง& - \\\hline
	44&-ระดับปริญญาเอก& 3  - \\\hline
	
	45&\cellcolor{red!10}{จำนวนอาจารย์ผู้รับผิดชอบหลักสูตรที่ดำรงตำแหน่งทางวิชาการ}&\cellcolor{red!10}{\textbf{5}}\\\hline
	46&-จำนวนอาจารย์ประจำหลักสูตรที่ไม่มีตำแหน่งทางวิชาการ&1\\\hline
	47&-จำนวนอาจารย์ประจำหลักสูตรที่มีตำแหน่งผู้ช่วยศาสตราจารย์& 3 \\\hline
	48&-จำนวนอาจารย์ประจำหลักสูตรที่มีตำแหน่งรองศาสตราจารย์ & 1\\\hline
	49&-จำนวนอาจารย์ประจำหลักสูตรที่มีตำแหน่งศาสตราจารย์& - \\\hline
	
	\multicolumn{3}{|l|}{\textbf{ชุดข้อมูลที่ 6}}\\\hline
	50&\cellcolor{red!10}{จำนวนรวมของผลงานทางวิชาการของอาจารย์ประจำหลักสูตร}&\cellcolor{red!10}{\textbf{11}}\\\hline
	51&-บทความวิจัยหรือบทความวิชาการฉบับสมบูรณ์ที่ตีพิมพ์ในรายงานสืบเนื่องจากการประชุมวิชาการระดับชาติ&-\\\hline
	52&-บทความฉบับสมบูรณ์ที่ตีพิมพ์ในรายงานสืบเนื่องจากการประชุมวิชาการระดับนานาชาติ หรือในวารสารทางวิชาการระดับชาติที่ไม่อยู่ในฐานข้อมูล ตามประกาศ ก.พ.อ. หรือระเบียบคณะกรรมการการอุดมศึกษาว่าด้วย หลักเกณฑ์การพิจารณาวารสารทางวิชาการสำหรับการเผยแพร่ผลงานทางวิชาการ พ.ศ.2556 แต่สถาบันนำเสนอสภาสถาบันอนุมัติและจัดทำเป็นประกาศให้ทราบเป็นการทั่วไป และแจ้งให้ กพอ./กกอ.ทราบภายใน 30 วันนับแต่วันที่ออกประกาศ
	& - \\\hline
	53&-ผลงานที่ได้รับการจดอนุสิทธิบัตร & -\\\hline
	54&-บทความวิจัยหรือบทความวิชาการที่ตีพิมพ์ในวารสารวิชาการที่ปรากฏในฐานข้อมูล TCI กลุ่มที่ 2& - \\\hline
	55&-บทความวิจัยหรือบทความวิชาการที่ตีพิมพ์ในวารสารวิชาการระดับนานาชาติที่ไม่อยู่ในฐานข้อมูล ตามประกาศ ก.พ.อ. หรือระเบียบคณะกรรมการการอุดมศึกษาว่าด้วย หลักเกณฑ์การพิจารณาวารสารทางวิชาการสำหรับการเผยแพร่ผลงานทางวิชาการ พ.ศ.2556 แต่สถาบันนำเสนอสภาสถาบันอนุมัติและจัดทำเป็นประกาศให้ทราบเป็นการทั่วไป และแจ้งให้  กพอ./กกอ.ทราบภายใน 30 วัน นับแต่วันที่ออกประกาศ (ซึ่งไม่อยู่ใน Beall’s list) หรือตีพิมพ์ในวารสารวิชาการที่ปรากฏ ในฐานข้อมูล TCI กลุ่มที่ 1& 1 \\\hline
	56&-บทความวิจัยหรือบทความวิชาการที่ตีพิมพ์ในวารสารวิชาการระดับนานาชาติที่ปรากฏในฐานข้อมูลระดับนานาชาติตามประกาศ ก.พ.อ. หรือระเบียบคณะกรรมการการอุดมศึกษา ว่าด้วย หลักเกณฑ์การพิจารณาวารสารทางวิชาการสำหรับการเผยแพร่ผลงานทางวิชาการ พ.ศ.2556 & 10 \\\hline
	57&-ผลงานได้รับการจดสิทธิบัตร& - \\\hline
	58&-ผลงานวิชาการรับใช้สังคมที่ได้รับการประเมินผ่านเกณฑ์การขอตำแหน่งทางวิชาการแล้ว& - \\\hline
	59&-ผลงานวิจัยที่หน่วยงานหรือองค์กรระดับชาติว่าจ้างให้ดำเนินการ& - \\\hline
	60&-ผลงานค้นพบพันธุ์พืช พันธุ์สัตว์ ที่ค้นพบใหม่และได้รับการจดทะเบียน&-\\\hline
	61&-ตำราหรือหนังสือหรืองานแปลที่ได้รับการประเมินผ่านเกณฑ์การขอตำแหน่งทางวิชาการแล้ว&-\\\hline
	62&-ตำราหรือหนังสือหรืองานแปลที่ผ่านการพิจารณาตามหลักเกณฑ์การประเมินตำแหน่งทางวิชาการแต่ไม่ได้นำมาขอรับการประเมินตำแหน่งทางวิชาการ&-\\\hline
	63&-จำนวนงานสร้างสรรค์ที่มีการเผยแพร่สู่สาธารณะในลักษณะใดลักษณะหนึ่ง หรือผ่านสื่ออิเลคทรอนิกส์ online&-\\\hline
	64&-จำนวนงานสร้างสรรค์ที่ได้รับการเผยแพร่ในระดับสถาบัน&-\\\hline
	65&-จำนวนงานสร้างสรรค์ที่ได้รับการเผยแพร่ในระดับชาติ&-\\\hline
	66&-จำนวนงานสร้างสรรค์ที่ได้รับการเผยแพร่ในระดับความร่วมมือระหว่างประเทศ&-\\\hline
	67&-จำนวนงานสร้างสรรค์ที่ได้รับการเผยแพร่ในระดับภูมิภาคอาเซียน&-\\\hline
	68&-จำนวนงานสร้างสรรค์ที่ได้รับการเผยแพร่ในระดับนานาชาติ &-\\\hline
	69&-จำนวนบทความของอาจารย์ประจำหลักสูตรปริญญาเอกที่ได้รับการอ้างอิงในฐานข้อมูล TCI และ Scopus ต่อจำนวนอาจารย์ประจำหลักสูตร &2.2\\\hline
	
	\multicolumn{3}{|l|}{\textbf{ชุดข้อมูลที่ 7}}\\\hline
	70&จำนวนบัณฑิตระดับปริญญาตรีทั้งหมด& 13\\\hline
	71&จำนวนบัณฑิตระดับปริญญาตรีที่ตอบแบบสำรวจเรื่องการมีงานทำภายใน 1 ปี หลังสำเร็จการศึกษา& 10\\\hline
	72&จำนวนบัณฑิตระดับปริญญาตรีที่ได้งานทำหลังสำเร็จการศึกษา (ไม่นับรวมผู้ที่ประกอบอาชีพอิสระ)& 10\\\hline
	73&จำนวนบัณฑิตระดับปริญญาตรีที่ประกอบอาชีพอิสระ &-\\\hline
	74&จำนวนผู้สำเร็จการศึกษาระดับปริญญาตรีที่มีงานทำก่อนเข้าศึกษา &-\\\hline
	75&จำนวนบัณฑิตระดับปริญญาตรีที่มีกิจการของตนเองที่มีรายได้ประจำอยู่แล้ว&-\\\hline
	76&จำนวนบัณฑิตระดับปริญญาตรีที่ศึกษาต่อระดับบัณฑิตศึกษา&-\\\hline
	77&จำนวนบัณฑิตระดับปริญญาตรีที่อุปสมบท&-\\\hline
	78&จำนวนบัณฑิตระดับปริญญาตรีที่เกณฑ์ทหาร&-\\\hline
	79&เงินเดือนหรือรายได้ต่อเดือน ของผู้สำเร็จการศึกษาระดับปริญญาตรีที่ได้งานทำหรือประกอบอาชีพอิสระ (ค่าเฉลี่ย)&-\\\hline
	80&ผลการประเมินจากความพึงพอใจของนายจ้างที่มีต่อผู้สำเร็จการศึกษาระดับปริญญาตรีตามกรอบ TQF เฉลี่ย (คะแนนเต็ม 5)& 4.67 \\\hline
	
	\multicolumn{3}{|l|}{\textbf{ชุดข้อมูลที่ 8}}\\\hline
	81&\cellcolor{red!10}{จำนวนรวมของผลงานนักศึกษาและผู้สำเร็จการศึกษาในระดับปริญญาโทที่ได้รับการตีพิมพ์หรือเผยแพร่}&\cellcolor{red!10}{\textbf{-}}\\\hline
	82&-จำนวนบทความฉบับสมบูรณ์ที่มีการตีพิมพ์ในลักษณะใดลักษณะหนึ่ง &-\\\hline
	83&-จำนวนบทความฉบับสมบูรณ์ที่ตีพิมพ์ในรายงานสืบเนื่องจากการประชุมวิชาการระดับชาติ&-\\\hline
	84&-จำนวนบทความฉบับสมบูรณ์ที่ตีพิมพ์ในรายงานสืบเนื่องจากการประชุมวิชาการระดับนานาชาติ หรือในวารสารทางวิชาการระดับชาติที่ไม่อยู่ในฐานข้อมูลตามประกาศ ก.พ.อ. หรือระเบียบคณะกรรมการอุดมศึกษาว่าด้วยหลักเกณฑ์การพิจารณาวารสารทางวิชาการว่าด้วยหลักเกณฑ์การพิจารณาวารสารทางวิชาการสำหรับการเผยแพร่ผลงานทางวิชาการ พ.ศ.2556 แต่สถาบันนำเสนอสภาสถาบันอนุมัติและจัดทำเป็นประกาศให้ทราบทั่วไปและแจ้ง ก.พ.อ./กกอ. ทราบภายใน 30 วัน  นับแต่วันที่ออกประกาศ&-\\\hline
	85&-ผลงานที่ได้รับการจดอนุสิทธิบัตร&-\\\hline
	86&-จำนวนบทความที่ตีพิมพ์ในวารสารวิชาการที่ปรากฏในฐานข้อมูล TCI กลุ่มที่ 2&-\\\hline
	87&-จำนวนบทความที่ตีพิมพ์ในวารสารวิชาการระดับนานาชาติ ที่ไม่อยู่ในฐานข้อมูลตามประกาศ ก.พ.อ.หรือระเบียบคณะกรรมการอุดมศึกษาว่าด้วยหลักเกณฑ์การพิจารณาวารสารทางวิชาการว่าด้วยหลักเกณฑ์การพิจารณาวารสารทางวิชาการสำหรับการเผยแพร่ผลงานทางวิชาการ พ.ศ.2556 แต่สถาบันนำเสนอสภาสถาบันอนุมัติและจัทำเป็นประกาศให้ทราบทั่วไปและแจ้ง ก.พ.อ./กกอ. ทราบภายใน 30 วัน  นับแต่วันที่ออกประกาศ (ซึ่งไม่อยู่ใน Beall's list) หรือตีพิมพ์ในวารสารวิชาการ ที่ปรากฏในฐานข้อมูล TCI กลุ่มที่ 1&-\\\hline
	88&-จำนวนบทความที่ตีพิมพ์ในวารสารวิชาการระดับนานาชาติ ที่ปรากฏอยู่ในฐานข้อมูลระดับนานานชาติตามประกาศ ก.พ.อ.หรือระเบียบคณะกรรมการอุดมศึกษาว่าด้วยหลักเกณฑ์การพิจารณาวารสารทางวิชาการว่าด้วยหลักเกณฑ์การพิจารณาวารสารทางวิชาการสำหรับการเผยแพร่ผลงานทางวิชาการ พ.ศ.2556&-\\\hline
	89&-ผลงานที่ได้รับการจดสิทธิบัตร&-\\\hline
	90&-จำนวนงานสร้างสรรค์ที่มีการเผยแพร่สู่สาธารณะในลักษณะใดลักษณะหนึ่ง หรือผ่านสื่ออิเลคทรอนิกส์ online&-\\\hline
	91&-จำนวนงานสร้างสรรค์ที่ได้รับการเผยแพร่ในระดับสถาบัน&-\\\hline
	92&-จำนวนงานสร้างสรรค์ที่ได้รับการเผยแพร่ในระดับชาติ&-\\\hline
	93&-จำนวนงานสร้างสรรค์ที่ได้รับการเผยแพร่ในระดับความร่วมมือระหว่างประเทศ&-\\\hline
	94&-จำนวนงานสร้างสรรค์ที่ได้รับการเผยแพร่ในระดับภูมิภาคอาเซียน&-\\\hline
	95&-จำนวนงานสร้างสรรค์ที่ได้รับการเผยแพร่ในระดับนานาชาติ &-\\\hline
	96&จำนวนผู้สำเร็จการศึกษาระดับปริญญาโททั้งหมด (ปีการศึกษาที่เป็นวงรอบประเมิน)&-\\\hline
	
	\multicolumn{3}{|l|}{\textbf{ชุดข้อมูลที่ 9}}\\\hline
	97&\cellcolor{red!10}{จำนวนรวมของผลงานนักศึกษาและผู้สำเร็จการศึกษาในระดับปริญญาเอกที่ได้รับการตีพิมพ์หรือเผยแพร่}&\cellcolor{red!10}{\textbf{-}}\\\hline
	98&-จำนวนบทความฉบับสมบูรณ์ที่ตีพิมพ์ในรายงานสืบเนื่องจากการประชุมวิชาการระดับชาติ&-\\\hline
	99&-จำนวนบทความฉบับสมบูรณ์ที่ตีพิมพ์ในรายงานสืบเนื่องจากการประชุมวิชาการระดับนานาชาติ หรือในวารสารทางวิชาการระดับชาติที่ไม่อยู่ในฐานข้อมูลตามประกาศ ก.พ.อ. หรือระเบียบคณะกรรมการอุดมศึกษาว่าด้วยหลักเกณฑ์การพิจารณาวารสารทางวิชาการว่าด้วยหลักเกณฑ์การพิจารณาวารสารทางวิชาการสำหรับการเผยแพร่ผลงานทางวิชาการ พ.ศ.2556 แต่สถาบันนำเสนอสภาสถาบันอนุมัติและจัดทำเป็นประกาศให้ทราบทั่วไปและแจ้ง ก.พ.อ./กกอ. ทราบภายใน 30 วัน  นับแต่วันที่ออกประกาศ&-\\\hline
	100&-ผลงานที่ได้รับการจดอนุสิทธิบัตร&-\\\hline
	101&-จำนวนบทความที่ตีพิมพ์ในวารสารวิชาการที่ปรากฏในฐานข้อมูล TCI กลุ่มที่ 2&-\\\hline
	102&-จำนวนบทความที่ตีพิมพ์ในวารสารวิชาการระดับนานาชาติ ที่ไม่อยู่ในฐานข้อมูลตามประกาศ ก.พ.อ.หรือระเบียบคณะกรรมการอุดมศึกษาว่าด้วยหลักเกณฑ์การพิจารณาวารสารทางวิชาการว่าด้วยหลักเกณฑ์การพิจารณาวารสารทางวิชาการสำหรับการเผยแพร่ผลงานทางวิชาการ พ.ศ.2556 แต่สถาบันนำเสนอสภาสถาบันอนุมัติและจัทำเป็นประกาศให้ทราบทั่วไปและแจ้ง ก.พ.อ./กกอ. ทราบภายใน 30 วัน  นับแต่วันที่ออกประกาศ (ซึ่งไม่อยู่ใน Beall's list) หรือตีพิมพ์ในวารสารวิชาการ ที่ปรากฏในฐานข้อมูล TCI กลุ่มที่ 1&-\\\hline
	103&-จำนวนบทความที่ตีพิมพ์ในวารสารวิชาการระดับนานาชาติ ที่ปรากฏอยู่ในฐานข้อมูลระดับนานานชาติตามประกาศ ก.พ.อ.หรือระเบียบคณะกรรมการอุดมศึกษาว่าด้วยหลักเกณฑ์การพิจารณาวารสารทางวิชาการว่าด้วยหลักเกณฑ์การพิจารณาวารสารทางวิชาการสำหรับการเผยแพร่ผลงานทางวิชาการ พ.ศ.2556&-\\\hline
	104&-ผลงานที่ได้รับการจดสิทธิบัตร&-\\\hline
	105&-จำนวนงานสร้างสรรค์ที่มีการเผยแพร่สู่สาธารณะในลักษณะใดลักษณะหนึ่ง หรือผ่านสื่ออิเลคทรอนิกส์ online&-\\\hline
	106&-จำนวนงานสร้างสรรค์ที่ได้รับการเผยแพร่ในระดับสถาบัน&-\\\hline
	107&-จำนวนงานสร้างสรรค์ที่ได้รับการเผยแพร่ในระดับชาติ&-\\\hline
	108&-จำนวนงานสร้างสรรค์ที่ได้รับการเผยแพร่ในระดับความร่วมมือระหว่างประเทศ&-\\\hline
	109&-จำนวนงานสร้างสรรค์ที่ได้รับการเผยแพร่ในระดับภูมิภาคอาเซียน&-\\\hline
	110&-จำนวนงานสร้างสรรค์ที่ได้รับการเผยแพร่ในระดับนานาชาติ &-\\\hline
	111&จำนวนผู้สำเร็จการศึกษาระดับปริญญาเอกทั้งหมด (ปีการศึกษาที่เป็นวงรอบประเมิน)&-\\\hline
	
	\multicolumn{3}{|l|}{\textbf{ชุดข้อมูลที่ 10}}\\\hline
	112&\cellcolor{red!10}{จำนวนนักศึกษาเต็มเวลาเทียบเท่า (FTES) รวมทุกหลักสูตร}&\cellcolor{red!10}{\textbf{-}}\\\hline
	113&-ระดับอนุปริญญา&-\\\hline
	114&-ระดับปริญญาตรี&-\\\hline
	115&-ระดับ ป.บัณฑิต&-\\\hline
	116&-ระดับปริญญาโท&-\\\hline
	117&-ระดับ ป.บัณฑิตขั้นสูง&-\\\hline
	118&-ระดับปริญญาเอก&-\\\hline
	
	\multicolumn{3}{|l|}{\textbf{ชุดข้อมูลที่ 11}}\\\hline
	119&\cellcolor{red!10}{จำนวนเงินสนับสนุนงานวิจัยหรืองานสร้างสรรค์จากภายในสถาบัน}&\cellcolor{red!10}{}\\\hline
	120&-กลุ่มสาขาวิชาวิทยาศาสตร์และเทคโนโลยี&-\\\hline
	121&-กลุ่มสาขาวิชาวิทยาศาสตร์สุขภาพ&-\\\hline
	122&-กลุ่มสาขาวิชามนุษยศาสตร์และสังคมศาสตร์&-\\\hline
	
	123&\cellcolor{red!10}{จำนวนเงินสนับสนุนงานวิจัยหรืองานสร้างสรรค์จากภายนอกสถาบัน}&\cellcolor{red!10}{}\\\hline
	124&-กลุ่มสาขาวิชาวิทยาศาสตร์และเทคโนโลยี&-\\\hline
	125&-กลุ่มสาขาวิชาวิทยาศาสตร์สุขภาพ&-\\\hline
	126&-กลุ่มสาขาวิชามนุษยศาสตร์และสังคมศาสตร์&-\\\hline
	
	127&\cellcolor{red!10}{จำนวนอาจารย์ประจำที่ปฏิบัติงานจริง (ไม่นับรวมผู้ลาศึกษาต่อ)}&\cellcolor{red!10}{}\\\hline
	128&-กลุ่มสาขาวิชาวิทยาศาสตร์และเทคโนโลยี&-\\\hline
	129&-กลุ่มสาขาวิชาวิทยาศาสตร์สุขภาพ&-\\\hline
	130&-กลุ่มสาขาวิชามนุษยศาสตร์และสังคมศาสตร์&-\\\hline
	
	131&\cellcolor{red!10}{จำนวนนักวิจัยประจำที่ปฏิบัติงานจริง (ไม่นับรวมผู้ลาศึกษาต่อ)}&\cellcolor{red!10}{\textbf{-}}\\\hline
	132&-กลุ่มสาขาวิชาวิทยาศาสตร์และเทคโนโลยี&-\\\hline
	133&-กลุ่มสาขาวิชาวิทยาศาสตร์สุขภาพ&-\\\hline
	134&-กลุ่มสาขาวิชามนุษยศาสตร์และสังคมศาสตร์&-\\\hline
	
	135&\cellcolor{red!10}{จำนวนอาจารย์ประจำที่ลาศึกษาต่อ}&\cellcolor{red!10}{\textbf{1}}\\\hline
	136&-กลุ่มสาขาวิชาวิทยาศาสตร์และเทคโนโลยี&-\\\hline
	137&-กลุ่มสาขาวิชาวิทยาศาสตร์สุขภาพ&-\\\hline
	138&-กลุ่มสาขาวิชามนุษยศาสตร์และสังคมศาสตร์&-\\\hline
	
	139&\cellcolor{red!10}{จำนวนอาจารย์ประจำที่ลาศึกษาต่อ}&\cellcolor{red!10}{\textbf{1}}\\\hline
	140&-กลุ่มสาขาวิชาวิทยาศาสตร์และเทคโนโลยี&-\\\hline
	141&-กลุ่มสาขาวิชาวิทยาศาสตร์สุขภาพ&-\\\hline
	142&-กลุ่มสาขาวิชามนุษยศาสตร์และสังคมศาสตร์&-\\\hline
	
	\multicolumn{3}{|l|}{\textbf{ชุดข้อมูลที่ 12}}\\\hline
	143&\cellcolor{red!10}{บทความวิจัยหรือบทความวิชาการฉบับสมบูรณ์ที่ตีพิมพ์ในรายงานสืบเนื่องจากการประชุมวิชาการระดับชาติ}&\cellcolor{red!10}{}\\\hline
	144&-กลุ่มสาขาวิชาวิทยาศาสตร์และเทคโนโลยี&-\\\hline
	145&-กลุ่มสาขาวิชาวิทยาศาสตร์สุขภาพ&-\\\hline
	146&-กลุ่มสาขาวิชามนุษยศาสตร์และสังคมศาสตร์&-\\\hline
	
	147&\cellcolor{red!10}{บทความวิจัยหรือบทความวิชาการฉบับสมบูรณ์ที่ตีพิมพ์ในรายงานสืบเนื่องจากการประชุมวิชาการระดับนานาชาติ หรือในวารสารทางวิชาการระดับชาติที่ไม่อยู่ในฐานข้อมูล 
		ตามประกาศ ก.พ.อ. หรือระเบียบคณะกรรมการการอุดมศึกษาว่าด้วย หลักเกณฑ์การพิจารณาวารสารทางวิชาการสำหรับการเผยแพร่ผลงานทางวิชาการ พ.ศ.2556 แต่สถาบันนำเสนอสภาสถาบันอนุมัติและจัดทำเป็นประกาศให้ทราบเป็นการทั่วไป และแจ้งให้ กพอ./กกอ.ทราบภายใน 30 วันนับแต่วันที่ออกประกาศ
	}&\cellcolor{red!10}{}\\\hline
	148&-กลุ่มสาขาวิชาวิทยาศาสตร์และเทคโนโลยี&-\\\hline
	149&-กลุ่มสาขาวิชาวิทยาศาสตร์สุขภาพ&-\\\hline
	150&-กลุ่มสาขาวิชามนุษยศาสตร์และสังคมศาสตร์&-\\\hline
	
	151&\cellcolor{red!10}{ผลงานที่ได้รับการจดอนุสิทธิบัตร}&\cellcolor{red!10}{\textbf{-}}\\\hline
	152&-กลุ่มสาขาวิชาวิทยาศาสตร์และเทคโนโลยี&-\\\hline
	153&-กลุ่มสาขาวิชาวิทยาศาสตร์สุขภาพ&-\\\hline
	154&-กลุ่มสาขาวิชามนุษยศาสตร์และสังคมศาสตร์&-\\\hline
	
	155&\cellcolor{red!10}{บทความวิจัยหรือบทความวิชาการฉบับสมบูรณ์ที่ตีพิมพ์ในวารสารทางวิชาการที่ปรากฏ
		ในฐานข้อมูล TCI กลุ่มที่ 2
	}&\cellcolor{red!10}{}\\\hline
	156&-กลุ่มสาขาวิชาวิทยาศาสตร์และเทคโนโลยี&-\\\hline
	157&-กลุ่มสาขาวิชาวิทยาศาสตร์สุขภาพ&-\\\hline
	158&-กลุ่มสาขาวิชามนุษยศาสตร์และสังคมศาสตร์&-\\\hline
	
	159&\cellcolor{red!10}{บทความวิจัยหรือบทความวิชาการฉบับสมบูรณ์ที่ตีพิมพ์ในวารสารทางวิชาการระดับนานาชาติที่ไม่อยู่ในฐานข้อมูล ตามประกาศ ก.พ.อ. หรือระเบียบคณะกรรมการการอุดมศึกษาว่าด้วย หลักเกณฑ์การพิจารณาวารสารทางวิชาการสำหรับการเผยแพร่ผลงานทางวิชาการ พ.ศ.2556 แต่สถาบันนำเสนอสภาสถาบันอนุมัติและจัดทำเป็นประกาศให้ทราบเป็นการทั่วไป และแจ้งให้ กพอ./กกอ.ทราบภายใน 30 วันนับแต่วันที่ออกประกาศ (ซึ่งไม่อยู่ใน Beall’s list) หรือตีพิมพ์ในวารสารวิชาการที่ปรากฏในฐานข้อมูล TCI กลุ่มที่ 1
	}&\cellcolor{red!10}{}\\\hline
	160&-กลุ่มสาขาวิชาวิทยาศาสตร์และเทคโนโลยี&-\\\hline
	161&-กลุ่มสาขาวิชาวิทยาศาสตร์สุขภาพ&-\\\hline
	162&-กลุ่มสาขาวิชามนุษยศาสตร์และสังคมศาสตร์&-\\\hline
	
	163&\cellcolor{red!10}{บทความวิจัยหรือบทความวิชาการฉบับสมบูรณ์ที่ตีพิมพ์ในวารสารทางวิชาการระดับนานาชาติที่ปรากฏในฐานข้อมูลระดับนานาชาติตามประกาศ ก.พ.อ. หรือระเบียบคณะกรรมการการอุดมศึกษา ว่าด้วยหลักเกณฑ์การพิจารณาวารสารทางวิชาการสำหรับ
		การเผยแพร่ผลงานทางวิชาการ พ.ศ.2556}&\cellcolor{red!10}{}\\\hline
	164&-กลุ่มสาขาวิชาวิทยาศาสตร์และเทคโนโลยี&-\\\hline
	165&-กลุ่มสาขาวิชาวิทยาศาสตร์สุขภาพ&-\\\hline
	166&-กลุ่มสาขาวิชามนุษยศาสตร์และสังคมศาสตร์&-\\\hline
	
	167&\cellcolor{red!10}{ผลงานได้รับการจดสิทธิบัตร}&\cellcolor{red!10}{\textbf{-}}\\\hline
	168&-กลุ่มสาขาวิชาวิทยาศาสตร์และเทคโนโลยี&-\\\hline
	169&-กลุ่มสาขาวิชาวิทยาศาสตร์สุขภาพ&-\\\hline
	170&-กลุ่มสาขาวิชามนุษยศาสตร์และสังคมศาสตร์&-\\\hline
	
	171&\cellcolor{red!10}{ผลงานวิชาการรับใช้สังคมที่ได้รับการประเมินผ่านเกณฑ์การขอตำแหน่งทางวิชาการแล้ว}&\cellcolor{red!10}{\textbf{-}}\\\hline
	172&-กลุ่มสาขาวิชาวิทยาศาสตร์และเทคโนโลยี&-\\\hline
	173&-กลุ่มสาขาวิชาวิทยาศาสตร์สุขภาพ&-\\\hline
	174&-กลุ่มสาขาวิชามนุษยศาสตร์และสังคมศาสตร์&-\\\hline
	
	175&\cellcolor{red!10}{ผลงานวิจัยที่หน่วยงานหรือองค์กรระดับชาติว่าจ้างให้ดำเนินการ}&\cellcolor{red!10}{}\\\hline
	176&-กลุ่มสาขาวิชาวิทยาศาสตร์และเทคโนโลยี&-\\\hline
	177&-กลุ่มสาขาวิชาวิทยาศาสตร์สุขภาพ&-\\\hline
	178&-กลุ่มสาขาวิชามนุษยศาสตร์และสังคมศาสตร์&-\\\hline
	
	179&\cellcolor{red!10}{ผลงานค้นพบพันธุ์พืช พันธุ์สัตว์ ที่ค้นพบใหม่และได้รับการจดทะเบียน}&\cellcolor{red!10}{\textbf{-}}\\\hline
	180&-กลุ่มสาขาวิชาวิทยาศาสตร์และเทคโนโลยี&-\\\hline
	181&-กลุ่มสาขาวิชาวิทยาศาสตร์สุขภาพ&-\\\hline
	182&-กลุ่มสาขาวิชามนุษยศาสตร์และสังคมศาสตร์&-\\\hline
	
	183&\cellcolor{red!10}{ตำราหรือหนังสือหรืองานแปลที่ได้รับการประเมินผ่านเกณฑ์การขอตำแหน่งทางวิชาการแล้ว}&\cellcolor{red!10}{\textbf{-}}\\\hline
	184&-กลุ่มสาขาวิชาวิทยาศาสตร์และเทคโนโลยี&-\\\hline
	185&-กลุ่มสาขาวิชาวิทยาศาสตร์สุขภาพ&-\\\hline
	186&-กลุ่มสาขาวิชามนุษยศาสตร์และสังคมศาสตร์&-\\\hline
	
	187&\cellcolor{red!10}{ตำราหรือหนังสือหรืองานแปลที่ผ่านการพิจารณาตามหลักเกณฑ์การประเมินตำแหน่งทางวิชาการแต่ไม่ได้นำมาขอรับการประเมินตำแหน่งทางวิชาการ}&\cellcolor{red!10}{\textbf{-}}\\\hline
	188&-กลุ่มสาขาวิชาวิทยาศาสตร์และเทคโนโลยี&-\\\hline
	189&-กลุ่มสาขาวิชาวิทยาศาสตร์สุขภาพ&-\\\hline
	190&-กลุ่มสาขาวิชามนุษยศาสตร์และสังคมศาสตร์&-\\\hline
	
	191&\cellcolor{red!10}{งานสร้างสรรค์ที่มีการเผยแพร่สู่สาธารณะในลักษณะใดลักษณะหนึ่ง หรือผ่านสื่ออิเล็กทรอนิกส์ online}&\cellcolor{red!10}{\textbf{-}}\\\hline
	192&-กลุ่มสาขาวิชาวิทยาศาสตร์และเทคโนโลยี&-\\\hline
	193&-กลุ่มสาขาวิชาวิทยาศาสตร์สุขภาพ&-\\\hline
	194&-กลุ่มสาขาวิชามนุษยศาสตร์และสังคมศาสตร์&-\\\hline
	
	195&\cellcolor{red!10}{งานสร้างสรรค์ที่ได้รับการเผยแพร่ในระดับสถาบัน}&\cellcolor{red!10}{\textbf{-}}\\\hline
	196&-กลุ่มสาขาวิชาวิทยาศาสตร์และเทคโนโลยี&-\\\hline
	197&-กลุ่มสาขาวิชาวิทยาศาสตร์สุขภาพ&-\\\hline
	198&-กลุ่มสาขาวิชามนุษยศาสตร์และสังคมศาสตร์&-\\\hline
	
	199&\cellcolor{red!10}{งานสร้างสรรค์ที่ได้รับการเผยแพร่ในระดับชาติ}&\cellcolor{red!10}{\textbf{-}}\\\hline
	200&-กลุ่มสาขาวิชาวิทยาศาสตร์และเทคโนโลยี&-\\\hline
	201&-กลุ่มสาขาวิชาวิทยาศาสตร์สุขภาพ&-\\\hline
	202&-กลุ่มสาขาวิชามนุษยศาสตร์และสังคมศาสตร์&-\\\hline
	
	203&\cellcolor{red!10}{งานสร้างสรรค์ที่ได้รับการเผยแพร่ในระดับความร่วมมือระหว่างประเทศ}&\cellcolor{red!10}{\textbf{-}}\\\hline
	204&-กลุ่มสาขาวิชาวิทยาศาสตร์และเทคโนโลยี&-\\\hline
	205&-กลุ่มสาขาวิชาวิทยาศาสตร์สุขภาพ&-\\\hline
	206&-กลุ่มสาขาวิชามนุษยศาสตร์และสังคมศาสตร์&-\\\hline
	
	207&\cellcolor{red!10}{งานสร้างสรรค์ที่ได้รับการเผยแพร่ในระดับภูมิภาคอาเซียน}&\cellcolor{red!10}{\textbf{-}}\\\hline
	208&-กลุ่มสาขาวิชาวิทยาศาสตร์และเทคโนโลยี&-\\\hline
	209&-กลุ่มสาขาวิชาวิทยาศาสตร์สุขภาพ&-\\\hline
	210&-กลุ่มสาขาวิชามนุษยศาสตร์และสังคมศาสตร์&-\\\hline
	
	211&\cellcolor{red!10}{งานสร้างสรรค์ที่ได้รับการเผยแพร่ในระดับนานาชาติ}&\cellcolor{red!10}{\textbf{-}}\\\hline
	212&-กลุ่มสาขาวิชาวิทยาศาสตร์และเทคโนโลยี&-\\\hline
	213&-กลุ่มสาขาวิชาวิทยาศาสตร์สุขภาพ&-\\\hline
	214&-กลุ่มสาขาวิชามนุษยศาสตร์และสังคมศาสตร์&-\\\hline
\end{longtable}

%%%%%%%%%%%%%%%%%%%%%%%
\clearpage
\authorization
{ผู้ช่วยศาสตราจารย์สมนึก ศรีสวัสดิ์}
{รองศาสตราจารย์ ดร.พงศกร สุนทรายุทธ์}
{ผู้ช่วยศาสตรจารย์ ดร.วงศ์วิศรุต เขื่องสตุ่ง}
{อาจารย์ ดร.รัฐพรหม พรหมคำ}
{ผู้ช่วยศาสตราจารย์มงคล ทาทอง}
{ผู้ช่วยศาสตราจารย์ ดร.พิเชฐ คุณากรวงศ์}
{ผู้ช่วยศาสตราจารย์ ดร.นิพัทธ์ จงสวัสดิ์}
























