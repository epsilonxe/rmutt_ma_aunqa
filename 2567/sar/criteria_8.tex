\newpage
\criteria{Output and Outcomes}

\subcriteria{The pass rate, drop-out rate, and average time to graduate are shown to be established, monitored, and benchmarked for improvement.}
 หลักสูตรมีการเก็บรวบรวมข้อมูลจำนวนผู้สำเร็จการศึกษาร้อยละของจำนวนผู้สำเร็จการศึกษาและร้อยละของจำนวนผู้ที่ตกออกของนักศึกษาที่รับเข้าในปีการศึกษา 2561-2564 โดยมีหลักสูตรคู่เทียบ คือ หลักสูตรชีววิทยาประยุกต์ ซึ่งมีรายละเอียดดังตาราง
 
 
\begin{center}
	{\footnotesize
		\begin{tabular}{|*{10}{x{0.07\textwidth}|}}
			\hline
			\textbf{หลักสูตร} &
			\textbf{ปีการศึกษา} &
			\textbf{A} & \multicolumn{4}{x{0.35\textwidth}|}{{\bf จำนวนผู้สำเร็จการศึกษาตามหลักสูตร}} & \textbf{B} & \textbf{C} & \textbf{D} \\
			\cline{4-7}     
			&     &      & \textbf{2564}   & \textbf{2565}  & \textbf{2566}   & \textbf{2567}  &    &  &   \\ \hline  	
			
			คณิตฯ & 2561     & 22    &  14  &   &  &   & 63.64   & 36.36   & 4    \\ 
			
			ชีวฯ &      & 52    &  44  &   &  &   & 84.62   & 15.38   & 4    \\
			
			\hline
			
			คณิตฯ  & 2562     & 17    &     &  13 &  &   & 76.47   & 23.53  & 4   \\ 		 
			ชีวฯ & 	     & 36    &     &  28 &  &   & 77.78   & 22.22  & 4      \\   
			
			
			
			\hline
			คณิตฯ & 2563    & 8     &     &    &  7 &  & 87.5   &  12.5  & 4      \\   
			
			ชีวฯ &	     & 18     &     &    &  14 &  & 77.78   &  22.22  & 4      \\   
			
			
			\hline
			
			คณิตฯ & 2564    & 33     &     &    &   &  23  & 69.70   &  30.30  & 4      \\  
			
			ชีวฯ &	     & 35     &     &    &   &  28  & 80   &  20  & 4      \\   
			\hline
		\end{tabular}  
	}
\end{center}


\textbf{หมายเหตุ} 
\begin{enumerate}
\item [\textbf{A}] คือ จำนวนรับเข้า (มีตัวตน)

\item [\textbf{B}] คือ ร้อยละของจำนวนผู้สำเร็จการศึกษาตามหลักสูตร

\item [\textbf{C}] คือ ร้อยละของจำนวนของผู้ที่ตกออก

\item [\textbf{D}] คือ ระยะเวลาเฉลี่ย (ปี) ในการสำเร็จการศึกษาตามหลักสูตร
\end{enumerate}


 จากตารางพบว่า
 \begin{enumerate}
	\item ปีการศึกษา 2561, 2562 และ 2564 ร้อยละของจำนวนผู้สำเร็จการศึกษาตามหลักสูตรของหลักสูตรชีววิทยาประยุกต์มากกว่าหลักสูตคณิตศาสตร์ประยุกต์ แต่ปีการศึกษา 2563 ร้อยละของจำนวนผู้สำเร็จการศึกษาตามหลักสูตรของหลักสูตรชีววิทยาประยุกต์น้อยกว่าหลักสูตรคณิตศาสตร์ 
	
	\item ร้อยละเฉลี่ยของจำนวนผู้สำเร็จการศึกษาตามหลักสูตรของหลักสูตรชีววิทยาประยุกต์ในปีการศึกษา 2561-2564 คิดเป็นร้อยละ 80.045 ซึ่งมากกว่า  ร้อยละเฉลี่ยของจำนวนผู้สำเร็จการศึกษาตามหลักสูตรของหลักสูตรคณิตศาสตร์ประยุกต์ ในปีการศึกษา 2561-2564 คิดเป็นร้อยละ 74.33
	
	\item ปีการศึกษา 2561, 2562 และ 2564 ร้อยละของจำนวนผู้ที่ตกออกของหลักสูตรชีววิทยาประยุกต์น้อยกว่าหลักสูตรคณิตศาสตร์ประยุกต์  แต่ปีการศึกษา 2563 ร้อยละของจำนวนผู้ที่ตกออกของหลักสูตรชีววิทยาประยุกต์มากกว่าหลักสูตรคณิตศาสตร์ประยุกต์
	
	\item ร้อยละเฉลี่ยของจำนวนผู้ที่ตกออกของหลักสูตรชีววิทยาประยุกต์ในปีการศึกษา 2561-2564 คิดเป็น\\ร้อยละ 19.955 ซึ่งน้อยกว่า ร้อยละเฉลี่ยของจำนวนผู้ที่ตกออกของหลักสูตรคณิตศาสตร์ประยุกต์ ในปีการศึกษา 2561-2564 คิดเป็นร้อยละ 25.67
	
	\item ระยะเวลาเฉลี่ยในการสำเร็จการศึกษาตามหลักสูตรเป็นไปตามแผนการศึกษาของหลักสูตร
\end{enumerate}

ทั้งนี้เป็นผลจากหลักสูตรชีววิทยาประยุกต์เป็นหลักสุตรที่เห็นเป็นรูปธรรมได้ง่ายกว่าหลักสูตรคณิตศาสตร์ประยุกต์ที่เป็นหลักสูตรที่เน้นทางด้านนามธรรม ทำให้นักศึกษาเข้าใจได้อยากกว่า ทางหลักสูตรคณิตศาสตร์ประยุกต์จึงได้มีการดำเนินการปรับปรุงบางรายวิชา ในเรื่องของการเรียนการสอนให้หลากหลายรูปแบบ เช่น การบรรยาย อภิปราย การสอนแบบ Active Learning เป็นต้น มีการจัดกิจกรรมที่ส่งเสริมให้นักศึกษาเกิดการคิดวิเคราะห์ ส่งเสริมให้มีการศึกษาค้นคว้าด้วยตัวเอง ซึ่งจะช่วยให้นักศึกษาได้เข้าใจเนื้อหาในรายวิชานั้นๆให้เป็นรูปธรรมมากยิ่งขึ้น นอกจากนี้ยังมีระบบอาจารย์ที่ปรึกษาที่ดีโดยอาจารย์ที่ปรึกษามีการกำกับติดตามและให้คำปรึกษาอย่างใกล้ชิดทั้งทางด้านผลการเรียนและในด้านอื่นๆ เพื่อทำให้นักศึกษาสามารถสำเร็จการศึกษาตามแผนที่หลักสูตรกำหนด และมีร้อยละการตกออกลดลง 


\begin{doclist}
\docitem{ข้อมูลจำนวนนักศึกษา จำนวนนักศึกษาตกออก}
\docitem{ข้อมูลจำนวนนักศึกษาที่สำเร็จการศึกษาตามแผน และระยะเวลาการสำเร็จการศึกษาเฉลี่ย}
\end{doclist}


\subcriteria{Employability as well as self-employment, entrepreneurship, and advancement to further studies, are shown to be established, monitored, and benchmarked for improvement.}

การเก็บรวบรวมข้อมูลเกี่ยวกับภาวะการมีงานทำภายใน 1 ปี และ รายได้เฉลี่ยต่อเดือนของบัณฑิตระดับปริญญาตรี  ทางมหาวิทยาลัยได้มอบหมายให้กองพัฒนานักศึกษา (กพน.) เป็นผู้เก็บรวบรวม วิเคราะห์ และส่งผลการสำรวจกลับมาให้ทางคณะและหลักสูตร โดยมีหลักสูตรคู่เทียบ คือ หลักสูตรพยาบาลศาสตร์  ซึ่งมีรายละเอียดของข้อมูลแสดงได้ดังนี้

 \begin{longtable}{|x{0.07\textwidth}|*{8}{x{0.07\textwidth}|}} % Adjusted column widths
	\hline
	\multicolumn{1}{|x{0.07\textwidth}|}{\textbf{ปีการศึกษา}} &
	\multicolumn{4}{x{0.4\linewidth}|}{\textbf{หลักสูตรคณิตศาสตร์ประยุกต์}} &
	\multicolumn{4}{x{0.4\linewidth}|}{\textbf{หลักสูตรพยาบาลศาสตร์(คู่เทียบ)}} \\
	\cline{2-9}
	\multicolumn{1}{|x{0.07\textwidth}|}{} &
	\multicolumn{1}{x{0.07\textwidth}|}{ จำนวนบัณฑิตทั้งหมด} &
	\multicolumn{1}{x{0.07\textwidth}|}{ จำนวนบัณฑิตที่ตอบฯ} &
	\multicolumn{1}{x{0.07\textwidth}|}{ ร้อยละการได้งานทำใน 1 ปี} &
	\multicolumn{1}{x{0.07\textwidth}|}{ รายได้เฉลี่ยต่อเดือน} &
	\multicolumn{1}{x{0.07\textwidth}|}{ จำนวนบัณฑิตทั้งหมด} &
	\multicolumn{1}{x{0.07\textwidth}|}{ จำนวนบัณฑิตที่ตอบฯ} &
	\multicolumn{1}{x{0.07\textwidth}|}{ ร้อยละการได้งานทำใน 1 ปี} &
	\multicolumn{1}{x{0.07\textwidth}|}{ รายได้เฉลี่ยต่อเดือน} \\
	\hline
	\endhead % End of table header
	
	\hline
	2564  &     14   &     14  &    76.92  &   17,300  &    65  &   65 &    95.38 &     24,576  \\
	\hline
	2565 &  13   &     10  &    80.00  &   20,444  &    69  &   32 &    84.38 &     23,914  \\
	\hline
	2566 &  6   &     6  &      83.33  &   17,433  &    79  &   73  &    91.78 &    31,218  \\
	\hline
	
\end{longtable}


 
 
 
 
จากตารางพบว่า 
 \begin{enumerate}
\item ร้อยละการได้งานทำใน 1 ปี ของหลักสูตรพยาบาลศาสตร์ มากกว่า หลักสูตรคณิตศาสตร์ประยุกต์ ในทุกปีการศึกษา

\item รายได้เฉลี่ยต่อเดือนของหลักสูตรพยาบาลศาสตร์ มากกว่า หลักสูตรคณิตศาสตร์ประยุกต์ ในทุกปีการศึกษา

 \end{enumerate}

ทั้งนี้เป็นผลมาจากหลักสูตรพยาบาลศาสตร์เป็นหลักสูตรที่ผลิตบัณฑิตที่มีอาชีพเฉพาะทาง คือ พยาบาลวิชาชีพ ในสถานบริการสุขภาพทุกระดับ ได้แก่ ระดับปฐมภูมิ ทุติยภูมิ และตติยภูมิ ทั้งในภาครัฐและเอกชน ซึ่งมีความต้องการเป็นจำนวนมาก ส่วนหลักสูตรคณิตศาสตร์ประยุกต์ เป็นหลักสูตรที่ผลิตบัณฑิตที่มีอาชีพที่หลากหลาย เช่น นักวิชาการ/นักวิจัย นักพัฒนาซอฟท์แวร์คอมพิวเตอร์ นักวิทยาศาสตร์ข้อมูล นักวิเคราะห์ข้อมูล เป็นต้น จึงทำให้บัณฑิตมีงานทำไม่ต่ำกว่าร้อยละ 75  ทุกปีการศึกษา
และรายได้เฉลี่ยต่อเดือนมากกว่า 1.2 เท่าของรายได้เฉลี่ยต่อเดือนมาตรฐานของผู้จบการศึกษาระดับปริญญาตรี 

\begin{doclist}
\docitem{ข้อมูลภาวะการมีงานทำภายใน 1 ปี ของบัณฑิต }
\docitem{ข้อมูลรายได้เฉลี่ยต่อเดือนของบัณฑิต}
\end{doclist}

\subcriteria{Research and creative work output and activities carried out by the academic staff and students, are shown to be established, monitored, and benchmarked for improvement.}
หลักสูตรส่งเสริมให้อาจารย์ผู้รับผิดชอบหลักสูตรมีผลงานตีพิมพ์ในวารสารทางวิชาการระดับนานาชาติที่อยู่ในฐานข้อมูล SCOPUS ที่มีคุณภาพระดับสูง (Q1) 

โดยมีการกำหนดตัวชี้วัดความสำเร็จในประเด็นนี้ คือ จำนวนงานวิจัยตีพิมพ์ในฐานข้อมูล SCOPUS ที่มีคุณภาพระดับสูง (Q1) ของอาจารย์ผู้รับผิดชอบหลักสูตรไม่ต่ำกว่าร้อยละ 20 ของจำนวนงานวิจัยตีพิมพ์ในฐานข้อมูล SCOPUS

ผลงานตีพิมพ์ในวารสารทางวิชาการระดับนานาชาติที่อยู่ในฐานข้อมูล SCOPUS ของอาจารย์ผู้รับผิดชอบหลักสูตรปีการศึกษา 2564-2567 โดยมีหลักสูตรคู่เทียบ คือ หลักสูตรสถิติประยุกต์  ซึ่งมีรายละเอียดของข้อมูลแสดงได้ดังนี้

\begin{longtable}{|>{\centering\arraybackslash}p{0.3\linewidth}|*{8}{c|}} % Adjusted column widths
	\hline
	\multicolumn{1}{|>{\centering\arraybackslash}p{0.3\linewidth}|}{\textbf{ระดับผลงาน}} &
	\multicolumn{4}{c|}{\textbf{หลักสูตรคณิตศาสตร์ประยุกต์}} &
	\multicolumn{4}{c|}{\textbf{หลักสูตรสถิติประยุกต์(คู่เทียบ)}} \\
	\cline{2-9}
	\multicolumn{1}{|c|}{} &
	\multicolumn{1}{c|}{2564} &
	\multicolumn{1}{c|}{2565} &
	\multicolumn{1}{c|}{2566} &
	\multicolumn{1}{c|}{2567} &
	\multicolumn{1}{c|}{2564} &
	\multicolumn{1}{c|}{2565} &
	\multicolumn{1}{c|}{2566} &
	\multicolumn{1}{c|}{2567} \\
	\hline
	\endhead % End of table header
	
	\hline
	Q1  &   	5  &     7  &  	9  &  4   &   	13  &  	10 &    9 &  	2  \\
	\hline
	Q2 & 	8  &     5  &  	0  &   2  &   	2  &  5 &    3 &  	5  \\
	\hline
	Q3 & 	2   &     0  &  1  &  1   &   8  &  	5  &    2 &  	3  \\
	\hline
	Q4 & 	5  &     2  &  0  &   0  &   	5  &  	2  &    1 &  	1  \\
	\hline
	รวม & 	20  &     14  &  10  &  7   &   28	  &  	22  &    15 &  	11  \\
	\hline
	ร้อยละของจำนวนงานวิจัยระดับ Q1 & 	25   &     50 &  	90  &   57.14  &   46.43	  &  	45.45  &    60 &  	18.18  \\
	\hline
	
	
	
\end{longtable}


	

จากตารางพบว่าจำนวนผลงานวิจัยของอาจารย์ผู้รับผิดชอบหลักสูตรของหลักสูตรสถิติประยุกต์มากกว่าหลักสูตรคณิตศาสตร์ประยุกต์ทุกปีการศึกษา และ ร้อยละของจำนวนงานวิจัยตีพิมพ์ในฐานข้อมูล SCOPUS ที่มีคุณภาพสูง (Q1) ของอาจารย์ผู้รับผิดชอบหลักสูตรคณิตศาสตร์สูงกว่าเกณฑ์ทุกปีการศึกษา

นอกจากนี้หลักสูตรยังส่งเสริมให้นักศึกษาเรียนรู้กระบวนการทำวิจัย เพื่อให้เกิดทักษะ กระบวนการคิด วิเคราะห์ คำนวณ การแก้ปัญหา การทำงานเป็นทีม การแสวงหาความรู้ และสามารถบูรณาการองค์ความรู้ที่ได้เรียนไปทั้งหมดในการทำโครงงานในรายวิชาโครงงานด้านคณิตศาสตร์ ซึ่งโครงงานของนักศึกษาในปีการศึกษา 2564-2567 มีรายละเอียดดังนี้

\begin{enumerate}
\item[]{\bf ปีการศึกษา 2564}
\begin{itemize}
\item[(1)]  On the  $(s,t)$-Pell and  $(s,t)$-Pell-Lucas Polynomials by Matrix Methods
\item[(2)]  A New Iterative Scheme for Approximation of Fixed Points in Banach Spaces
\item[(3)]  Classes of Matrices over a Commutative Ring with Identity whose Determinant are Zero 
\item[(4)]  On Some Diophantine Equations of The Form $\frac{a}{x}+\frac{b}{y}+\frac{c}{z}=d$
\item[(5)]  การประมาณค่าที่หายไปของดัชนีคุณภาพอากาศจากสถานีวัด
\item[(6)]  การพัฒนาแบบจำลองทางคณิตศาสตร์สำหรับการจัดการความเสี่ยงและการประยุกต์ใช้
\end{itemize}

\item[]\textbf{ปีการศึกษา 2565}
\begin{itemize}
\item[(1)] Bivariate Vieta-Fibonacci-like polynomials

\item[(2)] Some new $(s,t)$-Pell and $(s,t)$-Pell-Lucas polynomials identities by matrix methods

\item[(3)]  Bi-Periodick-Pell Sequence

\item[(4)]  Convergence Theorems for Modified Three-Step Iterations in Uniformly Convex Metric Spaces 

\item[(5)]  การประมาณค่าดัชนีคุณภาพอากาศ ณ จุดที่ไม่มีสถานีวัด
\end{itemize}

\item[]\textbf{ปีการศึกษา 2566}
\begin{itemize}
\item[(1)] Bivariate Vieta-Jacobsthal-like polynomials

\item[(2)] Some Properties of Determinant of Matrices over Generalized Fibonacci Numbers and Generalized Gaussian Fibonacci Numbers

\item[(3)] การวิเคราะห์เกี่ยวกับจำนวนเพลล์และจำนวนเพลล์ลูคัส
\end{itemize}


\item[]\textbf{ปีการศึกษา 2567}
\begin{itemize}
	\item[(1)] การลงทุนในหุ้นร่วมกับออปชั่น (Investing in stocks with options)
	
	\item[(2)] A Multi-Day Multi-Hub Delivery Planning
	
	\item[(3)] Fixed point methodologies for logistic regression problem with application to Alzheimer’s disease screening     
	
	\item[(4)] Generating Music Variation through Chaotic Dynamical System Exploration
	
	\item[(5)] Generalized Vieta-Fibonacci-Type Polynomials and Generalized Vieta Pell-Type Polynomials
	
	\item[(6)] เว็บไซต์ระบบการจัดการทุนการศึกษา
	
	
	\item[(7)] On the Generalized Vieta-Pell and Vieta- Pell-Lucas polynomials by matrix methods  
	
	
	\item[(8)] On the Diophantine Equation $F^n_{x-1} + F^n_{x+1} = y^2$
	
\end{itemize}
\end{enumerate}


\begin{doclist}
\docitem{ข้อมูลงานวิจัยตีพิมพ์ของอาจารย์ผู้รับผิดชอบหลักสูตร}
\docitem{ข้อมูลโครงงานของนักศึกษา}
\end{doclist}

\subcriteria{Data are provided to show directly the achievement of the programme outcomes, which are established and monitored.}

\printprogram{} ได้กำหนดวิธีการประเมินผลการบรรลุผลลัพธ์การเรียนรู้ที่คาดหวังของหลักสูตร  (PLOs) โดยใช้แบบสอบถาม {\bf แบบประเมินการบรรลุผลลัพธ์การเรียนรู้ที่คาดหวังของหลักสูตร (PLOs)} ให้นักศึกษาชั้นปีสุดท้ายจำนวน 23 คนเป็นผู้ประเมินตนเองวาตนเองสามารถบรรลุใน PLO นั้น ๆ ไดในระดับใด โดยกําหนดระดับการบรรลุ PLO เปนดังนี้\\
- คาเฉลี่ย 4.51 – 5.00 หมายถึงบรรลุ PLO นั้น ๆ ในระดับดีมาก\\
- คาเฉลี่ย 3.51 – 4.50 หมายถึง บรรลุ PLO นั้น ๆ ในระดับดี\\
- คาเฉลี่ย 2.51 – 3.50 หมายถึงบรรลุ PLO นั้น ๆ ในระดับปานกลาง\\
- คาเฉลี่ย 1.51 – 2.50 หมายถึง บรรลุ PLO นั้น ๆ ในระดับนอย\\
- คาเฉลี่ย 1.00 – 1.50 หมายถึง บรรลุ PLO นั้น ๆ ในระดับนอยที่สุด

โดยหลักสูตรกำหนดเกณฑ์การบรรลุแต่ละผลลัพธ์การเรียนรู้ที่คาดหวังของหลักสูตร (PLOs) ต้องมีผลการประเมินฉลี่ยไม่ต่ำกว่า 2.5 คะแนน จากคะแนนเต็ม 5 คะแนน หรือร้อยละ 50 โดยในปีการศึกษา 2567 เป็นปีแรกของนักศึกษาชั้นปีสุดท้ายที่จบการศึกษา \printprogram{} ผลการประเมินตนเองสำหรับการบรรลุผลลัพธ์การเรียนรู้ที่คาดหวังของหลักสูตร (PLOs) แสดงได้ดังตาราง \ref{Table:8.4} จากตาราง \ref{Table:8.4} จะเห็นว่าผลการประเมินโดยเฉลี่ยของ PLO1--PLO10 มีคะแนนที่สูงมาก (ค่าเฉลี่ยมากกว่า 4.51) นั่นคือการบรรลุ PLOs ในภาพรวมอยู่ในระดับที่ดีมาก ซึ่งสะท้อนว่านักศึกษาบรรลุผลลัพธ์การเรียนรู้ที่คาดหวังของหลักสูตร (PLOs) \printprogram{}

\begin{longtable} 
{|>{\centering\arraybackslash}p{0.14\textwidth} 
|>{\centering\arraybackslash}p{0.12\textwidth}
|>{\centering\arraybackslash}p{0.12\textwidth}
|>{\centering\arraybackslash}p{0.12\textwidth}
|>{\centering\arraybackslash}p{0.10\textwidth}
|>{\centering\arraybackslash}p{0.12\textwidth}|}
	\caption{ผลการประเมินตนเองสำหรับการบรรลุผลลัพธ์การเรียนรู้ที่คาดหวังของหลักสูตร (PLOs)}	
	\label{Table:8.4}\\
		\hline
\multicolumn{1}{|c|}{\textbf{PLOs}} 
& \multicolumn{2}{c|}{\textbf{การบรรลุ PLOs}} 
& \multicolumn{3}{c|}{\textbf{รุ่นสำเร็จการศึกษาปีการศึกษา 2567}} \\\cline{2-6}

& \textbf{บรรลุ} & \textbf{ไม่บรรลุ} 
& \textbf{ค่าเฉลี่ย} & \textbf{S.D.} & \textbf{แปลผล} \\\hline
\endhead

PLO1 & \checkmark & & 4.91 & 0.28 & ดีมาก \\\hline
PLO2 & \checkmark &  & 4.69 & 0.62 & ดีมาก \\\hline
PLO3 & \checkmark & & 4.87 & 0.33 & ดีมาก \\\hline
PLO4 & \checkmark &  & 4.78 & 0.50 & ดีมาก \\\hline
PLO5 & \checkmark & & 4.91 & 0.28 & ดีมาก \\\hline
PLO6 & \checkmark & & 4.73 & 0.52 & ดีมาก \\\hline
PLO7 & \checkmark & & 4.87 & 0.33 & ดีมาก \\\hline
PLO8 & \checkmark &  & 4.91 & 0.28 & ดีมาก \\\hline
PLO9 & \checkmark & & 4.82 & 0.37 & ดีมาก \\\hline
PLO10 & \checkmark & & 4.95 & 0.20 & ดีมาก \\\hline
PLO11 & \checkmark & & 4.88 & 0.30 & ดีมาก \\\hline
\multicolumn{3}{|c|}{\textbf{หมายเหตุ: คะแนนเฉลี่ยสะท้อนการประเมินตนเองของนักศึกษา}} 
& 4.85 & 0.35 & ดีมาก \\\hline

\end{longtable}


\begin{doclist}
\docitem{แบบประเมินการบรรลุผลลัพธ์การเรียนรู้ที่คาดหวังของหลักสูตร (PLOs)}
\end{doclist}












\subcriteria{Satisfaction level of the various stakeholders are shown to be established, monitored, and benchmarked for improvement.}

หลักสูตรรวบรวมข้อมูลย้อนกลับและการผลประเมินความพึงพอใจของผู้มีส่วนได้ส่วนเสียของหลักสูตร เพื่อนำมาพิจารณาวางแผนปรับปรุงกระบวนการพัฒนาบัณฑิตสำหรับปีการศึกษาต่อไป โดยผู้มีส่วนได้ส่วนเสียของหลักสูตรดังกล่าว ประกอบไปด้วย

\begin{enumerate}
	\item นักศึกษาทุกชั้นปี
	\item นักศึกษาชั้นปีสุดท้าย
	\item ผู้ใช้บัณฑิต
	\item อาจารย์ผู้รับผิดชอบหลักสูตร
\end{enumerate}
หลักสูตรเก็บรวบรวมข้อมูลผลการประเมินความพึงพอใจในแต่ละด้านดังนี้
\begin{enumerate}
	\item ประเมินความพึงพอใจของนักศึกษาทุกชั้นปีที่มีต่อหลักสูตร
	\item ประเมินความพึงพอใจของนักศึกษาชั้นปีสุดท้ายที่มีต่อคุณภาพหลักสูตร
	\item ประเมินความพึงพอใจของผู้ใช้บัณฑิตที่มีต่อบัณฑิตใหม่
	\item ประเมินความพึงพออาจารย์ของผู้รับผิดชอบหลักสูตรต่อการบริหารจัดการหลักสูตร
\end{enumerate}


\begin{longtable}{|>{\centering\arraybackslash}p{0.3\linewidth}|*{8}{c|}} 
\caption{ผลการประเมินความพึงพอใจหลักสูตรและคุณภาพบัณฑิตของผู้มีส่วนได้ส่วนเสีย}	
	\label{Table:8.5-1}\\
	\hline
	\multicolumn{1}{|>{\centering\arraybackslash}p{0.3\linewidth}|}{\textbf{การประเมินความพึงพอใจของผู้มีส่วนได้ส่วนเสีย}} &
	\multicolumn{4}{c|}{\textbf{หลักสูตรคณิตศาสตร์ประยุกต์}} &
	\multicolumn{4}{c|}{\textbf{หลักสูตรสถิติประยุกต์(คู่เทียบ)}} \\
	\cline{2-9}
	\multicolumn{1}{|c|}{} &
	\multicolumn{1}{c|}{2564} &
	\multicolumn{1}{c|}{2565} &
	\multicolumn{1}{c|}{2566} &
	\multicolumn{1}{c|}{2567} &
	\multicolumn{1}{c|}{2564} &
	\multicolumn{1}{c|}{2565} &
	\multicolumn{1}{c|}{2566} &
	\multicolumn{1}{c|}{2567} \\
	\hline
	\endhead
	
	นักศึกษาทุกชั้นปี & 4.28 & 4.29 & 4.62 & 4.69 & 4.75 & 4.77 & 4.70 & 4.73 \\
	\hline
	นักศึกษาชั้นปีสุดท้าย & 4.60 & 4.43 & 4.38 & 4.79 & 4.48 & 4.24 & 4.26 & 4.09 \\
	\hline
	ผู้ใช้บัณฑิต & 4.46 & 4.58 & 4.67 & - & 4.24 & 4.62 & 4.87 & - \\
	\hline
	อาจารย์ผู้รับผิดชอบหลักสูตร & 4.62 & 4.68 & 4.51 & 4.73 & 4.98 & 4.99 & 4.94 & 4.95 \\
	\hline

\end{longtable}




จากตาราง \ref{Table:8.5-1} ผลการประเมินความพึงพอใจของผู้มีส่วนได้ส่วนเสียเปรียบเทียบ 4 ปีย้อนหลัง พบว่าระดับความพึงพอใจอยู่ในระดับพึงพอใจมากถึงมากที่สุดในเกือบทุกด้าน โดยเฉพาะระดับความพึงพอใจของนักศึกษาทุกชั้นปีที่มีต่อหลักสูตรและระดับความพึงพอใจของผู้ใช้บัณฑิตสูงขึ้นมีแนวโน้มสูงขึ้น อย่างไรก็ตามเมื่อเทียบระดับความพึงพอใจของนักศึกษาทุกชั้นปีที่มีต่อหลักสูตรคณิตศาสตร์ประยุกต์กับหลักสูตรสถิติประยุกต์จะพบว่าระดับความพึงพอใจมีค่าน้อยกว่าอยู่เล็กน้อย เพื่อให้เกิดการพัฒนาระดับความพึงพอใจทางหลักสูตรจึงได้วิเคราะห์แบบสอบถามของนักศึกษา พบว่านักศึกษามีข้อเสนอแนะเพิ่มเติม ดังนี้
\begin{enumerate}
	\item อยากให้หลักสูตรเสริมรายวิชาที่เกี่ยวข้องกับวิชาที่เกี่ยวข้องกับ AI หรือ ML
    \item อยากให้หลักสูตรเน้นการปฎิบัติจริงให้มากขึ้น เช่น การเขียนโปรแกรมคอมพิวเตอร์
\end{enumerate}
นอกจากนี้เมื่อเทียบระดับความพึงพอใจของอาจารย์ผู้รับผิดชอบหลักสูตรต่อการบริหารจัดการหลักสูตรกับหลักสูตรสถิติประยุกต์จะพบว่าระดับความพึงพอใจมีค่าน้อยกว่ามาก และเมื่อพิจารณารายประเด็นจากแบบสอบพบว่า อาจารย์ผู้รับผิดชอบหลักสูตรมีข้อเสนอแนะว่า 

\begin{enumerate}
	\item ควรพัฒนากระบวนการเตรียมความพร้อมอาจารย์ใหม่เพื่อรองรับการพิจารณาเสนอชื่อเป็นอาจารย์ประจำหลักสูตร 
    \item ควรกำกับติดตามการเสนอขอตำแหน่งทางวิชาการของอาจารย์ประจำหลักสูตรเป็นรายบุคคลอย่าง
     ใกล้ชิดยิ่งขึ้น
      \item ควรเร่งการก้าวเข้าสู่การเสนอขอตำแหน่งทางวิชาการของอาจารย์ประจำหลักสูตรอย่างเร่งด่วน
\end{enumerate}
จากรายประเด็นข้อเสนอแนะดังกล่าวอาจารย์ผู้รับผิดชอบหลักสูตรได้ร่วมกันพิจารณาและปรับปรุงข้อเสนอแนะเพื่อให้เกิดการพัฒนาหลักสูตรให้ทันสมัยและดียิ่งขึ้นซึ่งจะเป็นแนวทางที่ใชในการปรับปรุง
หลักสูตรตอไปในอนาคต

%%%%%%%%%%%%%%%%%%%


\begin{doclist}
\docitem{ผลการประเมินความพึงพอใจของนักศึกษาทุกชั้นปีที่มีต่อหลักสูตร}
\docitem{ผลการประเมินความพึงพอใจของนักศึกษาชั้นปีสุดท้ายที่มีต่อคุณภาพหลักสูตร}
\docitem{ผลการประเมินความพึงพอใจของผู้ใช้บัณฑิตที่มีต่อบัณฑิตใหม่}
\docitem{ผลการประเมินความพึงพอใจของอาจารย์ผู้รับผิดชอบหลักสูตรต่อการบริหารจัดการหลักสูตร}
\end{doclist}


