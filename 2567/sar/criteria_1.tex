\criteria{Expected Learning Outcomes}

\subcriteria{The programme to show that the expected learning outcomes are appropriately formulated in accordance with an established learning taxonomy, are aligned to the vision and mission of the university, and are known to all stakeholders.}


\begin{center}
\begin{tabular}{|p{0.45\textwidth}|p{0.45\textwidth}|}
\hline
\multicolumn{1}{|c|}{\textbf{มหาวิทยาลัยเทคโนโลยีราชมงคลธัญบุรี}} & \multicolumn{1}{|c|}{\textbf{คณะวิทยาศาสตร์และเทคโนโลยี}} \\
\hline
\multicolumn{2}{|c|}{วิสัยทัศน์ (Vision)} \\
\hline
มหาวิทยาลัยนวัตกรรมที่สร้างคุณค่าสู่สังคมและประเทศ & 
เป็นคณะที่มุ่งเน้นการสร้างนวัตกรและนวัตกรรมด้านวิทยาศาสตร์และเทคโนโลยีที่มีคุณค่าสู่สังคมและประเทศ  \\
\hline
\multicolumn{2}{|c|}{พันธกิจ (Mission)} \\
\hline
\begin{enumerate}
\item ผลิตและพัฒนากำลังคนให้มีความสามารถทางวิชาการ วิชาชีพ คิดสร้างสรรค์และเรียนรู้ตลอดชีวิต
\item สร้างงานวิจัย สิ่งประดิษฐ์ งานสร้างสรรค์ และนวัตกรรม สู่การนำไปใช้ประโยชน์ในภาคอุตสาหกรรม สังคม ชุมชน หรือสร้างมูลค่าเชิงพาณิชย์
\item ให้บริการวิชาการแก่ชุมชนในพื้นที่เป้าหมายหรือภาคประกอบการเพื่อการพัฒนาอย่างยั่งยืน
\item ทำนุบำรุงศาสนา ศิลปวัฒนธรรม และอนุรักษ์สิ่งแวดล้อม
\item บริหารจัดการอย่างมีธรรมาภิบาล เพิ่มประสิทธิภาพและประสิทธิผลด้วยนวัตกรรม เพื่อการพัฒนาอย่างต่อเนื่องและยั่งยืน
\end{enumerate}
&
\begin{enumerate}
\item ผลิตนักนวัตกรที่ปฏิบัติงานได้จริง สามารถประยุกต์ใช้ประโยชน์หรือพัฒนาเทคโนโลยีและสร้างนวัตกรรม
\item ผลิตผลงานวิจัย สร้างสรรค์เทคโนโลยีและนวัตกรรมเพื่อการพัฒนาประเทศ
\item บริการวิชาการที่ตอบสนองต่อความต้องการ สร้างคุณค่า เป็นประโยชน์ เป็นที่ยอมรับและสร้างความเข้มแข็งให้ชุมชนและสังคมอย่างยั่งยืน
\end{enumerate} \\
\hline
\end{tabular}
\end{center}

ผลการเรียนรู้ที่คาดหวัง (PLOs) ของ \printprogram{} ได้รับการออกแบบให้สอดคล้องกับวิสัยทัศน์และพันธกิจของมหาวิทยาลัยเทคโนโลยีราชมงคลธัญบุรีและของคณะวิทยาศาสตร์และเทคโนโลยี อีกทั้งยังคำนึงถึงความต้องการของผู้มีส่วนได้เสียทุกฝ่าย เพื่อให้มั่นใจว่าผู้สำเร็จการศึกษามีคุณสมบัติที่ตรงตามความต้องการของตลาดแรงงานและสังคม ดังแสดงในตาราง \ref{table: req 1.1} นอกจากนี้ยังได้แสดงความสอดคล้องกับกรอบมาตรฐานคุณวุฒิระดับอุดมศึกษาแห่งชาติ พ.ศ. 2565 ในด้านความรู้ (Knowledge), ทักษะ (Skill), จริยธรรม (Ethic), และลักษณะบุคคล (Character) ดังตาราง \ref{table: plo_ksec}

\begin{longtable}{| >{\raggedright}p{0.5\textwidth} | p{0.1\textwidth} | p{0.1\textwidth} | p{0.1\textwidth} | p{0.1\textwidth} |}
\caption{แสดงความสอดคล้องระหว่างผลลัพธ์การเรียนรู้ที่คาดหวัง (PLOs) กับวิสัยทัศน์และพันธกิจของมหาวิทยาลัยและคณะฯ}
\label{table: req 1.1}
\\
\hline
\multicolumn{1}{|c|}{\textbf{ผลลัพธ์การเรียนรู้ที่คาดหวัง (PLOs)}} & \multicolumn{2}{c|}{\textbf{ระดับมหาวิทยาลัย}} & \multicolumn{2}{c|}{\textbf{ระดับคณะ}} \\
\cline{2-5}
\multicolumn{1}{|c|}{} & \textbf{วิสัยทัศน์} & \textbf{พันธกิจ} & \textbf{วิสัยทัศน์} & \textbf{พันธกิจ} \\
\hline
\endhead
PLO1. ปฏิบัติตามจรรยาบรรณทางวิชาการ กฎระเบียบ และข้อบังคับขององค์กร (Affective Domain) & \multicolumn{1}{c|}{\checkmark} & \multicolumn{1}{c|}{1, 5} & \multicolumn{1}{c|}{\checkmark} & \multicolumn{1}{c|}{1} \\
\hline
PLO2. อธิบายบทนิยาม หลักการ และทฤษฎีบททางด้านคณิตศาสตร์และวิทยาศาสตร์ที่สำคัญได้อย่างถูกต้อง (Understanding) & \multicolumn{1}{c|}{\checkmark} & \multicolumn{1}{c|}{1} & \multicolumn{1}{c|}{\checkmark} & \multicolumn{1}{c|}{1} \\
\hline
PLO3. คำนวณเพื่อแก้ปัญหาทางด้านคณิตศาสตร์ตามหลักการ บทนิยาม และทฤษฎีบทได้อย่างถูกต้องเหมาะสม (Analyzing) & \multicolumn{1}{c|}{\checkmark} & \multicolumn{1}{c|}{1} & \multicolumn{1}{c|}{\checkmark} & \multicolumn{1}{c|}{1} \\
\hline
PLO4. พิสูจน์ข้อความและทฤษฎีบททางคณิตศาสตร์ได้อย่างถูกต้องและสมเหตุสมผลตามหลักตรรกศาสตร์และการให้เหตุผล (Evaluating) & \multicolumn{1}{c|}{\checkmark} & \multicolumn{1}{c|}{1} & \multicolumn{1}{c|}{\checkmark} & \multicolumn{1}{c|}{1} \\
\hline
PLO5. ประยุกต์ใช้ความรู้ ทักษะ และเทคโนโลยีทางคณิตศาสตร์ในการแก้ปัญหาทางด้านวิทยาศาสตร์ วิศวกรรมศาสตร์ ธุรกิจ อุตสาหกรรม หรือศาสตร์ที่เกี่ยวข้อง (Applying) & \multicolumn{1}{c|}{\checkmark} & \multicolumn{1}{c|}{1, 2, 3} & \multicolumn{1}{c|}{\checkmark} & \multicolumn{1}{c|}{1, 2, 3} \\
\hline
PLO6. สร้างหรือปรับปรุงกระบวนการคิดทางคณิตศาสตร์และการวิจัยที่นำไปสู่องค์ความรู้ใหม่หรือนวัตกรรมทางด้านคณิตศาสตร์ คณิตศาสตร์ประยุกต์ หรือด้านที่เกี่ยวข้อง (Creating) & \multicolumn{1}{c|}{\checkmark} & \multicolumn{1}{c|}{1, 2} & \multicolumn{1}{c|}{\checkmark} & \multicolumn{1}{c|}{1, 2} \\
\hline
PLO7. ปรับตัวเข้ากับสถานการณ์และวัฒนธรรมขององค์กร มีความรับผิดชอบ และทำงานร่วมกับผู้อื่นในฐานะผู้นำหรือสมาชิกที่ดี (Affective Domain) & \multicolumn{1}{c|}{\checkmark} & \multicolumn{1}{c|}{1, 5} & \multicolumn{1}{c|}{\checkmark} & \multicolumn{1}{c|}{1} \\
\hline
PLO8. ใช้คณิตศาสตร์หรือสถิติเพื่อการวิเคราะห์ ประมวลผลข้อมูล และนำเสนอได้อย่างเหมาะสม (Evaluating) & \multicolumn{1}{c|}{\checkmark} & \multicolumn{1}{c|}{1, 2} & \multicolumn{1}{c|}{\checkmark} & \multicolumn{1}{c|}{1, 2} \\
\hline
PLO9. รู้วิธีแสวงหา และถ่ายทอดความรู้ได้อย่างถูกต้องตามหลักวิชาการ ร่วมกับการใช้เทคโนโลยี เพื่อการนำเสนองานทางด้านคณิตศาสตร์หรือด้านที่เกี่ยวข้อง (Remembering) & \multicolumn{1}{c|}{\checkmark} & \multicolumn{1}{c|}{1} & \multicolumn{1}{c|}{\checkmark} & \multicolumn{1}{c|}{1} \\
\hline
PLO10. ใช้อุปกรณ์และเครื่องมือพื้นฐานทางด้านวิทยาศาสตร์ และเขียนหรือใช้โปรแกรมคอมพิวเตอร์สำหรับงานทางด้านคณิตศาสตร์ได้ (Applying) & \multicolumn{1}{c|}{\checkmark} & \multicolumn{1}{c|}{1, 2} & \multicolumn{1}{c|}{\checkmark} & \multicolumn{1}{c|}{1, 2} \\
\hline
\end{longtable}

\begin{longtable}{| >{\raggedright}p{0.6\textwidth} | >{\centering}p{0.07\textwidth} | >{\centering}p{0.07\textwidth} | >{\centering}p{0.07\textwidth} | >{\centering\arraybackslash}p{0.07\textwidth} |}
\caption{แสดงความสอดคล้องระหว่าง PLOs ที่เชื่อมโยงกับกับกรอบมาตรฐานคุณวุฒิระดับอุดมศึกษาแห่งชาติ}
\label{table: plo_ksec}
\\
\hline
\multicolumn{1}{|c|}{\textbf{ผลลัพธ์การเรียนรู้ที่คาดหวัง (PLOs)}} & \textbf{K} & \textbf{S} & \textbf{E} & \textbf{C} \\
\hline
\endhead

PLO1. ปฏิบัติตามจรรยาบรรณทางวิชาการ กฎระเบียบ และข้อบังคับขององค์กร & & & \checkmark &  \\
\hline
PLO2. อธิบายบทนิยาม หลักการ และทฤษฎีบททางด้านคณิตศาสตร์และวิทยาศาสตร์ที่สำคัญได้อย่างถูกต้อง & \checkmark & & & \\
\hline
PLO3. คำนวณเพื่อแก้ปัญหาทางด้านคณิตศาสตร์ตามหลักการ บทนิยาม และทฤษฎีบทได้อย่างถูกต้องเหมาะสม & & \checkmark & & \\
\hline
PLO4. พิสูจน์ข้อความและทฤษฎีบททางคณิตศาสตร์ได้อย่างถูกต้องและสมเหตุสมผลตามหลักตรรกศาสตร์และการให้เหตุผล & \checkmark &  & & \\
\hline
PLO5. ประยุกต์ใช้ความรู้ ทักษะ และเทคโนโลยีทางคณิตศาสตร์ในการแก้ปัญหาทางด้านวิทยาศาสตร์ วิศวกรรมศาสตร์ ธุรกิจ อุตสาหกรรม หรือศาสตร์ที่เกี่ยวข้อง &  & \checkmark & & \\
\hline
PLO6. สร้างหรือปรับปรุงกระบวนการคิดทางคณิตศาสตร์และการวิจัยที่นำไปสู่องค์ความรู้ใหม่หรือนวัตกรรม &  & \checkmark & &  \\
\hline
PLO7. ปรับตัวเข้ากับสถานการณ์และวัฒนธรรมขององค์กร มีความรับผิดชอบ และทำงานร่วมกับผู้อื่น & &  &  & \checkmark \\
\hline
PLO8. ใช้คณิตศาสตร์หรือสถิติเพื่อการวิเคราะห์ ประมวลผลข้อมูล และนำเสนอได้อย่างเหมาะสม & & \checkmark & & \\
\hline
PLO9. รู้วิธีแสวงหา และถ่ายทอดความรู้ได้อย่างถูกต้องตามหลักวิชาการ ร่วมกับการใช้เทคโนโลยี &  &  & & \checkmark \\
\hline
PLO10. ใช้อุปกรณ์และเครื่องมือพื้นฐานทางด้านวิทยาศาสตร์ และเขียนหรือใช้โปรแกรมคอมพิวเตอร์สำหรับงานทางด้านคณิตศาสตร์ได้ & & \checkmark & & \\
\hline
\multicolumn{5}{l}{\footnotesize K: Knowledge, S: Skill, E: Ethic, C: Character} \\
\end{longtable}

ทั้งนี้ PLOs ของหลักสูตรได้บรรจุใน มคอ.2 และเผยแพร่ให้แก่ผู้มีส่วนได้ส่วนเสียในช่องทางต่างๆ ที่เข้าถึงได้ง่าย โดยหลักสูตรมีกลไกในการตรวจสอบเพื่อให้มั่นใจว่าผู้มีส่วนได้ส่วนเสียได้รับรู้และเข้าใจในข้อมูลดังกล่าว ดังตาราง \ref{Table:C_to-SH}

\begin{longtable}{|>{\raggedright}p{0.35\textwidth}|>{\raggedright}p{0.3\textwidth}|>{\raggedright\arraybackslash}p{0.3\textwidth}|}
\caption{ตารางแสดงการสื่อสาร PLOs กับผู้มีส่วนได้ส่วนเสียและกลไกการประเมินผล}
\label{Table:C_to-SH}
\\
\hline
\multicolumn{1}{|c|}{\bf ช่องทางการสื่อสาร}&\multicolumn{1}{c|}{\bf ผู้มีส่วนได้ส่วนเสีย}&\multicolumn{1}{c|}{\bf การประเมินผล/การรับทราบ}\\
\hline
\endhead
เอกสาร\printprogram{} (มคอ. 2) & มหาวิทยาลัย\newline อาจารย์ผู้รับผิดชอบหลักสูตร\newline อาจารย์ผู้สอน & บันทึกการประชุมคณะกรรมการประจำคณะ/สาขาวิชาคณิตศาสตร์ \newline บันทึกการรับทราบจากหน่วยงานที่เกี่ยวข้อง \\
\hline
เว็บไซต์ของสำนักส่งเสริมวิชาการและงานทะเบียน, คณะ, สาขาวิชา & นักศึกษาปัจจุบัน\newline นักศึกษาใหม่\newline ผู้สนใจเข้าศึกษา & สถิติการเข้าชมเว็บไซต์ (Website analytics)\newline แบบสอบถามการรับรู้ข้อมูลของนักศึกษาใหม่\newline ช่องทางติดต่อสอบถามบนเว็บไซต์ \\
\hline
Facebook Page ของหลักสูตร & นักศึกษาปัจจุบัน\newline ศิษย์เก่า\newline ผู้สนใจทั่วไป & การมีส่วนร่วมกับโพสต์ (Engagement rate) \newline การสำรวจผ่าน Facebook Polls\newline ข้อความสอบถามผ่าน Inbox \\
\hline
การประชุม/สัมมนาผู้ใช้บัณฑิต & ผู้ใช้บัณฑิต\newline สถานประกอบการ & บันทึกการประชุม/สรุปผลการสัมมนา\newline แบบประเมินผลการจัดกิจกรรม\newline ข้อเสนอแนะจากผู้เข้าร่วม \\
\hline
วันปฐมนิเทศนักศึกษาใหม่ & นักศึกษาใหม่\newline ผู้ปกครอง & แบบสอบถามความเข้าใจหลังจบกิจกรรม\newline การตอบคำถามในช่วง Q\&A \\
\hline
\end{longtable}

\begin{doclist}
\docitem{เอกสาร\printprogram{} (มคอ. 2)}
\docitem{เอกสารการปรับปรุงแก้ไข\printprogram{} (สมอ. 08) }
\end{doclist}

\subcriteria{The programme to show that the expected learning outcomes for all courses are appropriately formulated and are aligned to the expected learning outcomes of the programme.}

\printprogram{} ได้กำหนดผลลัพธ์การเรียนรู้ที่คาดหวังของรายวิชา (Course Learning Outcomes: CLOs) ให้รับผิดชอบกับผลลัพธ์การเรียนรู้ที่คาดหวังของหลักสูตร (Programme Learning Outcomes: PLOs) การกำหนด CLOs ได้รับการออกแบบอย่างละเอียดและพิจารณาความเชื่อมโยงกับ PLOs เพื่อให้มั่นใจว่านักศึกษาจะได้รับความรู้และทักษะที่จำเป็นสำหรับการบรรลุ PLOs ของหลักสูตร

ตาราง \ref{table: closvsplos} แสดงตัวอย่างการกระจายผลลัพธ์การเรียนรู้ที่คาดหวังของหลักสูตร (PLOs) ใน\printprogram{} ลงสู่ผลลัพธ์การเรียนรู้ที่คาดหวังของรายวิชา (CLOs)  

\begin{longtable}{|>{\raggedright}p{0.45\textwidth}|>{\raggedright\arraybackslash}p{0.45\textwidth}|}
\caption{ตัวอย่างการกระจายผลลัพธ์การเรียนรู้ที่คาดหวังของหลักสูตร (PLOs) ลงสู่ผลลัพธ์การเรียนรู้ที่คาดหวังของรายวิชา (CLOs)}
\label{table: closvsplos}
\\
\hline
\multicolumn{1}{|c|}{\textbf{CLOs}} & \multicolumn{1}{|c|}{\textbf{PLOs}} \\
\hline
\endfirsthead

\caption[]{(ต่อ) ตัวอย่างการกระจายผลลัพธ์การเรียนรู้ที่คาดหวังของหลักสูตร (PLOs) ลงสู่ผลลัพธ์การเรียนรู้ที่คาดหวังของรายวิชา (CLOs)}
\\
\hline
\textbf{รายวิชา} & \textbf{PLOs} \\
\hline
\endhead

\underline{09-114-205 กำหนดการเชิงคณิตศาสตร์เบื้องต้น} 
\begin{enumerate}[label={CLO\arabic*}]
\item เขียนปัญหาทางวิทยาศาสตร์ วิศวกรรมและการเงินในรูปแบบกําหนดการเชิงคณิตศาสตร์ได้  
\item อธิบายตัวแบบกําหนดการเชิงเส้นและไม่เชิงเส้นได้ 
\item หาผลเฉลยของตัวแบบกําหนดการเชิงคณิตศาสตร์เบื้องต้นด้วยโปรแกรมได้ 
\item เขียนโปรแกรมเพื่อหาผลเฉลยของตัวแบบกําหนดการเชิงคณิตศาสตร์เบื้องต้นด้วยโปรแกรมได้ 
\item ประยุกต์ใช้ตัวแบบกำหนดการเชิงคณิตศาสตร์ในการแก้ปัญหาได้             
\end{enumerate}

&

\begin{enumerate}[label={}]
	\item PLO2 อธิบายบทนิยาม หลักการ และทฤษฎีบททางด้านคณิตศาสตร์และวิทยาศาสตร์ที่สำคัญได้อย่างถูกต้อง
\item PLO3 คำนวณเพื่อแก้ปัญหาทางด้านคณิตศาสตร์ตามหลักการ บทนิยาม และทฤษฎีบทได้อย่างถูกต้องเหมาะสม
\item PLO5 ประยุกต์ใช้ความรู้ ทักษะ และเทคโนโลยีทางคณิตศาสตร์ในการแก้ปัญหาทางด้านวิทยาศาสตร์ วิศวกรรมศาสตร์ ธุรกิจ อุตสาหกรรม หรือศาสตร์ที่เกี่ยวข้อง
\item PLO10 ใช้อุปกรณ์และเครื่องมือพื้นฐานทางด้านวิทยาศาสตร์ และเขียนหรือใช้โปรแกรมคอมพิวเตอร์สำหรับงานทางด้านคณิตศาสตร์ได้
\end{enumerate}
\\
\hline

\end{longtable}

\begin{doclist}
\docitem{เอกสาร\printprogram{} (มคอ. 2)}
\docitem{เอกสารปรับปรุง\printprogram{} (สมอ. 08)}
\docitem{เอกสารคำอธิบายรายวิชาใน\printprogram{} (มคอ. 3)}
\end{doclist}


\subcriteria{The programme to show that the expected learning outcomes consist of both generic outcomes (related to written and oral communication, problem-solving, information technology, team building skills, etc) and subject specific outcomes (related to knowledge and skills of the study discipline).}

PLOs ของ\printprogram{} ได้รับการออกแบบให้ครอบคลุมทั้งผลลัพธ์ทั่วไปและผลลัพธ์เฉพาะด้าน แสดงโดยสรุปดังตาราง \ref{table: req 1.1} 

\begin{center}
	\begin{longtable}{| >{\raggedright}p{0.6\textwidth} | p{0.1\textwidth} | p{0.1\textwidth} |}
		\caption{ความสัมพันธ์ของ PLOs กับ Specific LO และ Generic LO} 
		\label{table: req 1.1}
		\\
		\hline
		\multicolumn{1}{|c|}{\textbf{ผลลัพธ์การเรียนรู้ที่คาดหวัง (PLOs)}} & \multicolumn{1}{c|}{\textbf{Generic LO}} &\multicolumn{1}{c|}{ \textbf{Specific LO}} \\
		\hline
		\endhead
		PLO1. ปฏิบัติตามจรรยาบรรณทางวิชาการ กฎระเบียบ และข้อบังคับขององค์กร (Affective Domain) &\multicolumn{1}{c|}{\checkmark}&\\
		\hline
		PLO2. อธิบายบทนิยาม หลักการ และทฤษฎีบททางด้านคณิตศาสตร์และวิทยาศาสตร์ที่สำคัญได้อย่างถูกต้อง (Understanding) &&\multicolumn{1}{c|}{\checkmark}\\
		\hline
		PLO3. คำนวณเพื่อแก้ปัญหาทางด้านคณิตศาสตร์ตามหลักการ บทนิยาม และทฤษฎีบทได้อย่างถูกต้องเหมาะสม (Analyzing)&&\multicolumn{1}{c|}{\checkmark}\\
		\hline
		PLO4. พิสูจน์ข้อความและทฤษฎีบททางคณิตศาสตร์ได้อย่างถูกต้องและสมเหตุสมผลตามหลักตรรกศาสตร์และการให้เหตุผล (Evaluating)&&\multicolumn{1}{c|}{\checkmark}\\
		\hline
		PLO5. ประยุกต์ใช้ความรู้ ทักษะ และเทคโนโลยีทางคณิตศาสตร์ในการแก้ปัญหาทางด้านวิทยาศาสตร์ วิศวกรรมศาสตร์ ธุรกิจ อุตสาหกรรม หรือศาสตร์ที่เกี่ยวข้อง (Applying)&&\multicolumn{1}{c|}{\checkmark}\\
		\hline
		PLO6. สร้างหรือปรับปรุงกระบวนการคิดทางคณิตศาสตร์และการวิจัยที่นำไปสู่องค์ความรู้ใหม่หรือนวัตกรรมทางด้านคณิตศาสตร์ คณิตศาสตร์ประยุกต์ หรือด้านที่เกี่ยวข้อง (Creating)&&\multicolumn{1}{c|}{\checkmark}\\
		\hline
		PLO7. ปรับตัวเข้ากับสถานการณ์และวัฒนธรรมขององค์กร มีความรับผิดชอบ และทำงานร่วมกับผู้อื่นในฐานะผู้นำหรือสมาชิกที่ดี (Affective Domain)&\multicolumn{1}{c|}{\checkmark}&\\
		\hline
		PLO8. ใช้คณิตศาสตร์หรือสถิติเพื่อการวิเคราะห์ ประมวลผลข้อมูล และนำเสนอได้อย่างเหมาะสม (Evaluating)&&\multicolumn{1}{c|}{\checkmark}\\
		\hline
		PLO9. รู้วิธีแสวงหา และถ่ายทอดความรู้ได้อย่างถูกต้องตามหลักวิชาการ ร่วมกับการใช้เทคโนโลยี เพื่อการนำเสนองานทางด้านคณิตศาสตร์หรือด้านที่เกี่ยวข้อง (Remembering)&\multicolumn{1}{c|}{\checkmark}&\\
		\hline
		PLO10. ใช้อุปกรณ์และเครื่องมือพื้นฐานทางด้านวิทยาศาสตร์ และเขียนหรือใช้โปรแกรมคอมพิวเตอร์สำหรับงานทางด้านคณิตศาสตร์ได้ (Applying) &&\multicolumn{1}{c|}{\checkmark}\\
		\hline
	\end{longtable}
\end{center}
\begin{doclist}
\docitem{เอกสาร\printprogram{} (มคอ. 2)}
\docitem{เอกสารปรับปรุง\printprogram{} (สมอ. 08)}
\end{doclist}



\subcriteria{The programme to show that the requirements of the stakeholders, especially the external stakeholders, are gathered, and that these are reflected in the expected learning outcomes.}

หลักสูตรได้ดำเนินการรวบรวมและวิเคราะห์ความต้องการของผู้มีส่วนได้ส่วนเสีย (stakeholders) ทั้งภายในและภายนอกอย่างเป็นระบบ เพื่อนำข้อมูลมาใช้ในการออกแบบและทบทวน PLOs ให้บัณฑิตมีคุณลักษณะที่พึงประสงค์และสอดคล้องกับความต้องการของตลาดแรงงานและสังคม โดยมีกระบวนการดังนี้

\begin{enumerate}
    \item \textbf{กำหนดกลุ่มและวิธีการเก็บข้อมูล:} หลักสูตรได้กำหนดกลุ่มผู้มีส่วนได้ส่วนเสียและวิธีการเก็บรวบรวมข้อมูลที่เหมาะสมกับแต่ละกลุ่ม ดังแสดงในตาราง \ref{table: m2-sh}
    
    \item \textbf{วิเคราะห์และสังเคราะห์ข้อมูล:} ข้อมูลความต้องการจากทุกกลุ่มได้ถูกนำมาวิเคราะห์และสังเคราะห์เป็นกลุ่มทักษะและความรู้ที่สำคัญที่บัณฑิตพึงมี
    
    \item \textbf{กำหนดและทบทวน PLOs:} ผลการวิเคราะห์ได้ถูกนำมาใช้ในการกำหนดและทบทวนผลลัพธ์การเรียนรู้ที่คาดหวังระดับหลักสูตร (PLOs) ทั้ง 10 ข้อ เพื่อให้มั่นใจว่าครอบคลุมความต้องการของผู้มีส่วนได้ส่วนเสียทุกกลุ่ม ดังสรุปในตาราง \ref{table:stakeholder_summary}
\end{enumerate}

\begin{longtable}{|>{\raggedright}p{0.3\textwidth}|>{\raggedright\arraybackslash}p{0.6\textwidth}|}
    \caption{กลุ่มผู้มีส่วนได้ส่วนเสียและแนวทางการเก็บรวบรวมข้อมูล}
    \label{table: m2-sh}
    \\
    \hline
    \multicolumn{1}{|c|}{\bf กลุ่มผู้มีส่วนได้ส่วนเสีย}&\multicolumn{1}{c|}{\bf วิธีการเก็บรวบรวมข้อมูล}\\
    \hline
    \endhead
    \multicolumn{2}{|l|}{\textbf{ผู้มีส่วนได้ส่วนเสียภายนอก (External Stakeholders)}} \\ \hline
    ผู้ใช้บัณฑิต/สถานประกอบการ & การจัดประชุมกลุ่มย่อย (Focus Group) และสัมภาษณ์เชิงลึกกับผู้ประกอบการ เพื่อรับฟังความต้องการโดยตรง และใช้แบบสอบถามความพึงพอใจต่อคุณภาพบัณฑิตเป็นประจำทุกปี \\ \hline
    บัณฑิตที่สำเร็จการศึกษาจากหลักสูตร & การสำรวจภาวะการมีงานทำของศิษย์เก่าผ่านแบบสอบถามออนไลน์ และการสัมภาษณ์กลุ่มเพื่อรวบรวมข้อเสนอแนะในการปรับปรุงหลักสูตรจากประสบการณ์ทำงานจริง \\ \hline
    \multicolumn{2}{|l|}{\textbf{ผู้มีส่วนได้ส่วนเสียภายใน (Internal Stakeholders)}} \\ \hline
    มหาวิทยาลัย/คณะฯ & การทบทวนความสอดคล้องกับวิสัยทัศน์ พันธกิจ และแผนยุทธศาสตร์ของมหาวิทยาลัยและคณะฯ ผ่านการประชุมร่วมกับผู้บริหาร \\ \hline
    อาจารย์ผู้รับผิดชอบหลักสูตร และอาจารย์ผู้สอน & การประชุมหลักสูตรอย่างสม่ำเสมอ การระดมสมองเพื่อทบทวนรายวิชา และการสัมภาษณ์รายบุคคลเพื่อรวบรวมมุมมองด้านการสอน \\ \hline
    นักศึกษาปัจจุบันและนักศึกษาชั้นปีสุดท้าย & การรวบรวมข้อมูลผ่านแบบประเมินการสอน, การจัดประชุมรับฟังความคิดเห็น (Student Voice), และการสัมภาษณ์กลุ่มย่อย (Focus Group) \\ \hline
\end{longtable}

\begin{longtable}{| >{\raggedright}p{0.18\textwidth} | >{\raggedright}p{0.27\textwidth} | >{\raggedright}p{0.3\textwidth} | >{\centering\arraybackslash}p{0.15\textwidth} |}
    \caption{สรุปการรวบรวมความต้องการของผู้มีส่วนได้ส่วนเสียและการสะท้อนในผลการเรียนรู้ที่คาดหวัง (PLOs) จัดเรียงตามลำดับความสำคัญของความต้องการของผู้มีส่วนได้เสีย}
    \label{table:stakeholder_summary}
    \\
    \hline
    \textbf{ผู้มีส่วนได้ส่วนเสีย} & \textbf{ความต้องการของกลุ่มผู้มีส่วนได้ส่วนเสียหลัก} & \textbf{สรุปความต้องการ} & \textbf{สะท้อนอยู่ใน PLOs} \\
    \hline
    \endhead
    \multicolumn{4}{|l|}{\textbf{ผู้มีส่วนได้ส่วนเสียภายนอก (External Stakeholders)}} \\
    \hline
    ผู้ใช้บัณฑิต/สถานประกอบการ & 
ต้องการบัณฑิตที่สามารถบูรณาการความรู้ทางคณิตศาสตร์เชิงทฤษฎีเข้ากับการแก้ปัญหาทางธุรกิจได้จริง โดยคาดหวังให้บัณฑิตสร้างแบบจำลองทางคณิตศาสตร์เพื่อการพยากรณ์หรือหาค่าที่เหมาะสมที่สุด (Optimization) ได้ นอกเหนือจากทักษะเฉพาะทางแล้ว Soft Skills ถือเป็นปัจจัยสำคัญอย่างยิ่ง โดยเฉพาะความสามารถในการทำงานร่วมกับทีมสหสาขาวิชา และทักษะการสื่อสารที่สามารถย่อยผลการวิเคราะห์ที่ซับซ้อนให้เป็นข้อมูลเชิงลึก (Insight) ที่ผู้บริหารสามารถนำไปใช้ตัดสินใจเชิงกลยุทธ์ได้จริง อีกทั้งต้องพร้อมเรียนรู้และปรับตัวเข้ากับเทคโนโลยีและเครื่องมือวิเคราะห์ข้อมูลใหม่ๆ อยู่เสมอ &
- ทักษะการแก้ปัญหาและการคิดวิเคราะห์เชิงลึก การสร้างแบบจำลองทางคณิตศาสตร์เพื่ออธิบายปัญหา, การคิดเชิงตรรกะและวิพากษ์เพื่อประเมินแนวทางการแก้ปัญหา, การวิเคราะห์ข้อมูลเชิงปริมาณเพื่อหาความสัมพันธ์ที่ซ่อนอยู่ \newline
- ทักษะการสื่อสารและการทำงานร่วมกับผู้อื่น ความสามารถในการนำเสนอข้อมูล (Data Storytelling), การประสานงานในทีมแบบสหวิทยาการ, การรับฟังและให้ข้อคิดเห็นอย่างสร้างสรรค์ \newline
- ทักษะการใช้เทคโนโลยีและเครื่องมือดิจิทัล ความชำนาญในการเขียนโปรแกรม (เช่น Python, R), การใช้ซอฟต์แวร์ทางสถิติและคณิตศาสตร์, ความเข้าใจในหลักการของฐานข้อมูล (SQL) \newline
- ความรับผิดชอบและจรรยาบรรณวิชาชีพ ความซื่อสัตย์ในการจัดการข้อมูล (Data Integrity), การบริหารจัดการเวลาและภาระงาน, การปฏิบัติตามกฎระเบียบและวัฒนธรรมองค์กร &
\begin{enumerate}[label={}]
	\item PLO1
	\item PLO3
	\item PLO5
	\item PLO7
	\item PLO8
	\item PLO10
\end{enumerate}
 \\    
\hline
บัณฑิตที่สำเร็จการศึกษา & 
บัณฑิตต้องการให้หลักสูตรเชื่อมโยงทฤษฎีกับการปฏิบัติ โดยเน้นทักษะการแปลงโจทย์ธุรกิจเป็นปัญหาทางคณิตศาสตร์และเลือกใช้เครื่องมือที่เหมาะสม นอกจากนี้ยังให้ความสำคัญกับประสบการณ์ทำโครงงานเพื่อสร้างแฟ้มผลงาน และการเรียนรู้เทคโนโลยีใหม่ๆ ที่จำเป็นต่อการทำงานในสายอาชีพคณิตศาสตร์และวิทยาการข้อมูล 
&
- ความรู้ทางคณิตศาสตร์ที่ทันสมัยและประยุกต์ได้ ความเข้าใจในทฤษฎีอย่างลึกซึ้ง และความสามารถในการนำไปสร้างแบบจำลองแก้ปัญหาจริง \newline
- ทักษะการวิจัยและสร้างนวัตกรรม กระบวนการตั้งคำถาม, การออกแบบการทดลอง, การวิเคราะห์และสรุปผลเพื่อสร้างองค์ความรู้ใหม่ \newline
- ทักษะการเรียนรู้ด้วยตนเองและการแสวงหาความรู้ ความสามารถในการศึกษาหัวข้อใหม่ๆ จากเอกสารทางวิชาการ, การติดตามความก้าวหน้าในสายงาน \newline
- การประยุกต์ใช้ความรู้เพื่อการสื่อสารและถ่ายทอด การสรุปและนำเสนอแนวคิดที่ซับซ้อนให้เข้าใจง่าย &
\begin{enumerate}[label={}]
	\item PLO2
	\item PLO5
	\item PLO6
	\item PLO9
\end{enumerate}
 \\
    \hline
    \multicolumn{4}{|l|}{\textbf{ผู้มีส่วนได้ส่วนเสียภายใน (Internal Stakeholders)}} \\
    \hline
มหาวิทยาลัย/คณะฯ & 
มหาวิทยาลัยและคณะฯ กำหนดให้หลักสูตรต้องมีบทบาทสำคัญในการบรรลุเป้าหมายเชิงกลยุทธ์สูงสุด คือการเป็นมหาวิทยาลัยนวัตกรรมที่สร้างคุณค่าสู่สังคมและประเทศ  ดังนั้น ความต้องการหลักคือให้หลักสูตรสามารถผลิตบัณฑิตที่มีอัตลักษณ์ของมหาวิทยาลัย  ได้อย่างแท้จริง บัณฑิตต้องไม่เพียงแต่มีความรู้ แต่ต้องสามารถสร้างสรรค์งานวิจัยและนวัตกรรมใหม่ๆ ที่สามารถนำไปใช้ประโยชน์ได้จริง สอดคล้องกับพันธกิจของมหาวิทยาลัย &
- การสร้างบัณฑิตที่สะท้อนอัตลักษณ์ บัณฑิตต้องเป็นนักปฏิบัติและนักสร้างสรรค์นวัตกร  \newline
- การตอบสนองต่อวิสัยทัศน์และพันธกิจ หลักสูตรต้องสอดคล้องกับเป้าหมายการเป็นมหาวิทยาลัยแห่งนวัตกรรม  \newline
- การส่งเสริมการวิจัยและนวัตกรรม ผลผลิตของหลักสูตรต้องนำไปสู่การสร้างองค์ความรู้และนวัตกรรมใหม่  &
\begin{enumerate}[label={}]
	\item PLO1
	\item PLO5
	\item PLO6
	\item PLO7
\end{enumerate}
 \\
    \hline
    อาจารย์ในหลักสูตร &
ในมุมมองของคณาจารย์ผู้สอน ความต้องการสำคัญคือการมีโครงสร้างหลักสูตรที่ร้อยเรียงเนื้อหาอย่างเป็นลำดับ (Scaffolding) เพื่อให้นักศึกษามีความรู้พื้นฐานที่แข็งแกร่งพอที่จะศึกษาต่อในรายวิชาขั้นสูงได้ราบรื่น อาจารย์ต้องการความมั่นใจว่านักศึกษาที่ผ่านวิชาพื้นฐานจะมีความพร้อมตามที่คาดหวัง เพื่อให้สามารถมุ่งเน้นการสอนเนื้อหาเชิงลึกได้เต็มที่  &
- การวางโครงสร้างหลักสูตรที่ดี รายวิชาพื้นฐานต้องส่งเสริมการเรียนรู้ในวิชาขั้นสูงได้อย่างเหมาะสม \newline
- คุณภาพความรู้พื้นฐานของนักศึกษา ความพร้อมในการต่อยอดองค์ความรู้ \newline
- ทักษะการพิสูจน์และการให้เหตุผล เป็นหัวใจสำคัญของการคิดทางคณิตศาสตร์ &
\begin{enumerate}[label={}]
	\item PLO2
	\item PLO3
	\item PLO4
\end{enumerate}
 \\
    \hline
    นักศึกษาปัจจุบันและนักศึกษาชั้นปีสุดท้าย & 
การเตรียมความพร้อมเพื่อการประกอบอาชีพในอนาคต ดังนั้นจึงต้องการให้หลักสูตรมุ่งเน้นทักษะเชิงปฏิบัติที่สามารถนำไปใช้ทำงานได้จริงและเป็นที่ต้องการของตลาดแรงงาน โดยเฉพาะทักษะการเขียนโปรแกรมคอมพิวเตอร์เพื่อแก้ปัญหาทางคณิตศาสตร์ การวิเคราะห์และประมวลผลข้อมูลขนาดใหญ่ และความสามารถในการใช้ซอฟต์แวร์ทางสถิติและคณิตศาสตร์ได้อย่างคล่องแคล่ว นักศึกษายังต้องการโอกาสในการทำโครงงานที่จำลองมาจากปัญหาในโลกธุรกิจจริง เพื่อสร้างแฟ้มสะสมผลงาน (Portfolio) และเพิ่มความสามารถในการแข่งขัน นอกจากนี้ ทักษะการสื่อสาร การทำงานเป็นทีม และการนำเสนอผลงานอย่างมืออาชีพ ถือเป็นสิ่งสำคัญที่จะช่วยให้พวกเขาปรับตัวเข้ากับวัฒนธรรมองค์กรได้ดี &
- ทักษะเชิงปฏิบัติที่พร้อมใช้งาน ความสามารถในการนำความรู้ไปใช้แก้ปัญหาจริงได้ทันที \newline
- การวิเคราะห์และประมวลผลข้อมูล ทักษะการจัดการข้อมูล การวิเคราะห์ และการนำเสนอผล \newline
- การใช้โปรแกรมและเครื่องมือเฉพาะทาง ความชำนาญในการใช้ซอฟต์แวร์ที่จำเป็นต่อสายงาน \newline
- ทักษะการสื่อสารและการทำงานเป็นทีม การทำงานร่วมกับผู้อื่นและการนำเสนออย่างมีประสิทธิภาพ &
\begin{enumerate}[label={}]
	\item PLO3
	\item PLO5
	\item PLO7
	\item PLO8
	\item PLO9
	\item PLO10
\end{enumerate}
 \\
    \hline
\end{longtable}

\begin{doclist}
\docitem{การวิเคราะห์ความต้องการของผู้มีส่วนได้ส่วนเสีย}
\end{doclist}




\subcriteria{The programme to show that the expected learning outcomes are achieved by the students by the time they graduate.}

หลักสูตรได้จัดทำระบบการประเมินที่ครอบคลุมและหลากหลายเพื่อตรวจสอบและติดตามว่าผู้สำเร็จการศึกษาสามารถบรรลุผลลัพธ์การเรียนรู้ที่คาดหวัง (PLOs) ได้จริง หลักสูตรใช้แนวทางการประเมินแบบสามเส้า (Triangulation) เพื่อให้ได้ข้อมูลที่รอบด้านและน่าเชื่อถือ โดยรวบรวมข้อมูลจาก 3 แหล่งหลัก ดังนี้:
\begin{enumerate}
\item \textbf{การประเมินผลโดยตรง (Direct Assessment)} ผ่านผลงานและการวัดผลในชั้นเรียนโดยคณาจารย์
\item \textbf{การประเมินตนเองของนักศึกษา (Graduate Self-Assessment)} ผ่านแบบสำรวจนักศึกษาชั้นปีที่ 4 เมื่อสิ้นภาคการศึกษา 2/\printyear{}
\item \textbf{การประเมินความพึงพอใจของผู้ใช้บัณฑิต (Employer Satisfaction Assessment)} ซึ่งเป็นมุมมองสะท้อนกลับจากผู้มีส่วนได้ส่วนเสียภายนอก
\end{enumerate}


\noindent\textbf{1. การประเมินผลโดยตรงโดยคณาจารย์ (Direct Assessment)}
เป็นการประเมินการบรรลุ PLOs ผ่านการวัดผลการเรียนรู้ในรายวิชาต่างๆ ตลอดหลักสูตร  โดยอาจารย์ผู้สอนจะประเมินผลงานของนักศึกษา เช่น รายงาน, การสอบ, การปฏิบัติ, และโครงงาน ซึ่งผลการประเมินในระดับรายวิชา (CLOs) จะถูกนำมาเชื่อมโยงเพื่อสะท้อนการบรรลุผลในระดับหลักสูตร (PLOs) ดังแสดงผลสรุปในตาราง \ref{table:direct_assessment}


\begin{longtable}{|>{\centering}p{0.1\textwidth}| >{\raggedright}p{0.5\textwidth} | >{\centering\arraybackslash}p{0.3\textwidth}|}
\caption{สรุปผลการประเมินการบรรลุ PLOs โดยคณาจารย์ (Direct Assessment) ประจำปีการศึกษา \printyear{}}
\label{table:direct_assessment}
\\
\hline
\multicolumn{1}{|c|}{\bf PLO} & \multicolumn{1}{c|}{\bf ผลลัพธ์การเรียนรู้ที่คาดหวัง} & {\bfseries ร้อยละของนักศึกษาที่บรรลุในระดับพอใช้ (เกรด C) ขึ้นไป} \\
\hline
\endhead
PLO1 & ปฏิบัติตามจรรยาบรรณทางวิชาการ กฎระเบียบ และข้อบังคับขององค์กร & 96.20 \\ \hline
	PLO2 & อธิบายบทนิยาม หลักการ และทฤษฎีบททางด้านคณิตศาสตร์และวิทยาศาสตร์ที่สำคัญได้อย่างถูกต้อง & 88.17 \\ \hline
	PLO3 & คำนวณเพื่อแก้ปัญหาทางด้านคณิตศาสตร์ตามหลักการ บทนิยาม และทฤษฎีบทได้อย่างถูกต้องเหมาะสม & 88.5 \\ \hline
	PLO4 & พิสูจน์ข้อความและทฤษฎีบททางคณิตศาสตร์ได้อย่างถูกต้องและสมเหตุสมผลตามหลักตรรกศาสตร์และการให้เหตุผล & 90.68 \\ \hline
	PLO5 & ประยุกต์ใช้ความรู้ ทักษะ และเทคโนโลยีทางคณิตศาสตร์ในการแก้ปัญหาทางด้านวิทยาศาสตร์ วิศวกรรมศาสตร์ ธุรกิจ อุตสาหกรรม หรือศาสตร์ที่เกี่ยวข้อง & 88.04 \\ \hline
	PLO6 & สร้างหรือปรับปรุงกระบวนการคิดทางคณิตศาสตร์และการวิจัยที่นำไปสู่องค์ความรู้ใหม่หรือนวัตกรรมทางด้านคณิตศาสตร์ คณิตศาสตร์ประยุกต์ หรือด้านที่เกี่ยวข้อง & 95.65 \\ \hline
	PLO7 & ปรับตัวเข้ากับสถานการณ์และวัฒนธรรมขององค์กร มีความรับผิดชอบ และทำงานร่วมกับผู้อื่นในฐานะผู้นำหรือสมาชิกที่ดี & 92.39 \\ \hline
	PLO8 & ใช้คณิตศาสตร์หรือสถิติเพื่อการวิเคราะห์ ประมวลผลข้อมูล และนำเสนอได้อย่างเหมาะสม & 94.93 \\ \hline
	PLO9 & รู้วิธีแสวงหา และถ่ายทอดความรู้ได้อย่างถูกต้องตามหลักวิชาการ ร่วมกับการใช้เทคโนโลยี เพื่อการนำเสนองานทางด้านคณิตศาสตร์หรือด้านที่เกี่ยวข้อง & 97.83 \\ \hline
	PLO10& ใช้อุปกรณ์และเครื่องมือพื้นฐานทางด้านวิทยาศาสตร์ และเขียนหรือใช้โปรแกรมคอมพิวเตอร์สำหรับงานทางด้านคณิตศาสตร์ได้ & 90.00 \\ \hline
\end{longtable}

\noindent\textbf{2. การประเมินตนเองของนักศึกษา (Graduate Self-Assessment)}
หลักสูตรทำการสำรวจนักศึกษาชั้นปีที่ 4 เมื่อสิ้นภาคการศึกษา 2/\printyear{} เพื่อให้นักศึกษาประเมินระดับความสามารถของตนเองตาม PLOs แต่ละข้อ ซึ่งเป็นข้อมูลสะท้อนกลับทางอ้อมที่สำคัญที่แสดงถึงความเชื่อมั่นของบัณฑิตต่อการจัดการเรียนการสอนของหลักสูตร 

\begin{longtable}{|>{\centering}p{0.1\textwidth}| >{\raggedright}p{0.6\textwidth} | >{\centering\arraybackslash}p{0.2\textwidth}|}
\caption{สรุปผลการประเมินตนเองต่อการบรรลุ PLOs ของนักศึกษาชั้นปีที่ 4 เมื่อสิ้นภาคการศึกษา 2/\printyear{}}
\label{table:self_assessment}
\\
\hline
\multicolumn{1}{|c|}{\bf PLO} & \multicolumn{1}{c|}{\bf ผลลัพธ์การเรียนรู้ที่คาดหวัง} & {\bf คะแนนประเมินเฉลี่ย (เต็ม 5)} \\
\hline
\endhead
PLO1 & ปฏิบัติตามจรรยาบรรณทางวิชาการ กฎระเบียบ และข้อบังคับขององค์กร & 4.68 \\ \hline
	PLO2 & อธิบายบทนิยาม หลักการ และทฤษฎีบททางด้านคณิตศาสตร์และวิทยาศาสตร์ที่สำคัญได้อย่างถูกต้อง & 4.05 \\ \hline
	PLO3 & คำนวณเพื่อแก้ปัญหาทางด้านคณิตศาสตร์ตามหลักการ บทนิยาม และทฤษฎีบทได้อย่างถูกต้องเหมาะสม & 4.32 \\ \hline
	PLO4 & พิสูจน์ข้อความและทฤษฎีบททางคณิตศาสตร์ได้อย่างถูกต้องและสมเหตุสมผลตามหลักตรรกศาสตร์และการให้เหตุผล & 4.05 \\ \hline
	PLO5 & ประยุกต์ใช้ความรู้ ทักษะ และเทคโนโลยีทางคณิตศาสตร์ในการแก้ปัญหาทางด้านวิทยาศาสตร์ วิศวกรรมศาสตร์ ธุรกิจ อุตสาหกรรม หรือศาสตร์ที่เกี่ยวข้อง & 4.45 \\ \hline
	PLO6 & สร้างหรือปรับปรุงกระบวนการคิดทางคณิตศาสตร์และการวิจัยที่นำไปสู่องค์ความรู้ใหม่หรือนวัตกรรมทางด้านคณิตศาสตร์ คณิตศาสตร์ประยุกต์ หรือด้านที่เกี่ยวข้อง & 4.23 \\ \hline
	PLO7 & ปรับตัวเข้ากับสถานการณ์และวัฒนธรรมขององค์กร มีความรับผิดชอบ และทำงานร่วมกับผู้อื่นในฐานะผู้นำหรือสมาชิกที่ดี & 4.59 \\ \hline
	PLO8 & ใช้คณิตศาสตร์หรือสถิติเพื่อการวิเคราะห์ ประมวลผลข้อมูล และนำเสนอได้อย่างเหมาะสม & 4.05 \\ \hline
	PLO9 & รู้วิธีแสวงหา และถ่ายทอดความรู้ได้อย่างถูกต้องตามหลักวิชาการ ร่วมกับการใช้เทคโนโลยี เพื่อการนำเสนองานทางด้านคณิตศาสตร์หรือด้านที่เกี่ยวข้อง & 4.32 \\ \hline
	PLO10& ใช้อุปกรณ์และเครื่องมือพื้นฐานทางด้านวิทยาศาสตร์ และเขียนหรือใช้โปรแกรมคอมพิวเตอร์สำหรับงานทางด้านคณิตศาสตร์ได้ & 4.32 \\ \hline
\end{longtable}

\noindent\textbf{3. การประเมินความพึงพอใจของผู้ใช้บัณฑิต (Employer Satisfaction Assessment)}
เนื่องจากหลักสูตรเพิ่งเปิดดำเนินการและยังไม่มีผู้สำเร็จการศึกษาในปีการศึกษา 2567 การประเมินความพึงพอใจจากผู้ใช้บัณฑิตจึงยังไม่สามารถดำเนินการได้ อย่างไรก็ตาม หลักสูตรได้วางแผนและจัดทำกลไกสำหรับการประเมินสมรรถนะของบัณฑิตในอนาคตไว้อย่างเป็นระบบ

หลักสูตรจะเริ่มดำเนินการสำรวจหลังจากบัณฑิตรุ่นแรกได้เข้าสู่ตลาดแรงงานเป็นระยะเวลา 6 เดือน ถึง 1 ปี เพื่อให้ผู้ใช้บัณฑิตมีเวลาประเมินการปฏิบัติงานจริงได้อย่างชัดเจน เครื่องมือที่ใช้คือแบบสำรวจความพึงพอใจออนไลน์ที่จะส่งไปยังผู้บังคับบัญชาโดยตรงของบัณฑิต ซึ่งหัวข้อการประเมินจะออกแบบมาให้สอดคล้องกับผลลัพธ์การเรียนรู้ที่คาดหวัง (PLOs) ของหลักสูตร เพื่อยืนยันว่าบัณฑิตมีคุณภาพและตอบสนองต่อความต้องการของตลาดแรงงานจริง ผลลัพธ์ที่ได้จะถูกนำมาใช้เป็นข้อมูลสำคัญในการปรับปรุงและพัฒนาหลักสูตรต่อไป


\subsection*{การสังเคราะห์ผลการประเมินและการปรับปรุง}

 
หลักสูตรได้นำข้อมูลจากการประเมินผลโดยตรง (Direct Assessment) โดยคณาจารย์ และข้อมูลการประเมินตนเอง (Self-Assessment) ของนักศึกษาชั้นปีที่ 4 มาวิเคราะห์เปรียบเทียบกัน เพื่อให้เห็นภาพรวมของการบรรลุผลลัพธ์การเรียนรู้ที่คาดหวัง (PLOs) และได้ทำการทดสอบความสัมพันธ์ของข้อมูลทั้งสองชุดโดยใช้ การทดสอบสหสัมพันธ์เชิงอันดับของสเปียร์แมน (Spearman's Rank Correlation Coefficient) เพื่อประเมินว่ามุมมองของอาจารย์และนักศึกษามีความสอดคล้องกันในทิศทางเดียวกันหรือไม่

\subsubsection*{ขั้นตอนการคำนวณค่าสหสัมพันธ์}
การคำนวณค่าสหสัมพันธ์ของสเปียร์แมน ($\rho$) มีขั้นตอนดังนี้
\begin{enumerate}
	\item จัดอันดับข้อมูลของผลการประเมินโดยตรง (Direct Assessment) และผลการประเมินตนเอง (Self-Assessment) ของแต่ละ PLO จากค่ามากไปน้อย
	\item คำนวณหาค่าผลต่างของอันดับ ($d$) ของแต่ละ PLO
	\item นำค่าผลต่างของอันดับในแต่ละ PLO มายกกำลังสอง ($d^2$) แล้วหาผลรวมของค่าที่ได้ ($\sum d^2$)
	\item นำค่าที่ได้มาคำนวณในสูตร:
    $$\rho = 1 - \frac{6 \sum d^2}{n(n^2 - 1)}$$
    โดยที่ $d$ คือผลต่างของอันดับ และ $n$ คือจำนวนคู่ของข้อมูล (ในที่นี้คือ 10 PLOs)
\end{enumerate}

\begin{longtable}{|c|c|c|c|c|c|c|}
    \caption{ตารางการคำนวณค่าสหสัมพันธ์ของสเปียร์แมน (ข้อมูลปีการศึกษา \printyear{})}
    \label{table:spearman_calc_revised}
    \\
    \hline
    \textbf{PLO} & \textbf{Direct (\%)} & \textbf{Self (Score)} & \textbf{Rank (Direct)} & \textbf{Rank (Self)} & \textbf{d} & \textbf{d\textsuperscript{2}} \\
    \hline
    \endhead
    1  & 96.20 & 4.68 & 2  & 1   & 1    & 1.00    \\ \hline
    2  & 88.17 & 4.05 & 9  & 9   & 0    & 0.00    \\ \hline
    3  & 88.50 & 4.32 & 8  & 5   & 3    & 9.00    \\ \hline
    4  & 90.68 & 4.05 & 6  & 9   & -3   & 9.00    \\ \hline
    5  & 88.04 & 4.45 & 10 & 3   & 7    & 49.00   \\ \hline
    6  & 95.65 & 4.23 & 3  & 7   & -4   & 16.00   \\ \hline
    7  & 92.39 & 4.59 & 5  & 2   & 3    & 9.00    \\ \hline
    8  & 94.93 & 4.05 & 4  & 9   & -5   & 25.00   \\ \hline
    9  & 97.83 & 4.32 & 1  & 5   & -4   & 16.00   \\ \hline
    10 & 90.00 & 4.32 & 7  & 5   & 2    & 4.00    \\ \hline
    \multicolumn{6}{|r|}{\textbf{ผลรวม ($\sum d^2$)}} & \textbf{138.0} \\ \hline
\end{longtable}


\subsubsection*{ผลการวิเคราะห์และข้อค้นพบ}

จากตารางคำนวณ จะได้ $\sum d^2 = 138.0$ และเมื่อแทนค่าในสูตร จะได้ค่าสหสัมพันธ์ของสเปียร์แมน ($\rho$) เท่ากับ +0.164 ค่าสัมประสิทธิ์ที่มีค่าใกล้ศูนย์นี้บ่งชี้ถึงความสัมพันธ์เชิงบวกที่อ่อนมาก และเมื่อทำการทดสอบนัยสำคัญทางสถิติ พบว่าค่า p-value เท่ากับ 0.65 ซึ่งสูงกว่าระดับนัยสำคัญที่กำหนดไว้ ($\alpha = 0.05$) อย่างมีนัยสำคัญ 
ดังนั้นจึงสรุปได้ว่า มุมมองของคณาจารย์ต่อผลสัมฤทธิ์ของนักศึกษาและมุมมองของนักศึกษาต่อความสามารถของตนเองนั้นไม่มีความสัมพันธ์กันอย่างมีนัยสำคัญทางสถิติ ซึ่งชี้ให้เห็นถึง ``ช่องว่างในการรับรู้'' (Perception Gap) ที่ชัดเจน
และเมื่อพิจารณาในรายละเอียดของแต่ละ PLO พบประเด็นที่น่าสนใจดังนี้
\begin{itemize}
	\item กลุ่มที่สอดคล้องกัน (Alignment): PLO2 เป็นเพียงข้อเดียวที่อันดับจากการประเมินทั้งสองฝั่งตรงกัน (อันดับที่ 9) ซึ่งแสดงให้เห็นว่าทั้งอาจารย์และนักศึกษาต่างมองว่าการอธิบายหลักการและทฤษฎียังเป็นจุดที่ต้องพัฒนาเพิ่มเติมเมื่อเทียบกับ PLO อื่น ๆ
	\item กลุ่มที่นักศึกษาประเมินตนเองสูงกว่าผลประเมินจริง (Overconfidence): PLO5 เป็นกรณีที่น่าสนใจที่สุด โดยนักศึกษาประเมินตนเองว่ามีความสามารถในการประยุกต์ใช้ความรู้สูงเป็นอันดับที่ 3 แต่ผลการประเมินโดยตรงจากอาจารย์กลับอยู่ในอันดับสุดท้าย (อันดับที่ 10) ซึ่งอาจเกิดจากนักศึกษาเข้าใจขอบเขตของการประยุกต์ใช้เพียงผิวเผิน
	\item กลุ่มที่นักศึกษาประเมินตนเองต่ำกว่าผลประเมินจริง (Underconfidence): PLO9 และ PLO8 เป็นตัวอย่างที่ชัดเจน ซึ่งนักศึกษาสามารถทำคะแนนได้ดีมากในการประเมินโดยตรง (อันดับที่ 1 และ 4) แต่กลับประเมินความสามารถของตนเองไว้ค่อนข้างต่ำ (อันดับที่ 5 และ 9) ซึ่งอาจบ่งชี้ว่านักศึกษาขาดความมั่นใจในการนำทักษะไปใช้งานจริงนอกห้องเรียน
\end{itemize}

\subsubsection*{แผนการปรับปรุง}
จากผลการสังเคราะห์ข้างต้น หลักสูตรได้กำหนดแนวทางการปรับปรุงเพื่อลดช่องว่างในการรับรู้ ดังนี้
\begin{enumerate}
	\item {\bfseries ปรับปรุงความเข้าใจและเกณฑ์การประเมินใน PLO5:} จัดกิจกรรม "Workshop on Real-World Application" โดยเชิญวิทยากรจากภาคอุตสาหกรรมมานำเสนอกรณีศึกษาและโจทย์ปัญหาจริง พร้อมทั้งปรับปรุง Rubrics ในการประเมินโครงงานที่เกี่ยวข้องกับ PLO5 ให้สะท้อนความซับซ้อนของปัญหาในโลกแห่งความเป็นจริงมากขึ้น เพื่อให้นักศึกษาเข้าใจความคาดหวังและประเมินตนเองได้แม่นยำขึ้น
	\item {\bfseries เสริมสร้างความมั่นใจในการปฏิบัติสำหรับ PLO8 และ PLO9:} เพิ่มกิจกรรมการนำเสนอผลการวิเคราะห์ข้อมูลต่อหน้าชั้นเรียนหรือในรูปแบบของการประชุมวิชาการจำลอง และจัดตั้งคลินิกให้คำปรึกษาด้านการใช้เทคโนโลยีเพื่อการถ่ายทอดความรู้ เพื่อให้นักศึกษาได้ฝึกฝนและเห็นคุณค่าของทักษะที่ตนเองมี
	\item {\bfseries จัดตั้งกลไกการติดตามและประเมินผลอย่างต่อเนื่อง:} นำกระบวนการประเมินแบบสามเส้านี้มาใช้เป็นส่วนหนึ่งของวงจรการพัฒนาคุณภาพหลักสูตร (PDCA) และจะทำการวิเคราะห์เปรียบเทียบข้อมูลในทุกปีการศึกษาเพื่อติดตามผลของการปรับปรุงอย่างต่อเนื่อง
\end{enumerate}

\begin{doclist}
\docitem{แบบประเมินการบรรลุผลลัพธ์การเรียนรู้ที่คาดหวังของหลักสูตร (PLOs)}
\end{doclist}




































