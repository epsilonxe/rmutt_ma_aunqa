\chapter{ผลการดำเนินงานของหลักสูตร}

\begin{center}
(เกณฑ์มาตรฐานหลักสูตรระดับอุดมศึกษา พ.ศ. 2558)\\[1cm]
การรายงานผลการดำเนินงานของ\\ \printprogram{} \\
\printfaculty{}  \printuniversity{}\\
ประจำปีการศึกษา \printyear{} วันที่รายงาน \printrepdate{}
\end{center}

\section{ข้อมูลทั่วไป}

\subsection*{รหัสหลักสูตร}
25511911104688

\subsection*{อาจารย์ผู้รับผิดชอบหลักสูตร}

\begin{center}
\begin{tabular}{|l|l|p{0.3\textwidth}|}
\hline
\multicolumn{1}{|c|}{\textbf{มคอ. 2}}                  & \multicolumn{1}{c|}{\textbf{ปัจจุบัน}}                    &\begin{tabular}[c]{@{}c@{}}\textbf{หมายเหตุ}\\ (วันที่เปลี่ยนแปลงพร้อมเหตุผล)\end{tabular}                                        \\ \hline


1. นายสมนึก ศรีสวัสดิ์ \dag{}      & 1. นายสมนึก ศรีสวัสดิ์ \dag{}     & \multirow{5}{0.3\textwidth}{\textbf{ปีการศึกษา 2564}\newline อาจารย์อัคเรศ สิงห์ทา ได้ลาศึกษาต่อ จึงมีการปรับเปลี่ยนอาจารย์ผู้รับผิดชอบหลักสูตรจำนวน 1 ท่าน โดยปรับเปลี่ยนจากอาจารย์อัคเรศ สิงห์ทา เป็น อาจารย์รัฐพรหม พรหมคำ ตั้งแต่ภาคการศึกษาที่ 1 ปีการศึกษา 2564 เป็นต้นไปโดยสภามหาวิทยาลัยให้การอนุมัติในการประชุมครั้งที่ 8/2564 เมื่อวันที่ 25 สิงหาคม 2564 และได้มีรับทราบหลักสูตรในระบบ CHE-CO จาก กระทรวงการอุดมศึกษา วิทยาศาสตร์ วิจัย และนวัตกรรม (อว.) เป็นที่เรียบร้อยแล้ว}   \\ \cline{1-2}
2. นายพงศกร สุนทรายุทธ์      & 2. นายพงศกร สุนทรายุทธ์      &                                                                                                                                                                                                                                                                                                                                                                                                                                                                                                               \\ \cline{1-2}
3. นายวงศ์วิศรุต เขื่องสตุ่ง & 3. นายวงศ์วิศรุต เขื่องสตุ่ง &                                                                    \\ \hhline{--~}
{\cellcolor{lightgray!50!}{4. นายอัคเรศ สิงห์ทา}} & {\cellcolor{lightgray!50!}{4. นายรัฐพรหม พรหมคำ}}       &                                                           \\\cline{1-2}
5. นายมงคล ทาทอง             & 5. นายมงคล ทาทอง              &                                                                                                                                                                                                                                                                                                                                                                                                                                                                                                               \\ 
\Gape[36mm]{} & & \\ \hline
\end{tabular}   
\end{center}
\par\dag{} \textbf{ประธานหลักสูตร} 

\subsection*{คุณวุฒิและตำแหน่งอาจารย์ผู้รับผิดชอบสูตร}

\begin{center}
{\footnotesize
\begin{longtable}{|p{0.22\textwidth}|p{0.13\textwidth}|>{\raggedright}p{0.27\textwidth}|p{0.18\textwidth}|>{\centering\arraybackslash}p{0.1\textwidth}|}
\hline
\multicolumn{1}{|c|}{\textbf{ชื่อ-นามสกุล}} & \textbf{ตำแหน่งทางวิชาการ} & \multicolumn{1}{c|}{\textbf{คุณวุฒิ-สาขา}}               & \textbf{สถาบันที่สำเร็จการศึกษา}  & \textbf{ปีที่สำเร็จการศึกษา} \\ \hline

\endhead

\multirow[c]{2}{*}{1. นายสมนึก ศรีสวัสดิ์}  & \multirow{2}{0.15\textwidth}{ผู้ช่วยศาสตราจารย์} & วท.ม. (คณิตศาสตร์ประยุกต์) & สถาบันเทคโนโลยีพระจอมเกล้าเจ้าคุณทหารลาดกระบัง & 2545 \\ \cline{3-5} 
             &                   & วท.บ. (คณิตศาสตร์)         & มหาวิทยาลัยรามคำแหง                   & 2532                \\ \hline
                          
\multirow{3}{*}{2. นายพงศกร สุนทรายุทธ์} & \multirow{3}{0.15\textwidth}{รองศาสตราจารย์}     & ปร.ด. (คณิตศาสตร์ประยุกต์) & มหาวิทยาลัยเทคโนโลยีพระจอมเกล้าธนบุรี          & 2558 \\ \cline{3-5} 
             &                   & วท.ม. (คณิตศาสตร์ประยุกต์) & มหาวิทยาลัยเทคโนโลยีพระจอมเกล้าธนบุรี & 2553                \\ \cline{3-5} 
             &                   & วท.บ. (คณิตศาสตร์)         & มหาวิทยาลัยเทคโนโลยีพระจอมเกล้าธนบุรี & 2551                \\ \hline
             
\multirow{3}{*}{3. นายวงศ์วิศรุต เขื่องสตุ่ง} & \multirow{3}{0.15\textwidth}{ผู้ช่วยศาสตราจารย์}     & ปร.ด. (คณิตศาสตร์ประยุกต์) & สถาบันเทคโนโลยีพระจอมเกล้าเจ้าคุณทหารลาดกระบัง          & 2559 \\ \cline{3-5} 
             &                   & วท.ม. (คณิตศาสตร์ประยุกต์) & สถาบันเทคโนโลยีพระจอมเกล้าเจ้าคุณทหารลาดกระบัง & 2555                \\ \cline{3-5} 
             &                   & วท.บ. (คณิตศาสตร์ประยุกต์)         & สถาบันเทคโนโลยีพระจอมเกล้าเจ้าคุณทหารลาดกระบัง & 2553                \\ \hline
             
             
\multirow{3}{*}{4. นายรัฐพรหม พรหมคำ} & \multirow{3}{0.15\textwidth}{อาจารย์}     & Dr.rer.nat. (Mathematik) & Universität Würzburg          & 2562 \\ \cline{3-5} 
             &                   & วท.ม. (คณิตศาสตร์) & มหาวิทยาลัยธรรมศาสตร์ & 2552                \\ \cline{3-5} 
             &                   & วท.บ. (คณิตศาสตร์)         & มหาวิทยาลัยธรรมศาสตร์ & 2550                \\ \hline
 
 \multirow{2}{*}{5. นายมงคล ทาทอง}  & \multirow{2}{0.15\textwidth}{ผู้ช่วยศาสตราจารย์} & วท.ม. (คณิตศาสตร์ประยุกต์) & สถาบันเทคโนโลยีพระจอมเกล้าเจ้าคุณทหารลาดกระบัง & 2547 \\ \cline{3-5} 
             &                   & วท.บ. (คณิตศาสตร์)         & มหาวิทยาลัยรามคำแหง                   & 2543                \\ \hline
             

\end{longtable}}
\end{center}

\subsection*{อาจารย์ประจำหลักสูตร}

\printprogram{} มีอาจารย์ประจำหลักสูตรเป็นอาจารย์ชุดเดียวกันกับอาจารย์ผู้รับผิดชอบหลักสูตร ซึ่งมีรายละเอียดดังที่แสดงไว้ในข้างต้น

\subsection*{อาจารย์ผู้สอน}
\begin{center}
	{\footnotesize     
		\begin{longtable}{|p{0.25\textwidth}|p{0.12\textwidth}|p{0.25\textwidth}|p{0.18\textwidth}|p{0.1\textwidth}|}
			\hline
			\multicolumn{1}{|c|}{\textbf{ชื่อ-นามสกุล}} & \textbf{ตำแหน่งทางวิชาการ} & \multicolumn{1}{c|}{\textbf{คุณวุฒิ-สาขา}}               & \textbf{สถาบันที่สำเร็จการศึกษา}               & \textbf{ปีที่สำเร็จการศึกษา} \\ \hline
			
			\endhead
		%%%%%%%%%%%%%%%%%%%%%%%%%%%%%%%%%%%%%%%%%%%%%%%%%%%%%
		\multirow{3}{*}{1. นายพงศกร สุนทรายุทธ์}
				&รองศาสตราจารย์	&ปร.ด.(คณิตศาสตร์ประยุกต์) &ม.เทคโนโลยีพระจอมเกล้าธนบุรี&2558\\
				\cline{3-5}
				&&วท.ม. (คณิตศาสตร์ประยุกต์)&ม.เทคโนโลยีพระจอมเกล้าธนบุรี&2553\\
				\cline{3-5}
				&& วท.บ. (คณิตศาสตร์)&ม.เทคโนโลยีพระจอมเกล้าธนบุรี&2551\\
				\hline
		
		
		\multirow{2}{*}{2. นางกุลประภา ศรีหมุด}
				&ผู้ช่วยศาสตราจารย์&วท.ม. (คณิตศาสตร์)&จุฬาลงกรณ์มหาวิทยาลัย&2545\\	\cline{3-5}
				&&วท.บ. (คณิตศาสตร์)&จุฬาลงกรณ์มหาวิทยาลัย&2542\\	
				\hline
				
		
		\multirow{2}{*}{3. นายสมนึก ศรีสวัสดิ์}
				&ผู้ช่วยศาสตราจารย์& วท.ม. (คณิตศาสตร์ประยุกต์)&สถาบันเทคโนโลยีพระจอมเกล้าเจ้าคุณทหารลาดกระบัง&2545\\
				\cline{3-5}
				&&วท.บ. (คณิตศาสตร์)&ม. รามคำแหง&2532\\
				\hline
				
				
		\multirow{2}{*}{4. นางสาวกมลรัตน์ สมบุตร}
				&ผู้ช่วยศาสตราจารย์&ปร.ด. (คณิตศาสตร์)&ม. นเรศวร&2556\\
				\cline{3-5}
				&&คบ. (คณิตศาสตร์)&ม. ราชภัฏอุตรดิตถ์&2549\\
				\hline
				
				
		\multirow{3}{*}{5. นายวงศ์วิศรุต เขื่องสตุ่ง}
				&ผู้ช่วยศาสตราจารย์&ปร.ด. (คณิตศาสตร์ประยุกต์)&สถาบันเทคโนโลยีพระจอมเกล้าเจ้าคุณทหารลาดกระบัง&2559\\
				\cline{3-5}
				
				&&วท.ม. (คณิตศาสตร์ประยุกต์)&สถาบันเทคโนโลยีพระจอมเกล้าเจ้าคุณทหารลาดกระบัง&2555\\
				\cline{3-5}
				&& วท.บ. (คณิตศาสตร์)&สถาบันเทคโนโลยีพระจอมเกล้าจ้าคุณทหารลาดกระบัง&2553\\
					\hline
					
\newpage %%%%%%%%%%%%%%%%%%%%%%%%%%%%%%%

		\multirow{3}{*}{6. นางภคีตา สุขประเสริฐ}&	ผู้ช่วยศาสตราจารย์ & ปร.ด. (คณิตศาสตร์ประยุกต์)& ม. เทคโนโลยี\newline พระจอมเกล้าธนบุรี& 2561\\\cline{3-5}
			&& วท.ม. (คณิตศาสตร์)& 	ม. ธรรมศาสตร์& 2554\\\cline{3-5}
			&& วท.บ. (คณิตศาสตร์)& 	ม. ธรรมศาสตร์&2550\\	\hline


		\multirow{3}{*}{7. นายปริญญวัฒน์ ชูสุวรรณ}&ผู้ช่วยศาสตราจารย์&	วท.ด.(คณิตศาสตร์)&จุฬาลงกรมหาวิทยาลัย&2561\\\cline{3-5}
		&&วท.ม. (คณิตศาสตร์)& 	จุฬาลงกรมหาวิทยาลัย& 2557\\\cline{3-5}
		&&วทบ. (คณิตศาสตร์)& ม. สงขานครินทร์&	2555\\	\hline
		
		
		\multirow{2}{*}{8. นายมงคล ทาทอง}&	ผู้ช่วยศาสตราจารย์&วท.ม. (คณิตศาสตร์ประยุกต์)&สถาบันเทคโนโลยีพระจอมเกล้าเจ้าคุณทหารลาดกระบัง&2547\\
		\cline{3-5}
		&&วท.บ. (คณิตศาสตร์)&ม.รามคำแหง&2543\\ \hline
		
		
		\multirow{3}{*}{9. นางสาวนนธิยา มากะเต}&อาจารย์&วท.ด.(คณิตศาสตร์ประยุกต์)&ม.เทคโนโลยีสุรนารี&2556\\\cline{3-5}
		&&วท.ม. (คณิตศาสตร์)&ม.เชียงใหม่&2545\\\cline{3-5}
		&&วท.บ. (คณิตศาสตร์)&	ม.นเรศวร&2543\\\hline
	

		\multirow{3}{*}{10.  นางวรรณา ศรีปราชญ์}& อาจารย์& ปร.ด. (คณิตศาสตร์)& ม. นเรศวร& 2554\\\cline{3-5}
		&& วท.ม. (คณิตศาสตร์)&	ม. นเรศวร&2548\\\cline{3-5}
		&& คบ. (คณิตศาสตร์)& มหาวิทยาลัยราชภัฏพระนครศรีอยุธยา& 2541\\\hline
		

		\multirow{3}{*}{11. นายรัฐพรหม พรหมคำ}&	อาจารย์&Dr.rer.nat (Mathematik)&Universi\"{a}t W\"{u}rzburg&2562\\\cline{3-5}
		&&วท.ม. (คณิตศาสตร์)& ม.ธรรมศาสตร์&2552\\\cline{3-5}
		&&วท.บ. (คณิตศาสตร์)& ม.ธรรมศาสตร์&2550\\\hline
		
		
		\multirow{2}{*}{12. นายอลงกต สุวรรณมณี}&อาจารย์&วท.ม.(คณิตศาสตร์ประยุกต์)& 	ม.มหิดล&2549\\\cline{3-5}
		&&วท.บ. (คณิตศาสตร์)&ม.มหิดล& 2546\\\hline
		
		
		\multirow{2}{*}{13. นายโอม สถิตยนาค}&อาจารย์	& วท.ม.  (คณิตศาสตร์)& จุฬาลงกรณ์มหาวิทยาลัย& 2551\\\cline{3-5}
		&&วท.บ. (คณิตศาสตร์)&ม. ธรรมศาสตร์&2547\\\hline
		
		\multirow{2}{*}{14. นางสาววาสนา ทองกำแหง}&อาจารย์&วท.ม. (คณิตศาสตร์)& ม. รามคำแหง&2551\\\cline{3-5}
		&&วท.บ. (คณิตศาสตร์)&ม. ศรีนครินทรวิโรฒประสานมิตร&2543\\\hline
	
\newpage%%%%%%%%%%%%%%%%%%%%%%%%%%%%%%%%%%%%%%%%%%%%%%%%%%%
	
		15. นายอัคเรศ สิงห์ทา\newline (ลาศึกษาต่อ)&อาจารย์&
		วท.ม. (คณิตศาสตร์)& ม. รามคำแหง&2551\\\cline{3-5}
		&& วท.บ. (คณิตศาสตร์)& 	ม.ศรีนครินทรวิโรฒประสานมิตร&2543\\\hline
	

	\multirow{2}{*}{16. นางอมราภรณ์ บำเพ็ญดี}&	อาจารย์&วท.ม.(คณิตศาสตร์)&ม.รามคำแหง&2550\\\cline{3-5}
	&&วท.บ.(คณิตศาสตร์)&ม.ศรีนครินทรวิโรฒประสานมิตร&2543\\\hline
	

	\multirow{2}{*}{17. นางสาวธาวัลย์ อัมพวา}&	อาจารย์&วท.ม.(คณิตศาสตร์)&ม.เทคโนโลยีราชมงคลธัญบุรี&2557\\\cline{3-5}
	&&วท.บ.(คณิตศาสตร์)&ม.รามคำแหง&2534\\\hline

    18. นางสาวปฤณท์ธพร\newline สงวนสุทธิกุล &อาจารย์&ปร.ด. (คณิตศาสตร์ประยุกต์)&ม.เทคโนโลยีพระจอมเกล้าธนบุรี&2563\\
	\cline{3-5}
	&&วท.ม. (คณิตศาสตร์ประยุกต์)&ม.เทคโนโลยีพระจอมเกล้าธนบุรี&2559 \\
	\cline{3-5}
	&&วท.บ. (คณิตศาสตร์) &ม.ศรีนครินทรวิโรฒประสานมิตร&2557\\\hline
		\end{longtable}}
\end{center}
%%%%%%%%%%%%%%%%%%%%%%%%%%

\subsection*{อาจารย์พิเศษ}

\begin{center}
\centering
\begin{tabular}{|p{0.24\textwidth}|p{0.24\textwidth}|p{0.24\textwidth}|p{0.24\textwidth}|}
\hline
\multicolumn{1}{|c|}{\bf ลำดับ} & \multicolumn{1}{c|}{\bf ชื่อ-สกุล} & \multicolumn{1}{c|}{\bf ตำแหน่ง} & \multicolumn{1}{c|}{\bf สถานที่ทำงาน} \\
\hline
-     & -         & -       & - \\
\hline          
\end{tabular}
\end{center}

\begin{remark*}
ในปีการศึกษา \printyear{} ไม่มีการเชิญอาจารย์พิเศษ    
\end{remark*}

\clearpage
\subsection*{สถานที่จัดการเรียนการสอน}

\begin{flushleft}
\begin{tabular}{lp{0.7\textwidth}}
อาคารเรียน &	อาคารเฉลิมพระเกียรติ ๖ รอบพระชนมพรรษา \printfaculty{}  \printuniversity{} \\        
จำนวนห้องเรียน	 &	3 ห้อง \\
จำนวนห้องปฏิบัติการ &	3 ห้อง  \\   
\end{tabular}    
\end{flushleft}
{\small
	\begin{tabular}{|p{0.15\textwidth}|p{0.4\textwidth}|>{\centering}p{0.1\textwidth}|>{\centering}p{0.1\textwidth}|>{\centering\arraybackslash}p{0.15\textwidth}|}
		\hline
		\multicolumn{1}{|c|}{\multirow{2}{*}{\textbf{ชื่ออาคาร}}} &
		\multicolumn{1}{c|}{\multirow{2}{*}{\textbf{ชื่อห้องเรียน/ห้องปฏิบัติการ}}} &
		\multicolumn{2}{c|}{\textbf{ประเภทห้อง}} &
		\textbf{ขนาดความจุ (คน)} \\
		\cline{3-4}
		& & ห้องเรียน & ห้องปฏิบัติการ & \\
		\hline 
		\multirow{6}{0.15\textwidth}{อาคารเฉลิมพระเกียรติ ๖ รอบพระชนมพรรษา} & 
		ห้องบรรยายรวม ST-1301 &
		\multicolumn{1}{c|}{$\checkmark$} &
		&
		80 \\ \cline{2-5}&
		ห้อง Research   and Discussion ST-1908 &
		\multicolumn{1}{c|}{} &
		$\checkmark$ &
		20 \\ \cline{2-5} 
		&
		ห้องปฏิบัติการคอมพิวเตอร์ ST-1905 &
		\multicolumn{1}{c|}{} &
		$\checkmark$ &
		25 \\ \cline{2-5} 
		&
		ห้อง   Smart Class Room ST-1906 &
		\multicolumn{1}{c|}{} &
		$\checkmark$ &
		40 \\ \cline{2-5} 
		&
		ห้องบรรยายรวม ST-1910 &
		\multicolumn{1}{c|}{$\checkmark$} &
		&	40  \\ \cline{2-5} 
		&
		ห้องบรรยายรวม ST-1911 &
		\multicolumn{1}{c|}{$\checkmark$} &
		&
		40 \\ \hline
\end{tabular}}
\begin{remark*}
สำหรับรายวิชาศึกษาทั่วไป หลักสูตรฯ ใช้ห้องเรียนที่อาคารปฏิบัติการเรียนรวม    
\end{remark*}


\cleardoublepage
%%%%%%%%%%% 2.2 %%%%%%%%%%%%%%%%%%%%%%%%%%%%%%%%%%
\section{การกำกับให้เป็นไปตามมาตรฐาน (องค์ประกอบที่ 1 การกำกับมาตรฐาน)}

\subsection*{1. จำนวนอาจารย์ผู้รับผิดชอบหลักสูตร}

\printprogram{} มีจำนวนอาจารย์ผู้รับผิดชอบหลักสูตร จำนวน 5 คน โดยอาจารย์ผู้รับผิดชอบหลักสูตรทุกคนมีคุณวุฒิตรงและสัมพันธ์กับหลักสูตรที่เปิดสอน มีผลงานทางวิชาการที่ไม่ใช่ส่วนหนึ่งของการศึกษาเพื่อรับปริญญา และเป็นผลงานทางวิชาการที่ได้รับการเผยแพร่ ตามหลักเกณฑ์ที่กำหนดในการพิจารณาแต่งตั้งให้บุคคลดำรงตำแหน่งทางวิชาการอย่างน้อย 1 รายการ ในรอบ 5 ปี และทุกคนเป็นอาจารย์ผู้รับผิดชอบหลักสูตรเพียงหลักสูตรเดียว และประจำหลักสูตรตลอดระยะเวลาที่จัดการศึกษาตามหลักสูตร

\begin{center}
\begin{tabular}{|P{0.4\textwidth}|P{0.1\textwidth}|P{0.1\textwidth}|P{0.1\textwidth}|P{0.1\textwidth}|}
\hline
ตำแหน่งทางวิชาการ/ วุฒิการศึกษา & อาจารย์ & ผศ. & รศ. & ศ. \\ \hline
ปริญญาตรี & - & - & - & - \\ \hline
ปริญญาโท  & - & 2 & - & - \\ \hline
ปริญญาเอก & 1 & 1 & 1 & - \\ \hline
\end{tabular}
\end{center}

\printselfeval   %ประเมินตนเอง

\subsection*{2. คุณสมบัติของอาจารย์ผู้รับผิดชอบหลักสูตร}

\printprogram{} มีอาจารย์ผู้รับผิดชอบหลักสูตรดำรงตำแหน่งทางวิชาการ รองศาสตราจารย์คุณวุฒิปริญญาเอก จำนวน 1 คน ผู้ช่วยศาสตราจารย์คุณวุฒิปริญญาเอก จำนวน 1  คน อาจารย์คุณวุฒิปริญญาเอก จำนวน 1 คน และผู้ช่วยศาสตราจารย์คุณวุฒิปริญญาโท จำนวน 2 คน โดยอาจารย์ผู้รับผิดชอบหลักสูตรทุกคนมีคุณวุฒิตรงและสัมพันธ์กับหลักสูตรที่เปิดสอน มีผลงานทางวิชาการที่ไม่ใช่ส่วนหนึ่งของการศึกษาเพื่อรับปริญญา และเป็นผลงานทางวิชาการที่ได้รับการเผยแพร่ตามหลักเกณฑ์ที่กำหนดในการพิจารณาแต่งตั้งให้บุคคลดำรงตำแหน่งทางวิชาการอย่างน้อย 1 รายการ ในรอบ 5 ปีย้อนหลัง ดังนี้
%%%%%%%%%%%%%%%%
\begin{center}
{\small
	\begin{longtable}{|p{0.26\textwidth}|p{0.2\textwidth}|p{0.28\textwidth}|P{0.18\textwidth}|}
		\hline
		\multicolumn{1}{|c|}{\textbf{ชื่อ-นามสกุล}} & \multicolumn{1}{c}{\textbf{ตำแหน่งทางวิชาการ}} & \multicolumn{1}{|c|}{\textbf{คุณวุฒิ-สาขา}} & \textbf{จำนวนผลงานวิจัยย้อนหลัง 5 ปี} \\ \hline
		\endhead
		1. นายสมนึก ศรีสวัสดิ์  
		& ผู้ช่วยศาสตราจารย์ 
		& วท.ม. (คณิตศาสตร์ประยุกต์) 
		& 7
		\\ \hline           
		2. นายพงศกร สุนทรายุทธ์
		& รองศาสตราจารย์
		& ปร.ด. (คณิตศาสตร์ประยุกต์)
		& 35 
		\\ \hline        
		3. นายวงศ์วิศรุต เขื่องสตุ่ง 
		& ผู้ช่วยศาสตราจารย์     
		& ปร.ด. (คณิตศาสตร์ประยุกต์) 
		& 18
		\\ \hline                
		4. นายรัฐพรหม พรหมคำ 
		& อาจารย์     
		& Dr.rer.nat. (Mathematik) 
		& 5          
		\\ \hline
		5. นายมงคล ทาทอง
		& ผู้ช่วยศาสตราจารย์ 
		& วท.ม. (คณิตศาสตร์ประยุกต์) 
		& 4 
		\\ \hline       
	\end{longtable}}
\end{center}

\noindent โดยอาจารย์ผู้รับผิดชอบหลักสูตรมีผลงานวิจัยย้อนหลัง 5 ปีดังนี้
%%%%%%%%%%%%%%%%%%%%%%%%%%%%%%%%%%%%%%%%%%%%%%%
{\small
\begin{center}
\begin{longtable}{|p{0.28\textwidth}|p{0.3\textwidth}|p{0.25\textwidth}|p{0.1\textwidth}|}
	\hline
	\multicolumn{1}{|c|}{\textbf{ชื่อ-นามสกุล}} &
	\multicolumn{1}{c|}{\textbf{ชื่อผลงาน}} &
	\multicolumn{1}{c|}{\textbf{แหล่งเผยแพร่/ตีพิมพ์}} &
	\textbf{ปีที่ตีพิมพ์}\\
	\hline
\endhead

\hline
\endfoot	


1. นายสมนึก ศรีสวัสดิ์&
1. Novel inertial methods for fixed point problems in reflexive Banach spaces with applications &
Rendiconti del Circolo Matematico di Palermo Series 2&
2023 
\\ \cline{2-4}

&2. On the Vieta-Jacobsthal-like polynomial &
Note on number Theory and Discrete Mathematics &
2022 \\ \cline{2-4}

&3. An Iterative Method for Solving Split Monotone   Variational Inclusion Problems and Finite Family of Variational Inequality Problems in Hilbert Spaces&
International Journal of Mathematics and Mathematical Sciences&2021 
\\ \cline{2-4}

&4. Vieta-Pell-like Polynomails and aome Identities &
Journal of Science and Arts & 2021 
\\ \cline{2-4}

&5. Vieta-Fibonacci-like polynomials and some identities &
Annales Mathematicae et Informaticae &2021 
\\ \cline{2-4}

&6. On the (s,t)-Pell and (s,t)-Pell-Lucas Polynomials&
Progress in Applied Science and Technology&
2021 \\ \cline{2-4}

&7. Weak and Strong Convergence of Hybrid Subgradient Method for Pseudomonotone Equilibrium Problems and Nonspreading-Type Mappings in Hilbert Spaces&
Kyungpook Mathematical Journal &
2019 
\\ \hline
%====================================================

2. นายพงศกร สุนทรายุทธ์&
1. Novel inertial methods \newline for   
fixed point problems in  
reflexive Banach spaces 
with applications &
Rendiconti del Circolo Matematico di Palermo Series 2&
2023 \\ \cline{2-4}

&2. Inertial-like Bregman\newline 
projection method for 
solving systems of 
variational inequalities
&Mathematical Methods in the Applied Sciences
& 2023 \\ \cline{2-4}		

&3. Inertial projection and \newline  
contraction methods for 
solving variational
inequalities with
applications to image   
restoration problems							
&
Carpathian Journal 
of Mathematics&
2023 
\\ \cline{2-4}

&4. Two-step inertial method
for solving split common
null point problem with
multiple output sets in
Hilbert spaces
&AIMS Mathematics&
2023 \\ \cline{2-4}	

&5. Modified accelerated \newline   
Bregman projection 
methods for solving quasi- 
monotone variational  
inequalities
&Optimization &
2023 \\ \cline{2-4}	

&6. Modified inertial 
extragradient methods for 
finding minimum-norm solution
of the variational inequality
problem with applications to
optimal control problem
&International 
Journal of Computer 
Mathematics&
2022 \\ \cline{2-4}

&7. Analysis of two \newline versions of
relaxed inertial algorithms
with Bregman divergences 
for solving variational 
inequalities
&Computational and 
Applied Mathematics
&2022\\\cline{2-4}	

&8. The Analysis of\newline  Fractional-Order System Delay 
Differential Equations
Using a Numerical Method					
&Complexity
&2022\\ \cline{2-4}	

&9.	Solving Fractional-Order 
Diffusion Equations in a 
Plasma and Fluids via a
Novel Transform 
&Journal of
Function Spaces
&2022 \\ \cline{2-4}

&10. Weak and strong \newline
convergence results for
solving monotone variational
inequalities in reflexive
Banach spaces
&Optimization
&2022 
\\ \cline{2-4}

&11. A Novel Multicriteria \newline
Decision-Making 
Approach for Einstein 
Weighted Average
Operator under
Pythagorean Fuzzy
Hypersoft Environment
&Journal of Mathematics
&2022 \\ \cline{2-4}		

&12. Phenomena of thermo-sloutal time’s relaxation in
mixed convection Carreau fluid with heat sink/Source						
&Waves in Random and Complex Media
&2022 \\ \cline{2-4}

&13. A New Self-Adaptive\newline
Method for the Multiple-Sets Split Common Null 
Point Problem in Banach Spaces
&Vietnam Journal of Mathematics	
&2022 
\\ \cline{2-4}
	
&14.	Analysis of non-singular
fractional bioconvection 
and thermal memory 
with generalized Mittag-Leffler kernel				
&Chaos, Solitons and Fractals
&2022 
\\ \cline{2-4}	

&15. Numerical solution \newline of
stochastic and fractional 
competition model in
Caputo derivative using 
Newton method
&AIMS Mathematics
&2022 \\ \cline{2-4}	

&16. Unsteady MHD Flow for
Fractional Casson Channel
Fluid in a Porous Medium:
An Application of the Caputo-Fabrizio Time  Fractional Derivative
&Journal of Function Spaces
&2022\\ \cline{2-4}	

&17. Impact of nanoparticle
aggregation on heat
transfer phenomena of 
second grade nanofluid
flow over melting surface
subject to homogeneous
heterogeneous reactions			
&Case Studies in Thermal Engineering
&2022\\\cline{2-4}

&18. Two New Inertial \newline
Algorithms for Solving 
Variational Inequalities in 
Reflexive Banach Spaces
&Numerical Functional Analysis 
and Optimization
&2021 
\\ \cline{2-4}

&19. An iterative algorithm
with inertial technique
for solving the split
common null point problem 
in Banach spaces
&Asian-European Journal of Mathematics
&2021 \\ \cline{2-4}		

&20. Convergence results of 
iterative algorithms for 
the sum of two monotone 
operators in reflexive 
Banach spaces					
&Applications of Mathematics
&2021 \\ \cline{2-4}	

&21. A Generalized Self-\newline Adaptive Algorithm for
the Split Feasibility
Problem in Banach Spaces
&Bulletin of the Iranian Mathematical Society		
&2021 \\ \cline{2-4}
	
&22. An inertial self-adaptive
algorithm for the 
generalized split common
null point problem in
Hilbert spaces			
&Rendiconti del
Circolo Matematico
di Palermo Series 2
&2021\\ \cline{2-4}	

&23. New Bregman \newline projection
methods for solving 
seudo-monotone
variational inequality
problem
&Journal of Applied Mathematics 
and Computing
&2021 \\ \cline{2-4}
	
&24. Mann-type algorithms for 
solving the monotone 
inclusion problem and 
the fixed point problem 
in reflexive Banach spaces
&Ricerche di Matematica
&2021\\ \cline{2-4}

&25. The Comparative Study
for Solving Fractional-
Order Fornberg–Whitham 
Equation via $\rho$-Laplace
Transform		
&Symmetry
&2021\\ \cline{2-4}

&26. A modified Popov’s
subgradient extragradient 
method for variational 
inequalities in Banach 
spaces	 
&Journal of
Nonlinear Functional 
Analysis
&2021
\\ \cline{2-4}

&27. Modified Tseng’s 
splitting algorithms for
the sum of two 
Monotone operators 
in Banach spaces
&AIMS Mathematics
&2021 
\\ \cline{2-4}

&28. Iterative Methods for Solving the Monotone
Inclusion Problem and the Fixed Point Problem
in Banach Spaces
&Thai Journal of 
Mathematic
&2020 
\\ \cline{2-4}		

&29. Strong convergence of a generalized forward–backward splitting
method in reflexive Banach spaces				
&Optimization
&2020 
\\ \cline{2-4}	

&30. The generalized viscosity explicit rules for solving variational inclusion
problems in Banach spaces
&Optimization
&2020 \\ \cline{2-4}
&31. Strong convergence of a general viscosity explicit
rule for the sum of two monotone operators in Hilbert spaces
&Journal of Applied Analysis and Computation
&2019 
\\ \cline{2-4}

&32. An explicit parallel 
algorithm for solving 
variational inclusion 
problem and fixed point 
problem in Banach spaces
&Banach Journal 
of Mathematical 
Analysis 
&2019 \\ \cline{2-4}
	
&33. A modified extragradient
method for variational
inclusion and fixed point 
problems in Banach 
spaces
&Ricerche di
Matematica
&2019\\ \cline{2-4}	

&34. Convergence theorems for
generalized viscosity explicit
methods for nonexpansive
mappings in Banach spaces 
and some applications	
&Mathematics
&2019
\\ \cline{2-4}

&35. An iterative method with
residual vectors for solving
the fixed point and the split
inclusion problems in Banach
spaces
&Computational and 
Applied Mathematics
&2019\\ \hline

3. วงศ์วิศรุต เขื่องสตุ่ง&
1. Self-adaptive CQ-type 
algorithms for the split 
feasibility problem 
involving two bounded
linear operators in
Hilbert spaces
&Carpathian 
Journal of 
Mathematics &
2024 \\ \cline{2-4}

&2. A regularization method
for solving the G-variational
inequality problem and 
fixed-point problems in 
Hilbert spaces endowed with
graphs
&Journal of 
Inequalities and 
Applications
&2024 
\\ \cline{2-4}


&3. An intermixed algorithm 
for solving fixed point 
problems of proximal
operators in Hilbert Spaces.
&Carpathian Journal 
of Mathematics
&2024 \\ \cline{2-4}		


&4. Impact of pretreatment 
with dielectric barrier
discharge plasma on the
drying characteristics and 
bioactive compounds of 
jackfruit slices
&Journal of the Science 
of Food and Agriculture
&2024 \\ \cline{2-4}
			
&5. An intermixed
method for solving
the combination 
of mixed variational 
inequality problems
and fixed-point
problems
&Journal of 
Inequalities and
Applications
& 2023 
\\ \cline{2-4}		

&6.	Strong Convergence 
for the Modified 
Split Monotone 
Variational Inclusion
and Fixed Point	Problem
&Thai Journal 
of Mathematics&
2022 
\\ \cline{2-4}	

&7.	On an Open
Problem in Complex 
Valued Rectangular 
b-Metric Spaces 
with an Application
&Science \& Technology Asia
&2022 \\ \cline{2-4}

	
&8. Convergence results
for modified SP-iteration
in uniformly convex metric spaces
&Journal of 
mathematics 
and computer 
science
&2021 \\ \cline{2-4}
	
&9. The Convergence Results for an 
AK-Generalized Nonexpansive
Mapping in Hilbert Spaces
&Thai Journal
of Mathematics
&2021 \\ \cline{2-4}

&10. A Method for Solving
the Variational Inequality 
Problem and Fixed-Point 
Problems in Banach Spaces
&Tamkang journal
of mathematics&
2021\\\cline{2-4}
	
&11.The Modification of 
Generalized Mixed 
Equilibrium Problems
for Convergence 
Theorem of Variational
Inequality Problems 
and Fixed-Point 
Problems
&Thai Journal 
of Mathematics
&2021\\ \cline{2-4}	

&12.Fixed Point Theorems 
for a Demicontractive
Mapping and Equilibrium 
Problems in Hilbert Spaces 
&Communications 
in Mathematics
and Applications
&2021 \\ \cline{2-4}

&13. The Convergence
Theorem for a Square 
$\alpha$-Nonexpansive
Mapping in a Hyperbolic Space
&Thai Journal of Mathematics
&2020 \\ \cline{2-4}

&14. The Rectangular
Quasi-Metric Space 
and Common
Fixed Point Theorem
for $\psi$-Contraction
and $\psi$-Kannan
Mappings
&Thai Journal 
of Mathematics
&2020 \\ \cline{2-4}

&15. The Method for 
Solving Fixed Point
Problem of G-Nonexpansive
Mapping in Hilbert 
Spaces Endowed 
with Graphs and 
Numerical Example
&Indian J Pure
Appl Math &
2020 \\ \cline{2-4}

&16. An iterative method for solving proximal split
feasibility problems and fixed point problems
&Comp. Appl. Math
&2019 \\ \cline{2-4}

&17. The Finite Family L-Lipschitzian 
Suzuki-Generalized Nonexpansive Mappings
&Communications in Mathematics and Applications
&2019 \\ \cline{2-4}

&18. The Generalized-Nonexpansive Mappings and 
Related Convergence Theorems in Hyperbolic Spaces 
&Journal of Informatics and Mathematical Sciences
&2019 \\ \hline


4. นายรัฐพรหม พรหมคำ&1. Novel inertial methods for fixed point problems in reflexive Banach spaces with applications
&Rendiconti del Circolo Matematico di Palermo Series 2
&2023 \\ \cline{2-4}

&2. New inertial self-adaptive algorithms for the split common null-point problem: application to data classifications
&Journal of Inequalities and Applications
&2023 \\ \cline{2-4}

&3. Two-step inertial method
for solving split common 
null point problem with 
multiple output sets in Hilbert spaces
&AIMS Mathematics
&2023 \\ \cline{2-4}

&4. Strong convergence of a
generalized forward–backward
splitting method in 
reflexive Banach spaces
&Optimization
&2022 \\ \cline{2-4}

&5. Convergence Results of Iterative Algorithms for the Sum of Two Monotone Operators in Reflexive Banach Spaces
&Applications  of Mathematics
&2021 \\ \cline{2-4}

5. นายมงคล ทาทอง
&1. The Differential Equation in Terms of Jacobsthal and Jacobsthal-Lucas Numbers
&Progress in Applied Science and Technology
&2023 \\ \cline{2-4}

&2. Some Identities of the Modified (s,t) Jacobsthal and 
Modified (s,t) Jacobsthal – Lucas Numbers by the Matrix Method 
&Burapha Science Journal
&2022 \\  \cline{2-4}

&3. Matrix Sequences in Terms
of Gaussian Pell Polynomial,
Gaussian Modified Pell Polynomial,
Gaussian Pell Number, 
Gaussian Pell-Lucas Number,
Gaussian Modified Pell Number,
Pell Polynomial, Pell-Lucas Polynomial 
and Modified Pell Polynomial
&Burapha Science Journal
&2021 \\ \cline{2-4}

&4. Generalized Identities for 
third order Pell Number,
Pell-Lucas Number 
and Modified Pell Number
&Science and Technology RMUTT Journal
&2020 \\ \hline

\end{longtable}
\end{center}

\printselfeval

\subsection*{3. คุณสมบัติของอาจารย์ประจำหลักสูตร}

\printprogram{}  มีอาจารย์ประจำหลักสูตรเป็นอาจารย์ชุดเดียวกันกับอาจารย์ผู้รับผิดชอบหลักสูตร จึงมีคุณสมบัติเช่นเดียวกับข้อ 2\\

\printselfeval

\subsection*{4. คุณสมบัติของอาจารย์ผู้สอน ที่เป็นอาจารย์ประจำ }

%\subsubsection*{คุณสมบัติของอาจารย์ผู้สอนที่เป็นอาจารย์ประจำ}
	อาจารย์ผู้สอนของ\printprogram{} เป็นอาจารย์ประจำที่มีคุณวุฒิปริญญาโทหรือเทียบเท่า หรือดำรงตำแหน่งทางวิชาการไม่ต่ำกว่าผู้ช่วยศาสตราจารย์ ในสาขาวิชาคณิตศาสตร์หรือคณิตศาสตร์ประยุกต์ ดังตารางต่อไปนี้้
\begin{center}
	{\small 
		\begin{longtable}{|p{0.4\textwidth}|p{0.2\textwidth}|p{0.3\textwidth}|}
			\hline
			\multicolumn{1}{|c|}{\textbf{ชื่อ-นามสกุล}} & \multicolumn{1}{c}{\textbf{ตำแหน่งทางวิชาการ}} & \multicolumn{1}{|c|}{\textbf{คุณวุฒิ-สาขา}} 
			\\\hline
			\endhead
			%%%%%%%%%%%%%%%%%%%%%%%%%%%%%%%%%%%%%%%%%%%%%%%%%%%%%
			1. นายพงศกร สุนทรายุทธ์
			&รองศาสตราจารย์	&ปร.ด.(คณิตศาสตร์ประยุกต์) \\
			&&วท.ม. (คณิตศาสตร์ประยุกต์)\\
			&& วท.บ. (คณิตศาสตร์)\\
			\hline
			
			
			2. นางกุลประภา ศรีหมุด
			&ผู้ช่วยศาสตราจารย์&วท.ม. (คณิตศาสตร์)\\
			&&วท.บ. (คณิตศาสตร์)\\	
			\hline
			
			
			3. นายสมนึก ศรีสวัสดิ์
			&ผู้ช่วยศาสตราจารย์& วท.ม. (คณิตศาสตร์ประยุกต์)\\
			&&วท.บ. (คณิตศาสตร์)\\
			\hline
			
			
			4. นางสาวกมลรัตน์ สมบุตร
			&ผู้ช่วยศาสตราจารย์&ปร.ด. (คณิตศาสตร์)\\
			&&คบ. (คณิตศาสตร์)\\
			\hline
			
			
			5. นายวงศ์วิศรุต เขื่องสตุ่ง
			&ผู้ช่วยศาสตราจารย์&ปร.ด. (คณิตศาสตร์ประยุกต์)\\
			&&วท.ม. (คณิตศาสตร์ประยุกต์)\\
			&& วท.บ. (คณิตศาสตร์)\\
			\hline
			
			6. นางภคีตา สุขประเสริฐ
			&	ผู้ช่วยศาสตราจารย์ 
			& ปร.ด. (คณิตศาสตร์ประยุกต์)\\
			&& วท.ม. (คณิตศาสตร์)\\
			&& วท.บ. (คณิตศาสตร์) 	\\	\hline
			
			
			7. นายปริญญวัฒน์ ชูสุวรรณ
			&ผู้ช่วยศาสตราจารย์
			&วท.ด.(คณิตศาสตร์)\\
			&&วท.ม. (คณิตศาสตร์)\\
			&&วทบ. (คณิตศาสตร์)\\\hline
			
			
			8. นายมงคล ทาทอง
			&	ผู้ช่วยศาสตราจารย์
			&วท.ม. (คณิตศาสตร์ประยุกต์)\\
			&&วท.บ. (คณิตศาสตร์)\\ \hline
			
			
			9. นางสาวนนธิย มากะเต
			&อาจารย์
			&วท.ด.(คณิตศาสตร์ประยุกต์)\\
			&&วท.ม. (คณิตศาสตร์)\\
			&&วท.บ. (คณิตศาสตร์)\\\hline
			
			
			10.  นางวรรณา ศรีปราชญ์
			& อาจารย์
			& ปร.ด. (คณิตศาสตร์)\\
			&& วท.ม. (คณิตศาสตร์)\\
			&& คบ. (คณิตศาสตร์)\\\hline
			
		\newpage%%%%%%%%%%%%%%%%%%%%%%%%%%%%%%%	
			11. นายรัฐพรหม พรหมคำ
			&	อาจารย์
			&Dr.rer.nat. (Mathematik)\\
			&&วท.ม. (คณิตศาสตร์)\\
			&&วท.บ. (คณิตศาสตร์)\\\hline
			
			
			12. นายอลงกต สุวรรณมณี
			&อาจารย์
			&วท.ม.(คณิตศาสตร์ประยุกต์)\\
			&&วท.บ. (คณิตศาสตร์)\\\hline
			
			
			13. นายโอม สถิตยนาค
			&อาจารย์	
			& วท.ม.  (คณิตศาสตร์) \\
			&&วท.บ. (คณิตศาสตร์)\\\hline
			
			14. นางสาววาสนา ทองกำแหง
			&อาจารย์
			&วท.ม. (คณิตศาสตร์)\\
			&&วท.บ. (คณิตศาสตร์)\\\hline
			
		
			15. นายอัคเรศ สิงห์ทา (ลาศึกษาต่อ)
			&อาจารย์
			&วท.ม. (คณิตศาสตร์)\\
			&& วท.บ. (คณิตศาสตร์)\\\hline
			
			
			16. นางอมราภรณ์ บำเพ็ญดี
			&อาจารย์
			&วท.ม.(คณิตศาสตร์)\\
			&&วท.บ.(คณิตศาสตร์)\\\hline
			
			
			17. นางสาวธาวัลย์ อัมพวา
			&	อาจารย์&วท.ม.(คณิตศาสตร์)\\
			&&วท.บ.(คณิตศาสตร์)\\\hline
			
			18. นางสาวปฤณท์ธพร สงวนสุทธิกุล 
			&อาจารย์&ปร.ด. (คณิตศาสตร์ประยุกต์)\\
			&&วท.ม. (คณิตศาสตร์ประยุกต์) \\
			&&วท.บ. (คณิตศาสตร์) \\\hline
	\end{longtable}}
\end{center} 

\printselfeval

\subsubsection*{คุณสมบัติของอาจารย์ผู้สอนที่เป็นอาจารย์พิเศษ (ถ้ามี)}

ในปีการศึกษา \printyear{} \printprogram{}ไม่มีการเชิญอาจารย์พิเศษมาร่วมสอนในหลักสูตรฯ

\printselfeval

\subsection*{10. การปรับปรุงหลักสูตรตามรอบระยะเวลาที่กำหนด}

\printprogram{} เป็นหลักสูตรที่ปรับปรุงมาจากหลักสูตรวิทยาศาสตรบัณฑิต สาขาวิชาคณิตศาสตร์ (หลักสูตรปรับปรุง พ.ศ. 2559) ในปีการศึกษา 2563 โดยมีกระบวนการในการปรับปรุงหลักสูตรตามระบบและกลไกของสำนักส่งเสริมวิชาการและงานทะเบียนมหาวิทยาลัยเทคโนโลยีราชมงคลธัญบุรี และเริ่มใช้ในปีการศึกษา 2564 ทั้งนี้สภามหาวิทยาลัยให้การอนุมัติหลักสูตรเมื่อวันที่ 25 พฤศจิกายน 2563 และได้รับการรับรองการพิจารณาความสอดคล้องหลักสูตรจากสำนักงานปลัดกระทรวงการอุดมศึกษา วิทยาศาสตร์ วิจัยและนวัตกรรมเมื่อวันที่ 6 สิงหาคม พ.ศ. 2565  ซึ่งหลักสูตรจะครบรอบปรับปรุงอีกครั้งในปีการศึกษา 2568 เพื่อเปิดรับนักศึกษาในปีการศึกษา 2569

\printselfeval

\subsection*{ผลการประเมิน องค์ประกอบที่ 1 การกำกับมาตรฐาน}

\evalbox{ตัวบ่งชี้ที่ 1.1 การบริหารจัดการหลักสูตรตามเกณฑ์มาตรฐานหลักสูตร ที่กำหนดโดยสำนักงานคณะกรรมการการอุดมศึกษา}

\begin{doclist}
	\docitem{วุฒิการศึกษาและตำแหน่งทางวิชาการของอาจารย์ผู้รับผิดชอบหลักสูตร  }
	\docitem{ผลงานวิจัยตีพิมพ์/เผยแพร่ของผู้รับผิดชอบหลักสูตร }
	\docitem{หนังสือแจ้งผลการพิจารณาให้ความเห็นชอบในการปรับปรุงหลักสูตรของหลักสูตรวิทยาศาสตรบัณฑิต สาขาวิชาคณิตศาสตร์ประยุกต์ (หลักสูตรปรับปรุงพ.ศ. 2564) }
	\docitem{หลักสูตรวิทยาศาสตรบัณฑิตสาขาวิชาคณิตศาสตร์ประยุกต์ (หลักสูตรปรับปรุง พ.ศ. 2564)}
	\docitem{หนังสือแจ้งผลการพิจารณาการให้การรับรองการพิจารณาความสอดคล้องหลักสูตรของหลักสูตรวิทยาศาสตรบัณฑิต สาขาวิชาคณิตศาสตร์ประยุกต์ หลักสูตรปรับปรุง พ.ศ. 2564) จากสำนักงานปลัดกระทรวงการอุดมศึกษาวิทยาศาสตร์ วิจัยและนวัตกรรม}
\end{doclist}
 

\cleardoublepage
\section{ผลการดำเนินงานของหลักสูตรตามเกณฑ์ AUN-QA}
























