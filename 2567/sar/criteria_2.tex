\newpage
\criteria{Programme Structure and Content}

\subcriteria{The specifications of the programme and all its courses are shown to be comprehensive, up-to-date, and made available and communicated to all stakeholders.}


\printprogram{} ได้เริ่มรับนักศึกษา
ตั้งแต่ภาคการศึกษาที่ 1 ปีการศึกษา 2564 โดยใน การจัดทำข้อมูลหลักสูตรได้จัดทำตามข้อกำหนดของหลักสูตร
(Programme specification) ตามแบบฟอร์ม มคอ.2 ของสำนักงานคณะกรรมการการอุดมศึกษา (สกอ.) ซึ่งแบ่งเนื้อหา
ออกเป็น 8 หมวด ในแต่ละหมวดมีการระบุข้อมูลที่ครบถ้วนและสอดคล้องกับข้อแนะนำจาก Guide to AUN-QA Assesment at Programme Level Version 4.0 หน้า 20 ดังตาราง \ref{Table:M2AUN}

\begin{longtable}{|p{0.6\textwidth}|p{0.35\textwidth}|}
\caption{ตารางแสดงความสอดคล้องระหว่าง มคอ.2 กับข้อมูลจาก Guide to AUN-QA Assesment at Programme Level Version 4.0 หน้า  20}
\label{Table:M2AUN}
\\
\hline
{\bf ข้อมูลจาก Guide to AUN-QA Assesment at\newline  Programme Level Version 4.0 หน้า  20}&\multicolumn{1}{c|}{\bf ข้อมูลใน มคอ.2}\\
\hline
\endfirsthead
\caption[]{(ต่อ) ตารางแสดงความสอดคล้องระหว่าง มคอ.2 กับข้อมูลจาก Guide to AUN-QA Assesment at Programme Level Version 4.0 หน้า  20}
\\
\hline
{\bf ข้อมูลจาก Guide to AUN-QA Assesment at\newline  Programme Level Version 4.0 หน้า  20}&\multicolumn{1}{c|}{\bf ข้อมูลใน มคอ.2}\\
\hline
\endhead
Awarding body/institution& หน้า 1 \\\hline
Teaching institution&หมวดที่ 1 ข้อที่ 10 (หน้า 4)\\\hline
Details of accreditation by professional or statutory bodies& หมวดที่ 1 ข้อที่ 6 (หน้า 2)\\\hline
Name of the final award&หมวดที่ 1 ข้อที่ 2 (หน้า 1)\\\hline
Programme title&หมวดที่ 1 ข้อที่ 1 (หน้า 1)\\\hline
Expected learning outcomes of the programme&หมวดที่ 2 ข้อที่ 1 (หน้า 7)\\\hline
Admission criteria or requirements&หมวดที่ 3 ข้อที่ 2 (หน้า 11)\\\hline
Relevant benchmark reports, external and internal reference points, that may
be used to provide information on programme learning outcomes&หมวดที่ 1 ข้อที่ 11,12 (หน้า 4-6)\\\hline
Programme structure and requirements including levels, courses, credits, etc&หมวดที่ 3 ข้อที่ 3 (หน้า 13-75)\\\hline
The date of writing the programme specifications.& หมวดที่ 1 (หน้า 2)\\\hline
\end{longtable}	

ทั้งนี้ข้อมูลหลักสูตรตาม มคอ.2 ได้มีการเผยแพร่ข้อมูลที่จำเป็นแก่ผู้ที่มีส่วนได้ส่วนเสียทุกกลุ่มในช่องทางต่างๆ ที่
เข้าถึงได้ง่าย เช่น เว็บไซต์ของสำนักส่งเสริมวิชาการและงานทะเบียน มทร.ธัญบุรี
เว็บไซต์ของคณะ เว็บไซต์ของสาขาวิชาฯ Facebook ของหลักสูตร แผ่นพับประชาสัมพันธ์ และคู่มือนักศึกษา รายละเอียดดังตาราง \ref{Table:M2-13}
\newpage
\begin{longtable}{|p{0.6\textwidth}|p{0.35\textwidth}|}
	\caption{ตารางแสดงการสื่อสาร The Programme specification กับผู้มีส่วนได้ส่วนเสีย}
	\label{Table:M2-13}
\\
\hline
\multicolumn{1}{|c|}{\bf ช่องทางการสื่อสาร}&\multicolumn{1}{c|}{\bf ผู้มีส่วนได้ส่วนเสีย}\\
\hline
\endhead
1) เว็บไซต์ของสำนักส่งเสริมวิชาการและงานทะเบียน มทร.ธัญบุรี\newline/คณะ/สาขา และเฟซบุ๊คของหลักสูตร& นักศึกษา ผู้ปกครอง อาจารย์\newline ผู้ที่สนใจสมัครเข้าศึกษาในหลักสูตร\\\hline
\hline
2) แผ่นพับ& นักศึกษา\newline 
ผู้ที่สนใจสมัครเข้าศึกษาในหลักสูตร\newline 
ฝ่ายแนะแนวของโรงเรียนระดับมัธยมศึกษา
\\\hline
\hline
3) คู่มือนักศึกษา& นักศึกษา\\\hline
\end{longtable}	
\noindent{\bf Crouse Spec :}\\
ในปีการศึกษา 2566 มีรายวิชาเปิดจำนวน 33  รายวิชา แบ่งเป็น
ภาคเรียนที่ 1 จำนวน 11 รายวิชา ภาคเรียนที่ 2 จำนวน 22 รายวิชา โดยอาจารย์ผู้รับผิดชอบรายวิชาทุกรายวิชามีการจัดทำ มคอ.3 ซึ่งมีรายละเอียดครบถ้วนตาม Guide to
AUN-QA Assesment at Programme Level Version 4.0 หน้า 20 ได้แก่
\begin{enumerate}
	\item Course title
	\item Course requirements such as pre-requisites, credits, etc
	\item Expected learning outcomes of the course in terms of knowledge, skills, and
	attitude
	\item Teaching, learning, and assessment methods that enable the expected learning outcomes to be achieved
	\item Course description, outline, or syllabus
	\item Details of student assessment
	\item Date on which the course specification was written or revised.
\end{enumerate}
และส่ง มคอ.3 ผ่านระบบของสำนักส่งเสริมวิชาการและงานทะเบียน ก่อนเปิดภาคการศึกษา

นอกจากนั้นทุกรายวิชามีการจัดทำ Course Syllabus ซึ่งมีรายละเอียดครบถ้วนตาม Guide to AUN-QA Assesment at Programme Level Version 4.0 หน้า 20 และมี Qr-Code มคอ.3 ของรายวิชาเพื่อสื่อสาร มคอ.3 ให้กับนักศึกษา โดยแจก Course Syllabus ให้กับนักศึกษาในคาบแรกของการจัดการเรียนการสอน

\begin{doclist}
\docitem{มคอ. 2}
\docitem{แผ่นพับประชาสัมพันธ์หลักสูตร}
\docitem{มคอ. 3}
\docitem{Course Syllabus}
\docitem{คู่มือนักศึกษา}

\end{doclist}


\subcriteria{The design of the curriculum is shown to be constructively aligned with achieving the expected learning outcomes.}

\printprogram{} มีการออกแบบที่สอดคล้องตามหลักการ Outcome Base Education (OBE) และ Backward Curriculum Design (BCD) โดยมีรายวิชาที่ส่งเสริมการบรรลุ PLOs แต่ละ PLOs ดังนี้\\
\textbf{Constructive Alignment}
\begin{longtable}{|p{0.1\textwidth}|p{0.15\textwidth}|p{0.15\textwidth}|p{0.15\textwidth}|p{0.15\textwidth}|p{0.15\textwidth}|}
		\hline
\multicolumn{1}{|c|}{\textbf{PLOs}}&\multicolumn{5}{c|}{\textbf{Contribution of Courses to PLOs}}\\\cline{2-6}
&\multicolumn{2}{c|}{\textbf{Knowledge}}&\multicolumn{2}{c|}{\textbf{Skills}}&\multicolumn{1}{c|}{\textbf{Attitudes}}\\\cline{2-5}
&\multicolumn{1}{c|}{\textbf{Generic}}&\multicolumn{1}{c}{\textbf{Specific}}&\multicolumn{1}{|c|}{\textbf{Generic}}&\multicolumn{1}{c|}{\textbf{Specific}}&\\\hline
\endhead
%%%%%%%%%%%%%%%%%%%%%%%%%%%%%%%%%
PLO1&
09-122-104 \newline
09-210-129 \newline
09-311-148
&&09-210-130\newline
 09-311-149\newline
 09-410-156
&09-115-401\newline
09-115-404
&\\\hline
 
PLO2& 
09-111-151\newline
09-111-152\newline
09-122-104\newline
09-210-129\newline
09-311-148\newline
09-410-155
&
09-111-253 \newline
09-111-257\newline
09-113-114\newline
09-113-201\newline
09-113-202\newline
09-113-305\newline
09-113-306\newline
09-114-205\newline
09-114-222\newline
09-114-223
&
09-090-016\newline
09-114-202\newline
09-210-130\newline
09-311-149\newline
09-410-156
&
09-114-204\newline
09-114-335\newline
09-115-401\newline
09-115-404
&\\\hline

PLO3&
09-111-151\newline
09-111-152\newline
09-122-104\newline
09-210-129\newline
09-311-148\newline
09-410-155
&
09-111-253\newline
09-111-257\newline
09-113-114\newline
09-113-202\newline
09-113-305\newline
09-114-205\newline
09-114-222\newline
09-114-223\newline
09-115-401\newline
09-115-404
&
09-210-130\newline
09-311-149\newline
09-410-156
&&\\\hline

PLO4&&
09-113-114\newline
09-113-201\newline
09-113-202\newline
09-113-305\newline
09-113-306
&&
09-115-401 \newline
09-115-404
&\\\hline

PLO5&
09-111-151\newline
09-210-129
&
09-114-205\newline
09-114-223
&
09-410-156
&
09-114-204\newline
09-115-401\newline
09-115-404
&\\\hline

PLO6&&&&09-115-404&\\\hline

PLO7&09-210-129&&09-210-130&
09-115-401\newline 
09-115-404
&\\\hline

PLO8&
09-122-104 \newline
09-210-129
&&
09-210-130\newline
09-410-156
&
09-115-401\newline
09-115-404
&\\\hline

PLO9&
09-311-148 
&&
09-311-149
&
09-115-401\newline
09-115-404
&\\\hline

PLO10&&
09-114-205\newline
09-114-222\newline
09-114-223
&
09-090-016\newline
09-114-202\newline
09-210-130\newline
09-311-149\newline
09-410-156
&
09-114-204\newline
09-114-335
&\\\hline
\end{longtable}

นอกจากนี้ทุกรายวิชายังมีวิธีการสอน และวิธีการประเมินผลที่ส่งเสริมการบรรลุ ClOs ของรายวิชา 09114202 ระเบียบวิธีเชิงตัวเลขเบื้องต้น ดังนี้

\begin{longtable}{|>{\raggedright}p{0.25\textwidth}|p{0.07\textwidth}|p{0.3\textwidth}|p{0.18\textwidth}|}
\hline
\multicolumn{1}{|c|}{\textbf{CLOs}}&\textbf{PLOs}&\textbf{วิธีการจัดการเรียนการสอน}&\textbf{วิธีการประเมินผล} \\\hline
\endhead
CLO1:\,\,บอกความหมายของความคลาดเคลื่อนได้&2,3,10&\multirow{9}{0.3\textwidth}{1. บรรยาย ยกตัวอย่าง ทำแบบฝึกในชั้นเรียน\newline
	2. ทำ workshop ในห้องปฏิบัติการคอมพิวเตอร์และการเขียนโปรแกรม\newline
	3.  ทำโครงงานกลุ่มเพื่อประยุกต์ ใช้ระเบียบวิธีเชิงตัวเลข และในการแก้ปัญหา}
	&สอบข้อเขียน\\\cline{1-1}

CLO2:\,\,คำนวณผลเฉลยของสมการไม่เชิงเส้นโดยวิธีแบ่งครึ่งช่วง วิธีวางผิดที่ วิธีทำซ้ำ วิธีนิวตัน วิธีซีแคนต์ได้&&&\\\cline{1-1}
CLO3:\,\,คำนวณผลเฉลยของระบบสมการเชิงเส้นได้&&&\\\cline{1-1}
CLO4:\,\,อธิบายการประมาณค่าในช่วงได้&&&\\\cline{1-1}
CLO5:\,\,คำนวณการประมาณค่าในช่วงโดยพหุนามได้&&&\\\cline{1-1}
CLO6:\,\,คำนวณการประมาณค่าในช่วงด้วยวิธีนิวตัน วิธีลา กรองจ์ได้&&&\\\cline{1-1}
CLO7:\,\,คำนวณการประมาณค่าแบบกำลังสองน้อยสุดได้&&&\\\hline
CLO8:\,\,คำนวณค่าปริพันธ์ด้วยวิธีสี่เหลี่ยมคางหมู วิธีสี่เหลี่ยมคางหมูหลายรูป วิธีซิมสันได้&&&\\\cline{1-1}\cline{4-4}
CLO9: เขียนหรือใช้โปรแกรมคอมพิวเตอร์ในการคํานวณด้านระเบียบวิธีเชิงตัวเลขเบื้องต้นได้&&&สอบปฏิบัติ\\\hline
\end{longtable}




\begin{doclist}
\docitem{เอกสาร\printprogram{} (มคอ. 2)}
\docitem{เอกสารการปรับปรุงแก้ไข\printprogram{} (สมอ. 08) }
\docitem{มคอ 3 }
\end{doclist}


\subcriteria{The design of the curriculum is shown to
include feedback from stakeholders, especially
external stakeholders.}

\printprogram{} ได้มีการนำผลการประเมินความพึงพอใจของผู้ใชับัณฑิตมาปรับปรุงหลักสูตรและกระบวนการจัดการเรียนการสอน โดยผลจากการประเมินความพึงพอใจผู้ใช้บัณฑิต (ประเมินจากแบบสอบถาม) ปีการศึกษา 2565 พบว่าผู้ใชับัณฑิตมีข้อเสนอแนะดังนี้
\begin{enumerate}
	\item บัณฑิตขาดความมั่นใจในการสื่อสารภาษาอังกฤษ หลักสูตรควรส่งเสริมด้านภาษาอังกฤษ
	\item ควรส่งเสริมความรู้ด้านการประกันภัย และทักษะต่าง ๆ ที่เกี่ยวข้องกับธุรกิจวินาศภัย
\end{enumerate}
หลักสูตรฯ ได้นำข้อเสนอแนะดังกล่าวมาวางแผนการปรับปรุงการดำเนินการในปีการศึกษา 2566 โดย
\begin{enumerate}
\item ส่งเสริมให้มีการใช้ภาษาอังกฤษในการจัดการเรียนการสอน
\item ส่งนักศึกษาเข้าร่วมอบรม/จัดกิจกรรม/โครงการที่ส่งเสริมทักษะภาษาอังกฤษ
\item เพิ่มรายวิชาด้านประกันภัยในกลุ่มรายวิชาชีพเลือก เพื่อให้นักศึกษาที่มีความสนใจทางด้านประกันภัย ได้เลือกศึกษา
\end{enumerate}


\begin{doclist}
\docitem{ผลประเมินความพึงพอใจผู้ใช้บัณฑิต}
\docitem{การเข้าร่วมกิจกรร/โครงการด้านภาษาอังกฤษของนักศึกษา}
\docitem{เอกสารการปรับปรุงแก้ไข\printprogram{} (สมอ. 08) }
\end{doclist}

\subcriteria{The contribution made by each couse in achieving the expected learning outcomes is shown to be clear.}

หลักสูตรฯ ได้นำผลลัพธ์การเรียนรู้ระดับหลักสูตร (PLOs) ทั้ง 10 ข้อ มาจัดทำแผนที่กระจายความรับผิดชอบผลลัพธ์การเรียนรู้ระดับหลักสูตร (PLOs) สู่รายวิชา (Curriculum Mapping) ดังตาราง \ref{table: Mapping}

	\begin{longtable}{|>{\raggedright}p{0.4\textwidth}|c|c|c|c|c|c|c|c|c|c|}
		\caption{แผนที่แสดงการกระจายความรับผิอชอบผลลัพธ์การเรียนรู้ระดับหลักสูตร (PLOs) สู่รายวิชา (Curriculum Mapping)}
		\label{table: Mapping}
		\\
		\hline
		\multicolumn{1}{|c|}{\textbf{รายวิชา}} & \multicolumn{10}{c|}{PLOs}\\
		\cline{2-11}
		&1&2&3&4&5&6&7&8&9&10\\
		\hline
		\endfirsthead
		
		\caption[]{(ต่อ) ความสอดคล้องระหว่าง CLOs ของรายวิชาสมการเชิงอนุพันธ์สามัญ และ PLOs ของ\printprogram{}}
		\\
		\hline
		\multicolumn{1}{|c|}{\textbf{รายวิชา}}  & \multicolumn{10}{c|}{PLOs}\\
		\cline{2-11}
		&1&2&3&4&5&6&7&8&9&10\\
		\hline
		\endhead
		
	09-090-016 พื้นฐานการเขียนโปรแกรม&&{\Large$\bullet$}&&&&&&&&{\Large$\bullet$}\\
	\hline
	09-111-151 แคลคูลัส 1&&{\Large$\bullet$}&{\Large$\bullet$}&&{\Large$\bullet$}&&&&&\\
	\hline
	09-111-152 แคลคูลัส 2
	&&{\Large$\bullet$}&{\Large$\bullet$}&&&&&&&\\
	\hline
	09-114-202 ระบบคอมพิวเตอร์สำหรับงานพีชคณิต&&{\Large$\bullet$}&&&&&&&&{\Large$\bullet$}\\
	\hline
	09-122-104 สถิติสำหรับวิทยาศาสตร์&{\Large$\bullet$}&{\Large$\bullet$}&{\Large$\bullet$}&&&&&{\Large$\bullet$}&&\\
	\hline
	09-210-129 เคมีพื้นฐาน&{\Large$\bullet$}&{\Large$\bullet$}&{\Large$\bullet$}&&{\Large$\bullet$}&&{\Large$\bullet$}&{\Large$\bullet$}&&\\
	\hline
	09-210-130 ปฏิบัติการเคมีพื้นฐาน&{\Large$\bullet$}&{\Large$\bullet$}&{\Large$\bullet$}&&&&{\Large$\bullet$}&{\Large$\bullet$}&&{\Large$\bullet$}\\
	\hline
	09-311-148 หลักชีววิทยา&{\Large$\bullet$}&{\Large$\bullet$}&{\Large$\bullet$}&&&&&&{\Large$\bullet$}&\\
	\hline
	09-311-149 ปฏิบัติการหลักชีววิทยา&{\Large$\bullet$}&{\Large$\bullet$}&{\Large$\bullet$}&&&&&&{\Large$\bullet$}&{\Large$\bullet$}\\
	\hline
	09-410-155 ฟิสิกส์เบื้องต้น&&{\Large$\bullet$}&{\Large$\bullet$}&&&&&&&\\
	\hline
	09-410-156 ปฏิบัติการฟิสิกส์เบื้องต้น&{\Large$\bullet$}&{\Large$\bullet$}&{\Large$\bullet$}&&{\Large$\bullet$}&&&{\Large$\bullet$}&&{\Large$\bullet$}\\
	\hline
	09-111-253 แคลคูลัส 3&&{\Large$\bullet$}&{\Large$\bullet$}&&&&&&&\\
	\hline
	09-111-257 สมการเชิงอนุพันธ์สามัญ&&{\Large$\bullet$}&{\Large$\bullet$}&&&&&&&\\
	\hline
	09-113-114 วิยุตคณิต&&{\Large$\bullet$}&{\Large$\bullet$}&{\Large$\bullet$}&&&&&&\\
	\hline
	09-113-201 หลักคณิตศาสตร์&&{\Large$\bullet$}&&{\Large$\bullet$}&&&&&&\\
	\hline
	09-113-202 พีชคณิตเชิงเส้น&&{\Large$\bullet$}&{\Large$\bullet$}&{\Large$\bullet$}&&&&&&\\
	\hline
	09-113-305 การวิเคราะห์เชิงคณิตศาสตร์&&{\Large$\bullet$}&{\Large$\bullet$}&{\Large$\bullet$}&&&&&&\\
	\hline
	09-113-306 พีชคณิตนามธรรม&&{\Large$\bullet$}&&{\Large$\bullet$}&&&&&&\\
	\hline
	09-114-204 การเขียนโปรแกรมคอมพิวเตอร์ทาง
	คณิตศาสตร์&&{\Large$\bullet$}&&&{\Large$\bullet$}&&&&&{\Large$\bullet$}\\
	\hline
	09-114-205 กำหนดการเชิงคณิตศาสตร์เบื้องต้น&&{\Large$\bullet$}&{\Large$\bullet$}&&{\Large$\bullet$}&&&&&{\Large$\bullet$}\\
	\hline
	09-114-222 ระเบียบวิธีเชิงตัวเลขเบื้องต้น&&{\Large$\bullet$}&{\Large$\bullet$}&&&&&&&{\Large$\bullet$}\\
	\hline
	09-114-223 การสร้างแบบจำลองทางคณิตศาสตร์
	เบื้องต้น&&{\Large$\bullet$}&{\Large$\bullet$}&&{\Large$\bullet$}&&&&&{\Large$\bullet$}\\
	\hline
	09-114-335 ระบบฐานข้อมูล&&{\Large$\bullet$}&&&&&&&&{\Large$\bullet$}\\
	\hline
	09-115-401 สัมมนาทางคณิตศาสตร์ประยุกต์&{\Large$\bullet$}&{\Large$\bullet$}&{\Large$\bullet$}&{\Large$\bullet$}&{\Large$\bullet$}&&{\Large$\bullet$}&{\Large$\bullet$}&{\Large$\bullet$}&\\
	\hline
	09-115-404 โครงงานด้านคณิตศาสตร์ประยุกต์&{\Large$\bullet$}&{\Large$\bullet$}&{\Large$\bullet$}&{\Large$\bullet$}&{\Large$\bullet$}&{\Large$\bullet$}&{\Large$\bullet$}&{\Large$\bullet$}&{\Large$\bullet$}&\\
	\hline
	\end{longtable}

\subcriteria{The curriculum to show that all its courses are logically structured, properly sequenced (progression from basic to intermediate to specialized courses) and are integrated.}

\printprogram{} ได้รับการออกแบบให้มีการจัดโครงสร้างหลักสูตรที่มีการจัดลำดับรายวิชาอย่างเป็นระบบและเหมาะสม โดยคำนึงถึงรายวิชาเรียนก่อน-หลัง  เรียนจากรายวิชาระดับพื้นฐานไปสู่รายวิชาระดับสูง และมีการบูรณาการเนื้อหาวิชาในแต่ละปีการศึกษา โดยมีโครงสร้างหลักสูตรดังตารางต่อไปนี้
\begin{longtable}{|>{\raggedright}p{0.6\textwidth}|c|}
	\caption{โครงสร้างหลักสูตร}
	\\
	\hline
%	\endfirsthead
	\multicolumn{1}{|c|}{{\bf หมวดวิชา}}&{\bf จำนวนหน่วยกิต}\\
	\hline
	{\bf หมวดวิชาเฉพาะ}&\textbf{94}\\
	- กลุ่มวิชาพื้นฐานวิชาชีพ&27\\
	- กลุ่มวิชาชีพบังคับ&40\\
	- กลุ่มวิชาชีพเลือก&27\\
	{\bf หมวดวิชาเสริมสร้างประสบการณ์ในวิชาชีพ}&\textbf{7}\\
	\hline
\end{longtable}
จากโครงสร้างหลักสูตรข้างต้น หลักสูตรฯได้นำมาออกแบบการจัดเรียงลำดับรายวิชาเป็นแผนการศึกษาในแต่ละภาคการศึกษา ดังตาราง \ref{table:Planyear} 
%>{\raggedright}p{0.6\textwidth}
\begin{longtable}{|c|c|>{\raggedright}p{0.52\textwidth}|c|}
	\caption{แผนการศึกษา}
	\label{table:Planyear}
	\\
	\hline
	{\bf ชั้นปี}&{\bf ภาคเรียน}&\multicolumn{1}{c|}{{\bf รายวิชา}}&{\bf จำนวนหน่วยกิต}\\
	\hline
	\endhead
	1&1&09-090-016	พื้นฐานการเขียนโปรแกรม	&	3\\
	&&09-111-151	แคลคูลัส 1			&3\\
	&&09-210-129	เคมีพื้นฐาน			&3\\
	&&09-210-130	ปฏิบัติการเคมีพื้นฐาน	&1\\
	&&09-122-104	สถิติสำหรับวิทยาศาสตร์	&3\\
	\cline{2-4}
	&2&09-111-152	แคลคูลัส 2	&3\\
	&&09-113-114	วิยุตคณิต	&3\\
	&&09-114-202	ระบบคอมพิวเตอร์สำหรับงานพีชคณิต	&3\\
	&&09-311-148	หลักชีววิทยา	&3\\
	&&09-311-149	ปฏิบัติการหลักชีววิทยา	&1\\
	\hline
	2&1&09-111-253	แคลคูลัส 3	&3\\
	&&09-113-201	หลักคณิตศาสตร์	&3\\
	&&09-410-155	ฟิสิกส์เบื้องต้น	&3\\
	&&09-410-156	ปฏิบัติการฟิสิกส์เบื้องต้น	&1\\
	\cline{2-4}
	&2&09-111-257	สมการเชิงอนุพันธ์สามัญ&3\\
	&&09-113-202	พีชคณิตเชิงเส้น&3\\
	&&09-114-204	การเขียนโปรแกรมคอมพิวเตอร์\newline ทางคณิตศาสตร์&3\\
	&&09-114-223	การสร้างแบบจำลองทางคณิตศาสตร์เบื้องต้น&3\\
	&&09-114-335	ระบบฐานข้อมูล&3\\
	\hline
	3&1&09-113-305	การวิเคราะห์เชิงคณิตศาสตร์&3\\
	&&09-114-205	กำหนดการเชิงคณิตศาสตร์เบื้องต้น&3\\
	&&09-114-222	ระเบียบวิธีเชิงตัวเลขเบื้องต้น&3\\
	&&09-xxx-xxx	เลือกจากรายวิชาชีพเลือก&3\\
	&&09-xxx-xxx	เลือกจากรายวิชาชีพเลือก&3\\
	&&09-xxx-xxx	เลือกจากรายวิชาชีพเลือก&3\\
	\cline{2-4}
	&2&09-113-306	พีชคณิตนามธรรม&3\\
	&&09-116-301	การเตรียมความพร้อมฝึกประสบการณ์วิชาชีพทางคณิตศาสตร์ประยุกต์&1\\
	&&09-xxx-xxx	เลือกจากรายวิชาชีพเลือก&3\\
	&&09-xxx-xxx	เลือกจากรายวิชาชีพเลือก&3\\
	&&09-xxx-xxx	เลือกจากรายวิชาชีพเลือก&3\\\hline
	&&09-xxx-xxx	เลือกจากรายวิชาชีพเลือก&3\\
	\hline
	4&1&09-116-402	สหกิจศึกษาทางคณิตศาสตร์ประยุกต์\newline
	หรือ
	\newline
	09-116-403	สหกิจศึกษาต่างประเทศทางคณิตศาสตร์ประยุกต์&6\\
	\cline{2-4}
	&2&09-115-401	สัมมนาทางคณิตศาสตร์ประยุกต์&1\\
	&&09-115-404	โครงงานด้านคณิตศาสตร์ประยุกต์&3\\
	&&09-xxx-xxx	เลือกจากรายวิชาชีพเลือก&3\\
	&&09-xxx-xxx	เลือกจากรายวิชาชีพเลือก&3\\
	\hline
\end{longtable}


\subcriteria{The curriculum to have option(s) for students to pursue major and/or minor specialisations.}

\printprogram{} ได้รับการออกแบบให้มีความยืดหยุ่นและมีทางเลือกให้นักศึกษาในการเลือกเรียนวิชาตามความต้องการ โดยนักศึกษาจะต้องเลือกเรียนกลุ่มวิชาชีพเลือก 27 หน่วยกิต โดยเลือกศึกษาจากกลุ่มวิชาต่อไปนี้ทุกกลุ่ม กลุ่มละไม่น้อยกว่า 6 หน่วยกิต

\subsection*{กลุ่มวิชาแบบจำลองทางคณิตศาสตร์ (Mathematical Modeling Courses)}

นักศึกษาสามารถเลือกเรียนรายวิชาในกลุ่มวิชาแบบจำลองทางคณิตศาสตร์ เพื่อพัฒนาทักษะในการสร้างและวิเคราะห์แบบจำลองทางคณิตศาสตร์สำหรับการประยุกต์ใช้ในด้านต่าง ๆ ตัวอย่างรายวิชาในกลุ่มนี้ได้แก่:
\begin{itemize}
    \item 09-111-338 สมการเชิงอนุพันธ์ย่อย (Partial Differential Equations)
    \item 09-114-206 ทฤษฎีกราฟและการประยุกต์ (Graph Theory and Applications)
    \item 09-114-316 คณิตศาสตร์ประกันภัย (Mathematics of Insurance)
    \item 09-114-318 คณิตศาสตร์การเงิน (Mathematics of Finance)
    \item 09-114-324 คณิตศาสตร์การลงทุน (Mathematics of Investment)
    \item 09-114-325 ระบบพลวัต (Dynamical Systems)
    \item 09-114-326 ระเบียบวิธีการประมาณค่าตามเส้น (Curve Fitting Methods)
    \item 09-114-327 การตัดสินใจอย่างชาญฉลาดด้วยกำหนดการเชิงคณิตศาสตร์ (Intelligence Decision Making with Mathematical Programming)
    \item 09-114-328 แบบจำลองทางคณิตศาสตร์ด้านชีววิทยา (Mathematical Modeling in Biology)
    \item 09-114-329 แบบจำลองทางคณิตศาสตร์ด้านระบาดวิทยา (Mathematical Modeling in Epidemiology)
    \item 09-115-409 หัวข้อพิเศษของแบบจำลองทางคณิตศาสตร์ (Special Topics in Mathematical Modeling)
\end{itemize}

\subsection*{กลุ่มวิชาเทคโนโลยีทางคณิตศาสตร์ (Mathematical Technology Courses)}

นักศึกษาสามารถเลือกเรียนรายวิชาในกลุ่มวิชาเทคโนโลยีทางคณิตศาสตร์ เพื่อเพิ่มพูนความรู้และทักษะในการใช้เทคโนโลยีและเครื่องมือทางคณิตศาสตร์ในการแก้ปัญหาต่าง ๆ ตัวอย่างรายวิชาในกลุ่มนี้ได้แก่:
\begin{itemize}
    \item 09-113-203 ทฤษฎีจำนวนและการประยุกต์ (Number Theory and Applications)
    \item 09-114-330 ระเบียบวิธีเชิงตัวเลขสำหรับระบบพลวัต (Numerical Methods for Dynamical Systems)
    \item 09-114-331 เทคนิคการหาค่าเหมาะสม (Optimization Techniques)
    \item 09-114-332 ระเบียบวิธีไฟไนต์เอลิเมนต์ (Finite Elements Methods)
    \item 09-114-333 วิทยาการเข้ารหัสลับเบื้องต้น (Introduction to Cryptography)
    \item 09-115-304 ทักษะการนำเสนอผลงานทางด้านคณิตศาสตร์ (Presentation Skills in Mathematics)
    \item 09-115-307 หัวข้อพิเศษของการคำนวณเชิงคณิตศาสตร์ (Special Topics in Computational Mathematics)
\end{itemize}

\subsection*{กลุ่มวิชาคอมพิวเตอร์สำหรับนักคณิตศาสตร์ (Computer Courses for Mathematicians)}

นักศึกษาสามารถเลือกเรียนรายวิชาในกลุ่มวิชาคอมพิวเตอร์สำหรับนักคณิตศาสตร์ เพื่อเพิ่มพูนทักษะการใช้คอมพิวเตอร์ในการคำนวณและการประยุกต์ใช้ทางคณิตศาสตร์ ตัวอย่างรายวิชาในกลุ่มนี้ได้แก่:
\begin{itemize}
    \item 09-114-319 โครงสร้างข้อมูลและอัลกอริทึม (Data Structures and Algorithms)
    \item 09-114-334 ระบบการจัดเตรียมเอกสารอย่างมืออาชีพ (Professional Document Preparation System)
    \item 09-114-336 รากฐานปัญญาประดิษฐ์ (Foundation in Artificial Intelligence)
    \item 09-114-337 การเรียนรู้ของจักรกล (Machine Learning)
    \item 09-114-338 การพัฒนาเว็บไซต์สมัยใหม่ (Modern Website Development)
    \item 09-114-339 วิทยาการข้อมูลสำหรับนักคณิตศาสตร์ (Data Sciences for Mathematicians)
    \item 09-115-308 หัวข้อพิเศษของคอมพิวเตอร์สำหรับคณิตศาสตร์ (Special Topics in Computer for Mathematics)
\end{itemize}

โดยในปีการศึกษา 2566 นักศึกษาชั้นปีที่ 3 ได้เลือกรายวิชาชีพเลือก แบ่งออกเป็น 3 แนวทางดังนี้
\begin{enumerate}
	\item[กลุ่มที่ 1:] นักศึกษาเลือกเรียนรายวิชา
\begin{itemize}
	\item 09-114-325 ระบบพลวัต (Dynamical Systems)
	\item 09-114-330 ระเบียบวิธีเชิงตัวเลขสำหรับระบบพลวัต (Numerical Methods for Dynamical Systems)
	\item 09-114-331 เทคนิคการหาค่าเหมาะสม (Optimization Techniques)
	\item 09-114-339 วิทยาการข้อมูลสำหรับนักคณิตศาสตร์ (Data Sciences for Mathematicians)
\end{itemize}
\item[กลุ่มที่ 2:] นักศึกษาเลือกเรียนรายวิชา
\begin{itemize}
 \item 09-113-203 ทฤษฎีจำนวนและการประยุกต์ (Number Theory and Applications)
 \item 09-114-206 ทฤษฎีกราฟและการประยุกต์ (Graph Theory and Applications)
  \item 09-114-316 คณิตศาสตร์ประกันภัย (Mathematics of Insurance)
 \item 09-114-324 คณิตศาสตร์การลงทุน (Mathematics of Investment)
\end{itemize}
\item[กลุ่มที่ 2:] นักศึกษาเลือกเรียนรายวิชา
\begin{itemize}
	\item 09-114-316 คณิตศาสตร์ประกันภัย (Mathematics of Insurance)
	\item 09-114-331 เทคนิคการหาค่าเหมาะสม (Optimization Techniques)
	\item 09-114-324 คณิตศาสตร์การลงทุน (Mathematics of Investment)
	\item 09-114-339 วิทยาการข้อมูลสำหรับนักคณิตศาสตร์ (Data Sciences for Mathematicians)
\end{itemize}
\end{enumerate}

\begin{doclist}
\docitem{แบบเปิดรายวิชาของนักศึกษาชั้นปีที่ 3}
\end{doclist}


\subcriteria{The programme to show that its curriculum is reviewed periodically following an established procedure and that it remains up-to-date and relevant to industry.}
หลักสูตรวิทยาศาสตรบัณฑิต สาขาวิชาคณิตศาสตร์ประยุกต์ มีระบบในการทบทวนหลักสูตร
โดยหลักสูตรฯ ดำเนินการรวมรวมข้อมูลจากข้อเสนอแนะของผู้มีส่วนได้ส่วนเสีย ความต้องการของตลาดแรงงาน แนวโน้มทางวิชาการ ทิศทางการพัฒนาประเทศ ข้อมูลจากการประเมินผลการเรียนการสอนและผลการปฏิบัติงานของนักศึกษา ฯลฯ 

หลังจากรวบรวมข้อมูลแล้ว จะทำการวิเคราะห์และประเมินผลข้อมูลที่ได้รับ เพื่อระบุจุดแข็งและจุดที่ควรปรับปรุงในหลักสูตร รวมถึงการประเมินผลการเรียนการสอนและการบรรลุผลลัพธ์การเรียนรู้ที่คาดหวัง (PLOs) ของนักศึกษา ผลการวิเคราะห์และประเมินจะถูกนำมาพิจารณาในการปรับปรุงหลักสูตร ปรับปรุงกิจกรรมการเรียนการสอน และปรับปรุงการประเมินผล เพื่อให้หลักสูตรมีความทันสมัยและสอดคล้องกับความต้องการของอุตสาหกรรม

หลักสูตรฯ มีการดำเนินการทบทวนและปรับปรุงทั้งปรับปรุงย่อยและปรับปรุงตามรอบ 5 ปี โดยหลักสูตรฯ จะดำเนินการปรับปรุงหลักสูตรตามรอบ 5 ปีในปีการศึกษา 2568 ซึ่งมีขั้นตอนและการดำเนินการตามแนวทางของการพัฒนาหลักสูตรแบบ Outcome-based Education(OBE) และจะเปิดรับนักศึกษาในภาคเรียนที่ 1 ปีการศึกษา 2569

ในปีการศึกษา 2565 หลักสูตรฯ มีการรวบรวมข้อมูลจากภาคอุตสาหกรรมและผู้ใช้บัณฑิต
โดยผลจากการประเมินความพึงพอใจผู้ใช้บัณฑิต (ประเมินจากแบบสอบถาม) ปีการศึกษา 2565 พบว่าผู้ใชับัณฑิตมีข้อเสนอแนะดังนี้
\begin{enumerate}
	\item บัณฑิตขาดความมั่นใจในการสื่อสารภาษาอังกฤษ หลักสูตรควรส่งเสริมด้านภาษาอังกฤษ
	\item ควรส่งเสริมความรู้ด้านการประกันภัย และทักษะต่าง ๆ ที่เกี่ยวข้องกับธุรกิจวินาศภัย
\end{enumerate}

นอกจากนี้ในปีการศึกษา 2566 คณะฯ และมหาวิทยาลัยมีนโยบายให้ทุกหลักสูตรฯ ปรับปรุงย่อยหลักสูตรเพื่อรองรับการตรวจประเมินตามเกณฑ์การประกันคุณภาพการศึกษา Asean University Network Quality Assurance (AUN-QA) 

หลักสูตรฯ ได้ประชุมพิจารณาข้อเสนอแนะของผู้ใช้บัณฑิต และร่วมกันวางแผนการดำเนินงานในปีการศึกษา 2566 โดย
\begin{enumerate}
	\item ส่งเสริมให้มีการใช้ภาษาอังกฤษในการจัดการเรียนการสอน
	\item ส่งนักศึกษาเข้าร่วมอบรม/จัดกิจกรรม/โครงการที่ส่งเสริมทักษะภาษาอังกฤษ
	\item เพิ่มรายวิชาด้านประกันภัยในกลุ่มรายวิชาชีพเลือก เพื่อให้นักศึกษาที่มีความสนใจทางด้านประกันภัย ได้เลือกศึกษา
	\item ปรับปรุงหลักสูตรเพื่อรองรับการตรวจประเมินตามเกณฑ์การประกันคุณภาพการศึกษา AUN-QA
\end{enumerate}
ผลการดำเนินการ\\
ในปีการศึกษา 2566 หลักสูตรดำเนินการปรับปรุงย่อยหลักสูตร และเริ่มใช้ตั้งแต่ภาคเรียนที่ ปีการศึกษา 2566 เป็นต้นไป นอกจากนี้ยัง กำหนดให้ผู้สอนทุกรายวิชาชีพใช้ภาษาอังกฤษในการจัดการเรียนการสอนในบางหัวข้อ และส่งนักศึกษาชั้นปีที่ 3 เข้าร่วมอบรมโครงการการพัฒนาทักษะความสามารถการใช้ภาษาอังกฤษของนักศึกษาเพื่อเตรียมพร้อมก่อนทำงาน

\begin{doclist}
\docitem{ผลการประเมินความพึงพอใจผู้ใช้บัณฑิต ปีการศึกษา 2565}
\docitem{เอกสารการปรับปรุงแก้ไข\printprogram{} (สมอ. 08)}
\docitem{หลักฐานการเข้าร่วมโครงการการพัฒนาทักษะความสามารถการใช้ภาษาอังกฤษของนักศึกษาเพื่อเตรียมพร้อมก่อนทำงาน}
\end{doclist}

































