\chapter{สรุปผลการประเมินตนเอง}

\section{สรุปผลการประเมินตนเองตามตัวบ่งชี้องค์ประกอบที่ 1 การกำกับมาตรฐาน}

สำหรับหลักสูตรที่ใช้เกณฑ์มาตรฐานหลักสูตรระดับอุดมศึกษา ปี พ.ศ. 2558 ที่สำนักงานคณะกรรมการการอุดมศึกษา (สกอ.) กำหนด
{\small
	\begin{longtable}{|p{0.55\textwidth}|p{0.2\textwidth}|p{0.2\textwidth}|}
		\hline
\multicolumn{1}{|c|}{\textbf{เกณฑ์}}&
		\textbf{ผลการประเมินโดยตนเอง}&
		\textbf{ผลการประเมินโดย คก.ตรวจประเมินฯ}\\\cline{2-3}
		&2566&2566\\\hline
\endhead
		องค์ประกอบที่ 1 การกำกับมาตรฐาน& $\checkmark$ หรือ $\times$&$\checkmark$ หรือ $\times$\\\hline
		\textbf{หลักสูตรระดับปริญญาตรี}\newline
		(หลักสูตรระดับบัณฑิตศึกษาตัดเนื้อหาส่วนนี้ออก)&&\\\hline
		1. จำนวนอาจารย์ผู้รับผิดชอบหลักสูตร	&$\checkmark$&\\\hline
		2. คุณสมบัติอาจารย์ผู้รับผิดชอบหลักสูตร	&$\checkmark$&\\\hline
	3. คุณสมบัติอาจารย์ประจำหลักสูตร&$\checkmark$&\\\hline
	4. คุณสมบัติอาจารย์ผู้สอน	&$\checkmark$&\\\hline	
		5. การปรับปรุงหลักสูตรตามกรอบระยะเวลาที่กำหนด	&$\checkmark$&\\\hline
		\textbf{หลักสูตรระดับบัณฑิตศึกษา}\newline
		(หลักสูตรระดับปริญญาตรี ตัดเนื้อหาส่วนนี้ออก)&&	\\\hline	
		1. จำนวนอาจารย์ผู้รับผิดชอบหลักสูตร	&&	\\\hline	
		2. คุณสมบัติอาจารย์ผู้รับผิดชอบหลักสูตร	&&	\\\hline	
		3. คุณสมบัติอาจารย์ประจำหลักสูตร	&&	\\\hline	
		4. คุณสมบัติอาจารย์ผู้สอน		&&	\\\hline
		5. คุณสมบัติของอาจารย์ที่ปรึกษาวิทยานิพนธ์หลัก\newline\hspace*{0.3cm} และอาจารย์ที่ปรึกษาการค้นคว้าอิสระ&&	\\\hline		
		6. คุณสมบัติของอาจารย์ที่ปรึกษาวิทยานิพนธ์ร่วม\newline\hspace*{0.3cm} (ถ้ามี)&&	\\\hline		
		7. คุณสมบัติของอาจารย์ผู้สอบวิทยานิพนธ์	&&	\\\hline	
		8. การตีพิมพ์เผยแพร่ผลงานของผู้สำเร็จการศึกษา	&&	\\\hline	
		9. ภาระงานอาจารย์ที่ปรึกษาวิทยานิพนธ์\newline\hspace*{0.3cm} และการค้นคว้าอิสระในระดับบัณฑิตศึกษา	&&	\\\hline	
		10. การปรับปรุงหลักสูตรตามรอบระยะเวลาที่กำหนด	&&	\\\hline	
		\end{longtable}}


\section{สรุปผลการประเมินตนเองตามเกณฑ์ AUN-QA}

การประเมินคุณภาพระดับหลักสูตรโดยใช้เกณฑ์ AUN-QA (Version 4.0) ประกอบด้วยเกณฑ์คุณภาพ จำนวน 8 เกณฑ์ ที่ต้องพิจารณาด้วยหลักการของการประกันคุณภาพการศึกษาในแต่ละเกณฑ์ย่อย เพื่อให้เกิดการพัฒนาอย่างต่อเนื่อง โดยมีระดับการประเมินคุณภาพในแต่ละเกณฑ์ย่อย จำนวน 7 ระดับ ดังนี้
\begin{longtable}{|c|p{0.85\textwidth}|}
\hline
\textbf{Rating}&\multicolumn{1}{c|}{\textbf{Description}}\\\hline
\endhead
1&\textbf{Absolutely Inadequate}\newline
The QA practice to fulfill the criterion is not implemented. There are no plans, documents, evidences or results available. Immediate improvement must be made\newline

ยังไม่ได้ดำเนินการตามเกณฑ์ ไม่มีการวางแผน ไม่มีหลักฐาน หรือผลจากการดำเนินการ 
ต้องปรับปรุงแก้ไข หรือพัฒนาโดยเร่งด่วน \\\hline

2&\textbf{Inadequate and Improvement is Necessary}\newline 
The QA practice to fulfill the criterion is still at its planning stage or is inadequate where improvement is necessary. There is little document or evidence available. Performance of the QA practice shows little or poor results.\newline

เริ่มมีการวางแผนที่จะพัฒนา มีเอกสารหรือหลักฐานบ้าง มีการดำเนินการบางส่วนจำเป็น
ต้องมีการปรับปรุง แก้ไข หรือพัฒนา\\\hline

3&\textbf{Inadequate but Minor Improvement Will Make It Adequate}\newline 
The QA practice to fulfill the criterion is defined and implemented but minor improvement is needed to fully meet them. Documents are available but no clear evidence to support that they have been fully used. Performance of the QA practice shows inconsistent or some results.\newline
 
มีการปฏิบัติตามเกณฑ์ แต่ยังต้องปรับปรุง แก้ไข หรือพัฒนาเพียงเล็กน้อย พบเอกสารหลักฐานแต่ยังไม่สอดคล้องกับเกณฑ์การปฏิบัติ\\\hline

4&\textbf{Adequate as Expected} \newline 
The QA practice to fulfill the criterion is adequate and evidences support that it has been fully implemented. Performance of the QA practice shows consistent results as expected.\newline

มีการปฏิบัติตามเกณฑ์ และพบหลักฐานที่เพียงพอ ผลการปฏิบัติเป็นไปตามเกณฑ์ที่กำหนด\\\hline

5&\textbf{Better Than Adequate}\newline  
The QA practice to fulfill the criterion is better than adequate. Evidences support that it has been efficiently implemented. Performance of the QA practice shows good results and positive improvement trend.\newline

มีการปฏิบัติสูงกว่าเกณฑ์ที่กำหนด พบหลักฐานที่ทำให้เกิดประสิทธิภาพ ผลดำเนินการดี และมีแนวโน้มในทางที่ดียิ่งขึ้น\\\hline

6&\textbf{Example of Best Practices}\newline 
The QA practice to fulfill the criterion is considered to be example of best practices in the field. Evidences support that it has been effectively implemented. Performance of QA practice shows very good results and positive improvement trend.\newline 
 
มีการปฏิบัติตามเกณฑ์ที่เป็นตัวอย่างที่ดี พบหลักฐานที่ทำให้เกิดประสิทธิภาพ ผลดำเนินการ
ดีมาก และมีแนวโน้มในทางที่ดียิ่งขึ้น \\\hline

7&\textbf{Excellent (Example of World-class or Leading Practices)}\newline 
The QA practice to fulfill the criterion is considered to be excellent or example of world-class practices in the field. Evidences support that it has been innovatively implemented. Performance of the QA practice shows excellent results and outstanding improvement trends.\newline

มีการปฏิบัติตามเกณฑ์ที่ดีเยี่ยม หรือเป็นตัวอย่างแนวปฏิบัติระดับโลก พบหลักฐานที่ทำให้เกิด
การสร้างสรรค์ที่มีประสิทธิภาพ ผลดำเนินการดีเยี่ยม และมีแนวโน้มการพัฒนาที่โดดเด่น \\\hline
\end{longtable}

\clearpage
\textbf{Self-rating for AUN-QA Assessment at Programme Level}
\begin{longtable}{|c| p{0.55\textwidth}|p{0.3cm}|p{0.3cm}|c|p{0.3cm}|p{0.3cm}|p{0.3cm}|p{0.3cm}|}
\hline
\multicolumn{2}{|c|}{\bf AUN-QA Criterion 1}&\multicolumn{7}{c|}{\bf Rating}\\\cline{3-9}
\multicolumn{2}{|c|}{\bf Expected Learning Outcomes}&1&2&3&4&5&6&7\\\hline
\endhead
1.1&The programme to show that the expected learning outcomes are appropriately formulated in accordance with an established learning taxonomy, are aligned to the vision and mission of the university, and are known to all stakeholders.&&& {\huge{$\bullet$}}&&&&\\\hline

1.2&The programme to show that the expected learning outcomes for all courses are appropriately formulated and are aligned to the expected learning outcomes of the programme.&&& {\huge{$\bullet$}}&&&&\\\hline 

1.3&The programme to show that the expected learning outcomes consist of both generic outcomes (related to written and oral communication, problem-solving, information technology, teambuilding skills, etc) and subject specific outcomes (related to knowledge and skills of the study discipline).&&&&{\huge{$\bullet$}}&&&\\\hline 

1.4&The programme to show that the requirements of the stakeholders, especially the external stakeholders, are gathered, and that these are reflected in the expected learning outcomes.&&& {\huge{$\bullet$}}&&&&\\\hline

1.5& The programme to show that the expected learning outcomes are achieved by the students by the time they graduate.&&& {\huge{$\bullet$}}&&&&\\\hline

\multicolumn{2}{|r|}{\bf Overall opinion}&&& {\huge{$\bullet$}}&&&&\\\hline
\end{longtable}


\begin{longtable}{|c| p{0.55\textwidth}|p{0.3cm}|p{0.3cm}|c|p{0.3cm}|p{0.3cm}|p{0.3cm}|p{0.3cm}|}
	\hline
	\multicolumn{2}{|c|}{\bf AUN-QA Criterion 2}&\multicolumn{7}{c|}{\bf Rating}\\\cline{3-9}
	\multicolumn{2}{|c|}{\bf Programme Structure and Content}&1&2&3&4&5&6&7\\\hline
	\endhead
2.1&The specifications of the programme and all its courses are shown to be comprehensive, up-to-date, and made available and communicated to all stakeholders.&&& {\huge{$\bullet$}}&&&&\\\hline 

2.2&The design of the curriculum is shown to be constructively aligned with achieving the expected learning outcomes.&&& {\huge{$\bullet$}}&&&&\\\hline 

2.3& The design of the curriculum is shown to include feedback from stakeholders, especially external stakeholders. &&& {\huge{$\bullet$}}&&&&\\\hline

2.4& The contribution made by each course in achieving the expected learning outcomes is shown to be clear. &&& {\huge{$\bullet$}}&&&&\\\hline

2.5& The curriculum to show that all its courses are logically structured, properly sequenced (progression from basic to intermediate to specialised courses), and are integrated. &&& {\huge{$\bullet$}}&&&&\\\hline

2.6&The curriculum to have option(s) for students to pursue major and/or minor specialisations. &&& {\huge{$\bullet$}}&&&&\\\hline

2.7&The programme to show that its curriculum is reviewed periodically following an established procedure and that it remains up-to-date and relevant to industry.&&& {\huge{$\bullet$}}&&&&\\\hline 

\multicolumn{2}{|r|}{\bf Overall opinion}&&& {\huge{$\bullet$}}&&&&\\\hline

\end{longtable}



\begin{longtable}{|c| p{0.55\textwidth}|p{0.3cm}|p{0.3cm}|c|p{0.3cm}|p{0.3cm}|p{0.3cm}|p{0.3cm}|}
	\hline
	\multicolumn{2}{|c|}{\bf AUN-QA Criterion 3}&\multicolumn{7}{c|}{\bf Rating}\\\cline{3-9}
	\multicolumn{2}{|c|}{\bf Teaching and Learning Approach}&1&2&3&4&5&6&7\\\hline
	\endhead

3.1&The educational philosophy is shown to be articulated and communicated to all stakeholders. It is also shown to be reflected in the teaching and learning activities. &&& {\huge{$\bullet$}}&&&&\\\hline 

3.2&The teaching and learning activities are shown to allow students to participate responsibly in the learning process. &&& {\huge{$\bullet$}}&&&&\\\hline

3.3&The teaching and learning activities are shown to involve active learning by the students.&&& {\huge{$\bullet$}}&&&&\\\hline

3.4&The teaching and learning activities are shown to promote learning, learning how to learn, and instilling in students a commitment for life-long learning (e.g. commitment to critical inquiry, information-processing skills, and a willingness to experiment with new ideas and practices). &&& {\huge{$\bullet$}}&&&&\\\hline

3.5&The teaching and learning activities are shown to inculcate in students, new ideas, creative thought, innovation, and an entrepreneurial mindset.&&& {\huge{$\bullet$}}&&&&\\\hline

3.6& The teaching and learning processes are shown to be continuously improved to ensure their relevance to the needs of industry and are aligned to the expected learning outcomes.&&& {\huge{$\bullet$}}&&&&\\\hline
	

\multicolumn{2}{|r|}{\bf Overall opinion}&&& {\huge{$\bullet$}}&&&&\\\hline

\end{longtable}



\begin{longtable}{|c| p{0.55\textwidth}|p{0.3cm}|p{0.3cm}|c|p{0.3cm}|p{0.3cm}|p{0.3cm}|p{0.3cm}|}
	\hline
	\multicolumn{2}{|c|}{\bf AUN-QA Criterion 4}&\multicolumn{7}{c|}{\bf Rating}\\\cline{3-9}
	\multicolumn{2}{|c|}{\bf Student Assessment}&1&2&3&4&5&6&7\\\hline
	\endhead

4.1&A variety of assessment methods are shown to be used and are shown to be constructively aligned to achieving the expected learning outcomes and the teaching and learning objectives. &&& {\huge{$\bullet$}}&&&&\\\hline

4.2&The assessment and assessment-appeal policies are shown to be explicit, communicated to students, and applied consistently.&&& {\huge{$\bullet$}}&&&&\\\hline

4.3&The assessment standards and procedures for student progression and degree completion, are shown to be explicit, communicated to students, and applied consistently. &&& {\huge{$\bullet$}}&&&&\\\hline

4.4&The assessments methods are shown to include rubrics, marking schemes, timelines, and regulations, and these are shown to ensure validity, reliability, and fairness in assessment.  &&& {\huge{$\bullet$}}&&&&\\\hline

4.5&The assessment methods are shown to measure the achievement of the expected learning outcomes of the programme and its courses.&&& {\huge{$\bullet$}}&&&&\\\hline

4.6&Feedback of student assessment is shown to be provided in a timely manner.&&& {\huge{$\bullet$}}&&&&\\\hline

4.7&The student assessment and its processes are shown to be continuously reviewed and improved to ensure their relevance to the needs of industry and alignment to the expected learning outcomes.&&& {\huge{$\bullet$}}&&&&\\\hline 
	
\multicolumn{2}{|r|}{\bf Overall opinion}&&& {\huge{$\bullet$}}&&&&\\\hline 
\end{longtable}

\begin{longtable}{|c| p{0.55\textwidth}|p{0.3cm}|p{0.3cm}|c|p{0.3cm}|p{0.3cm}|p{0.3cm}|p{0.3cm}|}
	\hline
	\multicolumn{2}{|c|}{\bf AUN-QA Criterion 5}&\multicolumn{7}{c|}{\bf Rating}\\\cline{3-9}
	\multicolumn{2}{|c|}{\bf Academic Staff}&1&2&3&4&5&6&7\\\hline
	\endhead

5.1& The programme to show that academic staff planning (including succession, promotion, re-deployment, termination, and retirement plans) is carried out to
ensure that the quality and quantity of the academic staff fulfil the needs for education, research, and service.&&& {\huge{$\bullet$}}&&&&\\\hline 

5.2&The programme to show that staff workload is measured and monitored to improve the quality of education, research, and service.&&& {\huge{$\bullet$}}&&&&\\\hline

5.3&The programme to show that the competences of the academic staff are determined, evaluated, and communicated.&&& {\huge{$\bullet$}}&&&&\\\hline

5.4&The programme to show that the duties allocated to the academic staff are appropriate to qualifications, experience, and aptitude.&&& {\huge{$\bullet$}}&&&&\\\hline

5.5&The programme to show that promotion of the academic staff is based on a merit system which accounts for teaching, research, and service.&&& {\huge{$\bullet$}}&&&&\\\hline

5.6&The programme to show that the rights and privileges, benefits, roles and relationships, and accountability of the academic staff, taking into account professional ethics and their academic freedom, are well defined and understood.&&& {\huge{$\bullet$}}&&&&\\\hline

5.7&The programme to show that the training and developmental needs of the academic staff are systematically identified, and that appropriate training anddevelopment activities are implemented to fulfil the identified needs.&&& {\huge{$\bullet$}}&&&&\\\hline

5.8&The programme to show that performance management including reward and recognition is implemented to assess academic staff teaching and research quality.&&& {\huge{$\bullet$}}&&&&\\\hline

\multicolumn{2}{|r|}{\bf Overall opinion}&&& {\huge{$\bullet$}}&&&&\\\hline
\end{longtable}


\begin{longtable}{|c| p{0.55\textwidth}|p{0.3cm}|p{0.3cm}|c|p{0.3cm}|p{0.3cm}|p{0.3cm}|p{0.3cm}|}
	\hline
	\multicolumn{2}{|c|}{\bf AUN-QA Criterion 6}&\multicolumn{7}{c|}{\bf Rating}\\\cline{3-9}
	\multicolumn{2}{|c|}{\bf Student Support Services}&1&2&3&4&5&6&7\\\hline
	\endhead

6.1&The student intake policy, admission criteria, and admission procedures to the programme are shown to be clearly defined, communicated, published, and
up-to-date.&&& {\huge{$\bullet$}}&&&&\\\hline

6.2&Both short-term and long-term planning of academic and non-academic support services are shown to be carried out to ensure sufficiency and quality of support services for teaching, research, and community service.&&& {\huge{$\bullet$}}&&&&\\\hline

6.3&An adequate system is shown to exist for student progress, academic performance, and workload monitoring. Student progress, academic performance, and workload are shown to be systematically recorded and monitored. Feedback to students and corrective actions are made where necessary.&&& {\huge{$\bullet$}}&&&&\\\hline

6.4&Co-curricular activities, student competition, and other student support services are shown to be available to improve learning experience and employability.&&& {\huge{$\bullet$}}&&&&\\\hline

6.5&The competences of the support staff rendering student services are shown to be identified for recruitment and deployment. These competences are shown to be evaluated to ensure their continued relevance to stakeholders needs. Roles and relationships are shown to be well-defined to ensure smooth delivery of the services.&&& {\huge{$\bullet$}}&&&&\\\hline

6.6&Student support services are shown to be subjected to evaluation, benchmarking, and enhancement.&&& {\huge{$\bullet$}}&&&&\\\hline

\multicolumn{2}{|r|}{\bf Overall opinion}&&& {\huge{$\bullet$}}&&&&\\\hline
\end{longtable}


\begin{longtable}{|c| p{0.55\textwidth}|p{0.3cm}|p{0.3cm}|c|p{0.3cm}|p{0.3cm}|p{0.3cm}|p{0.3cm}|}
	\hline
	\multicolumn{2}{|c|}{\bf AUN-QA Criterion 7}&\multicolumn{7}{c|}{\bf Rating}\\\cline{3-9}
	\multicolumn{2}{|c|}{\bf Facilities and Infrastructure}&1&2&3&4&5&6&7\\\hline
	\endhead

7.1&The physical resources to deliver the curriculum, including equipment, material, and information technology, are shown to be sufficient.&&& {\huge{$\bullet$}}&&&&\\\hline

7.2& The laboratories and equipment are shown to be up-to-date, readily available, and effectively deployed.&&& {\huge{$\bullet$}}&&&&\\\hline

7.3&A digital library is shown to be set-up, in keeping with progress in information and communication technology.&&& {\huge{$\bullet$}}&&&&\\\hline

7.4&The information technology systems are shown to be set up to meet the needs of staff and students.&&& {\huge{$\bullet$}}&&&&\\\hline

7.5&The university is shown to provide a highly accessible computer and network infrastructure that enables the campus community to fully exploit information technology for teaching, research, service, and administration.&&& {\huge{$\bullet$}}&&&&\\\hline

7.6&The environmental, health, and safety standards and access for people with special needs are shown to be defined and implemented.&&& {\huge{$\bullet$}}&&&&\\\hline

7.7&The university is shown to provide a physical, social, and psychological environment that is conducive for education, research, and personal well-being.&&& {\huge{$\bullet$}}&&&&\\\hline

7.8&The competences of the support staff rendering services related to facilities are shown to be identified and evaluated to ensure that their skills remain relevant to stakeholder needs.&&& {\huge{$\bullet$}}&&&&\\\hline

7.9&The quality of the facilities (library, laboratory, IT, and student services) are shown to be subjected to evaluation and enhancement.&&& {\huge{$\bullet$}}&&&&\\\hline

\multicolumn{2}{|r|}{\bf Overall opinion}&&& {\huge{$\bullet$}}&&&&\\\hline
\end{longtable}

\begin{longtable}{|c| p{0.55\textwidth}|p{0.3cm}|p{0.3cm}|c|p{0.3cm}|p{0.3cm}|p{0.3cm}|p{0.3cm}|}
	\hline
	\multicolumn{2}{|c|}{\bf AUN-QA Criterion 8}&\multicolumn{7}{c|}{\bf Rating}\\\cline{3-9}
	\multicolumn{2}{|c|}{\bf Output and Outcomes}&1&2&3&4&5&6&7\\\hline
	\endhead

8.1&The pass rate, dropout rate, and average time to graduate are shown to be established, monitored, and benchmarked for improvement.&&& {\huge{$\bullet$}}&&&&\\\hline

8.2&Employability as well as self-employment, entrepreneurship, and advancement to further studies, are shown to be established, monitored, and benchmarked for improvement.&&& {\huge{$\bullet$}}&&&&\\\hline

8.3&Research and creative work output and activities carried out by the academic staff and students, are shown to be established, monitored, and benchmarked for improvement.&&& {\huge{$\bullet$}}&&&&\\\hline

8.4&Data are provided to show directly the achievement of the programme outcomes, which are established and monitored.&&& {\huge{$\bullet$}}&&&&\\\hline

8.5&Satisfaction level of the various stakeholders are shown to be established, monitored, and benchmarked for improvement.&&& {\huge{$\bullet$}}&&&&\\\hline
	
\multicolumn{2}{|r|}{\bf Overall opinion}&&& {\huge{$\bullet$}}&&&&\\\hline
\end{longtable}

\section{การวิเคราะห์จุดแข็งและจุดที่ควรพัฒนา}
\begin{enumerate}
	\item[]\textbf{จุดแข็ง}
	\begin{itemize}[label=-]
		\item หลักสูตรมีแผนพัฒนาอาจารย์ซึ่งสามารถดำเนินการแล้วส่งผลให้อาจารย์ผู้รับผิดชอบหลักสูตรสามารถผลิตผลงานทางวิชาการที่ตีพิมพ์ในฐานข้อมูลนานาชาติเป็นจำนวนมาก และเข้าสู่ตำแหน่งทางวิชาการได้ตามเป้าหมายที่ตั้งไว้
		\item มีอัตราการได้งานทำของบัณฑิตภายในระยะเวลาหนึ่งปีไม่ต่ำกว่าร้อยละ 90
		\item รายวิชาของหลักสูตรมีความทันสมัยสอดคล้องกับความต้องการของตลาดแรงงาน
	\end{itemize}
\item[]\textbf{จุดอ่อน}
	\begin{itemize}[label=-]
	\item จำนวนนักศึกษาแรกเข้าต่ำกว่าแผนรับ
	\end{itemize}
\end{enumerate}
\section{แผนหรือแนวทางพัฒนาคุณภาพ}
พิจารณาออกแบบรูปแบบประชาสัมพันธ์เพื่อสื่อสารถึงกลุ่มเป้าหมาย โดยใช้จุดเด่นของหลักสูตรที่สามารถนำไปประกอบอาชีพ ศิษย์เก่าที่ประสบความสำเร็จ บนแพลตฟอร์มออนไลน์

\section{สรุปผลการดำเนินการปรับปรุงตามข้อเสนอแนะจากปีการศึกษาที่ผ่านมา}
เนื่องจากเป็นการประเมินตามเกณฑ์ AUN-QA เป็นปีแรก จึงไม่มีการดำเนินการในประเด็นนี้
\begin{longtable}{|p{0.6\textwidth}|c|}
\caption{ผลการดำเนินการปรับปรุงตามข้อเสนอแนะของคณะกรรมการตรวจประเมินฯ ในปีการศึกษาที่ผ่านมาในแต่ละ Criteria}	
\\
\hline
	\multicolumn{1}{|c|}{\textbf{ข้อเสนอแนะของคณะกรรมการตรวจประเมิน}}
	&\multicolumn{1}{c|}{\bf ผลการดำเนินการ}\\
	\multicolumn{1}{|c|}{\textbf{ปีการศึกษาที่ผ่านมาในแต่ละ Criteria}}&\\\hline
\endhead
	1. Expected Learning Outcomes\newline-&-\\\hline
	2. Programme Structure and Content\newline-&-\\\hline
	3. Teaching and Learning Approach
\newline-&-\\\hline
4. Student Assessment\newline-&-\\\hline
5. Academic Staff\newline-&-\\\hline
6. Student Support Services\newline-&-\\\hline
7. Facilities and Infrastructure\newline-&-\\\hline
8. Output and Outcomes\newline-&-\\\hline
	 \end{longtable}


