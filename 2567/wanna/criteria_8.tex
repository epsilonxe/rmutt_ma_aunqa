\newpage
\criteria{Output and Outcomes}

\subcriteria{The pass rate, drop-out rate, and average time to graduate are shown to be established, monitored, and benchmarked for improvement.}
%%%%%%%%%%%%%%%%%%%%%%%%%%%%%%%%%%%%%%%%%%
หลักสูตรได้มีการพิจารณาจำนวนผู้สำเร็จการศึกษาและการตกออกของนักศึกษาทุก ๆ ปี มีการวิเคราะห์ข้อมูลการตกออก เพื่อติดตามดูแลให้คำปรึกษาแก่นักศึกษา และบันทึกจำนวนนักศึกษาที่คงอยู่และที่ตกออกในแต่ละปีการศึกษา จัดทำข้อมูลสรุปสาเหตุการตกออกของนักศึกษา ส่งให้ประธานหลักสูตรและหัวหน้าสาขาเพื่อใช้ในการวางแนวทางแก้ไขปัญหาต่อไป 
%%%%%%%%%%%%%%%%%%%%%%%%%%%%%%%%%%%%%%%%%%
ซึ่งมีรายละเอียดจำนวน ร้อยละของผู้สำเร็จการศึกษาและผู้ที่ตกออก ของนักศึกษาที่รับเข้าในปีการศึกษา 2561-2564 ดังตาราง\\[0.2cm]
  {\footnotesize
 	\begin{tabular}{|c|c|c|c|c|c|c|c|}
 		\hline
 		{\multirow{2}{0.11\textwidth}{ปีการศึกษา}} & {\multirow{2}{0.12\textwidth}{จำนวนรับเข้า\newline  \hspace*{0.2cm}(มีตัวตน)}} & \multicolumn{4}{c|}{ผู้สำเร็จการศึกษาตามหลักสูตร} & \multirow{2}{0.1\textwidth}{\centering{ร้อยละของผู้สำเร็จการศึกษาตามหลักสูตร}} & \multirow{2}{0.11\textwidth}{\centering{ร้อยละของ\newline ผู้ที่ตกออก}} \\
 		\cline{3-6}          &      &\multicolumn{1}{p{0.05\textwidth}|}{{ 4 ปี\newline จำนวน}}  & \multicolumn{1}{p{0.058\textwidth}|}{{ 4 ปี\newline ร้อยละ}}  & \multicolumn{1}{p{0.105\textwidth}|}{{มากกว่า 4 ปี\newline จำนวน}}     &\multicolumn{1}{p{0.105\textwidth}|}{{มากกว่า 4 ปี\newline ร้อยละ}}&      & \\\hline
 		2561     & 22    &  14  & 63.64&0 &  0  & 63.64   & 36.36   \\  \hline
 		2562     & 17    &  13  & 76.47 &0&  0 & 76.47   & 23.53  \\   \hline
 		2563     & 8     &  6   &  75  & 1 & 12.50 & 87.5   &  12.5  \\   \hline
 		2564& 33     &  22   & 66.67   &   0&  0  & 66.67   &  33.33  \\  \hline
 	\multicolumn{6}{|c|}{}&	73.57   &  27.11 \\\hline
 	\end{tabular}   
 }
 

 จากตาราง พบว่า อัตราการสำเร็จการศึกษาตามหลักสูตรโดยเฉลี่ยคิดเป็นร้อยละ 73.57 และมีส่วนอัตราการตกออกเฉลี่ยคิดเป็นร้อยละ 27.11 หลักสูตร ได้วิเคราะห์ถึงสาเหตุการตกออกของนักศึกษา พบว่านักศึกษาส่วนใหญ่จะตกออกในช่วงชั้นปีที่ 1 และชั้นปีที่ 2 ทั้งนี้มีปัญหาหลักเนื่องมากจาก 
 \begin{itemize}
 	\item นักศึกษาบางส่วนไม่ได้สนในที่จะเรียนในหลักสูตร แต่เข้ามาเรียนเนื่องจากสอบไม่ติดในหลักสูตรอื่นที่ตนเองสนใจทำให้เลือกที่จะไปสอบเข้าเรียนใหม่
 	\item นักศึกษาบางส่วนปรับตัวกับการเรียนในรั้วมหาวิทยาลัยไม่ได้ รู้สึกท้อ เข้ากับเพื่อนๆ ไม่ได้ ทำให้อยากลาออก หรือผลการเรียนไม่ดี ทำให้ถูกรีไทร์
 \end{itemize}
  
  หลักสูตรจึงหาแนวทางในการแก้ปัญหาดังกล่าว รวมถึงศึกษาแนวทางการดำเนินงานของหลักสูตรอื่นที่มีผลการดำเนินงานที่ดี เพื่อแก้ปัญหาการตกออกของนักศึกษา โดยเลือกหลักสูตรวิทยาศาสตรบัณฑิต สาขาวิชาสถิติประยุกต์ เป็นคู่เทียบ ซึ่งมีรายละเอียดผลการดำเนินงานของหลักสูตรดังนี้\\[0.25cm]
   {\footnotesize
  	\begin{tabular}{|c|c|c|c|c|c|c|c|}
  		\hline
  		{\multirow{2}{0.11\textwidth}{ปีการศึกษา}} & {\multirow{2}{0.12\textwidth}{จำนวนรับเข้า\newline  \hspace*{0.2cm}(มีตัวตน)}} & \multicolumn{4}{c|}{ผู้สำเร็จการศึกษาตามหลักสูตร} & \multirow{2}{0.1\textwidth}{\centering{ร้อยละของผู้สำเร็จการศึกษาตามหลักสูตร}} & \multirow{2}{0.11\textwidth}{\centering{ร้อยละของ\newline ผู้ที่ตกออก}} \\
  		\cline{3-6}          &      &\multicolumn{1}{p{0.05\textwidth}|}{{ 4 ปี\newline จำนวน}}  & \multicolumn{1}{p{0.058\textwidth}|}{{ 4 ปี\newline ร้อยละ}}  & \multicolumn{1}{p{0.105\textwidth}|}{{มากกว่า 4 ปี\newline จำนวน}}     &\multicolumn{1}{p{0.105\textwidth}|}{{มากกว่า 4 ปี\newline ร้อยละ}}&      & \\\hline
  		2561     & 10    &  10  & 100 &0 &  0  & 100   & 0   \\  \hline
  		2562     & 17    &  13  & 76.47 &0&  0 & 76.47   & 23.53  \\   \hline
  		2563     & 32    &  25  &  78.13  & 0 & 0 & 78.13   &  21.87  \\   \hline
  		2564& 40     &  35   & 87.50   &   1&  2.5  & 87.50   &  10  \\  \hline
  		\multicolumn{6}{|c|}{}&	85.53   &  13.85 \\\hline
  	\end{tabular}   
  }
 
 จากตาราง พบว่าหลักสูตรวิทยาศาสตรบัณฑิต สาขาวิชาสถิติประยุกต์มีอัตราการสำเร็จการศึกษาตามหลักสูตรโดยเฉลี่ยคิดเป็นร้อยละ 85.53 และมีส่วนอัตราการตกออกเฉลี่ยคิดเป็นร้อยละ 13.85 ซึ่งมีอัตราการสำเร็จการศึกษาตามหลักสูตรเฉลี่ยสูงกว่า และอัตราการตกออกเฉลี่ยต่ำกว่าหลักสูตรวิทยาศาสตรบัณฑิต สาขาวิชาคณิตศาสตร์ประยุกต์ และจากการศึกษาแนวทางการดำเนินงานของหลักสูตรวิทยาศาสตรบัณฑิต สาขาวิชาสถิติประยุกต์ ซึ่งพบว่ามีระบบการให้คำปรึกษาที่ดี มีการปรับรูปแบบการเรียนการสอนที่เน้นการประยุกต์ใช้และนักศึกษาได้ฝึกปฏิบัติมากขึ้น และมีการจัดอบรมความรู้ที่เกียวข้องกับรายวิชาในหลักสูตรเพื่อเพิมพูนองค์ความรู้ที่จําเป็นสําหรับนักศึกษา 

หลักสูตร จึงนำแนวมาปรับปรุงการดำเนินงานของหลักสูตรโดยจัดทำเป็นแผนระยะสั้นและแผนระยะยาว ดังปรากฎใน criterion 6.2 ดังนี้
 \begin{itemize}
\item จัดกิจกรรม/โครงการเพื่อสร้างแรงบัลดาลใจ
 \begin{itemize}
 	\item นำนักศึกษาไปศึกษาดูงานที่สถานประกอบการ  เพื่อให้เห็นแนวทางการนำคณิตศาสตร์ไปใช้ในการทำงานจริง
  	\item เชิญศิษย์เก่าที่ประสบความสำเร็จมาบรรยายเพื่อสร้างแรงบัลดาลใจในการเรียน
 	\item จัดกิจกรรม สานสัมพันธ์ น้อง-พี่ สาขาวิชาคณิตศาสตร์
 	\item จัดกิจกรรมแสดงความยินดีกับพี่บัณฑิตสาขาวิชาคณิตศาสตร์
 	\item จัดกิจกรรมเตรียมความพร้อมเข้าสู่รั้วมหาวิทยาลัยและปฐมนิเทศนักศึกษาใหม่สาขาวิชาคณิตศาสตร์
 	 \end{itemize}
 \item จัดกิจกรรม/โครงการเพื่อส่งเสริมความรู้ทางด้านวิชาการและวิชาชีพ
 \begin{itemize}
 	\item โครงการถ่ายทอดประสบการณ์จริงสู่การฝึกประสบการณ์วิชาชีพสาขาวิชาคณิตศาสตร์
 	\item โครงการการพัฒนาศักยภาพด้านการเรียนรู้เชิงลึกสำหรับการแก้ปัญหาในศตวรรษที่ 21
 	\item โครงการพัฒนาทักษะกระบวนการคิดและการเรียนรู้ในการส่งเสริมความเป็นนวัตกรของนักศึกษาสาขาวิชาคณิตศาสตร
 \end{itemize}
 \end{itemize}

นอกจากนี้หลักสูตรยังมีระบบการให้คำปรึกษา มีการกำกับติดตามนักศึกษาในด้านต่างๆ ทั้งทางด้านการเรียน และการใช้ชีวิตเป็นต้น เพื่อลดอัตราการตกออกและส่งเสริมให้นักศึกษาสำเร็จการศึกษาตามหลักสูตรมากขึ้น
\begin{doclist}
\docitem{ข้อมูลจำนวนนักศึกษา จำนวนนักศึกษาตกออก}
\docitem{ข้อมูลจำนวนนักศึกษาที่สำเร็จการศึกษาตามแผน และระยะเวลาการสำเร็จการศึกษาเฉลี่ย}
\end{doclist}


\subcriteria{Employability as well as self-employment, entrepreneurship, and advancement to further studies, are shown to be established, monitored, and benchmarked for improvement.}

การเก็บรวบรวมข้อมูลเกี่ยวกับภาวะการมีงานทำภายใน 1 ปี และ รายได้เฉลี่ยต่อเดือนของบัณฑิตระดับปริญญาตรี  ทางมหาวิทยาลัยได้มอบหมายให้กองพัฒนานักศึกษา (กพน.) เป็นผู้เก็บรวบรวม วิเคราะห์ และส่งผลการสำรวจกลับมาให้ทางคณะและหลักสูตร 

ในปีการศึกษา 2567 \printprogram{}
ยังไม่มีนักศึกษาสำเร็จการศึกษา หลักสููตรจึงนำข้อมูลภาวะการมีงานทำของหลักสูตร วิทยาศาสตรบัณฑิต สาขาวิชาคณิตศาสตร์(หลักสูตรปรับปรุง พ.ศ.2559) มาพิจารณา โดยมีคู่เทียบคือ หลักสูตรวิทยาศาสตรบัณฑิต สาขาวิชาคณิตศาสตร์ประยุกต์ มหาวิทยาลัยธรรมศาสตร์ โดยนำข้อมูลภาวะการมีงานทำที่เผยแพร่ผ่านเว็บไซต์ของสำนักงานปลัดกระทรวงการอุดมศึกษา วิทยาศาสตร์ วิจัยและนวัตกรรม (https://employ.mhesi.go.th/) มาใช้ในการวิเคราะห์ ซึ่งมีรายละเอียดดังตารางต่อไปนี้
 \begin{longtable}{|x{0.1\textwidth}|*{7}{x{0.07\textwidth}|}} % Adjusted column widths
	\hline
	\multicolumn{1}{|x{0.2\textwidth}|}{\textbf{ปีการศึกษา}} &
	\multicolumn{4}{x{0.4\linewidth}|}{\textbf{วท.บ.(คณิตศาสตร์) มทร.ธัญบุรี}} &
	\multicolumn{3}{x{0.3\linewidth}|}{\textbf{วท.บ.(คณิตศาสตร์ประยุกต์) ม.ธรรมศาสตร์ (คู่เทียบ)}} \\
	\cline{2-8}
	\multicolumn{1}{|x{0.07\textwidth}|}{} &
	\multicolumn{1}{x{0.07\textwidth}|}{ จำนวนบัณฑิตทั้งหมด} &
	\multicolumn{1}{x{0.07\textwidth}|}{ จำนวนบัณฑิตที่ตอบฯ} &
	\multicolumn{1}{x{0.07\textwidth}|}{ ร้อยละการได้งานทำใน 1 ปี} &
	\multicolumn{1}{x{0.07\textwidth}|}{ รายได้เฉลี่ยต่อเดือน} &
	\multicolumn{1}{x{0.07\textwidth}|}{ จำนวนบัณฑิตที่ได้งานทำ} &
	\multicolumn{1}{x{0.07\textwidth}|}{ ร้อยละการได้งานทำใน 1 ปี} &
	\multicolumn{1}{x{0.07\textwidth}|}{ รายได้เฉลี่ยต่อเดือน} \\
	\hline
	\endhead % End of table header
	
	\hline
	2564  &     14   &     11  &    71.43  &   17,300  &    16  &   66.67 & 21,591  \\
	\hline
	2565 &  14   &     11  &    100.00  &   20,272  &    26  &   72.22 & 17,955\\
	\hline
	2566 &  6   &     6  &      83.33  &   17,000  &    19  &   67.85  &  24,167 \\
	\hline
	\multicolumn{3}{|l|}{\bf เฉลี่ย}&84.92&18,191&&68.91&21,238\\	\hline
\end{longtable}
จากตารางพบว่า อัตราการได้งานทำหลังสำเร็จการศึกษาของหลักสูตรวิทยาศาสตรบัณฑิต สาขาวิชาคณิตศาสตร์ (หลักสูตรปรับปรุง พ.ศ.2559) เฉลี่ย 3 ปีย้อนหลังคิดเป็นร้อยละ 84.92 แสดงให้เห็นว่าบัณฑิตที่จบการศึกษาจาหลักสูตร ยังเป็นที่ต้องการของตลาดแรงงาน  แต่เมื่อพิจารณาเงินเดือนเฉลี่ย 3 ปีย้อนหลัง เทียบกับบัณฑิตที่จบการศึกษาหลักสูตรวิทยาศาสตรบัณฑิต สาขาวิชาคณิตศาตร์ประยุกต์ มหาวิทยาลัยธรรรมศาสตร์ พบว่า บัณฑิตที่จบการศึกษาหลักสูตรวิทยาศาสตรบัณฑิต สาขาวิชาคณิตศาตร์ประยุกต์ มหาวิทยาลัยธรรรมศาสตร์มีเงินเดือนเฉลี่ยที่สูงกว่า

อาจารย์ผู้รับผิดชอบหลักสูตรจึงร่วมกันวิเคราะห์ในประเด็นดังกล่าวซึ่งพบว่า หลักสูตรวิทยาศาสตรบัณฑิต สาขาวิชาคณิตศาสตร์ (หลักสูตรปรับปรุง พ.ศ. 2559) เป็นหลักสูตรทางด้านคณิตศาสตร์บริสุทธิ์ รายวิชาทางด้านคณิตศาสตร์ประยุกต์มีน้อย ทำให้บัณฑิตที่จบการศึกษาจากหลักสูตรส่วนใหญ่ประกอบอาชีพเป็นผู้สอนในสถาบันการศึกษาภาคเอกชน ส่วนบัณฑิตที่จบจากหลักสูตรของมหาวิทยาลัยธรรมศาสตร์ เนื่องจากหลักสูตรเป็นด้านคณิตศาสตร์ประยุกต์จึงตรงตามความต้องการของภาคธุรกิจและอุตสาหกรรมมากกว่าทำให้บัณฑิตในหลักสูตรส่วนใหญ่ทำงานในบริษัทภาคเอกชนทำให้มีเงินเดือนที่สูงกว่า

อาจารย์ผู้รับผิดชอบหลักสูตรจึงปรับปรุงกระบวนการโดย
\begin{itemize}
	\item ปรับหลักสูตรให้มีความทันสมัยโดยปรับเป็น \printprogram{} ซึ่งเป็นหลักสูตรที่ใช้อยู่ในปัจจุบัน และจะมีนักศึกษาจบการศึกษารุ่นแรกในปีการศึกษา 2567
	\item ส่งเสริมการจัดกิจกรรม/โครงการพัฒนาศักยภาพนักศึกษาในด้านต่างๆ ที่ตรงตามความต้องการของภาคอุตสาหกรรม ดังรายละเอียดการดำเนินงานใน criterion 6.2
	\item ส่งสริมให้นักศึกษาออกฝึกประสบการณ์วิชาชีพในสถานประกอบการภาคธุรกิจและอุตสาหกรรมมากขึ้น
\end{itemize}
\begin{doclist}
\docitem{ข้อมูลภาวะการมีงานทำภายใน 1 ปี ของบัณฑิต }
\docitem{ข้อมูลรายได้เฉลี่ยต่อเดือนของบัณฑิต}
\end{doclist}

\subcriteria{Research and creative work output and activities carried out by the academic staff and students, are shown to be established, monitored, and benchmarked for improvement.}
หลักสูตรมีการส่งเสริมและสนับสนุนอาจารย์ในหลักสูตรในด้านต่าง ๆ รวมทั้งด้านงานวิจัย ดังรายละเอียดใน criterion 5 โดยหลักสูตรได้มีการกำหนดตัวชี้วัดความสำเร็จในประเด็นนี้ คือ จำนวนงานวิจัยตีพิมพ์ในฐานข้อมูลสากล (scopus) ที่มีคุณภาพระดับสูง (Q1) ของอาจารย์ในหลักสูตรไม่ต่ำกว่าร้อยละ 20 ของจำนวนงานวิจัยตีพิมพ์ทั้งหมด

ซึ่งผลงานตีพิมพ์ของอาจารย์ในหลักสูตรปีการศึกษา 2564-2567 โดยมีหลักสูตรวิทยาศาสตรบัณฑิต สาขาวิชาสถิติประยุกต์ เป็นคู่เทียบ มีรายละเอียดดังตาราง 
\begin{longtable}{| >{\raggedright}p{0.3\textwidth} |c|c|c|c|c|c|c|c|} 
\hline
\multicolumn{1}{|c|}{ระดับผลงาน}&\multicolumn{4}{c|}{วท.บ.(คณิตศาสตร์ประยุกต์)}&\multicolumn{4}{c|}{วท.บ.(สถิติประยุกต์)}\\\cline{2-9}
&2564&2565&2566&2567&2564&2565&2566&2567\\\hline
Q1&14&13&6&4&19&13&9&2\\
Q2&9&6&5&1&3&5&3&5\\
Q3&4&3&2&0&8&7&3&4\\
Q4&2&2&2&0&5&2&0&1\\
TCI1&3&3&1&0&1&2&2&1\\
TCI2&2&0&0&0&3&7&2&0\\
\hline
รวม&36&27&16&5&39&36&19&13\\\hline
ร้อยละของผลงานระดับ Q1 &39&48&38&80&49&36&47&15\\\hline
จำนวนผลงานตีพิมพ์ต่ออาจารย์ในหลักสูตร&2&1.5&0.94&0.28&3.25&3&1.58&1.18\\\hline
\end{longtable}
จากตารางพบว่า ร้อยละของผลงานระดับ Q1 ของอาจารย์ในหลักสูตร ระหว่างปีการศึกษา 2564-2567 สูงกว่าเกณฑ์ที่หลักสูตรกำหนดทุกปี แต่เมื่อเทียบกับหลักสูตรวิทยาศสาสตรบัณฑิต สาขาวิชาสถิติประยุกต์ พบว่าจำนวนผลงานวิจัยตีพิมพ์ต่ออาจารย์ในหลักสูตร ของหลักสูตรวิทยาศาสตรบัณฑิต สาขาวิชาสถิติประยุกต์สูงกว่าในทุกปี 
หลักสูตรจึงทบทวนกระบวนการการดำเนินงาน ตลอดจนศึกษาแนวทางการดำเนินงานของหลักสูตรวิทยาศาสตรบัณฑิต สาขาวิชาสถิติประยุกต์ มาประกอบการปรับปรุงกระบวนการ การดำเนินงานในปีการศึกษาถัดไป โดยกำหนด KPI ให้อาจารย์ทุกท่านมีงานวิจัยตีพิมพ์ ส่งเสริมการจัดตั้งระบบพี่เลี้ยงในการเขียนงานวิจัยตีพิมพ์ในฐานข้อมูลสากล (scopus) รวมทั้งการรวมกลุ่มเพื่อทำวิจัย เป็นต้น ทั้งนี้เพื่อเป็นการส่งเสริมและสนับสนุนให้อาจารย์ทุกท่านมีผลงานวิจัยตีพิมพ์และเตรียมพร้อมสำหรับการเข้าสู่ตำแหน่งทางวิชาการในระดับที่สูงขึ้น

สำหรับการส่งเสริมนักศึกษาด้านการวิจัย หลักสูตรมีการส่งเสริมให้นักศึกษาเรียนรู้กระบวนการทำวิจัย เพื่อให้เกิดทักษะ กระบวนการคิด วิเคราะห์ คำนวณ การแก้ปัญหา การทำงานเป็นทีม การแสวงหาความรู้ และสามารถบูรณาการองค์ความรู้ที่ได้เรียนมาใช้ในการทำโครงงานในรายวิชาโครงงานด้านคณิตศาสตร์ ซึ่งโครงงานของนักศึกษาในปีการศึกษา 2564-2567 มีรายละเอียดดังนี้

\begin{enumerate}
\item[]{\bf ปีการศึกษา 2564}
\begin{itemize}
\item[(1)]  On the  $(s,t)$-Pell and  $(s,t)$-Pell-Lucas Polynomials by Matrix Methods
\item[(2)]  A New Iterative Scheme for Approximation of Fixed Points in Banach Spaces
\item[(3)]  Classes of Matrices over a Commutative Ring with Identity whose Determinant are Zero 
\item[(4)]  On Some Diophantine Equations of The Form $\frac{a}{x}+\frac{b}{y}+\frac{c}{z}=d$
\item[(5)]  การประมาณค่าที่หายไปของดัชนีคุณภาพอากาศจากสถานีวัด
\item[(6)]  การพัฒนาแบบจำลองทางคณิตศาสตร์สำหรับการจัดการความเสี่ยงและการประยุกต์ใช้
\end{itemize}

\item[]\textbf{ปีการศึกษา 2565}
\begin{itemize}
\item[(1)] Bivariate Vieta-Fibonacci-like polynomials

\item[(2)] Some new $(s,t)$-Pell and $(s,t)$-Pell-Lucas polynomials identities by matrix methods

\item[(3)]  Bi-Periodick-Pell Sequence

\item[(4)]  Convergence Theorems for Modified Three-Step Iterations in Uniformly Convex Metric Spaces 

\item[(5)]  การประมาณค่าดัชนีคุณภาพอากาศ ณ จุดที่ไม่มีสถานีวัด
\end{itemize}

\item[]\textbf{ปีการศึกษา 2566}
\begin{itemize}
\item[(1)] Bivariate Vieta-Jacobsthal-like polynomials

\item[(2)] Some Properties of Determinant of Matrices over Generalized Fibonacci Numbers and Generalized Gaussian Fibonacci Numbers

\item[(3)] การวิเคราะห์เกี่ยวกับจำนวนเพลล์และจำนวนเพลล์ลูคัส
\end{itemize}


\item[]\textbf{ปีการศึกษา 2567}
\begin{itemize}
	\item[(1)] การลงทุนในหุ้นร่วมกับออปชั่น (Investing in stocks with options)
	
	\item[(2)] A Multi-Day Multi-Hub Delivery Planning
	
	\item[(3)] Fixed point methodologies for logistic regression problem with application to Alzheimer’s disease screening     
	
	\item[(4)] Generating Music Variation through Chaotic Dynamical System Exploration
	
	\item[(5)] Generalized Vieta-Fibonacci-Type Polynomials and Generalized Vieta Pell-Type Polynomials
	
	\item[(6)] เว็บไซต์ระบบการจัดการทุนการศึกษา
	
	
	\item[(7)] On the Generalized Vieta-Pell and Vieta- Pell-Lucas polynomials by matrix methods  
	
	
	\item[(8)] On the Diophantine Equation $F^n_{x-1} + F^n_{x+1} = y^2$
	
\end{itemize}
\end{enumerate}

นอกจากนี้ในปีการศึกษา 2567 หลักสูตรยังมีการส่งเสริมให้นักศึกษาร่วมนำเสนอผลงานวิจัยในงานประชุมวิชาการระดับชาติ ดังรายละเอียด criterion 6.2

\begin{doclist}
\docitem{ข้อมูลงานวิจัยตีพิมพ์ของอาจารย์ในหลักสูตร}
\docitem{ข้อมูลโครงงานของนักศึกษา}
\docitem{ข้อมูลการส่งนักศึกษาเข้าร่วมนำเสนอผลงานในงานประชุมวิชาการ}
\end{doclist}

\subcriteria{Data are provided to show directly the achievement of the programme outcomes, which are established and monitored.}
หลักสูตรมีกระบวนการประเมินความสำเร็จของหลักสูตร โดยวิเคราะห์จากผลการบรรลุ PLOs ของนักศึกษาที่สำเร็จการศึกษาตามหลักสูตร ซึ่งหลักสูตรกำหนดเป้าหมายความสำเร็จไว้ดังนี้
\begin{itemize}
	\item นักศึกษาที่สำเร็จการศึกษาตามหลักสูตรต้องบรรลุ PLOs ของหลักสูตร ครบทุกข้อ
	\item จำนวนนักศึกษาที่บรรลุ PLOs ตั้้งแต่ระดับดีขึ้นไปไม่น้อยกว่าร้อยละ 40
\end{itemize}

ในปีการศึกษา 2567 มีนักศึกษาสำเร็จการศึกษาตามหลักสูตรจำนวน 22 คน โดยนักศึกษาทั้ง 22 คน
บรรลุ PLOs ทั้งหมดของหลักสูตร ดังรายละเอียดใน criterion 1.5  และมีจำนวนนักศึกษาที่บรรลุ PLOs ในแต่ละระดับแสดงดังตาราง
\begin{center}
	\begin{tabular}{|c|c|c|}
		\hline
		\textbf{ช่วงคะแนน}&\textbf{ระดับการบรรลุ PLOs}&\textbf{จำนวนนักศึกษา (คน)}\\\hline
		3.50-4.00&ดีมาก&3\\\hline
		3.00-3.49&ดี&8\\\hline
		2.50-2.99&ปานกลาง&10\\\hline
		2.00-2.49&น้อย&1\\\hline
		1.00-1.99&น้อยที่สุด&0\\\hline
	\end{tabular}
\end{center}

จากตารางพบว่ามีนักศึกษาที่บรรลุ PLOs ตั้้งแต่ระดับดีขึ้นไปจำนวน 11 คน คิดเป็นร้อยละ 40 และมีค่าเฉลี่ยระดับการบรรลุ PLOs ของนักศึกษาในภาพรวมอยู่ในระดับดี (คะแนนเฉลี่ย 3.04 จากคะแนนเต็ม 4) ซึ่งเห็นได้ว่าหลักสูตรมีความสำเร็จตามเป้าหมายที่กำหนดไว้


\textbf{คู่เทียบ} หลักสูตรได้นำหลักสูตรวิทยาศาสตรบัณฑิต  สาขาวิชาสถิติประยุกต์มาเป็นคู่เทียบในการดำเนินงานด้านการประเมินผลการบรรลุ PLOs ของนักศึกษา ซึ่งมีผลการประเมินการบรรลุ PLOs ของนักศึกษาที่สำเร็จการศึกษาปีการศึกษา 2567 ที่ประเมินแบบทางตรง ในภาพรวมอยู่ในระดับดี (คะแนนเฉลี่ย 4.17 จากคะแนนเต็ม 5)

จากการเปรียบเทียบพบว่าระดับคะแนนเฉลี่ยของการบรรลุ PLOs ของหลักสูตรวิทยาศาสตรบัณฑิต สาขาวิชาสถิติประยุกต์สูงกว่า นอกจากนี้เมื่อวิเคราะห์ถึงระดับการบรรลุ PLOs ของหลักสูตร พบว่านักศึกษาที่สำเร็จการศึกษาในหลักสูตรร้อยละ 50 มีคะแนนประเมินต่ำกว่าระดับดี และมีนักศึกษาร้อยละ 4.56 มีผลการประเมินอยู่ในระดับน้อย 

อาจารย์ผู้รับผิดชอบหลักสูตรจึงร่วมกันพิจารณาและทบทวนกระบวนการดำเนินงานรวมถึงการศึกษาแนวทางการดำเนินงานของหลักสูตรวิทยาศาสตรบัณฑิต สาขาวิชาสถิติประยุกต์ และวางแผนปรับปรุงการดำเนินงานด้านต่างๆ ทั้งด้านการเรียนการสอน การวัดประเมินผล และการจัดโครงการ/กิจกรรมพัฒนานักศึกษา เป็นต้นดังรายละเอียดการดำเนินการใน criterion 3,4 และ criterion 6 เพื่อส่งเสริมให้นักศึกษาบรรลุ PLOs ในระดับที่สูงขึ้น

%%%%%%%%%%%%%%%%%%%%%%%%%%%%%%%%%%%%%%%%%%%%%%%%%%%%%%%%%%%%%%%%%%%%%

\begin{doclist}
\docitem{ผลการประเมินการบรรลุผลลัพธ์การเรียนรู้ที่คาดหวังของหลักสูตร (PLOs) ของ\printprogram{}}
\docitem{ผลการประเมินการบรรลุผลลัพธ์การเรียนรู้ที่คาดหวังของหลักสูตร (PLOs) ของหลักสูตรวิทยาศาสตรบัณฑิต สาขาวิชาสถิติประยุกต์}
\end{doclist}












\subcriteria{Satisfaction level of the various stakeholders are shown to be established, monitored, and benchmarked for improvement.}

หลักสูตรรวบรวมข้อมูลย้อนกลับและการผลประเมินความพึงพอใจของผู้มีส่วนได้ส่วนเสียของหลักสูตร เพื่อนำมาพิจารณาวางแผนปรับปรุงกระบวนการพัฒนาบัณฑิตสำหรับปีการศึกษาต่อไป โดยมีเป้าหมายให้ผู้มีส่วนได้ส่วนเสียมีความพึงใจต่อหลักสูตรไม่น้อยกว่าระดับพึงพอใจมาก (คะแนนเฉลี่ยไม่น้อยกว่า 4.00) ซึ่งผู้มีส่วนได้ส่วนเสียของหลักสูตร ประกอบด้วย
\begin{enumerate}
	\item นักศึกษาทุกชั้นปี
	\item นักศึกษาชั้นปีสุดท้าย
	\item ผู้ใช้บัณฑิต
	\item อาจารย์ผู้รับผิดชอบหลักสูตร
\end{enumerate}
หลักสูตรเก็บรวบรวมข้อมูลผลการประเมินความพึงพอใจในแต่ละด้านดังนี้
\begin{enumerate}
	\item ประเมินความพึงพอใจของนักศึกษาทุกชั้นปีที่มีต่อหลักสูตร
	\item ประเมินความพึงพอใจของนักศึกษาชั้นปีสุดท้ายที่มีต่อคุณภาพหลักสูตร
	\item ประเมินความพึงพอใจของผู้ใช้บัณฑิตที่มีต่อบัณฑิตใหม่
	\item ประเมินความพึงพออาจารย์ของผู้รับผิดชอบหลักสูตรต่อการบริหารจัดการหลักสูตร
\end{enumerate}
โดยมีผลการประเมินดังตาราง \ref{Table:8.5-1}
\begin{longtable}{|>{\centering\arraybackslash}p{0.3\linewidth}|*{8}{c|}} 
\caption{ผลการประเมินความพึงพอใจหลักสูตรและคุณภาพบัณฑิตของผู้มีส่วนได้ส่วนเสีย ปีการศึกษา 2564-2567}	
	\label{Table:8.5-1}\\
	\hline
\multicolumn{1}{|>{\centering\arraybackslash}p{0.3\linewidth}|}{\textbf{การประเมินความพึงพอใจของผู้มีส่วนได้ส่วนเสีย}} &
	\multicolumn{4}{c|}{\textbf{วท.บ.(คณิตศาสตร์ประยุกต์)}} &
	\multicolumn{4}{c|}{\textbf{วท.บ.(สถิติประยุกต์)(คู่เทียบ)}} \\
	\cline{2-9}
	\multicolumn{1}{|c|}{} &
	\multicolumn{1}{c|}{2564} &
	\multicolumn{1}{c|}{2565} &
	\multicolumn{1}{c|}{2566} &
	\multicolumn{1}{c|}{2567} &
	\multicolumn{1}{c|}{2564} &
	\multicolumn{1}{c|}{2565} &
	\multicolumn{1}{c|}{2566} &
	\multicolumn{1}{c|}{2567} \\
	\hline
	\endhead
	
	นักศึกษาทุกชั้นปี & 4.28 & 4.29 & 4.62 & 4.69 & 4.75 & 4.77 & 4.70 & 4.73 \\
	\hline
	นักศึกษาชั้นปีสุดท้าย & 4.60 & 4.43 & 4.38 & 4.79 & 4.48 & 4.24 & 4.26 & 4.09 \\
	\hline
	ผู้ใช้บัณฑิต & 4.46 & 4.58 & 4.67 & - & 4.24 & 4.62 & 4.87 & - \\
	\hline
	อาจารย์ผู้รับผิดชอบหลักสูตร & 4.62 & 4.68 & 4.51 & 4.73 & 4.98 & 4.99 & 4.94 & 4.95 \\
	\hline

\end{longtable}




จากตาราง \ref{Table:8.5-1} ผลการประเมินความพึงพอใจของผู้มีส่วนได้ส่วนเสียเปรียบเทียบ 4 ปีย้อนหลัง พบว่าระดับความพึงพอใจอยู่ในระดับพึงพอใจมากถึงมากที่สุดในทุกด้าน และมีแนวโน้มสูงขึ้นในเกือบทุกกลุ่มของผู้มีส่วนได้ส่วนเสีย ซึ่งเป็นไปตามเป้าหมายที่หลักสูตรกำหนด แสดงให้เห็นว่าการบริหารและการดำเนินงานของหลักสูตรมีประสิทธิภาพ และบัณฑิตมีคุณภาพ

อย่างไรก็ตามเมื่อพิจารณาเทียบกับผลการดำเนินงานของหลักสูตรวิทยาศาสตรบัณฑิต สาขาวิชาสถิติประยุกต์พบว่า ระดับความพึงพอใจของผู้มีส่วนได้ส่วนเสียของหลักสูตรวิทยาศาสตรบัณฑิต สาขาวิชาสถิติประยุกต์ สูงกว่าในเกือบทุกกลุ่มของผู้มีส่วนได้ส่วนเสีย 

%หลักสูตรจึงได้ร่วมกันทบทวนและพิจารณาในรายประเด็นเพื่อปรับปรุงกระบวนการต่าง ๆ ให้ดียิ่งขึ้น ซึ่ง
%จากผลการวิเคราะห์แบบสอบถาพบว่านักศึกษามีข้อเสนอแนะ ดังนี้
%\begin{enumerate}
%	\item อยากให้หลักสูตรเสริมรายวิชาที่เกี่ยวข้องกับวิชาที่เกี่ยวข้องกับ AI หรือ ML
%    \item อยากให้หลักสูตรเน้นการปฎิบัติจริงให้มากขึ้น เช่น การเขียนโปรแกรมคอมพิวเตอร์
%\end{enumerate}
%และอาจารย์ผู้รับผิดชอบหลักสูตรมีข้อเสนอแนะ ดังนี้
%\begin{enumerate}
%	\item ควรพัฒนากระบวนการเตรียมความพร้อมอาจารย์ใหม่เพื่อรองรับการพิจารณาเสนอชื่อเป็นอาจารย์ประจำหลักสูตร 
%    \item ควรกำกับติดตามการเสนอขอตำแหน่งทางวิชาการของอาจารย์ประจำหลักสูตรเป็นรายบุคคลอย่าง
%     ใกล้ชิดยิ่งขึ้น
%      \item ควรเร่งการก้าวเข้าสู่การเสนอขอตำแหน่งทางวิชาการของอาจารย์ประจำหลักสูตรอย่างเร่งด่วน
%\end{enumerate}

จากการวิเคราะห์ผลการประเมินเป็นรายประเด็นและข้อเสนอแนะ รวมทั้งจากการศึกษาแนวทางการดำเนินงานของหลักสูตรวิทยาศาตรบันฑิต สาขาวิชาสถิติประยุกต์ อาจารย์ผู้รับผิดชอบหลักสูตรได้ร่วมกันทบทวนในแต่ละประเด็นเพื่อวางแผนปรับปรุงกระบวนการต่างๆ ให้ดีขึ้น ประกอบด้วยปรับปรุงกระบวนการจัดการเรียนการสอน การประเมินผล  การจัดโครงการส่งเสริมนักศึกษา และรวมรวมข้อมูลสำหรับใช้ในการปรับปรุงหลักสูตรต่อไป 



\begin{doclist}
\docitem{ผลการประเมินความพึงพอใจของนักศึกษาทุกชั้นปีที่มีต่อหลักสูตร}
\docitem{ผลการประเมินความพึงพอใจของนักศึกษาชั้นปีสุดท้ายที่มีต่อคุณภาพหลักสูตร}
\docitem{ผลการประเมินความพึงพอใจของผู้ใช้บัณฑิตที่มีต่อบัณฑิตใหม่}
\docitem{ผลการประเมินความพึงพอใจของอาจารย์ผู้รับผิดชอบหลักสูตรต่อการบริหารจัดการหลักสูตร}
\end{doclist}


