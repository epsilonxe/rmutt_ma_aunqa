\newpage
\criteria{Programme Structure and Content}

\subcriteria{The specifications of the programme and all its courses are shown to be comprehensive, up-to-date, and made available and communicated to all stakeholders.}


\printprogram{} ได้เริ่มรับนักศึกษา
ตั้งแต่ภาคการศึกษาที่ 1 ปีการศึกษา 2564 โดยใน การจัดทำข้อมูลหลักสูตรได้จัดทำตามข้อกำหนดของหลักสูตร
(Programme specification) ตามแบบฟอร์ม มคอ.2 ของสำนักงานคณะกรรมการการอุดมศึกษา (สกอ.) ซึ่งแบ่งเนื้อหา
ออกเป็น 8 หมวด ในแต่ละหมวดมีการระบุข้อมูลที่ครบถ้วนและสอดคล้องกับข้อแนะนำจาก Guide to AUN-QA Assesment at Programme Level Version 4.0 หน้า 20 ดังตาราง \ref{Table:M2AUN}

\begin{longtable}{|p{0.6\textwidth}|p{0.35\textwidth}|}
\caption{ความสอดคล้องระหว่าง มคอ.2 กับข้อมูลจาก Guide to AUN-QA Assesment at Programme Level Version 4.0 หน้า  20}
\label{Table:M2AUN}
\\
\hline
{\bf ข้อมูลจาก Guide to AUN-QA Assesment at\newline  Programme Level Version 4.0 หน้า  20}&\multicolumn{1}{c|}{\bf ข้อมูลใน มคอ.2}\\
\hline
\endfirsthead
\caption[]{ความสอดคล้องระหว่าง มคอ.2 กับข้อมูลจาก Guide to AUN-QA Assesment at Programme Level Version 4.0 หน้า  20 (ต่อ)}
\\
\hline
{\bf ข้อมูลจาก Guide to AUN-QA Assesment at\newline  Programme Level Version 4.0 หน้า  20}&\multicolumn{1}{c|}{\bf ข้อมูลใน มคอ.2}\\
\hline
\endhead
Awarding body/institution& หน้า 1 \\\hline
Teaching institution&หมวดที่ 1 ข้อที่ 10 (หน้า 4)\\\hline
Details of accreditation by professional or statutory bodies& หมวดที่ 1 ข้อที่ 6 (หน้า 2)\\\hline
Name of the final award&หมวดที่ 1 ข้อที่ 2 (หน้า 1)\\\hline
Programme title&หมวดที่ 1 ข้อที่ 1 (หน้า 1)\\\hline
Expected learning outcomes of the programme&หมวดที่ 2 ข้อที่ 1 (หน้า 7)\\\hline
Admission criteria or requirements&หมวดที่ 3 ข้อที่ 2 (หน้า 11)\\\hline
Relevant benchmark reports, external and internal reference points, that may
be used to provide information on programme learning outcomes&หมวดที่ 1 ข้อที่ 11,12 (หน้า 4-6)\\\hline
Programme structure and requirements including levels, courses, credits, etc&หมวดที่ 3 ข้อที่ 3 (หน้า 13-75)\\\hline
The date of writing the programme specifications.& หมวดที่ 1 (หน้า 2)\\\hline
\end{longtable}	

ทั้งนี้ข้อมูลหลักสูตรตาม มคอ.2 ได้มีการเผยแพร่ข้อมูลที่จำเป็นแก่ผู้ที่มีส่วนได้ส่วนเสียทุกกลุ่มในช่องทางต่างๆ ที่เข้าถึงได้ง่าย เช่น เว็บไซต์ของสำนักส่งเสริมวิชาการและงานทะเบียน มทร.ธัญบุรี
เว็บไซต์ของคณะ เว็บไซต์ของสาขาวิชาฯ Facebook ของหลักสูตร แผ่นพับประชาสัมพันธ์ และคู่มือนักศึกษา\printprogram{} รายละเอียดดังตาราง \ref{Table:M2-13}
\newpage
\begin{longtable}{|p{0.6\textwidth}|p{0.35\textwidth}|}
	\caption{การสื่อสาร The Programme specification กับผู้มีส่วนได้ส่วนเสีย}
	\label{Table:M2-13}
\\
\hline
\multicolumn{1}{|c|}{\bf ช่องทางการสื่อสาร}&\multicolumn{1}{c|}{\bf ผู้มีส่วนได้ส่วนเสีย}\\
\hline
\endhead
1) เว็บไซต์ของสำนักส่งเสริมวิชาการและงานทะเบียน มทร.ธัญบุรี\newline/คณะ/สาขา และเฟซบุ๊คของหลักสูตร& นักศึกษา ผู้ปกครอง อาจารย์\newline ผู้ที่สนใจสมัครเข้าศึกษาในหลักสูตร\\\hline
2) แผ่นพับ& นักศึกษา\newline 
ผู้ที่สนใจสมัครเข้าศึกษาในหลักสูตร\newline 
ฝ่ายแนะแนวของโรงเรียนระดับมัธยมศึกษา
\\\hline
3) คู่มือนักศึกษา& นักศึกษา\\\hline
\end{longtable}	
\noindent{\bf Crouse Spec :}\\
ในปีการศึกษา 2567 มีรายวิชาเปิดจำนวน 31  รายวิชา แบ่งเป็น
ภาคเรียนที่ 1 จำนวน 11 รายวิชา ภาคเรียนที่ 2 จำนวน 20 รายวิชา โดยอาจารย์ผู้รับผิดชอบรายวิชาทุกรายวิชามีการจัดทำ มคอ.3 ซึ่งมีรายละเอียดครบถ้วนตาม Guide to
AUN-QA Assesment at Programme Level Version 4.0 หน้า 20 ได้แก่
\begin{enumerate}
	\item Course title
	\item Course requirements such as pre-requisites, credits, etc
	\item Expected learning outcomes of the course in terms of knowledge, skills, and
	attitude
	\item Teaching, learning, and assessment methods that enable the expected learning outcomes to be achieved
	\item Course description, outline, or syllabus
	\item Details of student assessment
	\item Date on which the course specification was written or revised.
\end{enumerate}
และส่ง มคอ.3 ผ่านระบบของสำนักส่งเสริมวิชาการและงานทะเบียน ก่อนเปิดภาคการศึกษา

นอกจากนั้นทุกรายวิชามีการจัดทำ Course Syllabus ซึ่งมีรายละเอียดครบถ้วนตาม Guide to AUN-QA Assesment at Programme Level Version 4.0 หน้า 20 และมี Qr-Code มคอ.3 ของรายวิชาเพื่อสื่อสาร มคอ.3 ให้กับนักศึกษา โดยแจก Course Syllabus ให้กับนักศึกษาในคาบแรกของการจัดการเรียนการสอน

\begin{doclist}
\docitem{\printprogram{} }
\docitem{แผ่นพับประชาสัมพันธ์หลักสูตร}
\docitem{รายละเอียดของรายวิชา (มคอ.3) ของหลักสูตรที่เปิดสอนในปีการศึกษา \printyear}
\docitem{Course Syllabus}
\docitem{คู่มือนักศึกษา\printprogram}

\end{doclist}


\subcriteria{The design of the curriculum is shown to be constructively aligned with achieving the expected learning outcomes.}

\printprogram{} มีการออกแบบที่สอดคล้องตามหลักการ Outcome Base Education (OBE) และ Backward Curriculum Design (BCD) 
ทั้งนี้หลักสูตรได้มีการกระจายความรับผิดชอบผลลัพธ์การเรียนรู้ระดับหลักสูตร (PLOs) สู่รายวิชา แสดงดังตาราง \ref{table: plostocourses}  และมีการกำหนดผลลัพธ์การเรียนรู้ระดับรายวิชา (CLOs)  แสดงตัวอย่างดังตาราง \ref{table: clos}


\begin{longtable}{|>{\centering}p{0.12\textwidth}|>{\centering}p{0.12\textwidth}|>{\centering}p{0.12\textwidth}|>{\centering}p{0.12\textwidth}|>{\centering}p{0.12\textwidth}|>{\centering}p{0.12\textwidth}|>{\centering\arraybackslash}p{0.12\textwidth}|}
\caption{การกระจายความรับผิดชอบผลลัพธ์การเรียนรู้ระดับหลักสูตร (PLOs) สู่รายวิชา}
\label{table: plostocourses}
\\
\hline
\textbf{PLOs} &\multicolumn{2}{c|}{\textbf{Knowledge}}& \multicolumn{2}{c|}{\textbf{Skills}}&\multicolumn{2}{c|}{\textbf{Attitudes}}\\ \cline{2-7}

  &\textbf{Generic} & \textbf{Specific} & \textbf{Generic} & \textbf{Specific} &\textbf{Generic} & \textbf{Specific} 
\\ 
\hline
\endfirsthead

\caption{(ต่อ) การกระจายความรับผิดชอบผลลัพธ์การเรียนรู้ระดับหลักสูตร (PLOs) สู่รายวิชา  }
\\
\hline
\textbf{PLOs} &\multicolumn{2}{c|}{\textbf{Knowledge}}&\multicolumn{2}{c|}{\textbf{Skills}}&\multicolumn{2}{c|}{\textbf{Attitudes}}\\ \cline{2-7}
  &\textbf{Generic}& \textbf{Specific}&\textbf{Generic} & \textbf{Specific}& \textbf{Generic}& \textbf{Specific} 
\\ 
\hline
\endhead

\hline
\endfoot

PLO1& & & & & 
09-122-104 \newline 09-210-129 \newline 09-210-130 \newline 09-311-148 \newline 09-311-149 \newline 09-410-156
    & 
09-115-401 \newline 09-115-404
\newline 09-116-301         
\newline 09-116-304
\newline 09-116-305
\newline 09-116-307   
\newline 09-116-402
\newline 09-116-403
\newline 09-116-406 \\
\hline
PLO2& 09-090-016 \newline 09-122-104\newline 09-210-129\newline 09-210-130 \newline 09-311-148 \newline 09-311-149 \newline 09-410-155 \newline 09-410-156 
& 09-111-151 \newline 09-111-152 \newline 09-114-202\newline 09-111-253 \newline 09-111-257 \newline 09-113-114 \newline 09-113-201 \newline 09-113-202 \newline 09-113-305 \newline 09-113-306 \newline 09-114-204 \newline 09-114-205 \newline 09-114-222 \newline 09-114-223 \newline 09-114-335 \newline 09-115-401 \newline 09-115-404 \newline 09-116-301 \newline 09-116-304 \newline 09-116-305 \newline 09-116-307 \newline 09-116-402 \newline 09-116-403 \newline 09-116-406 & & & & \\ \hline
PLO3& 09-122-104 \newline 09-210-129 \newline 09-210-130 \newline 09-311-148 \newline  09-311-149 \newline 09-410-155 \newline 09-410-156 
	& 09-111-151 \newline 09-111-152 \newline 09-111-253 \newline 09-111-257 \newline 09-113-114 \newline 09-113-202 \newline 09-113-305 \newline 09-114-205 \newline 09-114-222 \newline 09-114-223 \newline 09-115-401 \newline 09-115-404 \newline 09-116-304 \newline 09-116-305 \newline 09-116-307 \newline 09-116-402 \newline 09-116-403 \newline 09-116-406 & & & & \\ \hline
PLO4& & 09-113-114 \newline 09-113-201 \newline 09-113-202 \newline 09-113-305 \newline 09-113-306 \newline 09-115-401 \newline 09-115-404& & & & \\ \hline
PLO5& 09-210-129 \newline 09-410-156 & 09-111-151 \newline 09-114-204 \newline 09-114-205 \newline 09-114-223 \newline 09-115-401 \newline 09-115-404 \newline 09-116-304 \newline 09-116-305 \newline 09-116-307 \newline 09-116-402 \newline 09-116-403 \newline 09-116-406& 09-410-156& 09-114-204 \newline 09-114-205 \newline 09-114-223 \newline 09-115-401 \newline 09-115-404 \newline 09-116-304 \newline 09-116-305 \newline 09-116-307 \newline 09-116-402 \newline 09-116-403 \newline 09-116-406& & \\ \hline
PLO6& & 09-115-404 \newline 09-116-402 \newline 09-116-403& & & & \\ \hline
PLO7& & & & & 09-210-129 \newline 09-210-130& 09-115-401 \newline 09-115-404 \newline 09-116-301 \newline 09-116-304 \newline 09-116-305 \newline 09-116-307 \newline 09-116-402 \newline 09-116-403 \newline 09-116-406 \\ \hline
PLO8& 09-122-104 \newline 09-210-129 \newline 09-210-130 \newline 09-410-156& 09-115-401 \newline 09-115-404 \newline 09-116-304 \newline 09-116-305 \newline 09-116-307 \newline 09-116-402 \newline 09-116-403& & & & \\ \hline
PLO9& 09-311-148 & 09-115-401 \newline  09-115-404 \newline 09-116-304 \newline 09-116-305 \newline 09-116-307 \newline 09-116-402 \newline 09-116-403 \newline 09-116-406&09-311-149 & 09-115-401 \newline 09-115-404 \newline 09-116-304 \newline 09-116-305 \newline 09-116-307 \newline 09-116-402 \newline 09-116-403 \newline 09-116-406& & \\ \hline
PLO10& & & 09-090-016 \newline 09-210-130 \newline 09-311-149 \newline 09-410-156& 09-114-202 \newline 09-114-204 \newline 09-114-205 \newline 09-114-222 \newline 09-114-223 \newline 09-114-335 \newline 09-116-304 \newline 09-116-305 \newline 09-116-307 \newline 09-116-402 \newline 09-116-403 \newline 09-116-406& & \\ 
\end{longtable}




%\begin{longtable}{|>{\centering}p{0.12\textwidth}|>{\centering}p{0.12\textwidth}|>{\centering}p{0.12\textwidth}|>{\centering}p{0.12\textwidth}|>{\centering}p{0.12\textwidth}|>{\centering}p{0.12\textwidth}|>{\centering\arraybackslash}p{0.12\textwidth}|}
%
%\caption{การกระจายความรับผิดชอบ PLO4 สู่รายวิชาซึ่งสะท้อนอยู่ใน CLOs ระดับรายวิชา}
%\label{table: clos}
%\\
%\hline
%\endfirsthead
%
%\caption{การกระจายความรับผิดชอบ CLOs ระดับรายวิชาใน\printprogram}
%\\
%\hline
%\endhead
%
%\hline
%\endfoot
%
%09-113-114 & 09-113-201 & 09-113-202 & 09-113-305 & 09-113-306 & 09-115-401 & 09-115-404\\ \hline
%CLO3 \\ CLO9 & CLO7 & CLO7 & CLO2\\CLO4\\CLO5\\CLO7\\CLO8\\CLO10\\CLO11\\CLO13\\CLO14 & CLO3\\CLO5 & CLO4 & CLO5\\ 
%\end{longtable}


\begin{longtable}{|>{\centering\raggedright}p{0.44\textwidth}|>{\centering\raggedright}p{0.28\textwidth}|>{\centering}p{0.05\textwidth}|>{\centering}p{0.05\textwidth}|>{\centering\arraybackslash}p{0.05\textwidth}|}
\caption{ความเชื่อมโยงระหว่างผลลัพธ์การเรียนรู้ระดับรายวิชา (CLOs) ของรายวิชา 09-113-202 พีชคณิตเชิงเส้น กับผลลัพธ์การเรียนรู้ระดับหลักสูตร (PLOs)}
\label{table: clos}
\\
\hline
\centering\textbf{คำอธิบายรายวิชา} & \centering\textbf{CLOs} & \multicolumn{3}{c|}{\textbf{PLOs}}\\ \cline{3-5}
& & 2 & 3 & 4 \\ \hline
\endfirsthead
\caption{(ต่อ) ความเชื่อมโยงระหว่างผลลัพธ์การเรียนรู้ระดับรายวิชา (CLOs) ของรายวิชา 09-113-202 พีชคณิตเชิงเส้น กับผลลัพธ์การเรียนรู้ระดับหลักสูตร (PLOs) }
\\
\hline
\textbf{คำอธิบายรายวิชา} & \textbf{CLOs} & \multicolumn{3}{c|}{\textbf{PLOs}}\\ \cline{3-5}
& & 2 & 3 & 4 \\ \hline
\endhead
\hline
\endfoot
\vspace{-0.4cm}
\multirow{3}{0.45\textwidth}{เมทริกซ์ การดำเนินการขั้นมูลฐาน ดีเทอร์มิแนนท์ ระบบสมการเชิงเส้น ปริภูมิเวกเตอร์ ปริภูมิย่อย การรวมเชิงเส้น
การแผ่ทั่ว อิสระเชิงเส้น ฐานหลักและมิติ การแปลงเชิงเส้น
และการประยุกต์
พิสัย ปริภูมิสู่ศูนย์ เมทริกซ์ของการแปลงเชิงเส้น ค่าเจาะจง
เวกเตอร์เจาะจง การทำให้เป็นเมทริกซ์แนวทแยง}&CLO1: อธิบายบทนิยามและทฤษฎีบทเกี่ยวกับเมทริกซ์ ตัวผกผันของเมทริกซ์ สมบัติพื้นฐานของเมทริกซ์ การดำเนินการเบื้องต้น
เมทริกซ์เป็นขั้นแบบแถว เมทริกซ์ลดรูปเป็นขั้นแบบแถวได้ & \checkmark& &\\ \cline{2-5}
& CLO2: คำนวณการดำเนินการบนเมทริกซ์ เมทริกซ์ผกผัน สมการเมทริกซ์ การดำเนินการขั้นมูลฐานได้& & \checkmark&  \\ \cline{2-5}
& CLO3: คำนวณค่าดีเทอร์มิแนนท์ของเมทริกซ์ได้& & \checkmark &   \\ \cline{2-5}
& CLO4: อธิบายสมบัติของดีเทอร์มิแนนท์ได้& \checkmark & &  \\ \cline{2-5}
& CLO5: คำนวณผลเฉลยของระบบสมการเชิงเส้นเอกพันธ์และไม่เอกพันธ์ได้& & \checkmark &  \\ \cline{2-5}
& CLO6: อธิบายบทนิยามและทฤษฎีบทของปริภูมิเวกเตอร์ ปริภูมิย่อย การรวมเชิงเส้น การแผ่ทั่วถึง ความเป็นอิสระเชิงเส้น ฐาน
หลักและมิติได้& \checkmark & &  \\ \cline{2-5}
& CLO7: พิสูจน์เกี่ยวกับปริภูมิเวกเตอร์ ปริภูมิย่อย การรวมเชิงเส้น การแผ่ทั่วถึง ความเป็นอิสระเชิงเส้น ฐานหลักและมิติได้& & & \checkmark  \\ \cline{2-5}
& CLO8: อธิบายบทนิยามและทฤษฎีบทเกี่ยวกับการแปลงเชิงเส้น พิสัย ปริภูมิสู่ศูนย์ เมทริกซ์ของการแปลงเชิงเส้น ค่าเจาะจง
เวกเตอร์เจาะจงและการทำให้เป็นเมทริกซ์แนวทแยงได้& \checkmark  & &  \\ \cline{2-5}
& CLO9: คำนวณพิสัย ปริภูมิสู่ศูนย์ เมทริกซ์ของการแปลงเชิงเส้น ค่าเจาะจง เวกเตอร์เจาะจง และการทำให้เป็นเมทริกซ์แนวทแยงได้ & & \checkmark & \\ \cline{2-5}
& CLO10: พิสูจน์เกี่ยวกับการแปลงเชิงเส้น พิสัย ปริภูมิสู่ศูนย์ เมทริกซ์ของการแปลงเชิงเส้น ค่าเจาะจง เวกเตอร์เจาะจงและการทำ
ให้เป็นเมทริกซ์แนวทแยงได้ & & & \checkmark \\ \end{longtable}

นอกจากนี้ทุกรายวิชามีการกำหนดวิธีการสอนและการประเมินผลเพื่อให้บรรลุผลลัพท์การเรียนรู้ระดับรายวิชา (CLOs) แสดงตัวอย่างดังตาราง \ref{table: c}

\begin{longtable}{|>{\centering}p{0.07\textwidth}|>{\centering}p{0.07\textwidth}|>{\centering\raggedright}p{0.29\textwidth}|>{\centering}p{0.15\textwidth}|>{\centering}p{0.15\textwidth}|>{\centering\arraybackslash}p{0.14\textwidth}|}
\caption{ตัวอย่างวิธีการสอนและการประเมินผลของรายวิชา 09-113-202 พีชคณิตเชิงเส้น}
\label{table: c}
\\
\hline
\textbf{สัปดาห์}&\textbf{CLOs}& \centering\textbf{หัวเรื่องที่สอน }& \textbf{กลยุทธ์/วิธีการสอน} & \textbf{กลยุทธ์/วิธีการประเมินผล} & \textbf{สัดส่วนการประเมิน (\%)}\\
\hline
\endfirsthead
\caption{ตัวอย่างวิธีการสอนและการประเมินผลของรายวิชา 09-113-202 พีชคณิตเชิงเส้น (ต่อ)}
\\
\hline
\textbf{สัปดาห์}&\textbf{CLO}& \textbf{หัวเรื่องที่สอน }& กลยุทธ์/วิธีการสอน & กลยุทธ์/วิธีการประเมินผล & สัดส่วนการประเมิน\\
\hline
\endhead
\endfoot
1 & 1 & บทที่ 1 เมทริกซ์ \newline
1.1 บทนิยามของเมทริกซ์ \newline
1.2 การดำเนินการบนเมทริกซ์ \newline
1.3 ตัวผกผันของเมทริกซ์& - การบรรยาย \newline - การอภิปราย& สอบข้อเขียน&  3\\ \cline{2-6}
 & 2 & บทที่ 1 เมทริกซ์ \newline
1.2 การดำเนินการบนเมทริกซ์ \newline
1.3 ตัวผกผันของเมทริกซ์& - การบรรยาย \newline - การอภิปราย& สอบข้อเขียน & 5\\ \hline
2& 1 & 1.4 สมบัติพื้นฐานของเมทริกซ์ \newline
1.5 การดำเนินการขั้นมูลฐานและเมทริกซ์มูลฐาน& - การบรรยาย \newline - การอภิปราย& สอบข้อเขียน&  5\\ \cline{2-6}
& 2 & 1.4 สมบัติพื้นฐานของเมทริกซ์ \newline
1.5 การดำเนินการขั้นมูลฐานและเมทริกซ์มูลฐาน& - การบรรยาย \newline - การอภิปราย& สอบข้อเขียน&  5 \\ \hline
\end{longtable}


%\begin{longtable}{|p{0.4\textwidth}|p{0.4\textwidth}|p{0.08\textwidth}|}
%\caption{การกำหนด CLOs ของตัวอย่างรายวิชาใน\printprogram}
%\label{table: closvsplos}
%\\
%\hline
%\endfirsthead
%
%\caption{การกำหนด CLOs ของรายวิชาใน\printprogram (ต่อ)}
%\\
%\hline
%\endhead
%
%\hline
%\endfoot
%
%\multicolumn{3}{|p{0.88\textwidth}|}{\bfseries 09-113-114 วิยุตคณิต} \\
%\hline
%\multirow{11}{0.4\textwidth}{
%\textbf{คำอธิบายรายวิชา}\par
%พื้นฐานทางตรรกศาสตร์และการพิสูจน์ อุปนัยทางคณิตศาสตร์ เซตและ
%ความสัมพันธ์ พื้นฐานการนับ พีชคณิตบูลีน ความน่าจะเป็นแบบไม่ต่อเนื่อง
%ความสัมพันธ์เวียนเกิด กราฟ ต้นไม้ เครื่องจักรแบบจำกัด ฟังก์ชันก่อกำเนิด
%}
%& CLO1 อธิบายความหมายของประพจน์ ตัวเชื่อมประพจน์ การสมมูลของประพจน์ สัจนิรันดร์ ประโยคเปิด ตัวบ่งปริมาณได้ & PLO2 \\ \cline{2-3}
%& CLO2 คำนวณค่าความจริงของประพจน์ได้ & PLO3 \\ \cline{2-3}
%& CLO3 พิสูจน์ประพจน์ที่กำหนดให้ตามหลักตรรกศาสตร์และหลักอุปนัยเชิงคณิตศาสตร์ได้ & PLO4 \\ \cline{2-3}
%& CLO4 อธิบายความหมายของเซต สมาชิกของเซต เซตว่าง เอกภพสัมพัทธ์ เซตย่อย การเท่ากันของเซต เซตกำลังและการดำเนินการบนเซตได้ & PLO2 \\ \cline{2-3}
%& CLO5 คำนวณเกี่ยวกับการดำเนินการบนเซตและเซตกำลังได้ & PLO3 \\ \cline{2-3}
%& CLO6 อธิบายบทนิยามของความสัมพันธ์และความสัมพันธ์สมมูลได้ & PLO2 \\ \cline{2-3}
%& CLO7 อธิบายกฎการบวก กฎการคูณ การเรียงสับเปลี่ยน การจัดหมู่และทฤษฎีบททวินามได้ & PLO2 \\ \cline{2-3}
%& CLO8 คำนวณการเรียงสับเปลี่ยน การจัดหมู่และทฤษฎีบททวินามได้ & PLO3 \\ \cline{2-3}
%& CLO9 พิสูจน์เกี่ยวกับพีชคณิตบูลีนโดยใช้กฎของพีชคณิตบูลีนได้ & PLO4 \\ \cline{2-3}
%& CLO10 คำนวณความน่าจะเป็นแบบไม่ต่อเนื่อง กราฟ ต้นไม้ เครื่องจักรแบบจำกัดได้ & PLO3 \\ \cline{2-3}
%& CLO11 คำนวณผลเฉลยของความสัมพันธ์เวียนเกิดและฟังก์ชันก่อกำเนิดได้ & PLO3 \\
%\hline
%
%\multicolumn{3}{|p{0.88\textwidth}|}{\bfseries 09-114-205 กำหนดการเชิงคณิตศาสตร์เบื้องต้น} \\
%\hline
%\multirow{11}{0.4\textwidth}{
%\textbf{คำอธิบายรายวิชา}\par
%ตัวแบบกำหนดการเชิงเส้น วิธีซิมเพลกซ์ ทฤษฎีภาวะคู่กัน การวิเคราะห์
%ความไว ตัวแบบกำหนดการไม่เชิงเส้น กำหนดการพลวัต กำหนดการเชิงเส้น
%จำนวนเต็ม กำหนดการเชิงเส้นทวิภาค กำหนดการเชิงเส้นแบบผสมจำนวนเต็ม
%การเขียนโปรแกรมในการหาผลเฉลยของตัวแบบกำหนดการเชิงคณิตศาสตร์
%เบื้องต้น และปฏิบัติการที่เกี่ยวข้อง
%}
%& CLO1 อธิบายแนวคิดของการทำแบบจำลองทางคณิตศาสตร์ และขั้นตอนการทำแบบจำลองได้ & PLO2 \\ \cline{2-3}
%& CLO2 จำแนกแบบจำลองทางคณิตศาสตร์ได้ & PLO2 \\ \cline{2-3}
%& CLO3 สร้างแบบจำลองทางคณิตศาสตร์ดีสครีตตัวแปรเดียว แบบจำลองทางคณิตศาสตร์ดีสครีตหลายตัวแปรเดียว และแบบจำลองทางคณิตศาสตร์ต่อเนื่องได้ & PLO2 \\ \cline{2-3}
%& CLO4 คำนวณผลเฉลยของแบบจำลองฯ & PLO3 \\ \cline{2-3}
%& CLO5 เขียนโปรแกรมคำนวณผลเฉลยของแบบจำลองฯ & PLO10 \\ \cline{2-3}
%& CLO6 ทดสอบแบบจำลองฯ & PLO3 \\ \cline{2-3}
%& CLO7 เขียนโปรแกรมทดสอบแบบจำลองฯ & PLO10 \\ \cline{2-3}
%& CLO8 คำนวณการประมาณค่าพารามิเตอร์ของแบบจำลองฯ & PLO3 \\ \cline{2-3}
%& CLO9 เขียนโปรแกรมคำนวณการประมาณค่าพารามิเตอร์ฯ & PLO10 \\ \cline{2-3}
%& CLO10 อธิบายตัวอย่างการใช้งานแบบจำลองที่สำคัญในยุคปัจจุบัน และปฏิบัติการที่เกี่ยวข้องได้ & PLO2 \\ \cline{2-3}
%& CLO11 สร้างตัวแบบจำลองทางคณิตศาสตร์ของปัญหาที่สนใจได้อย่างถูกต้องตามหลักวิชาการทางด้านคณิตศาสตร์ได้ & PLO5 \\
%\hline
%
%\multicolumn{3}{|p{0.88\textwidth}|}{\bfseries 09-115-404 โครงงานด้านคณิตศาสตร์ประยุกต์} \\
%\hline
%\multirow{8}{0.4\textwidth}{
%\textbf{คำอธิบายรายวิชา}\par
%เตรียมความพร้อมในการฝึกทำวิจัยในสาขาคณิตศาสตร์ประยุกต์ หรือสาขาที่
%เกี่ยวข้อง
%}
%& CLO1 ปฏิบัติตามกฎ ระเบียบ ข้อบังคับ ข้อตกลงของชั้นเรียน และจรรยาบรรณทางวิชาชีพของนักคณิตศาสตร์ได้ & PLO1 \\ \cline{2-3}
%& CLO2 ส่งงานที่ได้รับมอบหมายครบ และตรงตามเวลาที่กำหนด & PLO7 \\ \cline{2-3}
%& CLO3 อธิบายหลักการและขั้นตอนการทำโครงงานด้านคณิตศาสตร์ หรือสาขาที่เกี่ยวข้องได้ & PLO2 \\ \cline{2-3}
%& CLO4 อธิบายแนวคิด บทนิยาม หลักการ ทฤษฎีบทพื้นฐานและงานวิจัยที่เกี่ยวข้องกับหัวข้อโครงงานได้ & PLO2 \\ \cline{2-3}
%& CLO5 ทำโครงงานด้านคณิตศาสตร์ฯ เพื่อสร้างหรือปรับปรุงกระบวนการคิดทางคณิตศาสตร์ที่นำไปสู่องค์ความรู้ใหม่หรือนวัตกรรมได้ & PLO2, 3, 4, 5, 6, 8, 10 \\ \cline{2-3}
%& CLO6 เขียนโครงร่างและรายงานฉบับสมบูรณ์ของโครงงานได้อย่างถูกต้องตามหลักวิชาการทางคณิตศาสตร์ & PLO9 \\ \cline{2-3}
%& CLO7 ใช้ภาษาเพื่อการค้นคว้า ใช้เทคโนโลยีเพื่อการสืบค้นและเก็บรวมรวมข้อมูลและสามารถทำงานเป็นทีมได้ & PLO7, 9 \\ \cline{2-3}
%& CLO8 นำเสนอโครงงานได้อย่างถูกต้องตามหลักวิชาการทางด้านคณิตศาสตร์ & PLO9 \\
%\hline
%
%\end{longtable}
%



%\begin{longtable}{| p{0.15\textwidth}|>{\raggedright}p{0.4\textwidth}|>{\raggedright\arraybackslash}p{0.4\textwidth}| }
%\caption{การกำหนด CLOs ของตัวอย่างรายวิชาใน\printprogram}
%\label{table: closvsplos}
%\\
%\hline
%\textbf{รายวิชา} & \textbf{CLOs}& \textbf{คำอธิบายรายวิชา} \\
%\hline 
%\endfirsthead
%
%\caption{การกำหนด CLOs ของรายวิชาใน\printprogram}
%\\ \hline
%\textbf{รายวิชา} & \textbf{CLOs}& \textbf{คำอธิบายรายวิชา} \\
%\endhead
%\endfoot
%
%
% 09-113-114 \newline วิยุตคณิต
%& CLO3  พิสูจน์ประพจน์ที่กำหนดให้ตามหลักตรรกศาสตร์และหลักอุปนัยเชิงคณิตศาสตร์ได้ \newline CLO9   พิสูจน์เกี่ยวกับพีชคณิตบูลีนโดยใช้กฎของพีชคณิตบูลีนได้ & 
%\underline{พื้นฐานทางตรรกศาสตร์และการพิสูจน์} \underline{อุปนัยทางคณิตศาสตร์} เซตและความสัมพันธ์ พื้นฐานการนับ \underline{พีชคณิตบูลีน} ความน่าจะเป็นแบบไม่ต่อเนื่อง
%ความสัมพันธ์เวียนเกิด กราฟ ต้นไม้ เครื่องจักรแบบจำกัด ฟังก์ชันก่อกำเนิด
% \\  
%
%\hline
%
% 09-113-201 หลักคณิตศาสตร์
%& CLO7: ใช้ระเบียบวิธีการพิสูจน์ทางคณิตศาสตร์ การอุปนัยเชิงคณิตศาสตร์ ในการพิสูจน์ข้อความ หรือ
%ทฤษฎีบทพื้นฐานที่สำคัญเกี่ยวกับเซต ผลคูณคาร์ทีเชียน ความสัมพันธ์และฟังก์ชัน ทฤษฎีจำนวนเบื้องต้นได้ & 
%คณิตตรรกศาสตร์ \underline{ระเบียบวิธีการพิสูจน์ทาง} \underline{คณิตศาสตร์ การอุปนัยเชิงคณิตศาสตร์ เซต} \underline{ผลคูณคาร์ทีเชียน ความสัมพันธ์ ฟังก์ชัน} \underline{ทฤษฎีจำนวนเบื้องต้น}
%
% \\
% \hline
% 09-113-202 พีชคณิตเชิงเส้น
%& CLO7: พิสูจน์เกี่ยวกับปริภูมิเวกเตอร์ ปริภูมิย่อย การรวมเชิงเส้น การแผ่ทั่วถึง ความเป็นอิสระเชิงเส้น ฐานหลักและมิติได้ & 
%เมทริกซ์ การดำเนินการขั้นมูลฐาน ดีเทอร์มิแนนท์ ระบบสมการเชิงเส้น \underline{ปริภูมิเวกเตอร์} \underline{ปริภูมิย่อย การรวมเชิงเส้น
%การแผ่ทั่ว อิสระ} \underline{เชิงเส้น ฐานหลักและมิติ} การแปลงเชิงเส้น
%และการประยุกต์
%พิสัย ปริภูมิสู่ศูนย์ เมทริกซ์ของการแปลงเชิงเส้น ค่าเจาะจง
%เวกเตอร์เจาะจง การทำให้เป็นเมทริกซ์แนวทแยง
%
% \\ 
% \hline
% 09-113-305 การวิเคราะห์เชิงคณิตศาสตร์
%& CLO2: พิสูจน์ทฤษฎีบทพื้นฐานที่สำคัญเกี่ยวกับระบบจำนวนจริงได้ 
%\newline CLO4: พิสูจน์ทฤษฎีบทพื้นฐานที่สำคัญเกี่ยวกับลำดับและอนุกรมของจริงได้ 
%\newline CLO5: นำทฤษฎีบทเกี่ยวกับลำดับและอนุกรมของจำนวนจริงไปใช้ในการแก้ปัญหาเกี่ยวกับลำดับและอนุกรมของจำนวนจริงได้ 
%\newline CLO7: พิสูจน์ทฤษฎีบทพื้นฐานที่สำคัญเกี่ยวกับลิมิตและความต่อเนื่องของฟังก์ชันค่าจริงหนึ่งตัวแปรได้ 
%\newline CLO8: นำทฤษฎีบทเกี่ยวกับลิมิตและความต่อเนื่องของฟังก์ชันค่าจริงหนึ่งตัวแปรไปใช้ในการแก้ปัญหาเกี่ยวกับลิมิตและความต่อเนื่องของฟังก์ชันค่าจริงหนึ่งตัวแปรได้
%\newline CLO10: พิสูจน์ทฤษฎีบทพื้นฐานที่สำคัญเกี่ยวกับอนุพันธ์ของฟังก์ชันค่าจริงหนึ่งตัวแปรได้ 
%\newline CLO11: นำทฤษฎีบทเกี่ยวกับอนุพันธ์ของฟังก์ชันค่าจริงหนึ่งตัวแปรไปใช้ในการแก้ปัญหาเกี่ยวกับอนุพันธ์ของฟังก์ชันค่าจริงหนึ่งตัวแปรได้
%\newline CLO13: พิสูจน์ทฤษฎีบทพื้นฐานที่สำคัญเกี่ยวกับปริพันธ์แบบรีมันน์ของฟังก์ชันค่าจริงหนึ่งตัวแปรได้ 
%\newline CLO14: นำทฤษฎีบทเกี่ยวกับปริพันธ์แบบรีมันน์ของฟังก์ชันค่าจริงหนึ่งตัวแปรไปใช้ในการแก้ปัญหาเกี่ยวกับปริพันธ์แบบรีมันน์ของฟังก์ชันค่าจริงหนึ่งตัวแปรได้ 
%& ระบบจำนวนจริง ลำดับของจำนวนจริง ลิมิตและความต่อเนื่อง อนุพันธ์ของฟังก์ชัน ปริพันธ์แบบรีมันน์ ลำดับและอนุกรมของจำนวนจริง
%\\ \hline
%\end{longtable}
%
%
%\begin{longtable}{|>{\centering}p{0.12\textwidth}|>{\centering}p{0.07\textwidth}|>{\centering}p{0.20\textwidth}|>{\centering}p{0.18\textwidth}|>{\centering}p{0.14\textwidth}|>{\centering\arraybackslash}p{0.16\textwidth}|}
%\caption{วิธีการเรียนการสอนและการประเมินผลเพื่อให้บรรลุผลลัพธ์การเรียนรู้ระดับรายวิชา}
%\label{table: plostocourses}
%\\
%\hline
%\textbf{รายวิชา} & \textbf{CLOs} & \textbf{วิธีการสอน} & \textbf{วิธีประเมิน} & \textbf{เกณฑ์การผ่าน(ร้อยละ)}& \textbf{สัดส่วนของการประเมิน(ร้อยละ)}
%  \\ 
%\hline
%\endfirsthead
%
%\caption{(ต่อ) วิธีการเรียนการสอนและการประเมินผลเพื่อให้บรรลุผลลัพธ์การเรียนรู้ระดับรายวิชา}
%\\
%\hline
%\textbf{รายวิชา} & \textbf{CLOs} & \textbf{วิธีการสอน} & \textbf{วิธีประเมิน} & \textbf{เกณฑ์การผ่าน(ร้อยละ)}& \textbf{สัดส่วนของการประเมิน(ร้อยละ)}
%\\ 
%\hline
%\endhead
%\hline
%\endfoot
% 09-113-114 & CLO3 & บรรยาย/อภิปราย& สอบข้อเขียน& 40& 15
%\\ \cline{2-6}
% & CLO9 & บรรยาย/อภิปราย& สอบข้อเขียน& 40& 5
%\\  \hline
% 09-113-201& CLO7 & บรรยาย/อภิปราย/Active learning& สอบข้อเขียน& 40& 20 \\ \hline
% 09-113-202& CLO7 & บรรยาย/อภิปราย& สอบข้อเขียน& 50& 30 \\ 
% 09-113-305 & CLO2 & บรรยาย/อภิปราย/Active learning& สอบข้อเขียน& 40& 3 \\ \cline{2-6}
% & CLO4 & บรรยาย/อภิปราย/Active learning& สอบข้อเขียน& 30&6 \\ \cline{2-6}
% & CLO5 & บรรยาย/อภิปราย/Active learning& สอบข้อเขียน& 30&3 \\ \cline{2-6}
% & CLO7 & บรรยาย/อภิปราย/Active learning& สอบข้อเขียน& 30&3 \\ \cline{2-6}
% &CLO8 & บรรยาย/อภิปราย/Active learning& สอบข้อเขียน& 30&3 \\ \cline{2-6}
% &CLO10 & บรรยาย/อภิปราย/Active learning& สอบข้อเขียน& 30&3 \\ \cline{2-6}
% &CLO11 & บรรยาย/อภิปราย/Active learning& สอบข้อเขียน& 30&3 \\ \cline{2-6}
% &CLO13 & บรรยาย/อภิปราย/Active learning& สอบข้อเขียน& 30&3 \\ \cline{2-6}
% &CLO14 & บรรยาย/อภิปราย/Active learning& สอบข้อเขียน& 30&3 \\ 
%%09-113-306 & CLO3 & & สอบข้อเขียน& & 15
%%\\ \cline{2-6}
%% & CLO5 & & สอบข้อเขียน& & 5
%%\\  \hline
%%09-115-401& CLO4 & & สอบข้อเขียน& & 20 \\ \hline
%% 09-115-404& CLO5 & & สอบข้อเขียน& & 30 \\ \hline
%
%\end{longtable}
%
%
%
%


\begin{doclist}
\docitem{\printprogram{}}
\docitem{เอกสารปรับปรุง\printprogram{} (สมอ. 08)}
\docitem{รายละเอียดของรายวิชา (มคอ.3)}
\end{doclist}


\subcriteria{The design of the curriculum is shown to
include feedback from stakeholders, especially
external stakeholders.}

\printprogram{} มีกระบวนการในการรวบรวมและนำข้อมูลป้อนกลับ (Feedback) จากผู้มีส่วนได้ส่วนเสีย (Stakeholders) ทุกกลุ่มมาใช้ในการออกแบบและปรับปรุงหลักสูตร เพื่อให้มั่นใจว่าหลักสูตรมีความทันสมัย และตรงตามความต้องการของตลาดแรงงาน โดยกำหนดกลุ่มผู้มีส่วนได้ส่วนเสียที่สำคัญและช่องทางในการรับฟังความคิดเห็น ดังตาราง \ref{table: m2-sh} และนำข้อมูลมากำหนดผลลัพธ์การเรียนรู้ระดับหลักสูตร  (PLOs)  ดังตาราง \ref{table:stakeholder_summary}\\
\indent นอกจากนี้หลักสูตรมีการนำข้อเสนอแนะมาใช้ในการออกแบบหลักสูตร ซึ่งสะท้อนให้เห็นอย่างเป็น\\รูปธรรมในการพัฒนาหลักสูตรดังตัวอย่างต่อไปนี้
\begin{longtable}{|>{\raggedright}p{0.2\textwidth}| >{\raggedright}p{0.34\textwidth} | >{\raggedright\arraybackslash}p{0.34\textwidth}|}
\caption{ข้อเสนอแนะจากผู้มีส่วนได้ส่วนเสียในการออกแบบ\printprogram{}}
\label{table:}
\\
\hline
\multicolumn{1}{|c|}{\bf กลุ่มผู้มีส่วนได้ส่วนเสีย} & \multicolumn{1}{c|}{\bf ข้อเสนอแนะข้อมูลป้อนกลับที่สำคัญ} &  {\bf การดำเนินการในหลักสูตรฉบับปรับปรุง พ.ศ. 2564} \\
\hline
\endfirsthead
\caption{(ต่อ) ข้อเสนอแนะจากผู้มีส่วนได้ส่วนเสียในการออกแบบ\printprogram{} }
\\
\hline
\multicolumn{1}{|c|}{\bf กลุ่มผู้มีส่วนได้ส่วนเสีย} & \multicolumn{1}{c|}{\bf ข้อเสนอแนะข้อมูลป้อนกลับที่สำคัญ} &  {\bf การดำเนินการในหลักสูตรฉบับปรับปรุง พ.ศ. 2564} \\
\hline
\endhead

\hline
\endfoot
1. ผู้ใช้บัณฑิต/สถานประกอบการ & บัณฑิตควรมีทักษะด้านวิทยาการข้อมูล (Data Science) การเขียนโปรแกรมคอมพิวเตอร์ (โดยเฉพาะ Python, R) และการใช้ฐานข้อมูล (SQL) ซึ่ง มีความสำคัญอย่างยิ่งสำหรับการทำงานในปัจจุบัน & 
\textbf{1. ปรับคำอธิบายรายวิชาชีพบังคับ}
\begin{itemize}
	\item 09-114-204 การเขียนโปรแกรมคอมพิวเตอร์ทางคณิตศาสตร์ 
	\begin{enumerate}[label={-}]
		\item ระบุให้ใช้ภาษา Python
	\end{enumerate}
	\item 09-114-335 ระบบฐานข้อมูล
	\begin{enumerate}[label={-}]
		\item เพิ่มเนื้อหาเกี่ยวกับ SQL ในคำอธิบายรายวิชา 
	\end{enumerate}
\end{itemize}\\
&&
\textbf{2. เพิ่มรายวิชาต่อไปนี้ในกลุ่มวิชาชีพเลือก }
\begin{itemize}
	\item 09-114-319 โครงสร้างข้อมูลและอัลกอริทึม
	\item 09-114-336 รากฐานปัญญาประดิษฐ์
	\item 09-114-337 การเรียนรู้ของจักรกล
	\item 09-114-339 วิทยาการข้อมูลสำหรับนักคณิตศาสตร์
	\item 09-115-308 หัวข้อพิเศษของคอมพิวเตอร์สำหรับคณิตศาสตร์
\end{itemize}
\\ \hline
2. ศิษย์เก่า & ควรเสริมสร้างทักษะการนำเสนอและการสื่อสาร (Presentation \& Communication Skills) & เพิ่มรายวิชา 09-115-304 ทักษะการนำเสนอผลงานทางด้านคณิตศาสตร์ ในกลุ่มวิชาชีพเลือก \\ 
\end{longtable}


นอกจากนี้ในแต่ละปีการศึกษายังมีการนำข้อเสนอแนะจากผู้มีส่วนได้ส่วนเสียมาปรับปรุงหลักสูตรให้มีความทันสมัยอยู่เสมอ โดยภาคเรียนที่ 1 ปีการศึกษา 2567 มีการสำรวจความคิดเห็นจากสถานประกอบการณ์ที่นักศึกษาออกปฏิบัติสหกิจศึกษา โดยสถานประกอบการมีข้อเสนอแนะว่า  ควรส่งเสริมให้นักศึกษาเห็นแนวทางการประยุกต์ใช้ความรู้ทางด้านคณิตศาสตร์กับการทำงานจริง 

หลักสูตรได้นำข้อเสนอแนะดังกล่าวมาวางแผนการดำเนินการจัดกิจกรรมเสริมหลักสูตรในภาคเรียนที่ 2 ปีการศึกษา 2567 ดังนี้
\begin{enumerate}
	\item จัดกิจกรรมศึกษาดูงานสำหรับนักศึกษา ณ บริษัท บ้านปู จำกัด (มหาชน) เมื่อวันที่ 11 กุมภาพันธ์ 2568 
	\item เชิญวิทยากรและผู้เชี่ยวชาญด้านการจัดการศึกษา การเงินและธนาคาร การวิเคราะห์ข้อมูล และเทคโนโลยีดิจิทัล มาถ่ายทอดประสบการณ์การทำงานให้กับนักศึกษา เมื่อวันที่ 23 กุมภาพันธ์ 2568 ณ คณะวิทยาศาสตร์และเทคโนโลยี มหาวิทยาลัยเทคโนโลยีราชมงคลธัญบุรี
\end{enumerate}
\begin{doclist}
%\docitem{ผล}
\docitem{\printprogram{} }
\docitem{โครงการการพัฒนาทักษะกระบวนการคิดและการเรียนรู้ในการส่งเสริมความเป็นนวัตกรของนักศึกษาเพื่อเพิ่มสมรรถนะสู่การ \newline ประกอบอาชีพ}
%\docitem{เอกสารการปรับปรุงแก้ไข\printprogram{} (สมอ. 08) }
\end{doclist}

\subcriteria{The contribution made by each couse in achieving the expected learning outcomes is shown to be clear.}

หลักสูตรได้จัดทำแผนที่กระจายความรับผิดชอบผลลัพธ์การเรียนรู้ระดับหลักสูตร (PLOs) สู่รายวิชา (Curriculum Mapping) ดังตาราง \ref{table: Mapping}

	\begin{longtable}{|>{\raggedright}p{0.48\textwidth}|c|c|c|c|c|c|c|c|c|c|}
		\caption{แผนที่แสดงการกระจายความรับผิดชอบผลลัพธ์การเรียนรู้ระดับหลักสูตร (PLOs) สู่รายวิชา (Curriculum Mapping)}
		\label{table: Mapping}
		\\
		\hline
		\multicolumn{1}{|c|}{\textbf{รายวิชา}} & \multicolumn{10}{c|}{PLOs}\\
		\cline{2-11}
		&1&2&3&4&5&6&7&8&9&10\\
		\hline
		\endfirsthead
		
		\caption{(ต่อ) แผนที่แสดงการกระจายความรับผิดชอบผลลัพธ์การเรียนรู้ระดับหลักสูตร (PLOs) สู่รายวิชา (Curriculum Mapping) }
		\\
		\hline
		\multicolumn{1}{|c|}{\textbf{รายวิชา}}  & \multicolumn{10}{c|}{PLOs}\\
		\cline{2-11}
		&1&2&3&4&5&6&7&8&9&10\\
		\hline
		\endhead
		
	09-090-016 พื้นฐานการเขียนโปรแกรม&&{\Large$\bullet$}&&&&&&&&{\Large$\bullet$}\\
	\hline
	09-111-151 แคลคูลัส 1&&{\Large$\bullet$}&{\Large$\bullet$}&&{\Large$\bullet$}&&&&&\\
	\hline
	09-111-152 แคลคูลัส 2
	&&{\Large$\bullet$}&{\Large$\bullet$}&&&&&&&\\
	\hline
	09-114-202 ระบบคอมพิวเตอร์สำหรับงานพีชคณิต&&{\Large$\bullet$}&&&&&&&&{\Large$\bullet$}\\
	\hline
	09-122-104 สถิติสำหรับวิทยาศาสตร์&{\Large$\bullet$}&{\Large$\bullet$}&{\Large$\bullet$}&&&&{\Large$\bullet$}&{\Large$\bullet$}&{\Large$\bullet$}&{\Large$\bullet$}\\
	\hline
	09-210-129 เคมีพื้นฐาน&{\Large$\bullet$}&{\Large$\bullet$}&{\Large$\bullet$}&&&&{\Large$\bullet$}&{\Large$\bullet$}&{\Large$\bullet$}&\\
	\hline
	09-210-130 ปฏิบัติการเคมีพื้นฐาน&{\Large$\bullet$}&{\Large$\bullet$}&{\Large$\bullet$}&&&&{\Large$\bullet$}&{\Large$\bullet$}&{\Large$\bullet$}&{\Large$\bullet$}\\
	\hline
	09-311-148 หลักชีววิทยา&{\Large$\bullet$}&{\Large$\bullet$}&{\Large$\bullet$}&&&&{\Large$\bullet$}&&{\Large$\bullet$}&\\
	\hline
	09-311-149 ปฏิบัติการหลักชีววิทยา&{\Large$\bullet$}&{\Large$\bullet$}&{\Large$\bullet$}&&&&{\Large$\bullet$}&&{\Large$\bullet$}&{\Large$\bullet$}\\
	\hline
	09-410-155 ฟิสิกส์เบื้องต้น&&{\Large$\bullet$}&{\Large$\bullet$}&&{\Large$\bullet$}&&&&&\\
	\hline
	09-410-156 ปฏิบัติการฟิสิกส์เบื้องต้น&{\Large$\bullet$}&{\Large$\bullet$}&{\Large$\bullet$}&&{\Large$\bullet$}&&{\Large$\bullet$}&{\Large$\bullet$}&&{\Large$\bullet$}\\
	\hline
	09-111-253 แคลคูลัส 3&&{\Large$\bullet$}&{\Large$\bullet$}&&&&&&&\\
	\hline
	09-111-257 สมการเชิงอนุพันธ์สามัญ&&{\Large$\bullet$}&{\Large$\bullet$}&&&&&&&\\
	\hline
	09-113-114 วิยุตคณิต&&{\Large$\bullet$}&{\Large$\bullet$}&{\Large$\bullet$}&&&&&&\\
	\hline
	09-113-201 หลักคณิตศาสตร์&&{\Large$\bullet$}&&{\Large$\bullet$}&&&&&&\\
	\hline
	09-113-202 พีชคณิตเชิงเส้น&&{\Large$\bullet$}&{\Large$\bullet$}&{\Large$\bullet$}&&&&&&\\
	\hline
	09-113-305 การวิเคราะห์เชิงคณิตศาสตร์&&{\Large$\bullet$}&{\Large$\bullet$}&{\Large$\bullet$}&&&&&&\\
	\hline
	09-113-306 พีชคณิตนามธรรม&&{\Large$\bullet$}&&{\Large$\bullet$}&&&&&&\\
	\hline
	09-114-204 การเขียนโปรแกรมคอมพิวเตอร์ทาง
	คณิตศาสตร์&&{\Large$\bullet$}&&&{\Large$\bullet$}&&&&&{\Large$\bullet$}\\
	\hline
	09-114-205 กำหนดการเชิงคณิตศาสตร์เบื้องต้น&&{\Large$\bullet$}&{\Large$\bullet$}&&{\Large$\bullet$}&&&&&{\Large$\bullet$}\\
	\hline
	09-114-222 ระเบียบวิธีเชิงตัวเลขเบื้องต้น&&{\Large$\bullet$}&{\Large$\bullet$}&&&&&&&{\Large$\bullet$}\\
	\hline
	09-114-223 การสร้างแบบจำลองทางคณิตศาสตร์เบื้องต้น&&{\Large$\bullet$}&{\Large$\bullet$}&&{\Large$\bullet$}&&&&&{\Large$\bullet$}\\
	\hline
	09-114-335 ระบบฐานข้อมูล&&{\Large$\bullet$}&&&&&&&&{\Large$\bullet$}\\
	\hline
	09-115-401 สัมมนาทางคณิตศาสตร์ประยุกต์&{\Large$\bullet$}&{\Large$\bullet$}&{\Large$\bullet$}&&&&{\Large$\bullet$}&{\Large$\bullet$}&{\Large$\bullet$}&\\
	\hline
	09-115-404 โครงงานด้านคณิตศาสตร์ประยุกต์&{\Large$\bullet$}&{\Large$\bullet$}&{\Large$\bullet$}&{\Large$\bullet$}&{\Large$\bullet$}&{\Large$\bullet$}&{\Large$\bullet$}&{\Large$\bullet$}&{\Large$\bullet$}&{\Large$\bullet$}\\
	\hline
	09-116-301 การเตรียมความพร้อมฝึกประสบการณ์วิชาชีพทางคณิตศาสตร์ประยุกต์ &{\Large$\bullet$}&{\Large$\bullet$}& & & & & {\Large$\bullet$} & & & \\
	\hline
	09-116-304 ฝึกงานทางคณิตศาสตร์ประยุกต์ &{\Large$\bullet$}&{\Large$\bullet$}&{\Large$\bullet$}& &{\Large$\bullet$}& &{\Large$\bullet$}&{\Large$\bullet$}&{\Large$\bullet$}&{\Large$\bullet$}\\
	\hline
    09-116-305 ฝึกงานต่างประเทศทางคณิตศาสตร์ประยุกต์ &{\Large$\bullet$}&{\Large$\bullet$}&{\Large$\bullet$}& &{\Large$\bullet$}& &{\Large$\bullet$}&{\Large$\bullet$}&{\Large$\bullet$}&{\Large$\bullet$}\\
	\hline
	09-116-307 ฝึกงานเฉพาะตำแหน่งทางคณิตศาสตร์ประยุกต์ &{\Large$\bullet$}&{\Large$\bullet$}&{\Large$\bullet$}& &{\Large$\bullet$}& &{\Large$\bullet$}&{\Large$\bullet$}&{\Large$\bullet$}&{\Large$\bullet$}\\
	\hline
	09-116-402 สหกิจศึกษาทางคณิตศาสตร์ประยุกต์ &{\Large$\bullet$}&{\Large$\bullet$}&{\Large$\bullet$}& &{\Large$\bullet$}&{\Large$\bullet$}&{\Large$\bullet$}&{\Large$\bullet$}&{\Large$\bullet$}&{\Large$\bullet$}\\
	\hline
	09-116-403 สหกิจศึกษาต่างประเทศทางคณิตศาสตร์ประยุกต์ &{\Large$\bullet$}&{\Large$\bullet$}&{\Large$\bullet$}& &{\Large$\bullet$}&{\Large$\bullet$}&{\Large$\bullet$}&{\Large$\bullet$}&{\Large$\bullet$}&{\Large$\bullet$}\\
	\hline
	09-116-406 ปัญหาพิเศษจากสถานประกอบการทางคณิตศาสตร์ประยุกต์ &{\Large$\bullet$}&{\Large$\bullet$}&{\Large$\bullet$}& &{\Large$\bullet$}& &{\Large$\bullet$}& &{\Large$\bullet$}&{\Large$\bullet$}\\
	\hline
		\end{longtable}
	
\begin{doclist}
\docitem{เอกสารการปรับปรุงแก้ไข\printprogram{} (สมอ. 08) }
\end{doclist}

\subcriteria{The curriculum to show that all its courses are logically structured, properly sequenced (progression from basic to intermediate to specialized courses) and are integrated.}

\printprogram{} ได้รับการออกแบบให้มีการจัดโครงสร้างหลักสูตรที่มีการจัดลำดับรายวิชาอย่างเป็นระบบและเหมาะสม โดยคำนึงถึงรายวิชาเรียนก่อน-หลัง  เรียนจากรายวิชาระดับพื้นฐานไปสู่รายวิชาระดับสูง และมีการบูรณาการเนื้อหาวิชาในแต่ละปีการศึกษา โดยมีโครงสร้างหลักสูตรดังตารางต่อไปนี้
\begin{longtable}{|>{\raggedright}p{0.6\textwidth}|c|}
	\caption{โครงสร้างหลักสูตร}
	\\
	\hline
%	\endfirsthead
	\multicolumn{1}{|c|}{{\bf หมวดวิชา}}&{\bf จำนวนหน่วยกิต}\\
	\hline
	{\bf หมวดวิชาเฉพาะ}&\textbf{94}\\
	- กลุ่มวิชาพื้นฐานวิชาชีพ&27\\
	- กลุ่มวิชาชีพบังคับ&40\\
	- กลุ่มวิชาชีพเลือก&27\\
	{\bf หมวดวิชาเสริมสร้างประสบการณ์ในวิชาชีพ}&\textbf{7}\\
	\hline
\end{longtable}
จากโครงสร้างหลักสูตรข้างต้น หลักสูตรฯได้นำมาออกแบบการจัดเรียงลำดับรายวิชาเป็นแผนการศึกษาในแต่ละภาคการศึกษา ดังตาราง \ref{table:Planyear} 
%>{\raggedright}p{0.6\textwidth}

\begin{longtable}{|c|c|>{\raggedright}p{0.52\textwidth}|c|}
	\caption{แผนการศึกษา}
	\label{table:Planyear}
	\\
	\hline
	{\bf ชั้นปี}&{\bf ภาคเรียน}&\multicolumn{1}{c|}{{\bf รายวิชา}}&{\bf จำนวนหน่วยกิต}\\
	\hline
	\endfirsthead
	
	\caption{(ต่อ) แผนการศึกษา}
	\\
	\hline
	{\bf ชั้นปี}&{\bf ภาคเรียน}&\multicolumn{1}{c|}{{\bf รายวิชา}}&{\bf จำนวนหน่วยกิต}\\
	\hline
	\endhead
	
	\hline
	\endfoot
	
	
	1&1&09-090-016	พื้นฐานการเขียนโปรแกรม	&	3\\
	&&09-111-151	แคลคูลัส 1			&3\\
	&&09-210-129	เคมีพื้นฐาน			&3\\
	&&09-210-130	ปฏิบัติการเคมีพื้นฐาน	&1\\
	&&09-122-104	สถิติสำหรับวิทยาศาสตร์	&3\\
	\cline{2-4}
	&2&09-111-152	แคลคูลัส 2	&3\\
	&&09-113-114	วิยุตคณิต	&3\\
	&&09-114-202	ระบบคอมพิวเตอร์สำหรับงานพีชคณิต	&3\\
	&&09-311-148	หลักชีววิทยา	&3\\
	&&09-311-149	ปฏิบัติการหลักชีววิทยา	&1\\
	\hline
	2&1&09-111-253	แคลคูลัส 3	&3\\
	&&09-113-201	หลักคณิตศาสตร์	&3\\
	&&09-410-155	ฟิสิกส์เบื้องต้น	&3\\
	&&09-410-156	ปฏิบัติการฟิสิกส์เบื้องต้น	&1\\
	\cline{2-4}
	&2&09-111-257	สมการเชิงอนุพันธ์สามัญ&3\\
	&&09-113-202	พีชคณิตเชิงเส้น&3\\
	&&09-114-204	การเขียนโปรแกรมคอมพิวเตอร์\newline ทางคณิตศาสตร์&3\\
	&&09-114-223	การสร้างแบบจำลองทางคณิตศาสตร์เบื้องต้น&3\\
	&&09-114-335	ระบบฐานข้อมูล&3\\
	\hline
	3&1&09-113-305	การวิเคราะห์เชิงคณิตศาสตร์&3\\
	&&09-114-205	กำหนดการเชิงคณิตศาสตร์เบื้องต้น&3\\
	&&09-114-222	ระเบียบวิธีเชิงตัวเลขเบื้องต้น&3\\
	&&09-xxx-xxx	เลือกจากรายวิชาชีพเลือก&3\\
	&&09-xxx-xxx	เลือกจากรายวิชาชีพเลือก&3\\
	&&09-xxx-xxx	เลือกจากรายวิชาชีพเลือก&3\\
	\cline{2-4}
	&2&09-113-306	พีชคณิตนามธรรม&3\\
	&&09-116-301	การเตรียมความพร้อมฝึกประสบการณ์วิชาชีพทางคณิตศาสตร์ประยุกต์&1\\
	&&09-xxx-xxx	เลือกจากรายวิชาชีพเลือก&3\\
	&&09-xxx-xxx	เลือกจากรายวิชาชีพเลือก&3\\
	&&09-xxx-xxx	เลือกจากรายวิชาชีพเลือก&3\\
	&&09-xxx-xxx	เลือกจากรายวิชาชีพเลือก&3\\
	\hline
	4&1&09-116-402	สหกิจศึกษาทางคณิตศาสตร์ประยุกต์\newline
	หรือ
	\newline
	09-116-403	สหกิจศึกษาต่างประเทศทางคณิตศาสตร์ประยุกต์&6\\
	\cline{2-4}
	&2&09-115-401	สัมมนาทางคณิตศาสตร์ประยุกต์&1\\
	&&09-115-404	โครงงานด้านคณิตศาสตร์ประยุกต์&3\\
	&&09-xxx-xxx	เลือกจากรายวิชาชีพเลือก&3\\
	&&09-xxx-xxx	เลือกจากรายวิชาชีพเลือก&3\\

\end{longtable}

\newpage

\begin{longtable}{| >{\centering}p{0.05\textwidth}|>{\raggedright}p{0.51\textwidth}|>{\centering}p{0.19\textwidth}|>{\centering\arraybackslash}p{0.1\textwidth}|}

\caption{รายวิชาชีพเลือกใน\printprogram}
\label{table: selective_courses}
\\
\hline
\textbf{ชั้นปี} & \multicolumn{1}{c|}{\bfseries รายวิชา} & \textbf{รายวิชาบังคับก่อน}&\textbf{จำนวนหน่วยกิต}  \\
 \hline
 \endfirsthead
 
 \caption{รายวิชาชีพเลือกใน\printprogram}
\\
\hline
\textbf{ชั้นปี} & \multicolumn{1}{c|}{\bfseries รายวิชา} & \textbf{รายวิชาบังคับก่อน}&\textbf{จำนวนหน่วยกิต}  \\
 \hline
 \endhead
 
 \hline
 \endfoot
   
   
\centering 2&09-114-206 ทฤษฎีกราฟและการประยุกต์ &- &3 \\ \cline{2-4}
&09-113-203 ทฤษฎีจำนวนและการประยุกต์ &09-113-201 &3 \\ \hline
\centering 3
&09-111-338 สมการเชิงอนุพันธ์ย่อย &09-111-257 &3\\ \cline{2-4}
&09-114-316 คณิตศาสตร์ประกันภัย &- &3\\ \cline{2-4}
&09-114-318 คณิตศาสตร์การเงิน &- &3\\ \cline{2-4}
&09-114-324 คณิตศาสตร์การลงทุน &09-114-318 &3\\ \cline{2-4}
&09-114-325 ระบบพลวัต &09-111-257\\ 09-114-223 &3\\ \cline{2-4}
&09-114-326 ระเบียบวิธีการประมาณค่าตามเส้น &09-114-223 &3\\ \cline{2-4} 
&09-114-327 การตัดสินใจอย่างชาญฉลาดด้วยกำหนดการเชิงคณิตศาสตร์ &09-114-205 &3\\ \cline{2-4} 
&9-114-328 แบบจำลองทางคณิตศาสตร์ด้านชีววิทยา&09-114-325 &3 \\ \cline{2-4} 
&09-114-329 แบบจำลองทางคณิตศาสตร์ด้านระบาดวิทยา &09-114-325 &3 \\ \cline{2-4} 
&09-114-330 ระเบียบวิธีเชิงตัวเลขสำหรับระบบพลวัต &09-114-222 &3 \\ \cline{2-4} 
&09-114-331 เทคนิคการหาค่าเหมาะสม &09-114-222 &3 \\ \cline{2-4} 
&9-114-332 ระเบียบวิธีไฟไนต์เอลิเมนต์ &09-114-330 &3 \\ \cline{2-4} 
&09-114-333 วิทยาการเข้ารหัสลับเบื้องต้น &09-113-203 &3 \\ \cline{2-4} 
&09-115-304\;ทักษะการนำเสนอผลงานทางด้านคณิตศาสตร์ &09-114-334 &3 \\ \cline{2-4} 
&09-115-307 หัวข้อพิเศษของการคำนวณเชิงคณิตศาสตร์ &- &3 \\ \cline{2-4} 
&09-114-319 โครงสร้างข้อมูลและอัลกอริทึม &09-114-204 &3 \\ \cline{2-4} 
&09-114-334 ระบบการจัดเตรียมเอกสารอย่างมืออาชีพ &- &3 \\ \cline{2-4} 
&09-114-336 รากฐานปัญญาประดิษฐ์ &09-114-204 &3 \\ \cline{2-4} 
&09-114-337 การเรียนรู้ของจักรกล &09-114-204 &3 \\ \cline{2-4} 
&09-114-338 การพัฒนาเว็บไซต์สมัยใหม่ &- &3 \\ \cline{2-4} 
&09-114-339 วิทยาการข้อมูลสำหรับนักคณิตศาสตร์ &09-114-204 &3 \\ \cline{2-4} 
&09-115-308\;หัวข้อพิเศษของคอมพิวเตอร์สำหรับคณิตศาสตร์ &- &3 \\ \hline
4 &09-115-409 หัวข้อพิเศษของแบบจำลองทางคณิตศาสตร์&- &3 \\ 

\end{longtable}

\begin{doclist}
\docitem{\printprogram{} }
\docitem{การปรับปรุงแก้ไข\printprogram{} (สมอ. 08) }
\end{doclist}




\subcriteria{The curriculum to have option(s) for students to pursue major and/or minor specialisations.}

\printprogram{} ได้รับการออกแบบให้มีความยืดหยุ่นและมีทางเลือกให้นักศึกษาในการเลือกเรียนวิชาตามความต้องการ โดยนักศึกษาจะต้องเลือกเรียนกลุ่มวิชาชีพเลือก 27 หน่วยกิต โดยเลือกศึกษาจากกลุ่มวิชาต่อไปนี้ทุกกลุ่ม กลุ่มละไม่น้อยกว่า 6 หน่วยกิต

\subsection*{กลุ่มวิชาแบบจำลองทางคณิตศาสตร์ (Mathematical Modeling Courses)}

นักศึกษาสามารถเลือกเรียนรายวิชาในกลุ่มวิชาแบบจำลองทางคณิตศาสตร์ เพื่อพัฒนาทักษะในการสร้างและวิเคราะห์แบบจำลองทางคณิตศาสตร์สำหรับการประยุกต์ใช้ในด้านต่าง ๆ ตัวอย่างรายวิชาในกลุ่มนี้ได้แก่:
\begin{itemize}
    \item 09-111-338 สมการเชิงอนุพันธ์ย่อย (Partial Differential Equations)
    \item 09-114-206 ทฤษฎีกราฟและการประยุกต์ (Graph Theory and Applications)
    \item 09-114-316 คณิตศาสตร์ประกันภัย (Mathematics of Insurance)
    \item 09-114-318 คณิตศาสตร์การเงิน (Mathematics of Finance)
    \item 09-114-324 คณิตศาสตร์การลงทุน (Mathematics of Investment)
    \item 09-114-325 ระบบพลวัต (Dynamical Systems)
    \item 09-114-326 ระเบียบวิธีการประมาณค่าตามเส้น (Curve Fitting Methods)
    \item 09-114-327 การตัดสินใจอย่างชาญฉลาดด้วยกำหนดการเชิงคณิตศาสตร์ (Intelligence Decision Making with Mathematical Programming)
    \item 09-114-328 แบบจำลองทางคณิตศาสตร์ด้านชีววิทยา (Mathematical Modeling in Biology)
    \item 09-114-329 แบบจำลองทางคณิตศาสตร์ด้านระบาดวิทยา (Mathematical Modeling in Epidemiology)
    \item 09-115-409 หัวข้อพิเศษของแบบจำลองทางคณิตศาสตร์ (Special Topics in Mathematical Modeling)
\end{itemize}

\subsection*{กลุ่มวิชาเทคโนโลยีทางคณิตศาสตร์ (Mathematical Technology Courses)}

นักศึกษาสามารถเลือกเรียนรายวิชาในกลุ่มวิชาเทคโนโลยีทางคณิตศาสตร์ เพื่อเพิ่มพูนความรู้และทักษะในการใช้เทคโนโลยีและเครื่องมือทางคณิตศาสตร์ในการแก้ปัญหาต่าง ๆ ตัวอย่างรายวิชาในกลุ่มนี้ได้แก่:
\begin{itemize}
    \item 09-113-203 ทฤษฎีจำนวนและการประยุกต์ (Number Theory and Applications)
    \item 09-114-330 ระเบียบวิธีเชิงตัวเลขสำหรับระบบพลวัต (Numerical Methods for Dynamical Systems)
    \item 09-114-331 เทคนิคการหาค่าเหมาะสม (Optimization Techniques)
    \item 09-114-332 ระเบียบวิธีไฟไนต์เอลิเมนต์ (Finite Elements Methods)
    \item 09-114-333 วิทยาการเข้ารหัสลับเบื้องต้น (Introduction to Cryptography)
    \item 09-115-304 ทักษะการนำเสนอผลงานทางด้านคณิตศาสตร์ (Presentation Skills in Mathematics)
    \item 09-115-307 หัวข้อพิเศษของการคำนวณเชิงคณิตศาสตร์ (Special Topics in Computational Mathematics)
\end{itemize}

\subsection*{กลุ่มวิชาคอมพิวเตอร์สำหรับนักคณิตศาสตร์ (Computer Courses for Mathematicians)}

นักศึกษาสามารถเลือกเรียนรายวิชาในกลุ่มวิชาคอมพิวเตอร์สำหรับนักคณิตศาสตร์ เพื่อเพิ่มพูนทักษะการใช้คอมพิวเตอร์ในการคำนวณและการประยุกต์ใช้ทางคณิตศาสตร์ ตัวอย่างรายวิชาในกลุ่มนี้ได้แก่:
\begin{itemize}
    \item 09-114-319 โครงสร้างข้อมูลและอัลกอริทึม (Data Structures and Algorithms)
    \item 09-114-334 ระบบการจัดเตรียมเอกสารอย่างมืออาชีพ (Professional Document Preparation System)
    \item 09-114-336 รากฐานปัญญาประดิษฐ์ (Foundation in Artificial Intelligence)
    \item 09-114-337 การเรียนรู้ของจักรกล (Machine Learning)
    \item 09-114-338 การพัฒนาเว็บไซต์สมัยใหม่ (Modern Website Development)
    \item 09-114-339 วิทยาการข้อมูลสำหรับนักคณิตศาสตร์ (Data Sciences for Mathematicians)
    \item 09-115-308 หัวข้อพิเศษของคอมพิวเตอร์สำหรับคณิตศาสตร์ (Special Topics in Computer for Mathematics)
\end{itemize}

โดยในปีการศึกษา \printyear{}  นักศึกษาชั้นปีที่ 3 ได้เลือกรายวิชาชีพเลือก แบ่งออกเป็น 2 แนวทางดังนี้
\begin{enumerate}
	\item[กลุ่มที่ 1:] นักศึกษาเลือกเรียนรายวิชา
\begin{itemize}
	\item 09-114-325 ระบบพลวัต (Dynamical Systems)
	\item 09-114-330 ระเบียบวิธีเชิงตัวเลขสำหรับระบบพลวัต (Numerical Methods for Dynamical Systems)
	\item 09-114-331 เทคนิคการหาค่าเหมาะสม (Optimization Techniques)
	\item 09-114-339 วิทยาการข้อมูลสำหรับนักคณิตศาสตร์ (Data Sciences for Mathematicians)
\end{itemize}
\item[กลุ่มที่ 2:] นักศึกษาเลือกเรียนรายวิชา
\begin{itemize}
% \item 09-113-203 ทฤษฎีจำนวนและการประยุกต์ (Number Theory and Applications)
% \item 09-114-206 ทฤษฎีกราฟและการประยุกต์ (Graph Theory and Applications)
  \item 09-114-316 คณิตศาสตร์ประกันภัย (Mathematics of Insurance)
 \item 09-114-324 คณิตศาสตร์การลงทุน (Mathematics of Investment)
\item 09-114-331 เทคนิคการหาค่าเหมาะสม (Optimization Techniques)
	\item 09-114-339 วิทยาการข้อมูลสำหรับนักคณิตศาสตร์ (Data Sciences for Mathematicians)
\end{itemize}
%\item[กลุ่มที่ 2:] นักศึกษาเลือกเรียนรายวิชา
%\begin{itemize}
%	\item 09-114-316 คณิตศาสตร์ประกันภัย (Mathematics of Insurance)
%	\item 09-114-331 เทคนิคการหาค่าเหมาะสม (Optimization Techniques)
%	\item 09-114-324 คณิตศาสตร์การลงทุน (Mathematics of Investment)
%	\item 09-114-339 วิทยาการข้อมูลสำหรับนักคณิตศาสตร์ (Data Sciences for Mathematicians)
%\end{itemize}
\end{enumerate}

\begin{doclist}
\docitem{แบบเปิดรายวิชาของนักศึกษาชั้นปีที่ 3 (AM65111) ปีการศึกษา \printyear }
\end{doclist}


\subcriteria{The programme to show that its curriculum is reviewed periodically following an established procedure and that it remains up-to-date and relevant to industry.}
อาจารย์ผู้รับผิดชอบหลักสูตรได้ดำเนินการทบทวนและปรับปรุง\printprogram{} ตามรอบระยะเวลา 5 ปี โดยครบรอบปรับปรุงในปีการศึกษา 2568 เพื่อเปิดรับนักศึกษาในปีการศึกษา 2569 ซึ่งมีขั้นตอนและการดำเนินการตามแนวทางของการพัฒนาหลักสูตรแบบ Outcome-based Education (OBE) \\
%\indent จึงได้ดำเนินการปรับปรุงหลักสูตรโดยมีเป้าหมายเพื่อยกระดับคุณภาพของบัณฑิตให้สอดคล้องกับความต้องการของตลาดแรงงานในปัจจุบัน โดยเฉพาะในด้านการประยุกต์ใช้คณิตศาสตร์ในภาคธุรกิจและอุตสาหกรรม และการบูรณาการทักษะด้านนวัตกรรมเข้ากับกระบวนการคิดเชิงคณิตศาสตร์
%
\indent ในปีการศึกษา 2567 อาจารย์ผู้รับผิดชอบหลักสูตรได้จัดโครงการพัฒนาหลักสูตรวิทยาศาสตรบัณฑิต สาขาวิชาคณิตศาสตร์ประยุกต์ (หลักสูตรปรับปรุง พ.ศ. 2569) เมื่อวันที่ 20 ธันวาคม พ.ศ. 2567 ณ ห้องประชุม ST1-306 คณะวิทยาศาสตร์และเทคโนโลยี มหาวิทยาลัยเทคโนโลยีราชมงคลธัญบุรี โดยมีวัตถุประสงค์เพื่อระดมความคิดเห็นและข้อเสนอแนะจากผู้ทรงคุณวุฒิทั้งจากภาครัฐและภาคเอกชน อันเป็นแนวทางสำคัญในการพัฒนาหลักสูตรให้ทันสมัย สอดคล้องกับแนวโน้มและความเปลี่ยนแปลงของตลาดแรงงานยุคใหม่ ซึ่งได้รับเกียรติจากผู้ทรงคุณวุฒิที่มีความเชี่ยวชาญในสาขาต่าง ๆ ได้แก่ 
\begin{enumerate}
	\item รองศาสตราจารย์ ดร.ฉัฐไชย์ ลีนาวงศ์ 
	\\อาจารย์ประจำภาควิชาคณิตศาสตร์ คณะวิทยาศาสตร์ สถาบันเทคโนโลยีพระจอมเกล้าเจ้าคุณทหารลาดกระบัง %ซึ่งให้ข้อเสนอแนะด้านการจัดการเรียนการสอนและการออกแบบหลักสูตรให้เหมาะสมกับบริบทของระดับอุดมศึกษา 
	\item คุณกมลทิพย์ ตั้งธรรมนิยม \\ผู้จัดการฝ่าย HR Digitalization บริษัท บ้านปู จำกัด (มหาชน) %ซึ่งเน้นย้ำถึงความสำคัญของ soft skills ควบคู่กับทักษะด้านเทคโนโลยีดิจิทัลในบริบทของภาคอุตสาหกรรม
	\item คุณปารเมษฐ์ เจริญกิจโรจน์ \\ผู้อำนวยการฝ่าย Insurance, Investment and Retail Deposit Product ธนาคารกรุงไทย จำกัด (มหาชน) %ซึ่งได้เสนอแนวทางการบูรณาการความรู้ทางคณิตศาสตร์ในธุรกิจการเงินและประกันภัย พร้อมแนะนำเนื้อหารายวิชาที่ควรเพิ่มเติมหรือปรับปรุงเพื่อให้สอดคล้องกับการนำไปใช้จริง
\end{enumerate}
%\indent\indent ข้อเสนอแนะจากผู้ทรงคุณวุฒิทั้งสามท่านได้สะท้อนให้เห็นถึงความจำเป็นในการออกแบบโครงสร้างหลักสูตรให้มีความยืดหยุ่น ทันสมัย และสามารถปรับตัวตามการเปลี่ยนแปลงของเทคโนโลยีและบริบทเศรษฐกิจได้อย่างมีประสิทธิภาพ เพื่อยกระดับความสามารถของบัณฑิตในการแข่งขัน และสร้างความพร้อมสำหรับการทำงานในบริบทสากล
%ต่อเนื่องจากการประชุมดังกล่าว คณะได้ดำเนินการปรับปรุงหลักสูตรตามข้อเสนอแนะที่ได้รับ 
จัดโครงการวิพากษ์และพัฒนาหลักสูตร เมื่อวันที่ 28 กุมภาพันธ์ พ.ศ. 2568 ณ คณะวิทยาศาสตร์และเทคโนโลยี มหาวิทยาลัยเทคโนโลยีราชมงคลธัญบุรี เพื่อเปิดรับข้อเสนอแนะเพิ่มเติมจากผู้ทรงคุณวุฒิจากภาครัฐและภาคเอกชน ซึ่งได้รับเกียรติจากผู้ทรงคุณวุฒิที่มีความเชี่ยวชาญในสาขาต่าง ๆ ได้แก่ 
\begin{enumerate}
	\item ผู้ช่วยศาสตราจารย์ ดร.ทศพร แถลงธรรม \\อาจารย์ประจำภาควิชาคณิตศาสตร์ คณะวิทยาศาสตร์ มหาวิทยาลัยขอนแก่น
	\item ผู้ช่วยศาสตราจารย์ ดร.วริสา ยมเสถียรกุล \\อาจารย์ประจำภาควิชาคณิตศาสตร์ คณะวิทยาศาสตร์ มหาวิทยาลัยเทคโนโลยีพระจอมเกล้าธนบุรี	
	\item คุณวีรพล อิทธิอมรกุลชัย \\ผู้อำนวยการกลยุทธ์ธุรกิจ บริษัท เบทาโกร จำกัด (มหาชน)
	\item คุณจุฬาภรณ์ พูลเอี่ยม \\ผู้ช่วยวิจัย หน่วยการจัดการข้อมูลเพื่อการวิจัย คณะแพทยศาสตร์ศิริราชพยาบาล มหาวิทยาลัยมหิดล
\end{enumerate}


\begin{doclist}
\docitem{หลักฐานการดำเนินการโครงการ}
\docitem{เอกสารขั้นตอนการนำเสนอหลักสูตรใหม่ และหลักสูตรปรับปรุง}
\end{doclist}

































