\chapter{ส่วนนำ}

\section{บทสรุปผู้บริหาร}

\printprogram{} \printfaculty{} \printuniversity{} มีผลการดำเนินงานการประเมินคุณภาพการศึกษาระดับหลักสูตร ในปีการศึกษา \printyear{} ตามเกณฑ์ AUN-QA ประกอบด้วย 8 Criteria 53 Requirements โดยมีผลการประเมินตนเองตามองค์ประกอบที่ 1 การกำกับมาตรฐาน คือ เป็นไปตามเกณฑ์ และมีผลการประเมินตนเองตามเกณฑ์ AUN-QA Criteria ได้คะแนนโดยรวม คือ 3 ซึ่งมีรายละเอียด ดังต่อไปนี้

\begin{center}
\begin{tabular}{clc}
Criteria 1 & Expected Learning Outcomes &	ระดับ 3 \\ 
Criteria 2 & Programme Structure and Content &	ระดับ 3 \\
Criteria 3 & Teaching and Learning Approach &	ระดับ 3 \\
Criteria 4 & Student Assessment &	ระดับ 3 \\
Criteria 5 & Academic Staff &	ระดับ 3 \\
Criteria 6 & Academic Staff Quality &	ระดับ 3 \\
Criteria 7 & Facilities and Infrastructure &	ระดับ 3 \\
Criteria 8 & Output and Outcomes &	ระดับ 3 
\end{tabular}    
\end{center}

\section{บทนำเกี่ยวกับมหาวิทยาลัย คณะและหลักสูตร}

สาขาคณิตศาสตร์เป็นสาขาวิชาหนึ่งในภาควิชาคณิตศาสตร์และวิทยาการคอมพิวเตอร์  สังกัดคณะวิทยาศาสตร์และเทคโนโลยี มหาวิทยาลัยเทคโนโลยีราชมงคลธัญบุรี มีประวัติความเป็นมาที่แสดงพัฒนาการของสาขาวิชา ดังนี้ พ.ศ. 2518 – 2538  สาขาวิชาคณิตศาสตร์สังกัดอยู่คณะศิลปะศาสตร์ กลุ่มวิทยาศาสตร์และคณิตศาสตร์ เมื่อสถาบันเทคโนโลยีราชมงคล มีประกาศจัดตั้งคณะวิทยาศาสตร์ ในวันที่ 21 มิถุนายน 2538 สาขาคณิตศาสตร์ จึงเป็นส่วนหนึ่งของภาควิชาคณิตศาสตร์ คณะวิทยาศาสตร์ สถาบันเทคโนโลยีราชมงคล ต่อมาในปี พ.ศ. 2548 ได้มีการเปลี่ยนแปลงระบบการจัดการศึกษา จากสถาบันเทคโนโลยีราชมงคลเป็นมหาวิทยาลัยเทคโนโลยีราชมงคลธัญบุรี และได้มีการเปลี่ยนชื่อจาก คณะวิทยาศาสตร์ เป็น คณะวิทยาศาสตร์และเทคโนโลยี ตามพระราชบัญญัติมหาวิทยาลัยเทคโนโลยีราชมงคลธัญบุรี  และเปลี่ยนภาควิชาคณิตศาสตร์ เป็นภาควิชาคณิตศาสตร์และวิทยาการคอมพิวเตอร์  ซึ่งประกอบไปด้วย 4 สาขาวิชา ได้แก่ สาขาวิชาคณิตศาสตร์ สาขาวิชาสถิติประยุกต์  สาขาเทคโนโลยีคอมพิวเตอร์  และสาขาวิทยาการคอมพิวเตอร์

หลักสูตรวิทยาศาสตรบัณฑิตสาขาวิชาคณิตศาสตร์ได้จัดทำขึ้นเมื่อปีการศึกษา 2544 โดยเริ่มรับนักศึกษาใน ภาคเรียนที่ 1 ปีการศึกษา 2545  และได้มีการปรับปรุงหลักสูตรตามกรอบระยะเวลาการปรับปรุงหลักสูตรเพื่อให้เป็นไปตามเกณฑ์มาตรฐานของหลักสูตร ได้แก่หลักสูตรวิทยาศาตรบัณฑิต สาขาวิชาคณิตศาสตร์ (หลักสูตรปรับปรุง พ.ศ.2553) หลักสูตรวิทยาศาตรบัณฑิต สาขาวิชาคณิตศาสตร์ (หลักสูตรปรับปรุง พ.ศ.2556) หลักสูตรวิทยาศาสตรบัณฑิต สาขาวิชาคณิตศาสตร์ (หลักสูตรปรับปรุง พ.ศ. 2559) และหลักสูตรปัจจุบันได้แก่หลักสูตร วิทยาศาสตรบัณฑิต สาขาวิชาคณิตศาสตร์ประยุกต์ (หลักสูตรปรับปรุง พ.ศ. 2564 ) เป็นหลักสูตรที่มุ่งมุ่งเน้นการผลิตนวัตกรผู้ใช้คณิตศาสตร์และเทคโนโลยีในการสร้างสรรค์ผลงาน ทางด้านวิชาการที่สามารถนำไปแก้ปัญหาสังคม ธุรกิจ และก่อประโยชน์ต่อประเทศชาติ โดยปัจจุบันมีบัณฑิตที่สำเร็จการศึกษาแล้วจำนวน 20 รุ่น


