\criteria{Expected Learning Outcomes}

\subcriteria{The programme to show that the expected learning outcomes are appropriately formulated in accordance with an established learning taxonomy, are aligned to the vision and mission of the university, and are known to all stakeholders.}
วิสัยทัศน์และพันธกิจของมหาวิทยาลัยเทคโนโลยีราชมงคลธัญบุรี และคณะวิทยาศาสตร์และเทคโนโลยี มีรายละเอียดดังตารางต่อไปนี้

\begin{center}
\begin{tabular}{|>{\raggedright}p{0.45\textwidth}|p{0.45\textwidth}|}
\hline
\multicolumn{1}{|c|}{\textbf{มหาวิทยาลัยเทคโนโลยีราชมงคลธัญบุรี}} & \multicolumn{1}{c|}{\textbf{คณะวิทยาศาสตร์และเทคโนโลยี}} \\
\hline
\multicolumn{2}{|c|}{วิสัยทัศน์ (Vision)} \\
\hline
มหาวิทยาลัยนวัตกรรมที่สร้างคุณค่าสู่สังคมและประเทศ & 
เป็นคณะที่มุ่งเน้นการสร้างนวัตกรและนวัตกรรมด้านวิทยาศาสตร์และเทคโนโลยีที่มีคุณค่าสู่สังคมและประเทศ  \\
\hline
\multicolumn{2}{|c|}{พันธกิจ (Mission)} \\
\hline
\vspace{-0.7cm}
\begin{enumerate}
\item ผลิตและพัฒนากำลังคนให้มีความสามารถทางวิชาการ วิชาชีพ คิดสร้างสรรค์และเรียนรู้ตลอดชีวิต
\item สร้างงานวิจัย สิ่งประดิษฐ์ งานสร้างสรรค์ และนวัตกรรม สู่การนำไปใช้ประโยชน์ในภาคอุตสาหกรรม สังคม ชุมชน หรือสร้างมูลค่าเชิงพาณิชย์
\item ให้บริการวิชาการแก่ชุมชนในพื้นที่เป้าหมายหรือภาคประกอบการเพื่อการพัฒนาอย่างยั่งยืน
\item ทำนุบำรุงศาสนา ศิลปวัฒนธรรม และอนุรักษ์สิ่งแวดล้อม
\item บริหารจัดการอย่างมีธรรมาภิบาล \newline เพิ่มประสิทธิภาพและประสิทธิผลด้วยนวัตกรรม เพื่อการพัฒนาอย่างต่อเนื่องและยั่งยืน
\end{enumerate}
&
\vspace{-0.7cm}
\begin{enumerate}
\item ผลิตนักนวัตกรที่ปฏิบัติงานได้จริง สามารถประยุกต์ใช้ประโยชน์หรือพัฒนาเทคโนโลยีและสร้างนวัตกรรม
\item ผลิตผลงานวิจัย สร้างสรรค์เทคโนโลยีและนวัตกรรมเพื่อการพัฒนาประเทศ
\item บริการวิชาการที่ตอบสนองต่อความต้องการ สร้างคุณค่า เป็นประโยชน์ เป็นที่ยอมรับและสร้างความเข้มแข็งให้ชุมชนและสังคมอย่างยั่งยืน
\end{enumerate} \\
\hline
\end{tabular}
\end{center}

ผลลัพธ์การเรียนรู้ระดับหลักสูตร (PLOs) ของ\printprogram{} ได้รับการออกแบบให้สอดคล้องกับวิสัยทัศน์และพันธกิจของมหาวิทยาลัยเทคโนโลยีราชมงคลธัญบุรีและของคณะวิทยาศาสตร์และเทคโนโลยี ดังตาราง \ref{table: 2.7} อีกทั้งยังออกแบบให้สอดคล้องกับ ความรู้ (knowledge) ทักษะ (skills) และทัศนคติ (attitudes) ดังตาราง \ref{table: plo_ksec}

\begin{longtable}{|p{0.05\textwidth} >{\raggedright}p{0.5\textwidth} | p{0.1\textwidth} | p{0.1\textwidth} | p{0.1\textwidth} | p{0.1\textwidth} |}
\caption{ความสอดคล้องระหว่างผลลัพธ์การเรียนรู้ระดับหลักสูตร (PLOs) กับวิสัยทัศน์และพันธกิจของมหาวิทยาลัยและคณะฯ}
\label{table: 2.7}
\\
\hline
\multicolumn{2}{|c|}{\textbf{ผลลัพธ์การเรียนรู้ระดับหลักสูตร (PLOs)}} & \multicolumn{2}{c|}{\textbf{ระดับมหาวิทยาลัย}} & \multicolumn{2}{c|}{\textbf{ระดับคณะ}} \\
\cline{3-6}
\multicolumn{2}{|c|}{} & \textbf{วิสัยทัศน์} & \textbf{พันธกิจ} & \textbf{วิสัยทัศน์} & \textbf{พันธกิจ} \\
\hline
\endhead
PLO1:& ปฏิบัติตามจรรยาบรรณทางวิชาการ กฎระเบียบ และข้อบังคับขององค์กร (Affective Domain) & \multicolumn{1}{c|}{\checkmark} & \multicolumn{1}{c|}{1, 5} & \multicolumn{1}{c|}{\checkmark} & \multicolumn{1}{c|}{1} \\
\hline
PLO2:&อธิบายบทนิยาม หลักการ และทฤษฎีบททางด้านคณิตศาสตร์และวิทยาศาสตร์ที่สำคัญได้อย่างถูกต้อง (Understanding) & \multicolumn{1}{c|}{\checkmark} & \multicolumn{1}{c|}{1} & \multicolumn{1}{c|}{\checkmark} & \multicolumn{1}{c|}{1} \\
\hline
PLO3:&คำนวณเพื่อแก้ปัญหาทางด้านคณิตศาสตร์ตามหลักการ บทนิยาม และทฤษฎีบทได้อย่างถูกต้องเหมาะสม (Analyzing) & \multicolumn{1}{c|}{\checkmark} & \multicolumn{1}{c|}{1} & \multicolumn{1}{c|}{\checkmark} & \multicolumn{1}{c|}{1} \\
\hline
PLO4:&พิสูจน์ข้อความและทฤษฎีบททางคณิตศาสตร์ได้อย่างถูกต้องและสมเหตุสมผลตามหลักตรรกศาสตร์และการให้เหตุผล (Evaluating) & \multicolumn{1}{c|}{\checkmark} & \multicolumn{1}{c|}{1} & \multicolumn{1}{c|}{\checkmark} & \multicolumn{1}{c|}{1} \\
\hline
PLO5:&ประยุกต์ใช้ความรู้ ทักษะ และเทคโนโลยีทางคณิตศาสตร์ในการแก้ปัญหาทางด้านวิทยาศาสตร์ วิศวกรรมศาสตร์ ธุรกิจ อุตสาหกรรม หรือศาสตร์ที่เกี่ยวข้อง (Applying) & \multicolumn{1}{c|}{\checkmark} & \multicolumn{1}{c|}{1, 2, 3} & \multicolumn{1}{c|}{\checkmark} & \multicolumn{1}{c|}{1, 2, 3} \\
\hline
PLO6:&สร้างหรือปรับปรุงกระบวนการคิดทางคณิตศาสตร์และการวิจัยที่นำไปสู่องค์ความรู้ใหม่หรือนวัตกรรมทางด้านคณิตศาสตร์ คณิตศาสตร์ประยุกต์ หรือด้านที่เกี่ยวข้อง (Creating) & \multicolumn{1}{c|}{\checkmark} & \multicolumn{1}{c|}{1, 2} & \multicolumn{1}{c|}{\checkmark} & \multicolumn{1}{c|}{1, 2} \\
\hline
PLO7:&ปรับตัวเข้ากับสถานการณ์และวัฒนธรรมขององค์กร มีความรับผิดชอบ และทำงานร่วมกับผู้อื่นในฐานะผู้นำหรือสมาชิกที่ดี (Affective Domain) & \multicolumn{1}{c|}{\checkmark} & \multicolumn{1}{c|}{1, 5} & \multicolumn{1}{c|}{\checkmark} & \multicolumn{1}{c|}{1} \\
\hline
PLO8:&ใช้คณิตศาสตร์หรือสถิติเพื่อการวิเคราะห์ ประมวลผลข้อมูล และนำเสนอได้อย่างเหมาะสม (Evaluating) & \multicolumn{1}{c|}{\checkmark} & \multicolumn{1}{c|}{1, 2} & \multicolumn{1}{c|}{\checkmark} & \multicolumn{1}{c|}{1, 2} \\
\hline
PLO9:&รู้วิธีแสวงหา และถ่ายทอดความรู้ได้อย่างถูกต้องตามหลักวิชาการ ร่วมกับการใช้เทคโนโลยี เพื่อการนำเสนองานทางด้านคณิตศาสตร์หรือด้านที่เกี่ยวข้อง (Remembering) & \multicolumn{1}{c|}{\checkmark} & \multicolumn{1}{c|}{1} & \multicolumn{1}{c|}{\checkmark} & \multicolumn{1}{c|}{1} \\
\hline
PLO10:&ใช้อุปกรณ์และเครื่องมือพื้นฐานทางด้านวิทยาศาสตร์ และเขียนหรือใช้โปรแกรมคอมพิวเตอร์สำหรับงานทางด้านคณิตศาสตร์ได้ (Applying) & \multicolumn{1}{c|}{\checkmark} & \multicolumn{1}{c|}{1, 2} & \multicolumn{1}{c|}{\checkmark} & \multicolumn{1}{c|}{1, 2} \\
\hline
\end{longtable}

\begin{longtable}{|p{0.052\textwidth}  >{\raggedright}p{0.6\textwidth} | >{\centering}p{0.07\textwidth} | >{\centering}p{0.07\textwidth} | p{0.07\textwidth} | }
\caption{ความเชื่อมโยงระหว่างผลลัพธ์การเรียนรู้ระดับหลักสูตร (PLOs) กับ KSA}
\label{table: plo_ksec}
\\
\hline
\multicolumn{2}{|c|}{\textbf{ผลลัพธ์การเรียนรู้ระดับหลักสูตร (PLOs)}} & \textbf{K} & \textbf{S} & \textbf{A}  \\
\hline
\endhead

PLO1:& ปฏิบัติตามจรรยาบรรณทางวิชาการ กฎระเบียบ และข้อบังคับขององค์กร & &  &\checkmark  \\
\hline
PLO2:& อธิบายบทนิยาม หลักการ และทฤษฎีบททางด้านคณิตศาสตร์และวิทยาศาสตร์ที่สำคัญได้อย่างถูกต้อง & \checkmark & & \\
\hline
PLO3:& คำนวณเพื่อแก้ปัญหาทางด้านคณิตศาสตร์ตามหลักการ บทนิยาม และทฤษฎีบทได้อย่างถูกต้องเหมาะสม & \checkmark &  & \\
\hline
PLO4:&พิสูจน์ข้อความและทฤษฎีบททางคณิตศาสตร์ได้อย่างถูกต้องและสมเหตุสมผลตามหลักตรรกศาสตร์และการให้เหตุผล & \checkmark &  & \\
\hline
PLO5:&ประยุกต์ใช้ความรู้ ทักษะ และเทคโนโลยีทางคณิตศาสตร์ในการแก้ปัญหาทางด้านวิทยาศาสตร์ วิศวกรรมศาสตร์ ธุรกิจ อุตสาหกรรม หรือศาสตร์ที่เกี่ยวข้อง & \checkmark  & \checkmark &  \\
\hline
PLO6:&สร้างหรือปรับปรุงกระบวนการคิดทางคณิตศาสตร์และการวิจัยที่นำไปสู่องค์ความรู้ใหม่หรือนวัตกรรม & \checkmark  &  & \\
\hline
PLO7:&ปรับตัวเข้ากับสถานการณ์และวัฒนธรรมขององค์กร มีความรับผิดชอบ และทำงานร่วมกับผู้อื่น & &  &  \checkmark \\
\hline
PLO8:&ใช้คณิตศาสตร์หรือสถิติเพื่อการวิเคราะห์ ประมวลผลข้อมูล และนำเสนอได้อย่างเหมาะสม & \checkmark &  & \\
\hline
PLO9:&รู้วิธีแสวงหา และถ่ายทอดความรู้ได้อย่างถูกต้องตามหลักวิชาการ ร่วมกับการใช้เทคโนโลยี & \checkmark & \checkmark & \\
\hline
PLO10:&ใช้อุปกรณ์และเครื่องมือพื้นฐานทางด้านวิทยาศาสตร์ และเขียนหรือใช้โปรแกรมคอมพิวเตอร์สำหรับงานทางด้านคณิตศาสตร์ได้ & & \checkmark & \\
\hline
\multicolumn{4}{l}{\footnotesize K: Knowledge, S: Skills, A: Attitudes } \\
\end{longtable}

ทั้งนี้ PLOs ของหลักสูตรได้บรรจุใน มคอ.2 และเผยแพร่ให้แก่ผู้มีส่วนได้ส่วนเสียในช่องทางต่างๆ ที่เข้าถึงได้ง่าย หลักสูตรมีการตรวจสอบการรับรู้ PLOs โดยใช้แบบสอบถาม รายงานการประชุมคณะกรรมการประจำคณะวิทยาศาสตร์และเทคโนโลยี และรายงานการประชุมสภามหาวิทยาลัย ซึ่งมีผลการรับรู้ดังตาราง \ref{Table:C_to-SH}

%\begin{longtable}{|>{\raggedright}p{0.35\textwidth}|>{\raggedright}p{0.3\textwidth}|>{\raggedright\arraybackslash}p{0.3\textwidth}|}
%\caption{ตารางแสดงการสื่อสาร PLOs กับผู้มีส่วนได้ส่วนเสียและกลไกการประเมินผล}
%\label{Table:C_to-SH}
%\\
%\hline
%\multicolumn{1}{|c|}{\bf ช่องทางการสื่อสาร}&\multicolumn{1}{c|}{\bf ผู้มีส่วนได้ส่วนเสีย}&\multicolumn{1}{c|}{\bf การประเมินผล/การรับทราบ}\\
%\hline
%\endhead
%เอกสาร\printprogram{} (มคอ. 2) & มหาวิทยาลัย\newline อาจารย์ผู้รับผิดชอบหลักสูตร\newline อาจารย์ผู้สอน & บันทึกการประชุมคณะกรรมการประจำคณะ/สาขาวิชาคณิตศาสตร์ \newline บันทึกการรับทราบจากหน่วยงานที่เกี่ยวข้อง \\
%\hline
%เว็บไซต์ของสำนักส่งเสริมวิชาการและงานทะเบียน, คณะ, สาขาวิชา & นักศึกษาปัจจุบัน\newline นักศึกษาใหม่\newline ผู้สนใจเข้าศึกษา & สถิติการเข้าชมเว็บไซต์ (Website analytics)\newline แบบสอบถามการรับรู้ข้อมูลของนักศึกษาใหม่\newline ช่องทางติดต่อสอบถามบนเว็บไซต์ \\
%\hline
%Facebook Page ของหลักสูตร & นักศึกษาปัจจุบัน\newline ศิษย์เก่า\newline ผู้สนใจทั่วไป & การมีส่วนร่วมกับโพสต์ (Engagement rate) \newline การสำรวจผ่าน Facebook Polls\newline ข้อความสอบถามผ่าน Inbox \\
%\hline
%การประชุม/สัมมนาผู้ใช้บัณฑิต & ผู้ใช้บัณฑิต\newline สถานประกอบการ & บันทึกการประชุม/สรุปผลการสัมมนา\newline แบบประเมินผลการจัดกิจกรรม\newline ข้อเสนอแนะจากผู้เข้าร่วม \\
%\hline
%วันปฐมนิเทศนักศึกษาใหม่ & นักศึกษาใหม่\newline ผู้ปกครอง & แบบสอบถามความเข้าใจหลังจบกิจกรรม\newline การตอบคำถามในช่วง Q\&A \\
%\hline
%\end{longtable}
%%%%%%%%%%%%%%%%%%%%%%%%%%%%%%%%%%%%%%%%%%%%%%%%%%%%%%%%%%%%%%%%%%%%%%%%%%%%%%%%%%%%%%%%%%%%%%%%%%%%%%%%%%%%%%%%
\begin{longtable}{|>{\raggedright}p{0.3\textwidth}|>{\raggedright\arraybackslash}p{0.38\textwidth}|>{\raggedright\arraybackslash}p{0.3\textwidth}|}
    \caption{ผลการรับรู้ PLOs ของผู้มีส่วนได้ส่วนเสีย}
 
    \label{Table:C_to-SH}    \\
    \hline
    \multicolumn{1}{|c|}{\bf ผู้มีส่วนได้ส่วนเสีย}&\multicolumn{1}{c|}{\bf ช่องทางการสื่อสาร}&\multicolumn{1}{c|}{\bf ร้อยละของการรับรู้ PLOs}\\
    \hline
    \endfirsthead
        \caption{(ต่อ) ผลการรับรู้ PLOs ของผู้มีส่วนได้ส่วนเสีย}
    
   \\
    \hline
    \multicolumn{1}{|c|}{\bf ผู้มีส่วนได้ส่วนเสีย}&\multicolumn{1}{c|}{\bf ช่องทางการสื่อสาร}&\multicolumn{1}{c|}{\bf ร้อยละของการรับรู้ PLOs}\\
    \hline
    \endhead
    %\multicolumn{2}{|l|}{\textbf{ผู้มีส่วนได้ส่วนเสียภายนอก (External Stakeholders)}} \\ \hline
    ผู้ใช้บัณฑิต/สถานประกอบการ & หนังสือประชาสัมพันธ์หลักสูตร & \multicolumn{1}{c|}{100} \\ \hline
    บัณฑิตที่สำเร็จการศึกษาจากหลักสูตร & \vspace{-0.7cm}
    \begin{enumerate}[label=-]
    	\item เว็บไซต์ของคณะ และสาขาวิชา
    	\item Facebook Page ของหลักสูตร
    \end{enumerate}
    & \multicolumn{1}{c|}{84.9} \\ \hline
    %\multicolumn{2}{|l|}{\textbf{ผู้มีส่วนได้ส่วนเสียภายใน (Internal Stakeholders)}} \\ \hline
    มหาวิทยาลัย/คณะฯ & \printprogram{} (มคอ. 2) &  \multicolumn{1}{c|}{100}  \\ \hline
    อาจารย์ผู้รับผิดชอบหลักสูตร และอาจารย์ผู้สอน & \printprogram{} (มคอ. 2) &  \multicolumn{1}{c|}{100}  \\ \hline
    นักศึกษาปัจจุบันและนักศึกษาชั้นปีสุดท้าย & 
    \begin{enumerate}[label=-] \vspace{-0.7cm}
    	\item เว็บไซต์ของสำนักส่งเสริมวิชาการและงานทะเบียน คณะ และสาขาวิชา
    	\item อาจารย์ที่ปรึกษา 
    	\item คู่มือนักศึกษา\printprogram
    \end{enumerate}
    &  \multicolumn{1}{c|}{100}  \\ \hline
\end{longtable}

\begin{doclist}
\docitem{\printprogram{}}
\docitem{เอกสารการปรับปรุงแก้ไข\printprogram{} (สมอ. 08) }
\docitem{รายงานผลแบบสอบถามการรับรู้ PLOs }
\end{doclist}

\subcriteria{The programme to show that the expected learning outcomes for all courses are appropriately formulated and are aligned to the expected learning outcomes of the programme.}

\printprogram{} ได้กำหนดผลลัพธ์การเรียนรู้ระดับรายวิชา\;(CLOs) ให้รับผิดชอบการบรรลุผลลัพธ์การเรียนรู้ระดับหลักสูตร\;(PLOs)  ดังตัวอย่างการกำหนด CLOs ของรายวิชา 09-114-205 กำหนดการเชิงคณิตศาสตร์เบื้องต้น ให้รับผิดชอบ PLO2 PLO3 PLO5 และ PLO10


\begin{longtable}{|>{\raggedright}p{0.4\linewidth}*{10}{|>{\centering\arraybackslash}p{0.025\linewidth}}|}

\caption{ความเชื่อมโยงระหว่างผลลัพธ์การเรียนรู้ระดับรายวิชา (CLOs) ของรายวิชา 09-114-205 กำหนดการเชิงคณิตศาสตร์เบื้องต้น กับผลลัพธ์การเรียนรู้ระดับหลักสูตร (PLOs)  ของรายวิชา 09-114-205 กำหนดการเชิงคณิตศาสตร์เบื้องต้น}
\label{table: closvsplos}
\\
\hline
\multicolumn{1}{|c|}{\multirow{2}{*}{\textbf{CLOs}}}
& \multicolumn{10}{c|}{\textbf{PLOs}}\\
\cline{2-11}
 & 1 & 2 & 3 & 4 & 5 & 6 & 7 & 8 & 9 & 10 
\\
\hline
\endfirsthead

\caption{(ต่อ) ความเชื่อมโยงระหว่างผลลัพธ์การเรียนรู้ระดับรายวิชา (CLOs) ของรายวิชา 09-114-205 กำหนดการเชิงคณิตศาสตร์เบื้องต้น กับผลลัพธ์การเรียนรู้ระดับหลักสูตร (PLOs)   }
\\
\hline
\multicolumn{1}{|c|}{\multirow{2}{*}{\textbf{CLOs}}}
& \multicolumn{10}{c|}{\textbf{PLOs}}\\
\cline{2-11}
 & 1 & 2 & 3 & 4 & 5 & 6 & 7 & 8 & 9 & 10 
\\
\hline
\endhead

\hline
\endfoot

CLO1: เขียนปัญหาทางวิทยาศาสตร์ วิศวกรรมและการเงินในรูปแบบกําหนดการเชิงคณิตศาสตร์ได้  &  & \checkmark &  &  &  &  &  &  &  &  \\ \hline
CLO2: อธิบายตัวแบบกําหนดการเชิงเส้นและไม่เชิงเส้นได้&  &  \checkmark&  &  &  &  &  &  &  &  \\ \hline
CLO3: หาผลเฉลยของตัวแบบกําหนดการเชิงคณิตศาสตร์เบื้องต้นด้วยโปรแกรมได้ &  &  & \checkmark &  &  &  &  &  &  &  \\ \hline
CLO4: เขียนโปรแกรมเพื่อหาผลเฉลยของตัวแบบกําหนดการเชิงคณิตศาสตร์เบื้องต้นด้วยโปรแกรมได้ &  &  &  &  &  &  &  &  &  & \checkmark \\ \hline
CLO5: ประยุกต์ใช้ตัวแบบกำหนดการเชิงคณิตศาสตร์ในการแก้ปัญหาได้&  &  &  &  & \checkmark &  &  &  &  &  \\ 



\end{longtable}
%\end{landscape}





\begin{doclist}
\docitem{\printprogram{}}
\docitem{เอกสารปรับปรุง\printprogram{} (สมอ. 08) }
%\docitem{รายละเอียดของรายวิชา (มคอ. 3) ของหลักสูตรที่เปิดสอนในปีการศึกษา \printyear}
%\docitem{ตารางแสดงความเชื่อมโยงระหว่างผลลัพธ์การเรียนรู้ระดับหลักสูตร (PLOs) กับผลลัพธ์การเรียนรู้ระดับรายวิชา (CLOs)}
\end{doclist}

\subcriteria{The programme to show that the expected learning outcomes consist of both generic outcomes (related to written and oral communication, problem-solving, information technology, team building skills, etc) and subject specific outcomes (related to knowledge and skills of the study discipline).}

PLOs ของ\printprogram{} ได้รับการออกแบบให้ครอบคลุมทั้งผลลัพธ์ทั่วไปและผลลัพธ์เฉพาะด้าน แสดงโดยสรุปดังตาราง \ref{table: req 1.1} 

\begin{center}
	\begin{longtable}{|p{0.052\textwidth}  >{\raggedright}p{0.6\textwidth} | p{0.1\textwidth} | p{0.1\textwidth} |}
		\caption{ความสัมพันธ์ระหว่างผลลัพธ์การเรียนรู้ระดับหลักสูตร (PLOs) กับ Generic outcomes (GLOs) และ  Subject specific outcomes (SSLOs)} 
		\label{table: req 1.1}
		\\
		\hline
		\multicolumn{2}{|c|}{\textbf{PLOs}} & \multicolumn{1}{c|}{\textbf{GLOs}} &\multicolumn{1}{c|}{ \textbf{SSLOs}} \\
		\hline
		\endfirsthead
			\caption{(ต่อ) ความสัมพันธ์ระหว่างผลลัพธ์การเรียนรู้ระดับหลักสูตร (PLOs) กับ Generic outcomes (GLOs) และ  Subject specific outcomes (SSLOs)} 
		\\
		\hline
		\multicolumn{2}{|c|}{\textbf{PLOs}} & \multicolumn{1}{c|}{\textbf{GLOs}} &\multicolumn{1}{c|}{ \textbf{SSLOs}} \\
		\hline
		\endhead
		
		PLO1:&ปฏิบัติตามจรรยาบรรณทางวิชาการ กฎระเบียบ และข้อบังคับขององค์กร (Affective Domain) &\multicolumn{1}{c|}{\checkmark}&\\
		\hline
		PLO2:&อธิบายบทนิยาม หลักการ และทฤษฎีบททางด้านคณิตศาสตร์และวิทยาศาสตร์ที่สำคัญได้อย่างถูกต้อง (Understanding) &&\multicolumn{1}{c|}{\checkmark}\\
		\hline
		PLO3:&คำนวณเพื่อแก้ปัญหาทางด้านคณิตศาสตร์ตามหลักการ บทนิยาม และทฤษฎีบทได้อย่างถูกต้องเหมาะสม (Analyzing)&&\multicolumn{1}{c|}{\checkmark}\\
		\hline
		PLO4:&พิสูจน์ข้อความและทฤษฎีบททางคณิตศาสตร์ได้อย่างถูกต้องและสมเหตุสมผลตามหลักตรรกศาสตร์และการให้เหตุผล (Evaluating)&&\multicolumn{1}{c|}{\checkmark}\\
		\hline
		PLO5:&ประยุกต์ใช้ความรู้ ทักษะ และเทคโนโลยีทางคณิตศาสตร์ในการแก้ปัญหาทางด้านวิทยาศาสตร์ วิศวกรรมศาสตร์ ธุรกิจ อุตสาหกรรม หรือศาสตร์ที่เกี่ยวข้อง (Applying)&&\multicolumn{1}{c|}{\checkmark}\\
		\hline
		PLO6:&สร้างหรือปรับปรุงกระบวนการคิดทางคณิตศาสตร์และการวิจัยที่นำไปสู่องค์ความรู้ใหม่หรือนวัตกรรมทางด้านคณิตศาสตร์ คณิตศาสตร์ประยุกต์ หรือด้านที่เกี่ยวข้อง (Creating)&&\multicolumn{1}{c|}{\checkmark}\\
		\hline
		PLO7:&ปรับตัวเข้ากับสถานการณ์และวัฒนธรรมขององค์กร มีความรับผิดชอบ และทำงานร่วมกับผู้อื่นในฐานะผู้นำหรือสมาชิกที่ดี (Affective Domain)&\multicolumn{1}{c|}{\checkmark}&\\
		\hline
		PLO8:&ใช้คณิตศาสตร์หรือสถิติเพื่อการวิเคราะห์ ประมวลผลข้อมูล และนำเสนอได้อย่างเหมาะสม (Evaluating)&&\multicolumn{1}{c|}{\checkmark}\\
		\hline
		PLO9:&รู้วิธีแสวงหา และถ่ายทอดความรู้ได้อย่างถูกต้องตามหลักวิชาการ ร่วมกับการใช้เทคโนโลยี เพื่อการนำเสนองานทางด้านคณิตศาสตร์หรือด้านที่เกี่ยวข้อง (Remembering)&\multicolumn{1}{c|}{\checkmark}&\\
		\hline
		PLO10:&ใช้อุปกรณ์และเครื่องมือพื้นฐานทางด้านวิทยาศาสตร์ และเขียนหรือใช้โปรแกรมคอมพิวเตอร์สำหรับงานทางด้านคณิตศาสตร์ได้ (Applying) &&\multicolumn{1}{c|}{\checkmark}\\
		\hline
	\end{longtable}
\end{center}
\begin{doclist}
\docitem{\printprogram{} }
\docitem{เอกสารปรับปรุง\printprogram{} (สมอ. 08)}
\end{doclist}



\subcriteria{The programme to show that the requirements of the stakeholders, especially the external stakeholders, are gathered, and that these are reflected in the expected learning outcomes.}

หลักสูตรได้ดำเนินการรวบรวมและวิเคราะห์ความต้องการของผู้มีส่วนได้ส่วนเสีย (stakeholders) ทั้งภายในและภายนอกอย่างเป็นระบบ เพื่อนำข้อมูลมาใช้ในการออกแบบและทบทวน PLOs ให้บัณฑิตมีคุณลักษณะที่พึงประสงค์และสอดคล้องกับความต้องการของตลาดแรงงานและสังคม โดยมีกระบวนการดังนี้

\begin{enumerate}
    \item \textbf{กำหนดกลุ่มและวิธีการเก็บข้อมูล:} หลักสูตรได้กำหนดกลุ่มผู้มีส่วนได้ส่วนเสียและวิธีการเก็บรวบรวมข้อมูลที่เหมาะสมกับแต่ละกลุ่ม ดังแสดงในตาราง \ref{table: m2-sh}
    
    \item \textbf{วิเคราะห์และสังเคราะห์ข้อมูล:} ข้อมูลความต้องการจากทุกกลุ่มได้ถูกนำมาวิเคราะห์และสังเคราะห์เป็นกลุ่มทักษะและความรู้ที่สำคัญที่บัณฑิตพึงมี
    
    \item \textbf{กำหนดและทบทวน PLOs:} ผลการวิเคราะห์ได้ถูกนำมาใช้ในการกำหนดและทบทวนผลลัพธ์การเรียนรู้ระดับหลักสูตร (PLOs) ทั้ง 10 ข้อ เพื่อให้มั่นใจว่าครอบคลุมความต้องการของผู้มีส่วนได้ส่วนเสียทุกกลุ่ม ดังสรุปในตาราง \ref{table:stakeholder_summary}
\end{enumerate}

\begin{longtable}{|>{\raggedright}p{0.3\textwidth}|>{\raggedright\arraybackslash}p{0.6\textwidth}|}
    \caption{กลุ่มผู้มีส่วนได้ส่วนเสียและแนวทางการเก็บรวบรวมข้อมูล}
    \label{table: m2-sh}
    \\
    \hline
    \multicolumn{1}{|c|}{\bf กลุ่มผู้มีส่วนได้ส่วนเสีย}&\multicolumn{1}{c|}{\bf วิธีการเก็บรวบรวมข้อมูล}\\
    \hline
    \endfirsthead
        \caption{กลุ่มผู้มีส่วนได้ส่วนเสียและแนวทางการเก็บรวบรวมข้อมูล (ต่อ)}
    \\
    \hline
    \multicolumn{1}{|c|}{\bf กลุ่มผู้มีส่วนได้ส่วนเสีย}&\multicolumn{1}{c|}{\bf วิธีการเก็บรวบรวมข้อมูล}\\
    \hline
    \endhead
    \multicolumn{2}{|l|}{\textbf{ผู้มีส่วนได้ส่วนเสียภายนอก (External Stakeholders)}} \\ \hline
    ผู้ใช้บัณฑิต/สถานประกอบการ & การจัดประชุมกลุ่มย่อย (Focus Group) และสัมภาษณ์เชิงลึกกับผู้ประกอบการ เพื่อรับฟังความต้องการโดยตรง และใช้แบบสอบถามความพึงพอใจต่อคุณภาพบัณฑิตเป็นประจำทุกปี \\ \hline
    บัณฑิตที่สำเร็จการศึกษาจากหลักสูตร & การสำรวจภาวะการมีงานทำของศิษย์เก่าผ่านแบบสอบถามออนไลน์ และการสัมภาษณ์กลุ่มเพื่อรวบรวมข้อเสนอแนะในการปรับปรุงหลักสูตรจากประสบการณ์ทำงานจริง \\ \hline
    \multicolumn{2}{|l|}{\textbf{ผู้มีส่วนได้ส่วนเสียภายใน (Internal Stakeholders)}} \\ \hline
    มหาวิทยาลัย/คณะฯ & การทบทวนความสอดคล้องกับวิสัยทัศน์ พันธกิจ และแผนยุทธศาสตร์ของมหาวิทยาลัยและคณะฯ ผ่านการประชุมร่วมกับผู้บริหาร \\ \hline
    อาจารย์ผู้รับผิดชอบหลักสูตร และอาจารย์ผู้สอน & การประชุมหลักสูตรอย่างสม่ำเสมอ การระดมสมองเพื่อทบทวนรายวิชา และการสัมภาษณ์รายบุคคลเพื่อรวบรวมมุมมองด้านการสอน \\ \hline
    นักศึกษาปัจจุบันและนักศึกษาชั้นปีสุดท้าย & การรวบรวมข้อมูลผ่านแบบประเมินการสอน, การจัดประชุมรับฟังความคิดเห็น (Student Voice), และการสัมภาษณ์กลุ่มย่อย (Focus Group) \\ \hline
\end{longtable}

\begin{longtable}{| >{\raggedright}p{0.18\textwidth} | >{\raggedright}p{0.27\textwidth} | >{\raggedright}p{0.3\textwidth} | >{\centering\arraybackslash}p{0.15\textwidth} |}
    \caption{สรุปการรวบรวมความต้องการของผู้มีส่วนได้ส่วนเสียและการสะท้อนในผลการเรียนรู้ระดับหลักสูตร (PLOs) จัดเรียงตามลำดับความสำคัญของความต้องการของผู้มีส่วนได้เสีย}
    \label{table:stakeholder_summary}
    \\
    \hline
    \textbf{ผู้มีส่วนได้ส่วนเสีย} & \textbf{ความต้องการของกลุ่มผู้มีส่วนได้ส่วนเสียหลัก} & \textbf{สรุปความต้องการ} & \textbf{สะท้อนอยู่ใน PLOs} \\
    \hline
    \endfirsthead
      \caption{(ต่อ) สรุปการรวบรวมความต้องการของผู้มีส่วนได้ส่วนเสียและการสะท้อนในผลการเรียนรู้ระดับหลักสูตร (PLOs) จัดเรียงตามลำดับความสำคัญของความต้องการของผู้มีส่วนได้เสีย}
    \\
    \hline
    \textbf{ผู้มีส่วนได้ส่วนเสีย} & \textbf{ความต้องการของกลุ่มผู้มีส่วนได้ส่วนเสียหลัก} & \textbf{สรุปความต้องการ} & \textbf{สะท้อนอยู่ใน PLOs} \\
    \hline
    \endhead
    
    
    \multicolumn{4}{|l|}{\textbf{ผู้มีส่วนได้ส่วนเสียภายนอก (External Stakeholders)}} \\
    \hline
    ผู้ใช้บัณฑิต/สถานประกอบการ & 
ต้องการบัณฑิตที่สามารถ\newline บูรณาการความรู้ทางคณิตศาสตร์เชิงทฤษฎีเข้ากับการแก้ปัญหาทางธุรกิจได้จริง โดยคาดหวังให้บัณฑิตสร้างแบบจำลองทางคณิตศาสตร์เพื่อการพยากรณ์หรือหาค่าที่เหมาะสมที่สุด (Optimization) ได้ นอกเหนือจากทักษะเฉพาะทางแล้ว Soft Skills ถือเป็นปัจจัยสำคัญอย่างยิ่ง โดยเฉพาะความสามารถในการทำงานร่วมกับทีมสหสาขาวิชา และทักษะการสื่อสารที่สามารถย่อยผลการวิเคราะห์ที่ซับซ้อนให้เป็นข้อมูลเชิงลึก (Insight) ที่ผู้บริหารสามารถนำไปใช้ตัดสินใจเชิงกลยุทธ์ได้จริง อีกทั้งต้องพร้อมเรียนรู้และปรับตัวเข้ากับเทคโนโลยีและเครื่องมือวิเคราะห์ข้อมูลใหม่ๆ อยู่เสมอ &
- ทักษะการแก้ปัญหาและการคิดวิเคราะห์เชิงลึก การสร้างแบบจำลองทางคณิตศาสตร์เพื่ออธิบายปัญหา, การคิดเชิงตรรกะและวิพากษ์เพื่อประเมินแนวทางการแก้ปัญหา, การวิเคราะห์ข้อมูลเชิงปริมาณเพื่อหาความสัมพันธ์ที่ซ่อนอยู่ \newline
- ทักษะการสื่อสารและการทำงานร่วมกับผู้อื่น ความสามารถในการนำเสนอข้อมูล (Data Storytelling), การประสานงานในทีมแบบสหวิทยาการ, การรับฟังและให้ข้อคิดเห็นอย่างสร้างสรรค์ \newline
- ทักษะการใช้เทคโนโลยีและเครื่องมือดิจิทัล ความชำนาญในการเขียนโปรแกรม (เช่น Python, R), การใช้ซอฟต์แวร์ทางสถิติและคณิตศาสตร์, ความเข้าใจในหลักการของฐานข้อมูล (SQL) \newline
- ความรับผิดชอบและจรรยาบรรณวิชาชีพ ความซื่อสัตย์ในการจัดการข้อมูล (Data Integrity), การบริหารจัดการเวลาและภาระงาน, การปฏิบัติตามกฎระเบียบและวัฒนธรรมองค์กร &
\vspace{-0.3cm}
\begin{enumerate}[label={}]
	\item PLO1
	\item PLO3
	\item PLO5
	\item PLO7
	\item PLO8
	\item PLO10
\end{enumerate}
 \\    
\hline
บัณฑิตที่สำเร็จการศึกษา & 
บัณฑิตต้องการให้หลักสูตรเชื่อมโยงทฤษฎีกับการปฏิบัติ โดยเน้นทักษะการแปลงโจทย์ธุรกิจเป็นปัญหาทางคณิตศาสตร์และเลือกใช้เครื่องมือที่เหมาะสม นอกจากนี้ยังให้ความสำคัญกับประสบการณ์ทำโครงงานเพื่อสร้างแฟ้มผลงาน และการเรียนรู้เทคโนโลยีใหม่ๆ ที่จำเป็นต่อการทำงานในสายอาชีพคณิตศาสตร์และวิทยาการข้อมูล 
&
- ความรู้ทางคณิตศาสตร์ที่ทันสมัยและประยุกต์ได้ ความเข้าใจในทฤษฎีอย่างลึกซึ้ง และความสามารถในการนำไปสร้างแบบจำลองแก้ปัญหาจริง \newline
- ทักษะการวิจัยและสร้างนวัตกรรม กระบวนการตั้งคำถาม, การออกแบบการทดลอง, การวิเคราะห์และสรุปผลเพื่อสร้างองค์ความรู้ใหม่ \newline
- ทักษะการเรียนรู้ด้วยตนเองและการแสวงหาความรู้ ความสามารถในการศึกษาหัวข้อใหม่ๆ จากเอกสารทางวิชาการ, การติดตามความก้าวหน้าในสายงาน \newline
- การประยุกต์ใช้ความรู้เพื่อการสื่อสารและถ่ายทอด การสรุปและนำเสนอแนวคิดที่ซับซ้อนให้เข้าใจง่าย &
\vspace{-0.5cm}
\begin{enumerate}[label={}]
	\item PLO2
	\item PLO5
	\item PLO6
	\item PLO9
\end{enumerate}
 \\
    \hline
    \multicolumn{4}{|l|}{\textbf{ผู้มีส่วนได้ส่วนเสียภายใน (Internal Stakeholders)}} \\
    \hline
มหาวิทยาลัย/คณะฯ & 
มหาวิทยาลัยและคณะฯ กำหนดให้หลักสูตรต้องมีบทบาทสำคัญในการบรรลุเป้าหมายเชิงกลยุทธ์สูงสุด คือการเป็นมหาวิทยาลัยนวัตกรรมที่สร้างคุณค่าสู่สังคมและประเทศ  ดังนั้น ความต้องการหลักคือให้หลักสูตรสามารถผลิตบัณฑิตที่มีอัตลักษณ์ของมหาวิทยาลัย  ได้อย่างแท้จริง บัณฑิตต้องไม่เพียงแต่มีความรู้ แต่ต้องสามารถสร้างสรรค์งานวิจัยและนวัตกรรมใหม่ๆ ที่สามารถนำไปใช้ประโยชน์ได้จริง สอดคล้องกับพันธกิจของมหาวิทยาลัย &
- การสร้างบัณฑิตที่สะท้อนอัตลักษณ์ บัณฑิตต้องเป็นนักปฏิบัติและนักสร้างสรรค์นวัตกร  \newline
- การตอบสนองต่อวิสัยทัศน์และพันธกิจ หลักสูตรต้องสอดคล้องกับเป้าหมายการเป็นมหาวิทยาลัยแห่งนวัตกรรม  \newline
- การส่งเสริมการวิจัยและนวัตกรรม ผลผลิตของหลักสูตรต้องนำไปสู่การสร้างองค์ความรู้และนวัตกรรมใหม่  &
\begin{enumerate}[label={}]
	\item PLO1
	\item PLO5
	\item PLO6
	\item PLO7
\end{enumerate}
 \\
    \hline
    อาจารย์ในหลักสูตร &
ในมุมมองของคณาจารย์ผู้สอน ความต้องการสำคัญคือการมีโครงสร้างหลักสูตรที่ร้อยเรียงเนื้อหาอย่างเป็นลำดับ (Scaffolding) เพื่อให้นักศึกษามีความรู้พื้นฐานที่แข็งแกร่งพอที่จะศึกษาต่อในรายวิชาขั้นสูงได้ราบรื่น อาจารย์ต้องการความมั่นใจว่านักศึกษาที่ผ่านวิชาพื้นฐานจะมีความพร้อมตามที่คาดหวัง เพื่อให้สามารถมุ่งเน้นการสอนเนื้อหาเชิงลึกได้เต็มที่  &
- การวางโครงสร้างหลักสูตรที่ดี รายวิชาพื้นฐานต้องส่งเสริมการเรียนรู้ในวิชาขั้นสูงได้อย่างเหมาะสม \newline
- คุณภาพความรู้พื้นฐานของนักศึกษา ความพร้อมในการต่อยอดองค์ความรู้ \newline
- ทักษะการพิสูจน์และการให้เหตุผล เป็นหัวใจสำคัญของการคิดทางคณิตศาสตร์ &
\begin{enumerate}[label={}]
	\item PLO2
	\item PLO3
	\item PLO4
\end{enumerate}
 \\
    \hline
    นักศึกษาปัจจุบันและนักศึกษาชั้นปีสุดท้าย & 
การเตรียมความพร้อมเพื่อการประกอบอาชีพในอนาคต ดังนั้นจึงต้องการให้หลักสูตรมุ่งเน้นทักษะเชิงปฏิบัติที่สามารถนำไปใช้ทำงานได้จริงและเป็นที่ต้องการของตลาดแรงงาน โดยเฉพาะทักษะการเขียนโปรแกรมคอมพิวเตอร์เพื่อแก้ปัญหาทางคณิตศาสตร์ การวิเคราะห์และประมวลผลข้อมูลขนาดใหญ่ และความสามารถในการใช้ซอฟต์แวร์ทางสถิติและคณิตศาสตร์ได้อย่างคล่องแคล่ว นักศึกษายังต้องการโอกาสในการทำโครงงานที่จำลองมาจากปัญหาในโลกธุรกิจจริง เพื่อสร้างแฟ้มสะสมผลงาน (Portfolio) และเพิ่มความสามารถในการแข่งขัน นอกจากนี้ ทักษะการสื่อสาร การทำงานเป็นทีม และการนำเสนอผลงานอย่างมืออาชีพ ถือเป็นสิ่งสำคัญที่จะช่วยให้พวกเขาปรับตัวเข้ากับวัฒนธรรมองค์กรได้ดี &
- ทักษะเชิงปฏิบัติที่พร้อมใช้งาน ความสามารถในการนำความรู้ไปใช้แก้ปัญหาจริงได้ทันที \newline
- การวิเคราะห์และประมวลผลข้อมูล ทักษะการจัดการข้อมูล การวิเคราะห์ และการนำเสนอผล \newline
- การใช้โปรแกรมและเครื่องมือเฉพาะทาง ความชำนาญในการใช้ซอฟต์แวร์ที่จำเป็นต่อสายงาน \newline
- ทักษะการสื่อสารและการทำงานเป็นทีม การทำงานร่วมกับผู้อื่นและการนำเสนออย่างมีประสิทธิภาพ &
\begin{enumerate}[label={}]
	\item PLO3
	\item PLO5
	\item PLO7
	\item PLO8
	\item PLO9
	\item PLO10
\end{enumerate}
 \\
    \hline
\end{longtable}

\begin{doclist}
\docitem{การวิเคราะห์ความต้องการของผู้มีส่วนได้ส่วนเสีย}
\end{doclist}




\subcriteria{The programme to show that the expected learning outcomes are achieved by the students by the time they graduate.}

หลักสูตรได้จัดทำระบบการประเมินที่ครอบคลุมและหลากหลายเพื่อตรวจสอบและติดตามว่าผู้สำเร็จการศึกษาสามารถบรรลุผลลัพธ์การเรียนรู้ระดับหลักสูตร (PLOs) ได้จริง หลักสูตรใช้แนวทางการประเมินแบบสามเส้า (Triangulation) เพื่อให้ได้ข้อมูลที่รอบด้านและน่าเชื่อถือ โดยรวบรวมข้อมูลจาก 3 แหล่งหลัก ดังนี้:
\begin{enumerate}
	\item การประเมินผลโดยตรง (Direct Assessment) ผ่านผลงานและการวัดผลในชั้นเรียนโดยอาจารย์\\
	เป็นการประเมินโดยพิจารณาจากผลการเรียนเฉลี่ยของนักศึกษาในกลุ่มของรายวิชาที่สนับสนุนการบรรลุ PLO นั้นๆ โดยกำหนดระดับการบรรลุ PLOs ไว้ดังนี้
	\begin{table}[h!]
	\caption{เกณฑ์การบรรลุผลลัพธ์การเรียนรู้ระดับหลักสูตร (PLOs)}
	\begin{center}
		\begin{tabular}{|c|c|}
			\hline
			\textbf{ช่วงคะแนน}&\textbf{ระดับการบรรลุ PLOs}\\\hline
		3.50-4.00&ดีมาก\\\hline
		3.00-3.49&ดี\\\hline
		2.50-2.99&ปานกลาง\\\hline
		2.00-2.49&น้อย\\\hline
		1.00-1.99&น้อยที่สุด\\\hline
	0-0.99&ไม่บรรลุ\\\hline
	\end{tabular}
	\end{center}
		\end{table}
	\item การประเมินตนเองของนักศึกษา (Graduate Self-Assessment) ผ่านแบบสำรวจนักศึกษาชั้นปีที่ 4 เมื่อสิ้นภาคการศึกษา 2/\printyear{}
	โดยให้นักศึกษาทำแบบประเมินตนเองตาม PLOs ว่าตนเองสามารถบรรลุใน PLO นั้น ๆ ได้ในระดับใด โดยกำหนดระดับการบรรลุ PLOs ไว้ดังนี้\\[-0.25cm]
		\begin{table}[h!]
		\caption{เกณฑ์การบรรลุผลลัพธ์การเรียนรู้ระดับหลักสูตร (PLOs)}
	\label{table:1.5}
	\begin{center}
	\begin{tabular}{|c|c|}
		\hline
		\textbf{ช่วงคะแนน}&\textbf{ระดับการบรรลุ PLOs}\\\hline
		4.51-5.00&ดีมาก\\\hline
		3.51-4.50&ดี\\\hline
		2.51-3.50&ปานกลาง\\\hline
		1.51-2.50&น้อย\\\hline
		1.00-1.50&น้อยที่สุด\\\hline
		0-0.99&ไม่บรรลุ\\\hline
	\end{tabular}
	\end{center}
	\end{table}
	\item การประเมินความพึงพอใจของผู้ใช้บัณฑิต (Employer Satisfaction Assessment) ซึ่งเป็นมุมมองสะท้อนกลับจากผู้มีส่วนได้ส่วนเสียภายนอก โดยใช้แบบสัมภาษณ์ผู้ใช้บัณฑิตโดยกำหนดเกณฑ์การบรรลุ PLOs ไว้ดังตาราง \ref{table:1.5}
\end{enumerate}

ในปีการศึกษา 2567 หลักสูตรดำเนินการประเมินการบรรลุุผลลัพธ์การเรียนรู้ระดับหลักสูตร (PLOs) ของนักศึกษาชั้นปีสุดท้ายที่จบหลักสูตร จำนวน 22 คน มีรายละเอียดต่อไปนี้
\begin{enumerate}
	\item การประเมินผลโดยตรงโดยอาจารย์ (Direct Assessment) มีผลการประเมินดังตาราง \ref{table:1.5-01}
\begin{longtable}{|>{\raggedright}p{0.12\linewidth}|c|c|c|c|c|c|c|}
	\caption{การบรรลุผลลัพธ์การเรียนรู้ระดับหลักสูตร (PLOs) ของนักศึกษาชั้นปีสุดท้าย ปีการศึกษา 2567}
	\label{table:1.5-01}
	\\ 
	\hline
	\centering\textbf{PLOs} & \multicolumn{7}{c|}{\textbf{ผลการบรรลุ PLOs }}\\\cline{2-8}
	&ดีมาก&ดี&ปานกลาง&น้อย&น้อยที่สุด&ค่าเฉลี่ย&การแปรผล\\\hline
	\endfirsthead
	\caption{(ต่อ)การบรรลุผลลัพธ์การเรียนรู้ระดับหลักสูตร (PLOs) ของนักศึกษาชั้นปีสุดท้าย}\\
	\hline
	\hline
\centering\textbf{PLOs} & \multicolumn{7}{c|}{\textbf{ผลการบรรลุ PLOs }}\\\cline{2-8}
&ดีมาก&ดี&ปานกลาง&น้อย&น้อยที่สุด&ค่าเฉลี่ย&การแปรผล\\\hline
	\endhead
	PLO1&2&10&10&0&0&3.06&ระดับดี\\\hline
	PLO2&3&0&11&8&0&2.70&ระดับปานกลาง\\\hline
	PLO3&3&0&12&7&0&2.70&ระดับปานกลาง\\\hline
	PLO4&3&3&10&5&1&2.78&ระดับปานกลาง\\\hline
	PLO5&3&2&13&4&0&2.84&ระดับปานกลาง\\\hline
	PLO6&22&0&0&0&0&3.75&ระดับดีมาก\\\hline
	PLO7&1&4&15&2&0&2.84&ระดับปานกลาง\\\hline
	PLO8&3&9&10&0&0&3.09&ระดับดี\\\hline
	PLO9&16&6&0&0&0&3.61&ระดับดีมาก\\\hline
	PLO10&3&3&10&6&0&2.81&ระดับปานกลาง\\\hline
	\multicolumn{6}{|l|}{\bf เฉลี่ยรวม}&{\bf 3.04}&{\bf ระดับดี}\\\hline
\end{longtable}
\item การประเมินตนเองของนักศึกษา (Graduate Self-Assessment)
 มีผลการประเมินดังตาราง \ref{table:8.4-2}
%%%%%%%%%%%%%%%%%%%%%%%%%%%%%%%%%%%%%%%%
\begin{longtable}{|>{\raggedright}p{0.12\linewidth}|c|c|c|}
	\caption{การบรรลุผลลัพธ์การเรียนรู้ระดับหลักสูตร (PLOs) ของนักศึกษาชั้นปีสุดท้าย ปีการศึกษา 2567}
	\label{table:8.4-2}
	\\ 
	\hline
	\centering\textbf{PLOs} & \multicolumn{3}{c|}{\textbf{ผลการบรรลุ PLOs }}\\\cline{2-4}
	&ค่าเฉลี่ย& SD& แปลผล\\\hline
	\endfirsthead
	\caption{(ต่อ) การบรรลุผลลัพธ์การเรียนรู้ระดับหลักสูตร (PLOs) ของนักศึกษาชั้นปีสุดท้าย}\\
	\hline
	\centering\textbf{PLOs} & \multicolumn{3}{c|}{\textbf{ผลการบรรลุ PLOs }}\\\cline{2-4}
&ค่าเฉลี่ย& SD& แปลผล\\\hline
	\endhead
	PLO1&4.68&0.31&ระดับดีมาก\\\hline
	PLO2&4.18&0.51&ระดับดี\\\hline
	PLO3&4.32&0.49&ระดับดี\\\hline
	PLO4&4.18&0.51&ระดับดี\\\hline
	PLO5&4.45&0.52&ระดับดี\\\hline
	PLO6&4.23&0.54&ระดับดี\\\hline
	PLO7&4.59&0.42&ระดับดีมาก\\\hline
	PLO8&4.18&0.51&ระดับดี\\\hline
	PLO9&4.32&0.49&ระดับดี\\\hline
	PLO10&4.32&0.49&ระดับดี\\\hline
	{\bf เฉลี่ยรวม}&{\bf 4.35}&{\bf 0.48}&{\bf ระดับดี}\\\hline
\end{longtable}
\item การประเมินความพึงพอใจของผู้ใช้บัณฑิต (Employer Satisfaction Assessment)\\
เนื่องจากหลักสูตรเพิ่งเปิดดำเนินการและยังไม่มีผู้สำเร็จการศึกษาในปีการศึกษา 2567 การประเมินความพึงพอใจจากผู้ใช้บัณฑิตจึงยังไม่สามารถดำเนินการได้ อย่างไรก็ตาม หลักสูตรได้วางแผนและจัดทำกลไกสำหรับการประเมินสมรรถนะของบัณฑิตในอนาคตไว้อย่างเป็นระบบ\\[0.2cm]
 \hspace*{0.5cm }หลักสูตรจะเริ่มดำเนินการสำรวจหลังจากบัณฑิตรุ่นแรกได้เข้าสู่ตลาดแรงงานเป็นระยะเวลา 6 เดือน ถึง 1 ปี เพื่อให้ผู้ใช้บัณฑิตมีเวลาประเมินการปฏิบัติงานจริงได้อย่างชัดเจน เครื่องมือที่ใช้คือแบบสำรวจความพึงพอใจออนไลน์ที่จะส่งไปยังผู้บังคับบัญชาโดยตรงของบัณฑิต ซึ่งหัวข้อการประเมินจะออกแบบมาให้สอดคล้องกับผลลัพธ์การเรียนรู้ที่คาดหวัง (PLOs) ของหลักสูตร เพื่อยืนยันว่าบัณฑิตมีคุณภาพและตอบสนองต่อความต้องการของตลาดแรงงานจริง ผลลัพธ์ที่ได้จะถูกนำมาใช้เป็นข้อมูลสำคัญในการปรับปรุงและพัฒนาหลักสูตรต่อไป
\end{enumerate}
\begin{doclist}
\docitem{รายงานสรุปผลการประเมินการบรรลุผลลัพธ์การเรียนรู้ระดับหลักสูตร (PLOs)}
\end{doclist}




































